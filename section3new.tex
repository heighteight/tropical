\subsection{Tropical linear algebra% and computation
}


At the basis of our approach is the observation that the \emph{tropical semiring} $([0,\infty], \min, +)$, which is at the heart of tropical mathematics, coincides with the \emph{Lawvere quantale} $\Lawv=([0,\infty], \geq, +)$ \cite{Hofmann2014, Stubbe2014}, the structure at the heart of the categorical study of metric spaces initiated by Lawvere himself \cite{Lawvere1973}.
Let us recall that a quantale is a complete lattice endowed with a continuous monoid action. In the case of $\Lawv$ the lattice is defined by the reverse order $\geq$ on $\BB R$, and the monoid action is provided by addition. Notice that the lattice join operation of $\Lawv$ coincides with the idempotent semiring operation $\min$. 
A consequence of these observations is that, as we discussed below, the tropical approach to linear algebra coincides with the study of ``$\Lawv$-valued matrices'', i.e.~of maps of the form $s: X\times Y\to \Lawv$ .
In particular, a (possibly $\infty$) metric on a set $X$ is nothing but a ``$\Lawv$-valued square matrix'' $d:X\times X\to \Lawv$ satisfying axioms like e.g.~the triangular law (indeed, such distance matrices correspond to $\Lawv$-\emph{enriched categories}, a viewpoint we explicitly take in Section \ref{section6}). 



The study of matrices with values over the tropical semiring can be seen as a special case of the
\emph{quantitative relational semantics} \cite{Manzo2013}, a well-studied semantics of the $\lambda$-calculus. 

For a fixed \emph{continuous} semi-ring $Q$ [Def. II.5, \cite{Manzo2013}], the category $\QREL$ has sets as objects and set-indexed matrices with coefficients in $Q$ as morphisms, i.e.~$\QREL(X,Y)=Q^{X\times Y}$ (\cite{Manzo2013} would call it $Q^\Pi$).
The identity morphism of $\QREL$ is the identity matrix $I:X\times X\to Q$,
% given by $i_{a,a}=1$ and $i_{a,b\neq a}=0$, 
 and composition of morphisms $t:X\times Y\to Q$ and $s:Y\times Z\to Q$ is given by 
$(st)_{a,c}:=\sum\limits_{b\in Y} s_{b,c}t_{a,b}$.
Notice that the assumption of $Q$ being continuous 
%(which can be weakened, see \cite{???})
 is used in order to make this potentially infinite sum converge.
 
 As it is expected, $Q^X$ is a $Q$-semimodule and the bijection $\hat{(\cdot)}$ identifies the set of linear maps from $Q^X$ to $Q^Y$ with 
 $\HOM{\QREL}{X}{Y}$, any map $f:Q^{X}\to Q^{Y}$ being of the form
   \begin{equation}
  f(x)_b:=\sum\limits_{a\in X} \matr f_{a,b}x_a
 \end{equation}
 for some matrix $\matr f:X\times Y\to Q$.
 
 \begin{remark}
 Following \cite{Manzo2013, Hofmann2014, Ehrhard2005}, we 
 chose to see a matrix from $X$ to $Y$ as a map $s:X\times Y\to Q$.
% 
% fix $\HOM{\QREL}{X}{Y}:=Q^{X\times Y}$ with composition $st:X\times Z\to Q$ of $s:Y\times Z\to Q$ and $t:X\times Y\to Q$ defined by $(st)_{a,c}:=\sum\limits_{b\in Y} s_{b,c}t_{a,b}$.
Notice that usual linear algebra conventions correspond to working in $\QREL^{\op}$: a matrix $X\times Y\to Q$ is usually called a ``$Y\times X$-matrix'', meaning $Y$ rows and $X$ columns, and the usual matrix-vector product defines a map $Q^Y\to Q^X$.
\end{remark}
%
%In $\QREL^{op}$ (which corresponds to systematically taking transpose matrices), composition coincides with the product matrix/matrix and $\hat{(\cdot)}$ with the product matrix/vector.
%In order to avoid confusion, we will refer to a $t\in Q^{X\times Y}$ just as a \emph{matrix from $X$ to $Y$}.
%
%
%
%We must fix a convention for matrices: following \cite{Manzo2013, Hofmann2014, Ehrhard2005}, we fix $\HOM{\QREL}{X}{Y}:=Q^{X\times Y}$ with composition $st:X\times Z\to Q$ of $s:Y\times Z\to Q$ and $t:X\times Y\to Q$ defined by $(st)_{a,c}:=\sum\limits_{b\in Y} s_{b,c}t_{a,b}$.
%In linear algebra, a map $X\times Y\to Q$ is usually called a ``$Y\times X$-matrix'', meaning $Y$ rows and $X$ columns.
%In particular, the product of such a matrix for a vector defines a map $Q^Y\to Q^X$.
%Instead, we prefer to see a $t\in\HOM{\QREL}{X}{Y}$ as giving rise to a map $\hat t:Q^X\to Q^Y$ defined by $\hat t(x)_a:=\sum\limits_{b\in Y} t_{a,b}x_a$.
%In $\QREL^{op}$ (which corresponds to systematically taking transpose matrices), composition coincides with the product matrix/matrix and $\hat{(\cdot)}$ with the product matrix/vector.
%In order to avoid confusion, we will refer to a $t\in Q^{X\times Y}$ just as a \emph{matrix from $X$ to $Y$}.
%As it is expected, $Q^X$ is a $Q$-semimodule and the bijection $\hat{(\cdot)}$ identifies $\HOM{\QREL}{X}{Y}$ with the set of linear maps from $Q^X$ to $Q^Y$.

\begin{remark}
 The category $\QREL$ is (equivalent to) a subcategory of the category $Q\SF{Mod}$ of \emph{complete} $Q$-semimodules.
 If $\QREL$ corresponds to considering semimodules (the $Q^X$'s) whose vectors are given in coordinates w.r.t.\ a \emph{fixed base} (the set $X$), $Q\SF{Mod}$ corresponds to considering semimodules in abstract, without fixing a base.
 We take this viewpoint in Section~\ref{section6}.
\end{remark}


The tropical relational model is thus provided by the category $\LREL$ of matrices with values over $\Lawv$ (which, being a quantale, is indeed a continuous semi-ring).
%, where $\Lawv$ is the already introduced Lawvere quantale, seen as the idempotent complete semiring $(\BB R_{\geq0}\cup\set{\infty},\inf,\infty,+,0)$.
%The category $\LREL$ is well-defined because $\Lawv$ is a continuous semiring (w.r.t.\ its quantale order $\preceq$.
%This amounts to check that $\min$ and $+$ commute with the $\inf$ (as operations on $\BB R_{\geq0}\cup\set{\infty}$, which is immediate), and that $(\Lawv,\preceq)$ is a cpo with $\infty$ as bottom element (which is immediate since in $\Lawv$ we have $\vee = \inf$) .
It is worth observing that the formula for composition in $\LREL$ %is the tropicalisation of the one defining it in $\QREL$, i.e.
reads as \ $(st)_{a,c}:=\inf\limits_{b\in Y}\set{s_{b,c}+t_{a,b}}$;
 similarly, the linear functions $f:\Lawv^X\to \Lawv^Y$ induced by matrices, which we call \emph{tropical linear}, are exactly those of shape $f(x)=\inf\limits_{b\in Y} \set{\matr f_{a,b}+x_a}$, for some matrix $\matr f$ from $X$ to $Y$.
%\end{remark}

Since $\Lawv$ is a continuous (commutative) semiring, [Proposition III.3, \cite{Manzo2013}] immediately applies and gives:

\begin{fact}
 $\LREL$ is a linear $\Lawv$-category.
\end{fact}

Unwrapping [Definition II.9, \cite{Manzo2013}], this means that:
$\HOM{\LREL}{X}{Y}$ is a continuous $\Lawv$-semimodule, with semimodule operations defined pointwise;
$\LREL$ is a continuous $\Lawv$-category, i.e.\ composition of morphisms commutes with $\inf$'s;
$\LREL$ is linear, i.e.\ pre- and post-composition with any morphism in any $\HOM{\LREL}{X}{Y}$ are automorphisms on it.

In the next sections we will see how $\LREL$ gives rise to denotational models of several variants of the $\lam$-calculus.



%
%
%
% 

%
%
%or more generally as a power series $f(x)=\sum_{n}\widehat f_{n}x^{n}$ with coefficient $\widehat f_{n}\in [0,1]$, we can define its \emph{tropicalization} $\trop f: \Lawv \to \Lawv$ as the function 
%\begin{align}
%\trop f(\alpha)= \inf_{n}\left\{ 
%\end{align}
%
%This correspondence can be made precise through the the so-called
% \emph{de Maslov dequantization} \cite{}.
% For each positive real $t$, any polynomial in $\BB R[x]$ can be written under the \emph{$t$-parameterized} form:
% \begin{align}
% p_{t}(x)= \sum_{i=1}^{k}t^{c_{i}}x^{i}
% \end{align}
% with the coefficients $c_{i}$ taken from $\Lawv$. 
% It is clear then that tropical polynomials and $t$-parameterized polynomials admit a one to one correspondence between their presentations.
% 
% Actually, the $\varphi$s and the $p_{t}$s can be related by passing through some intermediate functions $\varphi_{t}$ introduced by Maslov.
%For any $t>1$, the functions $\phi_{t}(x)=-\log_{t}x$ and $\psi_{t}(\alpha)=t^{-\alpha}$ are inverse of each other and define thus continuous (w.r.t.\ the usual topologies) bijections between the space of probabilities $[0,1]$ and $\BB R_{\geq 0}\cup\set{\infty}$ (we write $\log$ for the natural logarithm).
%Moreover, if we set $\alpha \widetilde+ \beta:= \frac{\alpha+\beta}{2}$, $\alpha\sumt{t}\beta=\phi_{t}(\psi_{t}(\alpha)\widetilde{+}\psi_{t}(\beta))=-\log (e^{-\alpha/t}+e^{-\beta/t})-\phi_{t}(2)$ and $\alpha\prodt{t} \beta:=\phi_{t}(\psi_{t}(\alpha)\psi_{t}(\beta))=\alpha+\beta$, it is known that: $\lim\limits_{t\to 0}\alpha\sumt{t} \beta= \min\{\alpha,\beta\}$.
%In this sense, setting $\Lawv_t:=([0,\infty],\sumt{t},\prodt{t})$, one says that $\Lawv_t\to_{t\to 0^+}\Lawv$.
%Moreover, setting $\widetilde\Lawv:=([0,\infty],\widetilde+,\cdot)$, it can be shown that $\Lawv_t\simeq\widetilde\Lawv$ for all $t>0$, so the $\Lawv_t$ are all isomorphic, whereas at the limit we have a discontinuity: it can be shown that $\Lawv_t\not\simeq\Lawv$.
% 

%
%{\color{red}Lista delle cose da dire:}
%
%1) Def di quantale, come lattice e come complete idempotent semiring.
%Among the the so-called \emph{tropical semirings}, we consider the \emph{Lawvere quantale/semiring}.
%
%2) Def di $\Lawv$, the \emph{Lawvere quantale}: seen as the idempotent complete semiring, it is $(\BB R_{\geq0}\cup\set{\infty},\inf,\infty,\cdot,0)$.
%Seen as lattice it is defined by the order $\preceq$, which is the reversed order $\geq$ of the usual order $\leq$ on $\BB R_{\geq0}\cup\set{\infty}$.
%
%3) Maslov dequantisation:
%
%First, let us recall that for any non-negative real $t$, the functions $\phi_{t}(x)=-t\log x$ and $\psi_{t}(\alpha)=e^{-\alpha/t}$ are inverse of each other and define thus continuous (w.r.t.\ the usual topologies) bijections between the space of probabilities $[0,1]$ and $\BB R_{\geq 0}\cup\set{\infty}$ (we write $\log$ for the natural logarithm).
%Moreover, if we set $\alpha \widetilde+ \beta:= \frac{\alpha+\beta}{2}$, $\alpha\sumt{t}\beta=\phi_{t}(\psi_{t}(\alpha)\widetilde{+}\psi_{t}(\beta))=-\log (e^{-\alpha/t}+e^{-\beta/t})-\phi_{t}(2)$ and $\alpha\prodt{t} \beta:=\phi_{t}(\psi_{t}(\alpha)\psi_{t}(\beta))=\alpha+\beta$, it is known that: $\lim\limits_{t\to 0}\alpha\sumt{t} \beta= \min\{\alpha,\beta\}$.
%In this sense, setting $\Lawv_t:=([0,\infty],\sumt{t},\prodt{t})$, one says that $\Lawv_t\to_{t\to 0^+}\Lawv$.
%Moreover, setting $\widetilde\Lawv:=([0,\infty],\widetilde+,\cdot)$, it can be shown that $\Lawv_t\simeq\widetilde\Lawv$ for all $t>0$, so the $\Lawv_t$ are all isomorphic, whereas at the limit we have a discontinuity: it can be shown that $\Lawv_t\not\simeq\Lawv$.
%
%4) Def di $\trop$ di un polinomio/serie, (\`e la stessa formula, dipende solo se gli indici sono finiti/infiniti).
%Come sta scritto sotto, giusto un po' pi\`u formale (per esempio, scriverlo come Definizione).
%
%The fundamental observation that led to the study of mathematics over the \emph{tropical semi-ring} $\Lawv=([0,\infty],\min,+)$ was that, by replacing everywhere the ``$+$'' by the ``$\min$'' and the ``$\times$'' by the ``$+$'', many algebraic and geometric objects becomes combinatorial and their computation simpler. 
%
%For instance, the tropicalization of a cubic polynomial $p(x)=ax^{3}+bx^{2}+cx+d$ yields a piecewise-linear function 
%\begin{align}
%\trop p(\alpha)= \min\{ 3\alpha+a, 2\alpha+b, \alpha+c,d\}
%\end{align}
%Notably, the \emph{tropical roots} (whose definition is recalled in Section \ref{section3}) of $\trop p(\alpha)$ can be found through a rather simple (indeed polytime \cite{}) algorithm, and can be used to \emph{approximate} the actual roots of $p(x)$ \cite{}. 
%More generally, the tropicalization of a power series $f(x)=\sum_{n}\widehat f_{n}x^{n}$ yields a \emph{tropical Laurent series} \cite{} 
%\begin{align}
%\trop f(\alpha)= \inf_{n}\left\{n\alpha+ \widehat f_{n}\right\}
%\end{align}
%a class of functions that we will study in detail in Section \ref{section4}.
%
%%- generalities about tropical maths (tropicalisation $\trop P$  of polynomials and of Laurent series, and their roots -- all that without $\LREL$)
%
%
%

