
The connections between the differential $\lambda$-calculus (and differential linear logic), the relational semantics, and the theory of non-idempotent intersection types is very well-studied (see \cite{}, and more recently, \cite{} for a more abstract perspective, and \cite{} for a 2-categorical, or proof-relevant, extension).
As we said, the relational semantics over the tropical semi-ring was quickly explored in \cite{}, to provide a ``best case'' resource analysis of a $\B{PCF}$-like language with non-deterministic choice. 
\emph{Probabilistic coherent spaces} \cite{}, a variant of  the relational semantics, provide an interpretation of higher-order probabilistic programs
as analytic functions. In \cite{} it was observed that such functions satisfy a local Lipschitz condition somehow reminiscent of our examples in Section \ref{section4}.


The study of linear or bounded type systems for sensitivity analysis was initiated in \cite{} and later developed \cite{}.
As recalled in the paper, the use bounded exponentials ensures that well-typed programs satisfy a Lipschitz condition.
Related approaches, although not based on metrics, are provided by \emph{differential logical relations} \cite{} and \emph{change action} models \cite{}.


More generally, the literature on program metrics in denotational semantics is vast. Since [6] metric spaces, also in Lawvere's generalized sense \cite{}, have been exploited as an alternative framework to standard, domain-theoretic, denotational semantics. 
While standard categories of metric spaces are not models of the full simply typed $\lambda$-calculus, several constructions of cartesian closed categories of metric spaces can be found in the literature. For instance, 
\emph{ultra}-metric spaces \cite{} form a CCC, and have been shown to model $\B{PCF}$ \cite{}.
Also \emph{partial} metrics, introduced in \cite{}, have been shown to provide models of $\STLC$, under suitable generalizations \cite{}.
More generally, \cite{} provides a general characterization of exponentiable objects in categories of (generalized) metric spaces, and \cite{} provides other ways to construct CCCs of
(generalized) metrics, including one based on locally Lipschitz maps, 
using ideas from {differential logical relations}.

Motivated by connections with computer science and fuzzy set-theory, 
the abstract study of generalized metric spaces in the framework of \emph{quantale}- or even \emph{quantaloid}-enriched categories has led to a vast literature in recent years \cite{}, 
and its connections with tropical mathematics are discussed in \cite{}. Moreover, applications of quantale-modules to both logic and computer science have also been studied \cite{}.

Connections between program metrics and the differential $\lambda$-calculus have been already suggested in \cite{}; moreover, \emph{cartesian difference categories} \cite{} have been proposed as a way to relate derivatives in differential categories with those found in change action models.



Finally, applications of tropical mathematics in computer science abound, notably in optimization methods for machine learning \cite{} (typically, for neural networks with peicewise linear activation), linear regression \cite{}, convex analysis \cite{}, as well as finite automata \cite{}.
While our discussion in Section \ref{section5} is inspired by a well-known application of tropical polynomials \cite{},  
the vast literature in this domain lets us think that other ways to 
apply tropical semantics to the analysis of higher-order programs might be studied.

%
%Other connections tropical/metric -> Fuji, ??
%Applications of tropical to computer science.
%Log-probabilities.
%Quantale-modules -> Abramski?
%
%
 









