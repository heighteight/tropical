
The connections between differential $\lambda$-calculus (and differential linear logic), relational semantics, and non-idempotent intersection types are very well-studied (see \cite{decarvalho2018}, and more recently, \cite{Mazza2016} for a more abstract perspective, and \cite{Olimpieri2021, Galal2021} for a 2-categorical, or proof-relevant, extension).
As we said, the relational semantics over the tropical semi-ring was quickly explored in \cite{Manzo2013}, to provide a ``best case'' resource analysis of a $\B{PCF}$-like language with non-deterministic choice. 
\emph{Probabilistic coherent spaces} \cite{Ehrhard2011}, a variant of  the relational semantics, provide an interpretation of higher-order probabilistic programs
as analytic functions. In \cite{Ehrhard2022} it was observed that such functions satisfy a local Lipschitz condition somehow reminiscent of our examples in Section \ref{section4}.


The study of linear or bounded type systems for sensitivity analysis was initiated in \cite{Girard92tcs} and later developed \cite{Schopp, SchoppDalLago, Reed2010}.
%As recalled in the paper, the use bounded exponentials ensures that well-typed programs satisfy a Lipschitz condition.
Related approaches, although not based on metrics, are provided by \emph{differential logical relations} \cite{dallago} and \emph{change action} models \cite{Picallo2019}.


More generally, the literature on program metrics in denotational semantics is vast. Since at least \cite{VANBREUGEL20011} metric spaces, also in Lawvere's generalized sense \cite{Lawvere1973}, have been exploited as an alternative framework to standard, domain-theoretic, denotational semantics. 
While standard categories of metric spaces are not models of the full simply typed $\lambda$-calculus, several constructions of cartesian closed categories of metric spaces can be found in the literature. For instance, 
\emph{ultra}-metric spaces form a CCC, and have been shown to model $\B{PCF}$ \cite{Escardo1999}.
Also \emph{partial} metrics, introduced in \cite{matthews}, have been shown to provide models of $\STLC$, under suitable generalizations \cite{Geoffroy2020}.
More generally, \cite{Clementino2006} provides a general characterization of exponentiable objects in categories of (generalized) metric spaces, and \cite{PistoneLICS, PistoneFSCD2022} provide other ways to construct CCCs of
(generalized) metrics, including one based on locally Lipschitz maps, 
using ideas from {differential logical relations} and quantitative equational theories \cite{Plotk}.

Motivated by connections with computer science and fuzzy set-theory, 
the abstract study of generalized metric spaces in the framework of \emph{quantale}- or even \emph{quantaloid}-enriched categories has led to a vast literature in recent years \cite{Hofmann2014, Stubbe2014}, 
and its connections with tropical mathematics have been explored e.g.~in \cite{Fuji, Willerton2013}. Moreover, applications of quantale-modules to both logic and computer science have also been studied \cite{Abramsky1993b, Russo2007}.

Connections between program metrics and the differential $\lambda$-calculus have been already suggested in \cite{PistoneLICS}; moreover, \emph{cartesian difference categories} \cite{Picallo2020} have been proposed as a way to relate derivatives in differential categories with those found in change action models.



Finally, applications of tropical mathematics in computer science abound, originally for automata theory \cite{Chua1992, Simon}, and more recently
 in optimization methods for machine learning and other statistical models (see e.g.~\cite{Maragos2021, Pachter2004, Zhang2018}), optimization \cite{Akian2011, Akian2012}, and convex analysis \cite{Lucet2009}. While our discussion in Section \ref{section5} is inspired by a well-known application of tropical polynomials to Hidden Markov Models \cite{Pachter2004},  
the vast literature in this domain lets us think that other ways to 
apply tropical semantics to the analysis of higher-order programs might be studied.

%
%Other connections tropical/metric -> Fuji, ??
%Applications of tropical to computer science.
%Log-probabilities.
%Quantale-modules -> Abramski?
%
%
 









