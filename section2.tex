%Recall the two approaches with more details on lambda-calculus and on existing challenges.


In this section we discuss in some more detail the two approaches in quantitative semantics recalled in the Introduction, at the same time providing an overview of how we aim at bridging them using tropical mathematics.

\subsection{Metric approach: bounded $\lambda$-terms as Lipschitz functions}


As recalled in the Introduction, in many situations (typically, when dealing with computationally expensive function) one does not look for algorithms to compute some function \emph{exactly}, but rather to approximate it (e.g.~in an efficient way) within some error bound. In other common situations (typically, in differential privacy \cite{}) one needs to verify that an algorithm is not \emph{too sensitive} to errors, that is, that a small error in the input will produce a comparably small error in the output. 

A natural way to account for these situations is to consider denotational semantics in which types are endowed with a \emph{behavioral metric}, that is, a metric on programs which measures differences of behavior. 
A fundamental insight coming from this line of works is that \emph{affine} programs, that is, programs that may use their input at most once, correspond to \emph{non-expansive} (or $1$-Lipschitz) maps, that is, to functions $f$ for which the distances
$d(f(x),f(y))$ produced in output are bounded by the distances $d(x,y)$ in input. 
A more formal way of stating this observation is that the category $\Met$ of pseudo-metric spaces and non-expansive provides a model of the \emph{affine} simply typed $\lambda$-calculus, being both a \emph{symmetric monoidal closed} category (i.e.~a model of \emph{linear} $\STLC$) and a cartesian category in which cartesian and monoidal unit coincide. 

As observed in \cite{Gabo2017}, the metric approach is not restricted to affine programs, but can be extended to programs with \emph{bounded} duplications. The fundamental intuition is that a program duplicating its input $K$ times will give rise to a $K$-\emph{Lipschitz-continuous} map.
For instance, the higher-order program $M=\lambda f.\lambda x.f(f(x))$, which duplicates the functional input $f$, yields a $2$-Lipschitz map between the metric space $\BB R\multimap \BB R$ of non-expansive real functions and itself: if $f,g$ are two non-expansive maps differing by at most $\epsilon$ (i.e.~for which $|f(x)-g(x)|\leq \epsilon$ holds for all $x\in \BB R$), then the application of $M$ to $f$ and $g$ will produce two maps differing by at most $2\epsilon$. 
By observing that a $r$-Lipschitz map between metric spaces $X$ and $Y$ is the same as a non-expansive map between the \emph{re-scaled} space $r\cdot X$ (i.e.~with distance $d_{r}(x,y)=r\cdot d(x,y)$) and $Y$, the program $M$ above 
can thus be interpreted as a non-expansive map from $2\cdot(\BB R\multimap \BB R)$ to $\BB R\multimap \BB R$.

These observations have led to the study of \emph{bounded} typed $\lambda$-calculi (like the system $\mathsf{Fuzz}$ \cite{Reed_2010}), which have been applied in the study of higher-order differential privacy \cite{}. The types of such systems are defined \cite{} by combining linear constructors with a \emph{graded exponential comonad} $!_{r}(-)$ \cite{}.
In the following sections we will sometimes make reference to a basic type system, that we call $\BSTLC$, for bounded higher-order programs, with types defined as below
\begin{align}
A::= o \  \mid \  (!_{\leq n}A) \multimap A  
\end{align}
where $n$ indicates a natural number. Intuitively, $!_{\leq n}A\multimap B$ is the type of functions from $A$ to $B$ that may use their input at most $n$ times. More details about $\BSTLC$ are provided in the Appendix.


Now, what about the good old, ``unbounded'', simply typed $\lambda$-calculus? Actually, by using unbounded duplications, one might lose the Lipschitz property. For instance, while the functions $M_{k}=\lambda x. k\cdot x: \BB R\to \BB R$ are all Lipschitz-continuous, with Lipschitz constant $k$, the function $M=\lambda x.x^{2}$ obtained by ``duplicating'' $x$ is not Lipschitz anymore: $M$ is, so to say, \emph{too} sensitive to errors. 
More abstractly, it is well-known that the category $\Met$ is \emph{not} cartesian closed, so it is not a model of $\STLC$ (yet, $\Met$ contains several cartesian closed \emph{sub-}categories, see e.g.~\cite{}).
Still, one might observe that the program $M$ above is actually Lipschitz-continuous, if not globally, at least \emph{locally} (i.e.~ over any compact set). However, constructing cartesian closed categories of locally Lipschitz maps is challenging. Yet, some examples exist in the literature (see \cite{EhrhardPCOHdiff, PistoneLICS2021}), and a new example will be exhibited in this paper.


\subsection{Resource approach: differential $\lambda$-terms as polynomials}

A different family of approaches to linearity and duplication arises from the study of the \emph{differential $\lambda$-calculus} (and differential linear logic) and its categorical models. 
The key ingredient is that a \emph{differential operator} $\Der$ is added to the usual syntax of the $\lambda$-calculus, with the intuition that, given $M$ of type $A\to B$ and $N$ of type $A$, the program $\Der[M]\cdot N$, which still has type $A\to B$, corresponds to a \emph{linear application} of $N$ to $M$: this means that $N$ is passed to $M$ so that the latter may use it exactly once (this is why $\Der[M]\cdot N$ has type $A\to B$, since $M$ might still ask for other copies of an input of type $A$). 

Interestingly, the categorical study of such differential operators has led to introduce a family of categories, called \emph{cartesian differential categories} $C\partial C$ \cite{}, which provide an abstract axiomatization of differentiation, in the usual mathematical sense. More precisely, a cartesian category $\C C$ is a $C\partial C$ when:
\begin{itemize}
\item $\C C$ is left-additive, i.e.~its hom-sets are commutative monoids, and the cartesian structure is well-behaved w.r.t.~this monoid structure;
\item $\C C$ is equipped with a differential operator $D:
\C C(X,Y)\to \C C(X\times X,Y)$ satisfying 7 axioms $D1,\dots, D7$, which translate usual properties of the differential (e.g.~the linearity of $D$ in one of its two variables, the chain rule,  Schwarz's theorem etc.).
\end{itemize}

Correspondingly, the syntax of the simply typed \emph{differential} $\lambda$-calculus ($\STDLC$) is defined by enriching $\STLC$ with a monoid structure $0,+$ over terms, as well as with the differential operator $\Der[-]\cdot (-)$, together with a notion of \emph{linear substitution} (see \cite{} for details).
The models of $\STDLC$ are the 
cartesian \emph{closed} differential categories ($CC\partial C$), 
which are defined as $C\partial C$ which are also cartesian closed, and in which the monoid structure and the differential are both well-behaved with respect to the closed structure (see \cite{}). 


Another suggestive similarity between program derivatives and 
actual derivatives is provided by the \emph{Taylor expansion}:
in $\STDLC$ one can prove that any application $MN$ can be expanded as a sum of \emph{linear} applications
$\Der^{(k)}[M]\cdot N^{k}$, i.e.~where $N$ is passed exactly $k$ times to $M$, as expressed below:
\begin{align}\label{eq:taylor}
MN  =  \sum_{k=0}^{\infty}\frac{1}{!k}\cdot (\Der^{(k)}[M] \cdot N^{k})0
\end{align}
In other words, terms with unbounded duplications can be seen as some sort of ``limits'' of bounded, but arbitrarily large, duplications.% (this perspective is made clearer by the related approach of the \emph{resource $\lambda$-calculus} \cite{}).

%More formally, the differential operator $\Der[-]$ transforms a function $M:A\to B$ into a function $\Der[M]: A\to (A\to B)$ which is linear in its first argument. 
%Since $M$ may rather ask for several copies of $N$, this requires a form of non-determinism: 
%For example, if $M$ is the term $\lambda fx.f(fx)$ considered before, $\Der[M]$ takes a first input $N$ and passes it linearly to $M$. Notice that there are two ways of doing so, corresponding to the two bound occurrences of $f$ in $M$: either by applying $N$ to $fx$, or by 
%applying $f$ linearly to $Nx$ (indeed, if $f$ were applied in an unrestricted way, it might duplicate $Nx$, so that $N$ would not be used linearly). This justifies the equation below, in which $\Der[M]$ is identified with the non-deterministic sum of the two possible linear choices:
%\begin{align}
%\Der\left[\lambda f x.f(fx)\right]\cdot N = 
%\lambda fx. N(fx) + \left(\Der[f]\cdot (Nx)\right)(fx)
%\end{align}
%More generally, one can define a notion of $k$-bounded application $\Der^{(k)}[M]\cdot N^{k}$, where $\Der^{(0)}[M]\cdot N^{0}= M$ and $\Der^{(k+1)}[M]\cdot N^{k+1}= \Der[ \Der^{(k)}[M]\cdot N^{k}]\cdot N$, corresponding to passing $N$ to $M$ exactly $k$ times.
%
%
%The name ``differential'' for the operator $\Der[-]$ is justified by the fact that it satisfies many properties of the usual differential operator of analysis $\Der[f]:= \lambda xy. \frac{\mathsf df(y)}{\mathsf dy}\cdot x$. Notably, it is additive in its first variable (i.e.~it commutes with the non-deterministic sum operator), and satisfies the chain rule.
%Most famously, the differential operator can be used to define a Taylor formula for $\lambda$-terms, which decomposes an unrestricted application into a formal non-deterministic sum of bounded applications:

At this point, given our previous discussion of bounded duplications and Lipschitz-continuity, a natural question would be whether the Taylor formula
could be used to decompose an unbounded $\lambda$-term 
into a sum of Lipschitz maps, thus bridging the metric and differential approach.  
While suggestive, an obstacle to realize this idea is that,
in general, a $C\partial C$ needs not satisfy the Taylor formula
\eqref{eq:taylor}, and in the \emph{relational semantics}, 
a well-studied and sort of canonical ``Taylor'' semantics for $\STDLC$, the bounded terms appearing in the Taylor expansion correspond to \emph{polynomials}, and thus to non-Lipschitz functions. 


Our question can thus be reformulated as follows: can we make the relational semantics \emph{Lipschitz}, hence amenable to metric and sensitivity analysis? The goal of this paper is to show that, by appealing to tropical mathematics, this is indeed possible and leads to a surprising 
bridge between the metric and differential study to higher-order programs.
%
%More generally, the relational semantics interprets unbounded programs as \emph{analytic functions}, that is, as functions admitting a representation as power series. For instance, observing that an analytic map $f: \BB R\to \BB R$, where $f(x)=\sum_{n}\widehat f_{n}\cdot x^{n}$ is uniquely determined by the sequence $\widehat f_{n}$, the program $M_{\infty}:=\lambda fx.fx: (\BB R\To \BB R)\To (\BB R\To \BB R)$ is represented by the power series below:
%\begin{align}
%F_{\infty}(f,x)= \sum_{n=0}^{\infty} \widehat f_{n} x^{n}
%\end{align}
%By restricting ourselves to bounded applications, the terms in the power series become finite, that is, the interpretation becomes a \emph{polynomial}: for instance, the program $M_{2}:=\lambda fx. \sum_{i=0}^{2}\Der^{(i)}[f]\cdot x^{i}$, corresponding to passing $x$ \emph{at most twice} to $f$, is represented by the polynomial
%\begin{align}
%F_{\leq 2}(f,x)=\widehat f_{2} x^{2}  + \widehat f_{1}x +  \widehat f_{0} 
%\end{align}
% In this framework the differential operator is naturally represented by formal differentiation of polynomials, where, as one would expect, 
% $\Der[\sum_{n}a_{n}x^{n}]=\sum_{n}\Der[a_{n}x^{n}]$ and $\Der[a_{0}x^{0}]=0$ and $\Der[a_{n+1}x^{n+1}]= (n+1)a_{n+1}x^{n}$, so that power series can be Taylor expanded. 


\subsection{Outline of the paper}

In the next section we recall some basic ideas from tropical mathematics, and its connection with the study of the Lawvere quantale.
%Since polynomials correspond to piecewise linear (hence Lipschitz) functions in tropical mathematics,
The reconstruction of the relational semantics over the tropical semi-ring, presented in Section \ref{section3} and Section \ref{section4}, will provide a metric semantics of the differential $\lambda$-calculus, bridging sensitivity and resource analysis. 
In Section \ref{section5} we suggest potential applications of this approach, relating well-studied applications of program metrics, resource analysis with current uses of tropical mathematics in computer science.  
Finally, in Section \ref{section6} we show that the connection between the metric and differential analysis of higher-order programs extends well beyond the relational semantics, through a more abstract correspondence between  
\emph{generalized metric spaces} (in the sense of Lawvere \cite{}) and modules over the tropical semi-ring.


 
\section{Tropical mathematics: a possible synthesis}

At the basis of our approach is the observation that the \emph{tropical semiring} $([0,\infty], \min, +)$, which is at the heart of tropical mathematics, coincides with the \emph{Lawvere quantale} $\Lawv=([0,\infty], \geq, +)$ \cite{}, the structure at the heart of the categorical study of metric spaces initiated by Lawvere himself \cite{}.
Let us recall that a quantale is a complete lattice endowed with a continuous monoid action. In the case of $\Lawv$ the lattice is defined by the reverse order $\geq$, and the monoid action is provided by addition. Notice that the lattice join operation of $\Lawv$ coincides with the idempotent semiring operation $\min$. 
A consequence of these observation is that, as discussed in Section \ref{section3}, the tropical approach to linear algebra coincides with the study of ``$\Lawv$-valued matrices'', i.e.~of maps of the form $s: X\times Y\to \Lawv$ .
In particular, a (pseudo-)metric on a set $X$ is nothing but a ``$\Lawv$-valued matrix'' $d:X\times X\to \Lawv$ satisfying axioms like e.g.~the triangular law (indeed, such distance matrices correspond to $\Lawv$-\emph{enriched categories}, a viewpoint we discuss in Section \ref{section6}). 

Going beyond linear algebra, a \emph{tropical polynomial} is defined as a piecewise linear function $\varphi:\Lawv\to \Lawv$ of the form 
\begin{align}\label{eq:polytrop}
\varphi(\alpha)= \min_{i=1,\dots,k}\left\{ i\alpha + c_{i}\right\}
\end{align}
where the coefficients $c_{i}$ are taken from $\Lawv$. 
%A value $\alpha\in \Lawv$ is a \emph{root} of the polynomial $P$ when
%the minimum at $\varphi(\alpha)$ is attained at least twice (equivalently, when 
% $\varphi$ is not differentiable at $\alpha$). In other words, the tropical roots of $\varphi$ coincide with the points where the slope of $\varphi$ changes. 
%
Intuitively, tropical polynomials look much like standard polynomials, although with ``$+$'' replaced by ``$\min$'', and ``$\times$'' replaced by ``$+$''. 
In fact, this intuition can be made precise as follows: for any positive real $t$, the tropical polynomials like $\varphi$ are in one-to-one correspondence with the functions $f:[0,1]\to [0,1]$ which can be written as a \emph{parameterized} polynomial 
$f_{t}(x)= \sum_{i=1}^{n}t^{c_{i}}x^{n}$, with the $c_{i}\in [0,\infty]$. 
%Hence, for any polynomial $p(x)= 
%For instance, the tropicalization of a cubic polynomial $p(x)=ax^{3}+bx^{2}+cx+d$ yields a piecewise-linear function 
%\begin{align}
%\trop p(\alpha)= \min\{ 3\alpha+a, 2\alpha+b, \alpha+c,d\}
%\end{align}






More generally, for any function $f:[0,1]\to [0,1]$ which can be written as a (parameterized) \emph{power series} of the form $f_{t}(x)= \sum_{n}t^{c_{n}}x^{n}$ (as we'll see, such functions arise naturally from the interpretation of probabilistic programs, as we'll recall in Section \ref{section5}), we can define its \emph{tropicalization} $\trop f: \Lawv \to \Lawv$
as the function defined as follows:
\begin{align}
\trop f(\alpha) =\inf_{n}\left\{ n\alpha+c_{n}\right\}
\end{align}
Such functions, called \emph{tropical Laurent series} \cite{}, will be discussed in more detail in Section \ref{section5}.

The functions $f$ and $\trop f$ can be related by a limit passage as follows: the functions $\phi_{t}(x)= -\log_{t}x$ and $\varphi_{t}(\alpha)= t^{-\alpha}$ define continuous bijections between $[0,1]$ and $[0,\infty]$ and, by letting
$\trop_{t}f: [0,\infty]\to [0,\infty]$ be defined by 
$\trop_{t}f(\alpha)= \phi_{t}\circ f \circ \psi_{t}$, one has that 
$\trop f= \lim_{t\to 0}\trop_{t}f$. 
Indeed, one can check that the ``parameterized'' sums and product $\alpha \sumt{t}\beta:= \phi_{t}(\psi_{t}(\alpha)+\psi_{t}(\beta))= \log_{t}(t^{-\alpha}+t^{-\beta})$ and 
$\alpha \prodt{t}\beta:= \phi_{t}( \psi_{t}(\alpha)\psi_{t}(\beta))=
\log_{t}(t^{-\alpha}t^{-\beta})$ converge respectively to $\min\{\alpha,\beta\}$ and $\alpha+\beta$, when $t\to 0$ (this remark is an example of the so-called \emph{Maslov dequantization}, relating the semi-ring $\BB R$ with the tropical semiring $((-\infty, \infty], \min, +)$, see e.g.~\cite{}).



Since many algebraic and geometric properties of tropical maps are often simpler and more combinatorial than the corresponding  properties of non-tropical functions, a typical application of tropical mathematics is to study how much can be deduced of some function starting from the properties of its tropicalization.
In Section \ref{section5} we will follow a similar direction, investigating what quantitative properties of a higher-order programs are revealed by the study of its tropical interpretation.

 

%
%
%or more generally as a power series $f(x)=\sum_{n}\widehat f_{n}x^{n}$ with coefficient $\widehat f_{n}\in [0,1]$, we can define its \emph{tropicalization} $\trop f: \Lawv \to \Lawv$ as the function 
%\begin{align}
%\trop f(\alpha)= \inf_{n}\left\{ 
%\end{align}
%
%This correspondence can be made precise through the the so-called
% \emph{de Maslov dequantization} \cite{}.
% For each positive real $t$, any polynomial in $\BB R[x]$ can be written under the \emph{$t$-parameterized} form:
% \begin{align}
% p_{t}(x)= \sum_{i=1}^{k}t^{c_{i}}x^{i}
% \end{align}
% with the coefficients $c_{i}$ taken from $\Lawv$. 
% It is clear then that tropical polynomials and $t$-parameterized polynomials admit a one to one correspondence between their presentations.
% 
% Actually, the $\varphi$s and the $p_{t}$s can be related by passing through some intermediate functions $\varphi_{t}$ introduced by Maslov.
%For any $t>1$, the functions $\phi_{t}(x)=-\log_{t}x$ and $\psi_{t}(\alpha)=t^{-\alpha}$ are inverse of each other and define thus continuous (w.r.t.\ the usual topologies) bijections between the space of probabilities $[0,1]$ and $\BB R_{\geq 0}\cup\set{\infty}$ (we write $\log$ for the natural logarithm).
%Moreover, if we set $\alpha \widetilde+ \beta:= \frac{\alpha+\beta}{2}$, $\alpha\sumt{t}\beta=\phi_{t}(\psi_{t}(\alpha)\widetilde{+}\psi_{t}(\beta))=-\log (e^{-\alpha/t}+e^{-\beta/t})-\phi_{t}(2)$ and $\alpha\prodt{t} \beta:=\phi_{t}(\psi_{t}(\alpha)\psi_{t}(\beta))=\alpha+\beta$, it is known that: $\lim\limits_{t\to 0}\alpha\sumt{t} \beta= \min\{\alpha,\beta\}$.
%In this sense, setting $\Lawv_t:=([0,\infty],\sumt{t},\prodt{t})$, one says that $\Lawv_t\to_{t\to 0^+}\Lawv$.
%Moreover, setting $\widetilde\Lawv:=([0,\infty],\widetilde+,\cdot)$, it can be shown that $\Lawv_t\simeq\widetilde\Lawv$ for all $t>0$, so the $\Lawv_t$ are all isomorphic, whereas at the limit we have a discontinuity: it can be shown that $\Lawv_t\not\simeq\Lawv$.
% 

%
%{\color{red}Lista delle cose da dire:}
%
%1) Def di quantale, come lattice e come complete idempotent semiring.
%Among the the so-called \emph{tropical semirings}, we consider the \emph{Lawvere quantale/semiring}.
%
%2) Def di $\Lawv$, the \emph{Lawvere quantale}: seen as the idempotent complete semiring, it is $(\BB R_{\geq0}\cup\set{\infty},\inf,\infty,\cdot,0)$.
%Seen as lattice it is defined by the order $\preceq$, which is the reversed order $\geq$ of the usual order $\leq$ on $\BB R_{\geq0}\cup\set{\infty}$.
%
%3) Maslov dequantisation:
%
%First, let us recall that for any non-negative real $t$, the functions $\phi_{t}(x)=-t\log x$ and $\psi_{t}(\alpha)=e^{-\alpha/t}$ are inverse of each other and define thus continuous (w.r.t.\ the usual topologies) bijections between the space of probabilities $[0,1]$ and $\BB R_{\geq 0}\cup\set{\infty}$ (we write $\log$ for the natural logarithm).
%Moreover, if we set $\alpha \widetilde+ \beta:= \frac{\alpha+\beta}{2}$, $\alpha\sumt{t}\beta=\phi_{t}(\psi_{t}(\alpha)\widetilde{+}\psi_{t}(\beta))=-\log (e^{-\alpha/t}+e^{-\beta/t})-\phi_{t}(2)$ and $\alpha\prodt{t} \beta:=\phi_{t}(\psi_{t}(\alpha)\psi_{t}(\beta))=\alpha+\beta$, it is known that: $\lim\limits_{t\to 0}\alpha\sumt{t} \beta= \min\{\alpha,\beta\}$.
%In this sense, setting $\Lawv_t:=([0,\infty],\sumt{t},\prodt{t})$, one says that $\Lawv_t\to_{t\to 0^+}\Lawv$.
%Moreover, setting $\widetilde\Lawv:=([0,\infty],\widetilde+,\cdot)$, it can be shown that $\Lawv_t\simeq\widetilde\Lawv$ for all $t>0$, so the $\Lawv_t$ are all isomorphic, whereas at the limit we have a discontinuity: it can be shown that $\Lawv_t\not\simeq\Lawv$.
%
%4) Def di $\trop$ di un polinomio/serie, (\`e la stessa formula, dipende solo se gli indici sono finiti/infiniti).
%Come sta scritto sotto, giusto un po' pi\`u formale (per esempio, scriverlo come Definizione).
%
%The fundamental observation that led to the study of mathematics over the \emph{tropical semi-ring} $\Lawv=([0,\infty],\min,+)$ was that, by replacing everywhere the ``$+$'' by the ``$\min$'' and the ``$\times$'' by the ``$+$'', many algebraic and geometric objects becomes combinatorial and their computation simpler. 
%
%For instance, the tropicalization of a cubic polynomial $p(x)=ax^{3}+bx^{2}+cx+d$ yields a piecewise-linear function 
%\begin{align}
%\trop p(\alpha)= \min\{ 3\alpha+a, 2\alpha+b, \alpha+c,d\}
%\end{align}
%Notably, the \emph{tropical roots} (whose definition is recalled in Section \ref{section3}) of $\trop p(\alpha)$ can be found through a rather simple (indeed polytime \cite{}) algorithm, and can be used to \emph{approximate} the actual roots of $p(x)$ \cite{}. 
%More generally, the tropicalization of a power series $f(x)=\sum_{n}\widehat f_{n}x^{n}$ yields a \emph{tropical Laurent series} \cite{} 
%\begin{align}
%\trop f(\alpha)= \inf_{n}\left\{n\alpha+ \widehat f_{n}\right\}
%\end{align}
%a class of functions that we will study in detail in Section \ref{section4}.
%
%%- generalities about tropical maths (tropicalisation $\trop P$  of polynomials and of Laurent series, and their roots -- all that without $\LREL$)
%
%
%

