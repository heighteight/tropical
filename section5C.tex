

\subsection{Resource Analysis for Differential Privacy}

The typical situation in differential privacy is where one considers a probabilistic query $f: \mathsf{db}\to [0,1]^{X}$ over some database, and one requires that $f$ should not be \emph{too sensitive} to small changes in the output, in other to prevent potential leaks of private information about individual items in $\mathsf{db}$ (for instance, an element $x\in\mathsf{db}$ could be the list of students of some university and $f(x)$ indicates the percentage of female students$\mathsf{db}$).

More formally, differential privacy is defined as follows \cite{Reed2010}:
\begin{definition}
Let $f: \mathsf{db}\to [0,1]^{X}$ and $\epsilon \in \BB R_{\geq 0}$. $f$ is said \emph{$\epsilon$-differentially private} if for all $x,x'\in \mathsf{db}$
differing by $L$ items, for all $s\in X$, 
$$
f(x)(s) \ = \ e^{\epsilon L} f(x')(s)
$$
\end{definition}

A well-studied approach to ensure differential privacy for higher-order programs is to use bounded type systems like $\mathsf{Fuzz}$ \cite{Reed2010}. Indeed, such systems come with an \emph{a priori} warrant that well-typed programs are Lipschitz.

The use of tropical semantics suggests how resource analysis could also be used to provide bounds for differential privacy. 
Suppose our probabilistic query $f$ can be expressed as a power series (this is what happens e.g.~in \emph{probabilistic coherent spaces} \cite{Ehrhard2011}). Then, if we discover, either by studying differential properties of $f$, or using methods from convex analysis as suggested in the previous section (e.g.~Theorem \ref{??}), 
 that the tropicalization $\trop f$ satisfies a Lipschitz condition, we may use this fact to deduce that $f$ is differentially private, as shown by the result below.

%
%We have seen how a higher-order probabilistic program, which is expressed as a polynomial, can be tropicalized. This operation can be extended to programs expressed as power series by taking the limit. 
%
\begin{proposition}
Let $f: [0,1]^{X}\to [0,1]^{Y}$ be an analytic function. If $\trop f$ is Lipschitz over some open set $U$, then $f$ is differentially private over $\psi_{t}(U)$, for small enough $t$.
\end{proposition}
\begin{proof}
We consider the case $X=Y=\B 1$ for simplicity. 
Express $f$ as a power series $f(x)=\sum_{n}\psi_{e}(\widehat f_{n})x^{n}$, so that $\trop f(\alpha)= \inf_{n}\left \{
n\alpha + \widehat f_{n}
\right\}$.
Let us define a family of intermediate functions $\trop_{t}f(\alpha)=\phi_{t}(\stackrel{t}{\sum_{n}}\widehat{f}_{n}\times_{t}n\alpha)$. These, by construction, satisfy
\begin{align}\label{eq:phit}
f(\phi_{t}(\alpha)) = \phi_{t}(\trop_{t}f(\alpha))
\end{align}
and, using the fact that $\alpha\sumt{t}\beta\to\min\{\alpha,\beta\}$ and $\alpha\prodt{t}\beta\to \alpha+\beta$, we can deduce that $\trop f(\alpha)=\lim_{t\to 0}\trop_{t}f(\alpha)$ (for this, we use the fact that for all $m\in \BB N$,
$\min_{i\leq m}\alpha_{i}\leq \stackrel{t}{\sum}_{i\leq m}\alpha_{i}\leq 
(\min_{i\leq m}\alpha_{i})+t\log m$).

Now, using the fact that $\trop f$ and $\trop_{t} f$ come arbitrarily close with $t$ small enough, together with \eqref{eq:phit}, the Lipschitz condition  
$|\trop f(\phi_{t}(x))-\trop f(\phi_{t}(y))|\leq L\cdot |\phi_{t}(x)-\phi_{t}(y)|$ translates into the differential privacy condition 
$f(x) \leq e^{L\cdot |\log x-\log y|} f(y)$ (supposing $y\geq x$ and $f(y)\neq 0$).
\end{proof}

%General discussion: optimization properties behave in a Lipschitz way.
%
%
%- differential privacy and Lipschitzness
%
%
%- log-probabilities and tropical roots 
%
%
%- counting computation steps (from Manzonetto, but add relational ``Lipschitz'' reasoning)
%
%
%- measuring duplications of discrete functions (needs finiteness!)



