
The study of matrices with values over the tropical semi-ring can be seen as a special case of the
\emph{quantitative relational semantics} \cite{Manzo2013}, a well-studied semantics of the $\lambda$-calculus. 

For a fixed \emph{continuous} semi-ring $Q$ [Def. II.5, \cite{Manzo2013}], the category $\QREL$ has sets as objects and set-indexed matrices with coefficients in $Q$ as morphisms (\cite{Manzo2013} would call it $Q^\Pi$).
The identity morphism of $\QREL$ is the matrix $i:X\times X\to Q$ given by $i_{a,a}=1$ and $i_{a,b\neq a}=0$, and composition of morphisms $t:X\times Y\to Q$ and $s:Y\times Z\to Q$ is given by 
$(st)_{a,c}:=\sum\limits_{b\in Y} s_{b,c}t_{a,b}$.
Notice that the assumption of $Q$ being continuous 
%(which can be weakened, see \cite{???})
 is used in order to make this potentially infinite sum converge.
 
 As it is expected, $Q^X$ is a $Q$-semimodule and the bijection $\hat{(\cdot)}$ identifies $\HOM{\QREL}{X}{Y}$ with the set of linear maps from $Q^X$ to $Q^Y$ via
 \begin{equation}
 \hat t(x)_b:=\sum\limits_{a\in X} t_{a,b}x_a
 \end{equation}
 
 
 \begin{remark}
 Following \cite{Manzo2013, Hofmann2014, Ehrhard2005}, we 
 chose to see a matrix from $X$ to $Y$ as a map $s:X\times Y\to Q$.
% 
% fix $\HOM{\QREL}{X}{Y}:=Q^{X\times Y}$ with composition $st:X\times Z\to Q$ of $s:Y\times Z\to Q$ and $t:X\times Y\to Q$ defined by $(st)_{a,c}:=\sum\limits_{b\in Y} s_{b,c}t_{a,b}$.
Notice that usual linear algebra conventions correspond to working in $\LREL^{\op}$: a matrix $X\times Y\to Q$ is usually called a ``$Y\times X$-matrix'', meaning $Y$ rows and $X$ columns, and the usual matrix-vector product defines a map $Q^Y\to Q^X$.
\end{remark}
%
%In $\QREL^{op}$ (which corresponds to systematically taking transpose matrices), composition coincides with the product matrix/matrix and $\hat{(\cdot)}$ with the product matrix/vector.
%In order to avoid confusion, we will refer to a $t\in Q^{X\times Y}$ just as a \emph{matrix from $X$ to $Y$}.
%
%
%
%We must fix a convention for matrices: following \cite{Manzo2013, Hofmann2014, Ehrhard2005}, we fix $\HOM{\QREL}{X}{Y}:=Q^{X\times Y}$ with composition $st:X\times Z\to Q$ of $s:Y\times Z\to Q$ and $t:X\times Y\to Q$ defined by $(st)_{a,c}:=\sum\limits_{b\in Y} s_{b,c}t_{a,b}$.
%In linear algebra, a map $X\times Y\to Q$ is usually called a ``$Y\times X$-matrix'', meaning $Y$ rows and $X$ columns.
%In particular, the product of such a matrix for a vector defines a map $Q^Y\to Q^X$.
%Instead, we prefer to see a $t\in\HOM{\QREL}{X}{Y}$ as giving rise to a map $\hat t:Q^X\to Q^Y$ defined by $\hat t(x)_a:=\sum\limits_{b\in Y} t_{a,b}x_a$.
%In $\QREL^{op}$ (which corresponds to systematically taking transpose matrices), composition coincides with the product matrix/matrix and $\hat{(\cdot)}$ with the product matrix/vector.
%In order to avoid confusion, we will refer to a $t\in Q^{X\times Y}$ just as a \emph{matrix from $X$ to $Y$}.
%As it is expected, $Q^X$ is a $Q$-semimodule and the bijection $\hat{(\cdot)}$ identifies $\HOM{\QREL}{X}{Y}$ with the set of linear maps from $Q^X$ to $Q^Y$.

\begin{remark}
 The category $\QREL$ is (equivalent to) a subcategory of the category $Q\SF{Mod}$ of \emph{complete} $Q$-semimodules.
 If $\QREL$ corresponds to considering semimodules (the $Q^X$'s) whose vectors are given in coordinates w.r.t.\ a \emph{fixed base} (the set $X$), $Q\SF{Mod}$ corresponds to considering semimodules in abstract, without fixing a base.
 We take this viewpoint in Section~\ref{section6}.
\end{remark}


The tropical relational semantics is thus provided by the category $\LREL$ of matrices with values over $\Lawv$ (which, being a quantale, is indeed a continuous semi-ring).
%, where $\Lawv$ is the already introduced Lawvere quantale, seen as the idempotent complete semiring $(\BB R_{\geq0}\cup\set{\infty},\inf,\infty,+,0)$.
%The category $\LREL$ is well-defined because $\Lawv$ is a continuous semiring (w.r.t.\ its quantale order $\preceq$.
%This amounts to check that $\min$ and $+$ commute with the $\inf$ (as operations on $\BB R_{\geq0}\cup\set{\infty}$, which is immediate), and that $(\Lawv,\preceq)$ is a cpo with $\infty$ as bottom element (which is immediate since in $\Lawv$ we have $\vee = \inf$) .
It is worth observing that the formula for composition in $\LREL$ is the tropicalisation of the one defining it in $\QREL$, i.e.\ $(st)_{a,c}:=\inf\limits_{b\in Y}\set{s_{b,c}+t_{a,b}}$;
 similarly, the linear functions $f:\Lawv^X\to \Lawv^Y$ induced by matrices, which we call \emph{tropical linear}, are exactly those of shape $f(x)=\inf\limits_{b\in Y} \set{\check f_{a,b}+x_a}$, for some matrix $\check f$ from $X$ to $Y$.
%\end{remark}

Since $\Lawv$ is a continuous (commutative) semiring, [Proposition III.3, \cite{Manzo2013}] immediately applies and gives:

\begin{fact}
 $\LREL$ is a linear $\Lawv$-category.
\end{fact}

Unwrapping [Definition II.9, \cite{Manzo2013}], this means that:
$\HOM{\LREL}{X}{Y}$ is a continuous $\Lawv$-semimodule, with semimodule operations defined pointwise;
$\LREL$ is a continuous $\Lawv$-category, i.e.\ composition of morphisms commutes with $\inf$'s;
$\LREL$ is linear, i.e.\ pre- and post-composition with any morphism in any $\HOM{\LREL}{X}{Y}$ are automorphisms on it.

In the next sections we will see how $\LREL$ gives rise to denotational models of several variants of the $\lam$-calculus.
