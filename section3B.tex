%\subsubsection{Unbounded duplications}

In order to interpret the full $\STLC$, we need a Cartesian closed category (CCC).
It is well-known \cite{Mellies2009} that it is always possible to construct a CCC by taking the \emph{co-Kleisli} $\C C_!$ of a so-called \emph{Lafont category} $\C C$.
%, a construction we now quickly recall.
%defined via a \emph{Lafont exponential} comonad $!$.
%Let us quickly recall the ideas behind these notions.
A SMCC is Lafont when it has finite products and it is equipped with a comonad $!$ (its \emph{Lafont exponential}) which, at level of objects, sends $X$ to an object $!X$ being the free commutative comonoid on $X$.
Such objects $!X$ represent the \emph{bang} connective of linear logic, granting infinite duplications via the infinite product $X^0\otimes X\otimes X^2\otimes X^3\otimes\cdots$, each factor representing a possible number of duplications.
It is well-known that, under mild conditions satisfied by $\QREL$, one can explicit this idea via the fact that the map $X\mapsto \finMS{X}$ (where $\finMS{X}$ is the set of finite multi-sets on $X$) lifts to a functor $!:\QREL\to\QREL$ which is a Lafont-exponential comonad.
Specializing [Corollary III.6, \cite{Manzo2013}] to our case, we have:

\begin{fact}
 $\LREL$ is Lafont.
\end{fact}

Let us recall that the coKleisli category $\C C_!$ of a category $\C C$ w.r.t.\ a comonad $!$ is the category whose elements are the same of $\C C$, and $\HOM{\C C_!}{X}{Y}:=\HOM{\C C}{!X}{Y}$, with composition $\circ_!$ defined by making use of the co-multiplication of $!$.
%We will explicit this constructions in our tropical setting in the next lines.
As already mentioned, one obtains a CCC from a Lafont category by defining the exponential objects as $X\to Y:=!X \multimap Y$.
In our case, specialising [Theorem III.7, \cite{Manzo2013}] we have indeed:
\begin{fact}
 The coKleisli $\LREL_!$ is CCC, i.e.\ a model of $\STLC$.
\end{fact}

%It is instructive what its CCC-structure looks like in our tropical world.
In our tropical setting, 
the exponential object $X\multimap Y$ is $!X\times Y$, and 
the coKleisli composition of $s\in\Lawv^{!Y\times Z}$ and $t\in\Lawv^{!X\times Y}$ is the matrix $s\circ_! t\in\Lawv^{!X\times Z}$ where $(s\circ_! t)_{\mu,c}$ is:
\begin{align}
 \inf\limits_{n\in\N, b_1\dots,b_n\in Y, \mu = \mu_1*\cdots *\mu_n}
 \left\{s_{[b_1,\dots,b_n],c} + \sum\limits_{i=1}^n t_{\mu_i,b_i}\right\}.
\end{align}
% (where $+$ is the disjoint union). 

%Remember that in $\Lawv$ the neutral element for addition is $\infty$ and the neutral for multiplication is $0$, so for instance the evaluation map is the matrix $\RM{eval}^{X,Y}\in\Lawv^{!((X\multimap Y) \times X)\times Y}\simeq\Lawv^{(!!X\times !Y\times !X)\times Y}$ given by $\RM{eval}^{X,Y}_{\rho_1\oplus\rho_2\oplus\mu,b}:=0$ if $\rho_1=[\mu]$ and $\rho_2=[b]$, and $\RM{eval}^{X,Y}_{\rho_1\oplus\rho_2\oplus\mu,b}:=\infty$ otherwise.

As it is well-known, the Cartesian closed structure %of a category $\C C$ 
allows to define a sound interpretation $\model{\Gamma\vdash M:A}\in\HOM{\LREL_{!}}{\model \Gamma}{\model A}$ of terms as morphisms.
In our case, we have:

\begin{proposition}\label{prop:descrete}
 The interpretation $\model M$ of a $\lam$-term $M$ in $\LREL$ is a \emph{discrete} matrix, i.e.\ its coefficients are either $0$ or $\infty$.
\end{proposition}
\begin{proof}
 First easily prove that composition preserves discreteness.
 Then go by straightforward induction on $M$, using that the projections and evaluation of  $\LREL$ are discrete.
\end{proof}

While ordinary $\lam$-terms yield discrete matrices, 
as discussed in Section~\ref{sec:app}, the other coefficients from $[0,\infty]$ will appear when considering quantitative effects like computation steps or probabilities.
%, give rise to  we will however discuss how 
%we can use this model in order to make it keep track of quantitative effects of programs.

\begin{remark}\label{rmk:ModelsOfBSTLC}
%Similarly to It is known {\color{red}(([reference??] e dire meglio)} 
The exponential $!$ can be ``decomposed'' into a family of \emph{graded} exponentials $!_n:\LREL\to\LREL$ ($n\in\BB N$), defined by functors $X\mapsto \C M_{\leq n}(X)$. %  lift to functors 
The sequence $(!_n)_{n\in\N}$ can be then turned into a $\N$-graded linear exponential comonad on (the SMC) $\LREL$, satisfying the adjunction:
$\LREL(Z\otimes !_{n}X,Y) \simeq \LREL(Z, !_{n}X\multimap Y)$.
Therefore, $\LREL$, together with $(!_n)_{n\in\N}$, is a model for $\BSTLC$. 
In particular, bounded arrow types are interpreted via
%Concretely, the types of this calculus (Equation~\ref{eq:Btyped}) are interpreted as: 
$\model o=\{\star\}$ and 
$\model{!_{n}A\multimap B}:= \C M_{\leq n}(\model A) \times \model B$.
Notice that, for any type $A$ of $\BSTLC$, its interpretation $\model{A}$ is a finite set.
\end{remark}
%Bounded types are interesting because of the following proposition:
%\begin{proposition}
%For all bounded types $A,B$, the morphisms from $\model A$ to $\model B$ (in all parametric relational models) correspond to polynomials.
%\end{proposition}
%\begin{proof}
%It suffices to check that $\model A$ is finite for all bounded types $A$. Indeed this implies that a morphism $t:\model A\to \model B$ is a finite matrix $t: \model A \times \model B \to \Lawv$.Hence, its corresponding map $\widehat t:\Lawv^{\model A} \to \Lawv^{\model B}$ is a polynomial.
%\end{proof}

%For example (here $!_{n}(\Lawv^{X}):= \Lawv^{\C M_{\leq n}(X)}$):
%\begin{itemize}
%\item a map $f\in \LREL( !_{1}\Lawv, \Lawv)$ is of the form $f(x)=\min \{x+a,b\}$;
%\item a map $f\in \LREL(!_{2}\Lawv, \Lawv)$ is a ``quadratic'' polynomial $f(x)=\min\{2x+a, x+b, c\}$.
%\end{itemize}


