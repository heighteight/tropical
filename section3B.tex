\subsubsection{Bounded duplications}

It is known ([reference??]) that the maps $X\mapsto \C M_{\leq n}(X)$ ($n\in\BB N$) lift to functors $!_n:\LREL\to\LREL$.
The sequence $(!_n)_{n\in\N}$ can be then turned into a $\N$-graded linear exponential comonad on (the SMC) $\LREL$, satisfying the adjunction:
$\LREL(Z\otimes !_{n}X,Y) \simeq \LREL(Z, !_{n}X\multimap Y)$.
Therefore, $\LREL$, together with $(!_n)_{n\in\N}$, is a model for $\BSTLC$. 
Concretely, a type $A::= o \mid !_{n}A \multimap A$ (i tipi meglio dirli in Section II) is in interpreted as $\model o=\B 1$ and 
$\model{!_{n}A\multimap B}:= \C M_{\leq n}(\model A) \times \model B$.

Bounded types are interesting because of the following proposition:
\begin{proposition}
For all bounded types $A,B$, the morphisms from $\model A$ to $\model B$ (in all parametric relational models) correspond to polynomials.
\end{proposition}
\begin{proof}
It suffices to check that $\model A$ is finite for all bounded types $A$. Indeed this implies that a morphism $t:\model A\to \model B$ is a finite matrix $t: \model A \times \model B \to \Lawv$.
Hence, its corresponding map $\widehat t:\Lawv^{\model A} \to \Lawv^{\model B}$ is a polynomial.
\end{proof}

For example (here $!_{n}(\Lawv^{X}):= \Lawv^{\C M_{\leq n}(X)}$):
\begin{itemize}
\item a map $f\in \LREL( !_{1}\Lawv, \Lawv)$ is of the form $f(x)=\min \{x+a,b\}$;
\item a map $f\in \LREL(!_{2}\Lawv, \Lawv)$ is a ``quadratic'' polynomial $f(x)=\min\{2x+a, x+b, c\}$.
\end{itemize}

- tropical polynomials and tropical roots ?? Meglio in Section II

\subsubsection{Unbounded duplications}

If in the previous subsection we considered \emph{bounded} multisets via the graded comonad $(!_n)_{n\in\N}$, it is immediate to see that \emph{unbounded} multisets yield a comonad able to interpret the $\STLC$:
following [Corollary III.6, Weighted] we have:

\begin{fact}
 $\LREL$ is Lafont.
\end{fact}

Therefore, it is well known that the map $X\mapsto \finMS{X}$ lifts to a functor $!:\LREL\to\LREL$ which %satisfies the adjunction $\LREL(!(Z\& X),Y) \simeq \LREL(!Z, !X\multimap Y)$, at 
is a Lafont-exponential.
In particular, [Theorem III.7, Weighted] says that:
\begin{fact}
 The coKleisli $\LREL_!$ is CCC, i.e.\ a model of $\STLC$.
\end{fact}

It is instructive at this point to see what its CCC-structure looks like in our tropical world.
In particular, the coKleisli composition of $s\in\Lawv^{!Y\times Z}$ and $t\in\Lawv^{!X\times Y}$ is the matrix $s\circ_! t:\Lawv^{!X\times Z}$ where $(s\circ_! t)_{\mu,c}$ is:
\[
 \inf\limits_{n\in\N, b_1\dots,b_n\in Y, \mu = \mu_1*\cdots *\mu_n}
 \set{s_{[b_1,\dots,b_n],c} + \sum\limits_{i=1}^n t_{\mu_i,b_i}}.
\]
The exponential object $X\multimap Y$ is $!X+Y$ (where $+$ is the disjoint union).
Remember that in $\Lawv$ the neutral element for addition is $\infty$ and the neutral for multiplication is $0$, so for instance the evaluation morphism is the matrix $\RM{eval}^{X,Y}\in\Lawv^{!((X\multimap Y) + X)\times Y}\simeq\Lawv^{(!!X\times !Y\times !X)\times Y}$ given by $\RM{eval}^{X,Y}_{\rho_1\oplus\rho_2\oplus\mu,b}:=0$ if $\rho_1=[\mu]$ and $\rho_2=[b]$, and $\RM{eval}^{X,Y}_{\rho_1\oplus\rho_2\oplus\mu,b}:=\infty$ otherwise.

\begin{remark}[Tropical Laurent series]
 As usual, a matrix $t\in\HOM{\LREL_!}{X}{Y}$ operates as a linear map $\widehat t:\Lawv^{!X}\to\Lawv^Y$.
 But we can also ``express it in the base $X$'', i.e.\ see it as a map $t^!:\Lawv^X\to\Lawv^Y$, by setting $t^!(x):=t\circ_! x$ (we are identifying $\Lawv^X$ with the set $\HOM{\LREL_!}{\emptyset}{X}$ of the \emph{points} of $X$).
 This is the notion of \emph{non-linear} map generated by the CCC-structure of $\LREL_!$.
 Concretely, we have:
 \[t^!(x)_b=\inf\limits_{\mu\in !X} \set{\mu x+ t_{\mu,b}}\] where $\mu x:=\sum\limits_{a\in X} \mu(a)x_a$.
 Since in the general case of $\QREL$, $t^!$ is a Laurent series with operations in $Q$, let us call \emph{tropical Laurent series} the functions of shape $t^!$ for some $t\in\HOM{\LREL_!}{X}{Y}$.
\end{remark}

\begin{remark}
 Indentifying $!\set{*}\simeq \N$ and $\Lawv^{\set{*}}\simeq\Lawv$, the tropical Laurent series generated by the morphisms in $\HOM{\LREL_!}{\set{*}}{\set{*}}$ are exactly the functions $f:\Lawv\to\Lawv$ of shape $f(x)=\inf\limits_{n\in\N}\set{nx+\widehat f(n)}$, for some $\widehat f:\N\to\Lawv$.
Remark that we find usual \emph{tropical polynomials} of tropical geometry as a particular case: they corrispond to the ones for which the support $\set{n\in\N\mid\widehat f(n)\neq\infty}$ of $\widehat f$ is \emph{finite}.
\end{remark}

\begin{proposition}\label{prop:descrete}
 The interpretation $\model M$ of a $\lam$-term $M$ in $\LREL$ is a \emph{descrete} matrix, i.e.\ its coefficients are either $0$ or $\infty$.
\end{proposition}
\begin{proof}
 First easily prove that composition preserves descreteness.
 Then go by straightforward induction on $M$, using that the projections and evaluation of  $\LREL$ are descrete.
\end{proof}

\begin{example}
 The function $f:\Lawv\to\Lawv$, $f(x):=\inf\limits_{n\in\N}\set{nx+\frac{1}{2^n}}$ is a tropical Laurent series: it is of shape $f=t^!$, for $t\in\Lawv^{!\set{*}\times\set{*}}$, $t_{\mu,*}:=2^{-\# \mu}$.
By Proposition~\ref{prop:descrete}, $f$ is not the interpretation of a $\lam$-term, because its matrix is not descrete. Therefore $\LREL_!$ is not a full-complete model of $\STLC$.
\end{example}


