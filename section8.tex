We have shown that:

- intuitively, the interplay between tropical mathematics and $\lam$-calculus \emph{could} relate the metric and differential approaches on approximations of $\lam$-calc (introduction+section 2)

- this relation \emph{can} take place, because the natural category $\LREL$ and its generalised versions are metric models of the differential $\lam$-calculus (section 3 and 6)

- this relation \emph{seems} to provide applications in different fields (section 5).

Therefore, we mainly set the basis for future interplays between all these areas, hopefully motivating the interest in such an interplay.
For instance, the general questions are:

- Can we improve the results of section 4 ?

- Can we develop and make the applications of section 5 useful ?

- What does the general setting of section 6 give in terms of theoretical and applied results ?

- Do tools from tropical \emph{geometry} provide something for $\lam$-calculus ? (for instance, the role of tropical roots, tropical varieties,...)

- Finally, there is a last point which we think of interest and we did not mention through the paper: the inclusion of finiteness spaces in the picture.
Say what could they do and why.