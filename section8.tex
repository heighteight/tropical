


%In our opinion,
The main goals of this paper are two. Firstly,  to
demonstrate the existence of a conceptual bridge between two different well-studied quantitative approaches to higher-order programs, and to highlight the possibility of transferring results and techniques from one approach to the other. 
Secondly, to suggest that tropical mathematics, a
field which has been largely and successfully applied in computer science, could be used to study quantitative properties of higher-order programs.

While the first goal was here developed in detail, and at different levels of abstraction, for the second goal we only sketched a few interesting directions (best-case analysis, log-probabilities, differential privacy). We believe that exploring these ideas in more depth could be a fruitful direction; moreover, 
since both generalized metrics and quantale-modules have been largely studied in computer science, 
a natural question is if the generalized approach of Section \ref{section6} could  lead to new applications of metric and tropical methods to the $\lambda$-calculus.

Finally, since bounds on the Taylor expansion translate into Lipschitz conditions, 
 two interesting directions to explore are provided by (non-idempotent) intersection types and finiteness spaces \cite{Ehrhard2005}, as both methods are in principle capable of capturing \emph{finitary} bounds on the Taylor expansion. %
Notably, knowing that the application of a program $M$ to $N$ expands as a finite sum of linear applications may allow one to predict how sensitive $M$ will be ``around $N$''. 

\newpage