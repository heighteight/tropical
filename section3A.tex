We now turn to discuss how different variants of the simply typed $\lambda$-calculus are interpreted in $\LREL$.

The linear simply typed $\lam$-calculus is a restriction of the ordinary $\lam$-calculus in which programs can only use their arguments \emph{exactly once}.
This is the notion of linearity taken into account by linear logic, and indeed this calculus can be embedded in its \emph{intuitionistic multiplicative} fragment \emph{IMLL}.

More precisely, the linear $\lam$-calculus is obtained from the ordinary one by adding the constraint that each $\lam$-abstracted variable appears exactly once in the scope of the $\lam$-abstraction.

In order to frame it in a category $\C C$, one needs a symmetric tensor product $\otimes$ together with internal hom-set objects $X\multimap Y$ s.t.\ the evaluation and curry maps yield a \emph{symmetric monoidal adjunction}: $\C C(Z\otimes X, Y) \simeq \C C (Z, X\multimap Y)$.
This gives the notion of Symmetric Monoidal Closed Category (SMCC).

[Section III.A, \cite{Manzo2013}], immediately gives:

\begin{fact}\label{fact:LREL_SMCC}
 $\LREL$ is a SMCC, thus a model of the linear $\STLC$.
\end{fact}

The monoidal structure of $\LREL$ is given by a tensor product $\otimes$ acting on the objects as the Cartesian product of sets, and as the \emph{Kronecker product}
 %{(\color{red} Ref?)} 
 of matrices on morphisms.
Its closed structures is also given by the Cartesian product on sets, with the usual evaluation and curry maps.
Actually, the SMCC $\LREL$ is even a model of IMLL.
