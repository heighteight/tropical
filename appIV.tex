


%We already know from Giulio's paper that $\LREL_!$ is a post-linear continuous $\Lawv$-ccc.

\begin{proposition}
$\LREL_!$ is a cartesian closed left-$\Lawv$-additive category.
\end{proposition}
\begin{proof} 
Since $\infty \circ f= f\circ \infty=\infty$ and $\min\{g,h\}\circ f= \min\{g\circ f, h\circ f \}$, $\LREL_!$ is {left-$\Lawv$-additive}.
A morphism $h\in \LREL_!(X,Y)$ that further satisfies $h\circ \min\{f,g\}=\min\{h\circ f, h\circ g\}$ for all object $X'$ and $f,g\in \LREL_!(X',X)$, is called \emph{additive}.
To show that $\LREL_!$ is cartesian closed left-$\Lawv$-additive we must also check that (1) products and projections of additive morphisms are additive, and that (2) $\Lambda(\min\{f,g\})=\min\{\Lambda(f),\Lambda(g)\}$, $\Lambda(\infty)=\infty$, where $\Lambda:\Lawv^{\C M_{\mathsf{fin}}(Z+X)\times Y}\to \Lawv^{\C M_{\mathsf{fin}}(Z)\times( \C M_{\mathsf{fin}}(X)\times Y)}$ is the isomorphism given by
$(\Lambda(f))_{\mu,\nu,y}= f_{\mu\oplus\nu,y}$, where 
 $\mu\oplus\nu$ is defined by $(\mu\oplus\nu)(\langle 0,x\rangle)=\mu(x)$ and $(\mu\oplus \nu)(\langle 1,x\rangle)=\nu(x)$.
	\begin{enumerate}
	\item  Let $f\in \Lawv^{\C M_{\mathsf{fin}}(X)\times Y}$ and $g\in \Lawv^{\C M_{\mathsf{fin}}(X)\times Z}$ be additive; then $\langle f,g\rangle \in \Lawv^{\C M_{\mathsf{fin}}(X)\times (Y+Z)}$, which is defined by 
	$$
	\langle f,g\rangle_{\mu,\langle i,a\rangle}= \begin{cases}
	f_{\mu,a} & \text{ if }i=0\\
	g_{\mu,a} & \text{ if }i=1
	\end{cases}
	$$
	is also additive. Indeed, for all $h\in \LREL_!(X',X)$, 
	in any cartesian category it holds that $\langle f,g\rangle \circ h=\langle f\circ h, g\circ h\rangle$. Now, 
		if $i=0$, then for all $h_{1},h_{2}\in \Lawv^{\C M_{\mathsf{fin}}(X')\times X}$, 
	\begin{align*}
	(\langle f,g\rangle\circ \min\{h_{1},h_{2}\})_{\rho,\langle i,z\rangle}&=
	(\langle f\circ  \min\{h_{1},h_{2}\},g\circ  \min\{h_{1},h_{2}\}\rangle)_{\rho,\langle i,z\rangle} 
	\\
&=	(f\circ \min\{h_{1},h_{2}\})_{\rho, z}\\ &
	= \min\{(f\circ h_{1})_{\rho,z},(f\circ h_{2})_{\rho,z}\} \\
	&=
	\min\{ \langle f\circ h_{1},g\circ h_{1}\rangle_{\rho,\langle i,z\rangle}, 
	\langle f\circ h_{2},g\circ h_{2}\rangle_{\rho,\langle i,z\rangle}
\}
	\\
	& = \min\{ (\langle f,g\rangle\circ h_{1})_{\rho,\langle i,z\rangle},(\langle f,g\rangle\circ h_{2})_{\rho,\langle i,z\rangle}\}	\end{align*}
	and similarly if $i=1$.
	
	Moreover, suppose $f\in \LREL_!(X, Y+Z)$ is additive, and let us show that $\pi_{1}( f)\in \LREL_!(X,Y)$, defined by 
	$
	(\pi_{1}( f))_{\mu,y}= f_{\mu, \langle 0,y\rangle}
	$, is also additive: first observe that $\pi_{1}(f)=\pi_{1}\circ f$, where $\pi_{1}\in \LREL_!(Y+Z,Y)$ is given by $(\pi_{1})_{\mu,y}=0$ if $\mu=[y]\oplus \emptyset$ and is $\infty$ otherwise; moreover, 
	$\pi_{1}(\min\{g,h\})=\min\{\pi_{1}(g),\pi_{1}(h)\}$, since 
	$(\pi_{1}(\min\{g,h\}))_{\mu, y}=(\min\{g,h\})_{\mu, \langle 0,y\rangle}= \min\{g_{\mu, \langle 0,y\rangle},h_{\mu, \langle 0,y\rangle}\}=( \min\{\pi_{1}(g),\pi_{1}(h)\})_{\mu, y}$. Now,
		given $h_{1},h_{2}\in \LREL_!(X',X)$, we have that 
	$\pi_{1}(f)\circ \min\{h_{1},h_{2}\}= (\pi_{1}\circ f)\circ \min\{h_{1},h_{2}\}= \pi_{1}\circ (f\circ \min\{h_{1},h_{2}\})= \pi_{1}\circ \min\{f\circ h_{1},f\circ h_{2}\}= \pi_{1}(\min\{f\circ h_{1},f\circ h_{2}\})= \min\{\pi_{1}(f\circ h_{1}), \pi_{1}(f\circ h_{2})\}= \min\{\pi_{1}(f)\circ h_{1},\pi_{1}(f)\circ h_{2})$.
	
	\item  It is clear then that $\Lambda(\infty)=\infty$, and moreover
	$$
	\Lambda (\min\{f,g\})_{\mu,\nu,y}= \min\{ f,g\}_{\mu\oplus \nu,y}= \min\{f_{\mu\oplus\nu,y},g_{\mu\oplus\nu,y}\}= \min(\Lambda(f),\Lambda(g))_{\mu,\nu,y}
	$$
	
	\end{enumerate}


\end{proof}



For any morphism $f\in \LREL_!(X,Y)$, let us define a morphism $D(f)\in \LREL_!(X+X,Y)$, i.e.~$D(f)\in \Lawv^{\C M_{\mathsf{fin}}(X+X),Y}$, by
$$
D(f)_{\mu,y}=
\begin{cases}
f_{\mu'+x,y}
& \text{ if }\mu=[x]\oplus \mu'
\\
\infty & \text{ otherwise}
\end{cases}
$$



\begin{proposition}
The category $\LREL_!$, endowed with the operator $D$, is a cartesian closed differential category.
\end{proposition}
\begin{proof}
We must check axioms (D1)-(D7) of cartesian differential categories  plus axiom (D-curry) (cf.~\cite{Manzo2012}).
\begin{description}
\item[(D1)]$D(\min\{f,g\})=\min \{D(f),D(g)\}$ and $D(\infty)=\infty$: 
while the latter is obvious, for the former we have 
$D(\min\{f,g\})_{[x]\oplus\nu,y}= \min\{f,g\}_{\nu+x,y}= \min\{f_{\nu+x,y},g_{\nu+x,y}\}=\min\{D(f),D(g)\}_{[x]\oplus \nu,y}$, and if $\mu\neq [x]\oplus \nu$, 
$D(\min\{f,g\})_{\mu,y}= \infty= \min\{\infty, \infty\}=\min\{ D(f),D(g)\}_{\mu,y}$. 
\item[(D2)]
$D(f)\circ \langle \min\{h,k\},v\rangle= \min\{ D(f)\circ \langle h,v\rangle, D(f)\circ \langle k,v\rangle\}$, and $D(f)\circ \langle \infty,v\rangle=\infty$: we can compute
	\begin{align*}
	(D(f)\circ \langle \min\{h,k\},v\rangle)_{\mu,y}&=
	\inf\Big\{ 
	\sum_{i=1}^{n}\min\{h,k\}_{\rho_{i},w_{i}}+
	\sum_{j=1}^{m}v_{\nu_{j},z_{j}}
	+
	f_{[z_{1},\dots, z_{m}]+w,y}\\
&	\qquad\qquad\mid
	\mu=\sum_{i=1}^{n}\rho_{i}+\sum_{j=1}^{m}\nu_{j},	[w]=[w_{1},\dots, w_{n}]
	\Big\}\\
	& 
=	\inf\Big\{ 
	\min\{h,k\}_{\rho,w}+
	\sum_{j=1}^{m}v_{\nu_{j},z_{j}}
	+
	f_{[z_{1},\dots, z_{m}]+w,y}\\
&	\qquad\qquad\mid
	\mu=\rho+\sum_{j=1}^{m}\nu_{j}
	\Big\}\\
	&=	\min\Big\{\inf\big\{ 
	h_{\rho,w}+
	\sum_{j=1}^{m}v_{\nu_{j},z_{j}}
	+
	f_{[z_{1},\dots, z_{m}]+w,y}	\mid
	\mu=\rho+\sum_{j=1}^{m}\nu_{j}\big\}, \\
	& \qquad\qquad \inf\big\{ 
	k_{\rho,w}+
	\sum_{j=1}^{m}v_{\nu_{j},z_{j}}
	+
	f_{[z_{1},\dots, z_{m}]+w,y}	\mid
	\mu=\rho+\sum_{j=1}^{m}\nu_{j}\big\}\Big\}\\
	& = \min\Big\{(D(f)\circ \langle h,v\rangle)_{\mu,y}, (D(f)\circ \langle k,v\rangle)_{\mu,y}\Big\}\\
		& = \Big(\min\big\{D(f)\circ \langle h,v\rangle, D(f)\circ \langle k,v\rangle\big\}\Big )_{\mu,y}
	\end{align*}
	where, in the first equation, the condition $[w_{1},\dots, w_{n}]=[w]$ (i.e.~$n=1$) is forced by the fact that, otherwise, the application of $D(f)$ would give $\infty$. Moreover, we have
\begin{align*}
(D(f) \circ \langle \infty, v\rangle)_{\mu,y}&=\inf\Big\{ 
	\infty+
	\sum_{j=1}^{m}v_{\nu_{j},z_{j}}
	+
	f_{[z_{1},\dots, z_{m}]+w,y}\mid
	\mu=\rho+\sum_{j=1}^{m}\nu_{j}
	\Big\} = \infty
\end{align*}

\item[(D3)] $D(\mathrm{id})=\pi_{1}$, $D(\pi_{i})=\pi_{i}\circ \pi_{1}$: 
recall that $\mathrm{id}_{[x],x}=0$ and $\mathrm{id}_{\mu,x}=\infty$, if $\mu\neq [x]$. 
Moreover $(\pi_{1})_{\mu,x}=0$ if $\mu=[x]\oplus \emptyset$, and is $\infty$ otherwise, and $\pi_{2}$ is defined similarly.
Hence 
$D(\mathrm{id})_{[x]\oplus\nu,y}=\mathrm{id}_{\nu+x,y}$ is $0$ precisely when $x=y$ and $\nu=\emptyset$, and in all other cases is $\infty$. This shows that $D(\mathrm{id})=\pi_{1}$. 

$D(\pi_{1})\in \Lawv^{\C M_{\mathsf{fin}}((X+Y)+(X+Y))\times Y}$ is given by
$D(\pi_{1})
_{
[x\oplus \emptyset] \oplus (\mu\oplus\nu),y
}= (\pi_{1})_{(\mu\oplus\nu)+\langle 0,x\rangle ,y }
$, 
which is 0 precisely when $(\mu\oplus\nu)+\langle 0,x\rangle= y\oplus\emptyset$, i.e.~when 
$x=y$ and $\mu=\nu=\emptyset$; in all other cases one can check that $D(\pi_{1})_{\rho,y}=\infty$, so we conclude $D(\pi_{1})=\pi_{1}\circ \pi_{1}$.
One can argue similarly for $\pi_{2}$.

\item[(D4)] $D(\langle f,g\rangle)=\langle D(f),D(g)\rangle$: 
we have
\begin{align*}
D(\langle f,g\rangle)_{[x]\oplus\mu, \langle 0,y\rangle}& = 
(\langle f,g\rangle)_{\mu+x,\langle 0,y\rangle}= f_{\mu+x,y}= D(f)_{[x]\oplus \mu,y}\\
D(\langle f,g\rangle)_{[x]\oplus\mu, \langle 1,y\rangle}& = 
(\langle f,g\rangle)_{\mu+x,\langle 1,y\rangle}= g_{\mu+x,y}= D(g)_{[x]\oplus \mu,y}
\end{align*}
from which we deduce
$D(\langle f,g\rangle)_{[x]\oplus\mu, \langle i,y\rangle}=\langle D(f),D(g)\rangle_{[x]\oplus\mu, \langle i,y\rangle}$ 
 by the definition of $\langle \_,\_\rangle$.
 If $\rho\neq [x]\oplus\mu$, then
 $D(\langle f,g\rangle)_{\rho, \langle i,y\rangle}=\infty=\langle \infty,\infty\rangle=\langle D(f),D(g)\rangle_{\rho, \langle i,y\rangle}$
 (where the equation $\infty=\langle \infty,\infty\rangle$ is to be read as an equality between the functions $ X+Y\longrightarrow Q$
 defined by $\langle i,y\rangle \mapsto \infty$ and by
 $\begin{matrix}
 \langle 0,x\rangle\mapsto\infty\\
  \langle 1,y\rangle\mapsto\infty
 \end{matrix}$, respectively).
 
\item[(D5)] $D(f\circ g)=D(f)\circ \langle D(g), g\circ \pi_{2}\rangle$: we can compute
\begin{align*}
\Big( D(f)\circ \langle D(g), g\circ \pi_{2}\rangle\Big)_{[x]\oplus\mu,z}
&=
\inf \Big\{
D(g)_{[x]\oplus\mu',w}+
\sum_{i}g_{\mu_{i},w_{i}}+
D(f)_{[w]\oplus [w_{1},\dots, w_{n}],z}\\
&\qquad\qquad \mid
w,w_{i}\in Y, 
 \mu=\mu'+ \sum_{i}\mu_{i},
\Big\}\\
&=
\inf \Big\{
g_{\mu'+x,w}+
\sum_{i}g_{\mu_{i},w_{i}}+
f_{ [w_{1},\dots, w_{n}]+w,z}\\
&\qquad\qquad \mid
w,w_{i}\in Y, 
 \mu= \mu'+\sum_{i}\mu_{i}
\Big\}
\\
&=\inf\Big\{
\sum_{i} g_{\mu_{i},w_{i}} + f_{[w_{1},\dots, w_{n}],z}
\mid
w_{1},\dots, w_{n}\in Y, 
\mu+x=\sum_{i}\mu_{i}
\Big\}
\\
&= (f\circ g)_{\mu+x,y} =D(f\circ g)_{[x]\oplus\mu,z}
\end{align*}
if $\rho\neq [x]\oplus \mu$, then $D(f\circ g)_{\rho,z}=\infty$ and 
from the first equation above it follows that 
also $( D(f)\circ \langle D(g), g\circ \pi_{2}\rangle)_{\rho,z}=\infty$.



\item[(D6)] $D(D(f))\circ \langle \langle g,\infty\rangle,\langle h,k\rangle\rangle=D(f)\circ  \langle g,k\rangle$:
observe that 
\begin{align*}
\Big(D(D(f))\Big)_{[\langle 1,x'\rangle]\oplus([x]\oplus \mu),z}&=
\big(D(f) \big)_{[x]\oplus (\mu+x'),z }= f_{\mu+x'+x,z}\\
\Big(D(D(f))\Big)_{[\langle 0,x\rangle]\oplus (\emptyset \oplus \mu),z}&=
\big(D(f) \big)_{[x]\oplus \mu,z }= f_{\mu+x,z}
\end{align*}
and in all other cases $(D(D(f)))_{\mu,z}=\infty$.
Using this fact we can compute:


{
\small
\begin{align*}
\Big(D(D(f))\circ \langle \langle g,\infty\rangle,\langle h,k\rangle\rangle\Big)_{\mu,z}&=
\min\left\{
\begin{matrix}
\inf\left\{
\begin{matrix}
\infty_{\rho_{1},x'} + h_{\rho_{2},x}+ \sum_{i}k_{\mu_{i},w_{i}}
+
f_{[w_{1},\dots, w_{n}]+x'+x,z}\\
\qquad \mid 
x,x',w_{i}\in Y, 
\mu=\rho_{1}+\rho_{2}+\sum_{i}\mu_{i}
\end{matrix}
\right\},\\
\inf\left\{
\begin{matrix}
g_{\rho,x}+ \sum_{i}k_{\mu_{i},w_{i}}
+f_{[w_{1},\dots, w_{n}]+x,z}\\
\qquad \mid 
x,w_{i}\in Y, 
\mu=\rho+\sum_{i}\mu_{i}
\end{matrix}
\right\}
\end{matrix}
\right\}\\
&=
\inf\left\{
g_{\rho,x}+ \sum_{i}k_{\mu_{i},w_{i}}
+f_{[w_{1},\dots, w_{n}]+x,z}
\mid 
x,w_{i}\in Y, 
\mu=\rho+\sum_{i}\mu_{i}
\right\}\\
&= \Big(D(f)\circ \langle g,k\rangle\Big)_{\mu,z}
\end{align*}
}

\item[(D7)] $D(D(f))\circ \langle\langle \infty,h\rangle, \langle g,k\rangle\rangle= D(D(f))\circ \langle \langle \infty,g\rangle, \langle h,k\rangle\rangle$:
by computations similar to the case above we obtain

{
\small
\begin{align*}
\Big(D(D(f))\circ \langle \langle & \infty,h\rangle,\langle g,k\rangle\rangle\Big)_{\mu,z}\\
%&=
%\inf\left\{
%\begin{matrix}
%h_{\mu',w} + g_{\mu'',x}+ \sum_{i}k_{\mu_{i},w_{i}}
%+
%f_{[w,w_{1},\dots, w_{n}]+x,z}\\
%\infty+ \sum_{i}k_{\mu_{i},w_{i}}
%+f_{[w,w_{1},\dots, w_{n}],z}
%\end{matrix}
%\mid 
%w,w_{i}\in Y, 
%\mu=\mu'+\mu''+\sum_{i}\mu_{i}
%\right\}\\
&=
\inf\left\{
h_{\rho',x'} + g_{\rho,x}+ \sum_{i}k_{\mu_{i},w_{i}}
+
f_{[w_{1},\dots, w_{n}]+x'+x,z}
\mid 
x,x',w_{i}\in Y, 
\mu=\rho'+\rho+\sum_{i}\mu_{i}
\right\}\\
&=
\inf\left\{
g_{\rho,x} + h_{\rho',x'}+ \sum_{i}k_{\mu_{i},w_{i}}
+
f_{[w_{1},\dots, w_{n}]+x+x',z}
\mid 
x,x',w_{i}\in Y, 
\mu=\rho+\rho'+\sum_{i}\mu_{i}
\right\}\\
&=
\Big(D(D(f))\circ \langle \langle\infty,g\rangle,\langle h,k\rangle\rangle\Big)_{\mu,z}
\end{align*}
}

\item[(D-curry)]  \ \ \ \ \  $D(\Lambda(f))=
\Lambda(D(f)\circ \langle \pi_{1}\times \infty, \pi_{2}\times \mathrm{id}\rangle)$:
by observing that both morphisms are 
in $\LREL_!(X+X, Z^{Y})= \Lawv^{\C M_{\mathsf{fin}}(X+X)\times \C M_{\mathsf{fin}}(Y)\times Z}$, 
and that $ \langle \pi_{1}\times \infty, \pi_{2}\times \mathrm{id}\rangle
\in \LREL_!( (X+X)+Y  , (X+Y)+(X+Y) )$,
we can compute:

{\small
\begin{align*}
\big(\Lambda(D(f)\circ \langle&  \pi_{1}\times \infty, \pi_{2}\times \mathrm{id}\rangle)\big)_{[x]\oplus\mu,\nu,z}\\
&=
\big(D(f)\circ \langle \pi_{1}\times \infty, \pi_{2}\times \mathrm{id}\rangle\big)_{([x]\oplus\mu)\oplus\nu, z}\\
&=
\inf\left\{
( \pi_{1})_{[x]\oplus \emptyset,x}+
\sum_{i}(\pi_{2})_{\emptyset\oplus [w_{i}],w_{i}}
+
\sum_{j}(\mathrm{id})_{[z_{j}],z_{j}}
 + D(f)_{([x]\oplus \emptyset)\oplus(\mu\oplus \nu)}
\mid
\begin{matrix}
\mu=[w_{1},\dots, w_{n}],\\
\nu=[z_{1},\dots, z_{m}]
\end{matrix}
\right\}\\
&=
\inf\left\{
0+0
+
0 + D(f)_{([x]\oplus \emptyset)\oplus(\mu\oplus \nu)}
\mid
\begin{matrix}
\mu=[w_{1},\dots, w_{n}],\\
\nu=[z_{1},\dots, z_{m}]
\end{matrix}
\right\}\\
&=
\big(D(f)\big)_{ ([x]\oplus\emptyset)\oplus(\mu\oplus\nu),z}\\
&=
f_{(\mu+x)\oplus\nu, z}\\
& =
\big(\Lambda(f)\big)_{\mu+x,\nu,z}=\big(D(\Lambda(f))\big)_{[x]\oplus\mu, \nu,z}
\end{align*}
}

If $\rho\neq [x]\oplus\mu$, then $(D(\Lambda(f)))_{\rho,\nu,z}=\infty$ and $
(\Lambda(D(f)\circ \langle  \pi_{1}\times \infty, \pi_{2}\times \mathrm{id}\rangle))_{\rho,\nu,z}=
(D(f)\circ \langle \pi_{1}\times \infty, \pi_{2}\times \mathrm{id}\rangle)_{\rho\oplus\nu, z}
$, and one can check that also this is $\infty$, using the second equation above and the fact that $(\pi_{1})_{\rho\oplus\emptyset,x}=\infty$.\end{description}


\end{proof}



\paragraph{Taylor Expansion }


To check the validity of Taylor expansion we must further check the following equation, given $f\in \LREL_!(C, B^{A})$ and $g\in \LREL_!(C,A)$:
$$
\mathsf{ev}\circ \langle f,g\rangle= \inf_{n\in \BB N}
(( \cdots (\Lambda^{-}(f) \underbrace{\star g)\cdots )\star g}_{n\text{ times}})\circ \langle \mathrm{id}, \infty\rangle
$$
where:
\begin{enumerate}
\item $\mathsf{ev}\in \LREL_!(B^{A}+A, B)$ is the canonical \emph{evaluation} morphism;

\item $\Lambda^{-}(\_):= \mathsf{ev}\circ (\_\times \mathrm{id})$ is the \emph{uncurry} operator;

\item given $f\in \LREL_!(C+A,B)$ and $g\in \LREL_!(C,A)$, 
$f\star g\in \LREL_!(C+A,B)$ is the morphism obtained by differentiating $f$ in its second component and applying $g$ in that component, i.e.~
$$
f\star g =  D(f)\circ \langle \langle \infty, g\circ \pi_{1}\rangle, \mathrm{id}_{C+A}\rangle.
$$ 

\end{enumerate}



Let us first compute the three morphisms $\mathsf{ev}, \Lambda^{-}$ and $\star$ explicitly:
\begin{enumerate}

\item $\mathsf{ev}\in \Lawv^{ \C M_{\mathsf{fin}}(  ( \C M_{\mathsf{fin}}(A)\times B)       +  A    ) \times B  }$ is given by
$$\mathsf{ev}_{\mu,y}=
 \begin{cases}
 0 & \text{ if } \mu=[ \langle\rho, y\rangle]  \oplus \rho \\
 \infty & \text{ otherwise}
 \end{cases}
 $$
and observe that, given $f\in \LREL_!(C, B^{A})$ and $g\in \LREL_!(C,A)$, 
$$
\big(\mathsf{ev}\circ \langle f,g\rangle \big)_{\chi, y}= 
\inf\Big \{ 
\sum_{i=1}^{m}g_{\chi_{i},x_{i}}+
f_{\chi', \langle [x_{1},\dots, x_{m}],y\rangle}
\mid 
x_{1},\dots, x_{m}\in A,
\chi= \chi'+\sum_{i=1}^{m}\chi_{i}, 
\Big \}
$$



\item given $g\in \LREL_!(C, B^{A})$, 
$\Lambda^{-}(g)\in \LREL_!(C+A, B)$ is given by 
$$
\big(\Lambda^{-}(g)\big)_{\rho\oplus\mu,y}=g_{\rho, \langle \mu,y\rangle}
$$


\item $f\star g$ is given by 
$$
(f\star g)_{\rho\oplus\mu,y}=
\inf\Big\{
g_{\rho',x}+
f_{\rho''\oplus(\mu+x)}
\mid
x\in A,
\rho= \rho'+\rho''
\Big\}
$$

\end{enumerate}


Given the definition of $\mathsf{ev}\circ \langle f,g\rangle$, to check the Taylor equation it is enough to check that, for all $N\in \BB N$, 
$$
\left((( \cdots (\Lambda^{-}(f) \underbrace{\star g)\cdots )\star g}_{N\text{ times}})\circ \langle \mathrm{id}, \infty\rangle\right)_{\chi,y}=
\inf\Big \{ 
\sum_{i=1}^{N}g_{\chi_{i},x_{i}}+
f_{\chi', \langle [x_{1},\dots, x_{N}],y\rangle}
\mid 
\begin{matrix}
x_{1},\dots, x_{N}\in A,\\
\chi= \chi'+\sum_{i=1}^{N}\chi_{i}
\end{matrix}
\Big \}
$$

Let us show, by induction on $N$, the following equality, from which the desired equality easily descends:
$$
\big(( \cdots (\Lambda^{-}(f) \underbrace{\star g)\cdots )\star g}_{N\text{ times}}\big)_{\chi\oplus\mu,y}=
\inf\Big \{ 
\sum_{i=1}^{N}g_{\chi_{i},x_{i}}+
f_{\chi', \langle\mu+ [x_{1},\dots, x_{N}],y\rangle}
\mid 
\begin{matrix}
x_{1},\dots, x_{N}\in A,\\
\chi= \chi'+\sum_{i=1}^{N}\chi_{i}
\end{matrix}
\Big \}
$$
\begin{itemize}

\item if $N=0$, the right-hand term reduces to 
$f_{\chi, \langle \mu, y\rangle}=(\Lambda^{-}(f))_{\chi\oplus\mu,y}$;

\item otherwise, let $F=(( \cdots (\Lambda^{-}(f) \underbrace{\star g)\cdots )\star g}_{N-1\text{ times}})$, so that by I.H.~we have
$$
F_{\chi\oplus\mu,y}=
\inf\Big \{ 
\sum_{i=1}^{N-1}g_{\chi_{i},x_{i}}+
f_{\chi', \langle\mu+ [x_{1},\dots, x_{N-1}],y\rangle}
\mid 
\begin{matrix}
x_{1},\dots, x_{N-1}\in A,\\
\chi= \chi'+\sum_{i=1}^{N-1}\chi_{i}
\end{matrix}
\Big \}
$$
Then we have
{\small
\begin{align*}
\big( F\star g\big)_{\chi\oplus\mu,y}
&=
\inf \left \{
g_{\chi',x}+F_{\chi''\oplus(\mu+x)}
\mid
x\in A, \chi=\chi'+\chi''
\right\}\\
&=
\inf\left \{ 
g_{\chi',x}+
\inf\left\{
\sum_{i=1}^{N-1}g_{\chi_{i},x_{i}}+
f_{\chi'', \langle\mu+ [x_{1},\dots, x_{N-1}]+x,y\rangle}
\mid 
\begin{matrix}
x_{1},\dots, x_{N-1}\in A,\\
\chi^{*}= \chi''+\sum_{i=1}^{N-1}\chi_{i}
\end{matrix}
\right\}
\ 
\Big\vert \ 
\begin{matrix}
x\in A,\\
\chi=\chi'+\chi^{*}
\end{matrix}
\right \}\\
&=
\inf\Big \{ 
g_{\chi',x}+
\sum_{i=1}^{N-1}g_{\chi_{i},x_{i}}+
f_{\chi'', \langle\mu+ [x_{1},\dots, x_{N-1}]+x,y\rangle}
\mid 
\begin{matrix}
x,x_{1},\dots, x_{N-1}\in A,\\
\chi= \chi'+\chi''+\sum_{i=1}^{N-1}\chi_{i}
\end{matrix}
\Big \}\\
&=
\inf\Big \{ 
\sum_{i=1}^{N}g_{\chi_{i},x_{i}}+
f_{\chi', \langle\mu+ [x_{1},\dots, x_{N}],y\rangle}
\mid 
\begin{matrix}
x_{1},\dots, x_{N}\in A,\\
\chi= \chi'+\sum_{i=1}^{N}\chi_{i}
\end{matrix}
\Big \}.
\end{align*}
}
\end{itemize}

