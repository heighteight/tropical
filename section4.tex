In this section we study the tropical Laurent series $\Lawv^X\to \Lawv ^Y$ from the viewpoint of analysis.

\subsection{Results on Tropical Laurent Series}
{\color{red}

- recall Trop Laurent Series (TLS) tropical roots

- Example 1

- The morphisms of $\LREL_{!}$ can be seen as generalizations of TLS

- Remark: for $f:\Lawv^{X}\to \Lawv^{Y}$, if $X$ is finite, $f$ is a finite variable TLS; if $f$ is written as a finite $\min$, it is a tropical polynomial. 

- Maybe Theorem 7 could even go here?? Saying something like: after the remark above, we can prove a result that confirms what seen in Example 1, for the case when $X$ is finite: $f$ can LOCALLY be seen as a polynomial, the degree of such polynomials potentially diverging when $\epsilon\to 0$. Indeed...


- bisognerebbe anche dire due parole sul differenziale e dell'espansione di Taylor come operazione sulle TLS invece che sulle matrici. 
}

Let us start by recalling Example~\ref{ex:famous_ex}.
By plotting its graph {\color{red}vogliamo plottarlo?}, we see first of all that the function is non-decreasing and concave.
It is easy to see that this is actually always the case:

\begin{proposition}\label{prop:nondecr+conc}
 Any tropical Laurent series $f:\Lawv^X\to\Lawv^Y$ is non-decreasing and concave, w.r.t.\ the pointwise order.
\end{proposition}

\subsection{$\Lawv^{X}$ as a normed space.}

{\color{red}Move here also the cone part}

The $f$ of Example~\ref{ex:famous_ex} is continuous on $\BB R_{\geq0}=\Lawv-\set{\infty}$ (w.r.t.\ the usual norm of real numbers).
By considering the usual norm $\norm{x}_\infty:=\sup\limits_{a\in X} \absv{x_a}$ on $\Lawv^X$, we could generalise this property by dropping the case of $x$ having some $0$ coordinate:

\begin{theorem}\label{thm:cont}
 Any tropical Laurent series $f:\Lawv^X\to\Lawv$ is continuous on $\BB R_{>0}^X$, w.r.t.\ to the norm $\norm{\cdot}_\infty$.
\end{theorem}
\begin{proof}
 The result follows after proving that if a real-valued function on a locally convex topological $\BB R$-vector space is, locally around $x$, concave and bounded by a finite constant, then it is continuous at $x$.
\end{proof}

Not only the $f$ of Example~\ref{ex:famous_ex} is continuous, but it is also locally Lipschitz on $\BB R_{>0}$.
Actually, we can prove the following:

\begin{theorem}\label{theorem:fepsilon}
 Let $f:\Lawv\to\Lawv$ a tropical Laurent series with matrix $\widehat f:\N\to\Lawv$.
 For all $0<\epsilon<\infty$, there is a \emph{finite} $\C F_\epsilon \subseteq \N$ s.t.:
 \begin{enumerate}
  \item If $\C F_\epsilon=\emptyset$ then $f=\infty$ on all $\Lawv$;
  \item If $f(x_0)=\infty$ for some $x_0<\infty$, then $\C F_\epsilon=\emptyset$;
  \item On all $[\epsilon,\infty]$, $f$ coincides with the tropical \emph{polynomial} $P_\epsilon(x):=\min\limits_{n\in \C F_\epsilon}\set{nx+\widehat f(n)}$.
 \end{enumerate}
 In particular, $f$ is locally Lipschitz on all $\BB R_{>0}$.
\end{theorem}
\begin{proof}
 We can let $\C F_\epsilon:=\set{n\in\N \mid
 \widehat f(n)<\infty \textit{ and } \widehat f(m)> \widehat f(m)+\epsilon \textit{ for all } m<n}$.
\end{proof}

We believe that a variant of such result holds in $\Lawv^X$ for $X$ finite.
We do not know what happens if $X$ is infinite.

Remark that, in coherence with the previous result, in Example~\ref{ex:famous_ex} $f(x)$ is indeed a $\min$ for all $x>0$.
At $x=0$ we have $f(x=0)=\inf\limits_{n\in\N} \frac{1}{2^n}=0$ which is \emph{not} a min.
Also, while the derivative of $f$ is bounded on all $\BB R_{>0}$, at $x=0$ it tends to $\infty$.
This phenomenon is reminiscent of [Example 7, PCoh]
%Differentials and Distances in Probabilistic Coherence Spaces. FSCD 2019
, which actually motivated our first investigations.




\subsection{$\Lawv^{X}$ as a metric space.}



The norm $\|-\|_\infty$ naturally induces a metric $\|\B x-\B y\|_{\infty}$ over the spaces $\Lawv^{X}$. 
We now study the Lipschitz properties of tropical Laurent series w.r.t.~these metrics. 

A first result is the following:
\begin{proposition}\label{prop:troplinear}
All tropical linear functions $f: \Lawv^{X}\to \Lawv^{Y}$ are non-expansive.  
\end{proposition}
\begin{proof}[Proof sketch]
Using the fact that $f(\B x)_{b}= \inf_{a\in X}\check f_{a,b}+\B x_{a}$,
the problem reduces to checking that $|(\check f_{a,b}-\B x_{a})- (\check f_{a,b}-\B y_{a})| = |\B x_{a}-\B y_{a}|\leq \| \B x-\B y\|_{\infty}$.\end{proof}
This result shows that, in analogy with that happens in usual metric semantics, linear programs are interpreted by non-expansive functions. 
%\begin{proof}
%Using $f(\B x)_{b}= \inf_{a\in X}\check f_{a,b}+\B x_{a}$,
%first observe that $|(\check f_{a,b}-\B x_{a})- (\check f_{a,b}-\B y_{a})| = |\B x_{a}-\B y_{a}|\leq \| \B x-\B y\|_{\infty}$; we now have
%$|f(\B x)_{b}-f(\B y)_{b}| \leq |(\inf_{a\in X}\check f_{a,b}-\B x_{a})-(\inf_{a\in X}\check f_{a,b}-\B y_{a})| \leq
%\sup_{a\in X}|(\check f_{a,b}-\B x_{a})- (\check f_{a,b}-\B y_{a})|\leq 
% \| \B x-\B y\|_{\infty}$.
%\end{proof}

Before looking at what happens in the case of non-linear programs, let us make the metric structure of $\LREL$ explicit. 

Using the fact that the tropical linear functions from $\Lawv^{X}$ to $\Lawv^{Y}$ are in bijection with the elements of the hom-set $\LREL(X,Y)\simeq \Lawv^{X\times Y}$ through the map $f\mapsto \check f$, the following proposition provides a useful characterization of the functional metrics in $\LREL$:
\begin{proposition}
For all sets $X,Y$ and tropical linear functions $f,g:\Lawv^{X}\to \Lawv^{Y}$, $\| \check f-\check g\|_{\infty} =  \sup_{\B x\in \Lawv^{X}}
\| f(\B x)-g(\B x)\|_{\infty}$.\end{proposition}

To study what happens in the case of tropical Laurent series, let us first consider the case of bounded exponentials $!_{\leq n}X$:
\begin{proposition}\label{prop:boundedlip}
All tropical linear functions $f: \Lawv^{\C M_{\leq n}(X)}\to \Lawv^{Y}$ are $n$-Lipschitz-continuous.
\end{proposition}
\begin{proof}[Proof sketch]
Using the fact that $f(\B x)_{b}=\inf_{\mu\in \C M_{\leq n}(X)}\{ \check f_{\mu,b}+ \mu (!_{n}\B x) \}$, where $!_{n}\B x\in \Lawv^{\C M_{\leq n}(X)}$ is given by 
$(!_{n}\B x)_{[a_{1},\dots, a_{k}]}=\sum_{i=1}^{k}\B x_{a_{i}}$, 
it suffices to check that $\| (!_{n}\B x)-(!_{n}\B y)\|_{\infty}\leq n\cdot \| \B x-\B y\|_{\infty}$ and apply Proposition \ref{prop:troplinear}.
\end{proof}
This result is perfectly analogous to what happens in standard models for sensitivity analysis based on the re-scaling trick recalled in Section \ref{section2}.

We can now look at what happens in the case of the full exponential comonad, i.e.~with the tropical Laurent series.
First, let us observe that, as expected, tropical polynomials 
correspond to Lipschitz functions:
\begin{proposition}
Any tropical polynomial $\varphi:\Lawv^{\{1,\dots, K\}}\to\Lawv$ is $\mathrm{deg}(\varphi)$-Lipschitz continuous.
\end{proposition}
\begin{proof}
This immediately follows from Prop.~\ref{prop:boundedlip}, since a tropical polynomial can be represented as a tropical linear function from $\Lawv^{\C M_{\mathrm{deg}(\varphi)(\{1,\dots, K\})}}$ to $\Lawv$.
\end{proof}

The proposition above, together with Theorem \ref{theorem:fepsilon}, can be used to show that tropical Laurent series with \emph{finitely many} variables are always locally Lipschitz over $(0,\infty)^{X}$. Yet, we can prove a more general statement, covering also the infinitary case.


\begin{theorem}
 All tropical Laurent series $\Lawv^X\to\Lawv$ are locally Lipschitz on $\BB R_{>0}$.
\end{theorem}
\begin{proof}[Proof sketch]
The core of the proof is a convex analysis argument (see the Appendix) showing that an arbitrary function $f:\Lawv^X\to\Lawv$ which is non-decreasing, concave and continuous, must be locally Lipschitz. 
\end{proof}


The results just presented translate into the following facts about the interpretation of higher-order programs:
\begin{corollary}
For any $\lambda$-term $M$:
\begin{enumerate}
\item if $\Gamma \vdash_{\BSTLC} M:A$, then $\model M:\model\Gamma \to \model A$ is a Lipschitz map.
\item if $\Gamma \vdash_{\STLC}M:A$, then $\model M: \model\Gamma \to \model A$ is a locally Lipschitz map.
\end{enumerate}
\end{corollary} 


Finally, let us discuss the differential structure. We show that the distance between two tropical maps can be approximated using the terms appearing in their Taylor expansions:
\begin{proposition}
For all tropical Laurent series $f,g: \Lawv^{X}\to \Lawv^{Y}$, and for all $n\in \BB N$, 
the functions
$\delta^{(n)}f, \delta^{(n)}g: \Lawv^{\C M_{\leq n}(X)}\to \Lawv ^{Y}$, with 
 $\delta^{(n)}f(\B x)= \Der^{(n)}f(\B x, \infty)$, appearing in their Taylor expansions, are $n$-Lipschitz-continuous, and moreover 
\begin{align}
\| \check f-\check g\|_{\infty}= \sup_{n} \| \check{\delta^{(n)}f}- \check{\delta^{(n)}g}\|_{\infty}
\end{align}
%where $\delta^{(n)}f:( \Lawv^{X})^{n}\to \Lawv^{X}$ is the tropical linear function $\delta^{(n)}f(\B x_{1},\dots, \B x_{n})=
%(\Der^{(n)}f)(\B x_{1},\dots, \B x_{n}, \infty)$. 
\end{proposition} 


%A consequence of this result is that the distance between two differential programs can always be approximated via bounded programs:
%\begin{corollary}
%For all $\lambda$-terms $M,N: A\to B$ of $\STDLC$, for all $\epsilon >0$, there exists $K\in \BB N$ such that 
%$|\|\model{M}-\model{N}\|_{\infty}- | \| \model{
%\end{corollary}


{\color{red}MOVE:
Let us end this section by mentioning another point of view on tropical Laurent series.
$\Lawv^X$ with the usual $+$ and the usual $\cdot$ is a $\BB R_{\geq0}$-semimodule.
Together with the norm $\norm{\cdot}_\infty$, it can be proved that it is a Scott-complete normed cone.
The normed cone structure induces an order on it, called its \emph{cone order}, by setting:
$x\leq y$ iff $y=x+z$ for some (unique) $z\in\Lawv^X$.
This order makes it a Scott-continuous dcpo.
Furthermore we have:

\begin{proposition}
  Tropical Laurent series $\Lawv^X\to\Lawv^Y$ are Scott-continuous on $\BB R_{>0}^X$, w.r.t.\ the cone orders on the domain and codomain.
\end{proposition}
}
