\documentclass[submission,%copyright,creativecommons
]{eptcs}
\providecommand{\event}{TLLA 2023} % Name of the event you are submitting to
%\usepackage{breakurl}             % Not needed if you use pdflatex only.
\usepackage{underscore}           % Only needed if you use pdflatex.

\usepackage{tikz}
  \usetikzlibrary{calc}
\usepackage{hyperref}
\usepackage{stmaryrd,amssymb,proof,cmll,mathrsfs,amsmath,amsthm}
\usepackage{todonotes}



% MATH TEXT STYLES
\newcommand{\B}[1]{\mathbf{#1}}
\newcommand{\BB}[1]{\mathbb{#1}}
\newcommand{\C}[1]{\mathcal{#1}}
\newcommand{\F}[1]{\mathfrak{#1}}
\newcommand{\TT}[1]{\mathtt{#1}}
\newcommand{\RM}[1]{\mathrm{#1}}
\newcommand{\SF}[1]{\mathsf{#1}}



%CATEGORIES

\newcommand{\Met}{\mathsf{Met}}
\newcommand{\Mod}{\Lawv\mathsf{Mod}}
\newcommand{\GMet}{\Lawv\mathsf{CCat}}
\newcommand{\Fun}{\mathsf{Fun}}
\newcommand{\colim}{\mathrm{colim}}
\newcommand{\Yon}{\B{Y}}
\newcommand{\Hom}{\mathrm{Hom}}
\newcommand{\Sym}{\mathrm{Sym}}
\newcommand{\matr}[1]{\hat{#1}}

\newcommand\pfun{\mathrel{\ooalign{\hfil$\mapstochar\mkern5mu$\hfil\cr$\to$\cr}}}



% LAMBDA CALCULI

\newcommand{\lamcalc}{$\lambda$-calculus}
\newcommand{\lam}{\lambda}

\newcommand{\STLC}{\RM{STLC}}
\newcommand{\BSTLC}{\mathsf b\RM{STLC}}
\newcommand{\RSTLC}{\C R\RM{STLC}}
\newcommand{\STDLC}{\RM{ST}\partial\RM{LC}}
\newcommand{\Real}{\SF{Real}}

\newcommand{\Der}{\SF D}
\newcommand{\To}{\Rightarrow}
\newcommand{\Diff}[2]{\Der[#1,#2]}

\newcommand{\finMS}[1]{\C M_{\RM{fin}}(#1)}

\newcommand{\true}{\prog{True}}
\newcommand{\false}{\prog{False}}
\newcommand{\bool}{\SF{Bool}}

\newcommand{\Te}[1]{\C T(#1)}

\newcommand{\prog}[1]{\mathtt{#1}}




% METRIC STUFF
\newcommand{\Lawv}{\BB L}
\newcommand{\QualREL}[1]{#1 \SF{Rel}}
\newcommand{\QREL}{\QualREL{Q}}
\newcommand{\LREL}{\QualREL{\Lawv}}
\newcommand{\LCAT}{\Lawv\SF{CCat}}

\newcommand{\op}{\mathrm{op}} 
\newcommand{\sk}{\mathrm{sk}} 
\newcommand{\sym}{\mathrm{sym}} 
\newcommand{\menus}{\dotdiv} 

\newcommand{\norm}[1]{\lVert#1\rVert}
\newcommand{\supnorm}[1]{\lVert#1\rVert_\infty}
\newcommand{\absv}[1]{\left\lvert#1\right\rvert}

% TROPICAL STUFF

\newcommand{\trop}[1]{\SF t #1}
\newcommand{\model}[1]{\llbracket#1\rrbracket}
\newcommand{\nodel}[1]{\langle #1\rangle}
\newcommand{\sumt}[1]{{+}^{#1}}
\newcommand{\prodt}[1]{{\times}^{#1}}


% MISCELLANEOUS

\newcommand{\HOM}[3]{{#1}(#2,#3)}
\newcommand{\N}{\BB N}
\newcommand{\R}{\BB R}
\newcommand{\set}[1]{\{#1\}}
\newcommand{\multiset}{\C M_{\mathrm{fin}}}

\newcommand{\eps}{\epsilon}

\newcommand{\twoheaddownarrow}{\mathrel{\rotatebox[origin=c]{270}{$\twoheadrightarrow$}}\!}


%LIST ENVIRONMENTS


\newenvironment{varenumerate}
{
	\begin{list}{\arabic{numberone}.}
		{
			\usecounter{numberone}
			\setlength{\itemsep}{0pt}
			\setlength{\topsep}{0pt}
			\setlength{\parsep}{0pt}
			\setlength{\partopsep}{0pt}
			\setlength{\leftmargin}{15pt}
			\setlength{\rightmargin}{0pt}
			\setlength{\itemindent}{0pt}
			\setlength{\labelsep}{5pt}
			\setlength{\labelwidth}{15pt}
	}}
	{
	\end{list} 
}





%MATH EVIRONMENTS











\title{Tropical methods in Lambda-Calculus?}
\author{Davide Barbarossa
\institute{DISI, Universit\`a di Bologna}
\email{davide.barbarossa@unibo.it}
\and
\qquad\qquad Paolo Pistone
\institute{\qquad\qquad\qquad DISI, Universit\`a di Bologna}
\email{\qquad\qquad\qquad paolo.pistone@unibo.it}
}
\def\titlerunning{Trends in Linear Logic and Applications}
\def\authorrunning{D.\ Barbarossa and P.\ Pistone}

\newcommand\eg{\textit{e.g.\ }}
\newcommand\etc{\textit{etc}}


\begin{document}
\maketitle

%\begin{abstract}We propose to study the interpretation of the $\lambda$-calculus in the framework of tropical mathematics, as a unified framework for both program metrics -- based on the analysis of program sensitivity via Lipschitz-conditions -- and for resource analysis -- based on higher-order program differentiation.We sketch the relation of this semantics to quantitative properties like differential privacy, convergence logprobabilities and Probabilistic Coherent spaces.Finally, we study the abstract correspondence between this tropical semantics and Lawvere’s generalised metric spaces.\end{abstract}

%\section*{Motivation}

Two different quantitative approaches have received considerable attention from the programming language community in recent years: %(e.g.\ \cite{decarvalho2018, Accattoli2022}, \cite{Breuvart2018, PistoneLICS2022}, \cite{Barthe_2012} \cite{Reed2010, dallago}, \cite{Ehrhard2011, Staton2017}, \cite{difflambda}), on the one hand one there is 
the approach of \emph{program metrics} \cite{Reed2010, Gaboardi2017, Gabo2019} and \emph{quantitative equational theories} \cite{Plotk} is based on the fact that when considering e.g.\ probabilistic %or approximate 
computation, %rather than asking whether two programs compute \emph{the same} function,
it is natural to ask whether two programs compute functions which %do not differ \emph{too much}
approximate each other (instead of equality of functions).
This led to denotational frameworks in which types are endowed with a metric \cite{Reed2010},\cite{Bonchi2018}, \cite{Geoffroy2020, PistoneLICS, PistoneFSCD2022}.
On the other hand, there is the approach based on \emph{differential} \cite{difflambda}, \cite{difflambda}, \cite{Manzo2013, Breuvart2018, PistoneLICS2022} or \emph{resource-aware} \cite{Boudol1993} extensions of the $\lambda$-calculus, which is well-connected to the so-called \emph{relational semantics} \cite{Manzo2012, Manzo2013, dill} and has a syntactic counterpart in the study of \emph{non-idempotent} intersection types \cite{decarvalho2018, Mazza2016}.
This led to syntactic or denotational frameworks in which one can define a \emph{Taylor expansion} of programs.

In both approaches the notion of \emph{linearity}, in the sense of linear logic \cite{girardLl} (i.e.~of using inputs exactly once), plays a crucial role.
In metric semantics, linear programs correspond to \emph{non-expansive} maps, that is, to functions that do not increase distances; moreover, the possibility of duplicating inputs leads to interpret programs with a fixed duplication bound as \emph{Lipschitz-continuous} maps \cite{Gaboardi2017}.
By contrast, in the standard semantics of the differential $\lambda$-calculus, linear programs correspond to linear maps, in the usual algebraic sense, while the possibility of duplicating inputs gives rise to \emph{power series}.

At a first glance, there seems to be a  ``logarithmic'' gap between the two approaches:
in metric models $n$ times duplication results in a  \emph{linear} $n$-Lipschitz function $n\cdot x$, while in differential models this results in a non-Lipschitz \emph{polynomial} function $x^{n}$.
The fundamental motivation of this work is the observation that 
this gap is naturally overcome once we interpret these functions in the framework of tropical mathematics where, for instance, the monomial $x^{n}$ precisely reads as the linear function $n\cdot x$.
Actually, a tropical variant of relational semantics has already been considered \cite{Manzo2013}, and shown capable of capturing \emph{best-case} quantitative properties, but has not yet been studied in detail. Furthermore, connections between tropical linear algebra and metric spaces have also been observed \cite{Fuji} within the abstract setting of \emph{quantale-enriched} categories \cite{Hofmann2014, Stubbe2014}.
However, a thorough investigation of the interpretation of the $\lambda$-calculus within tropical mathematics has not yet been undertaken. 

Tropical mathematics \cite{Simon} is a well established algebraic and geometrical framework, with tight connections with optimisation theory \cite{Sturmfelds}, where the usual ring structure of numbers based on addition and multiplication is replaced by the semiring structure given, respectively, by ``$\min$'' and ``$+$''.
For instance, the polynomial $p(x,y)=x^{2}+xy^{2}+y^{3}$, when interpreted over the tropical semiring, translates as the piecewise linear function
$
\trop f(\alpha,\beta)=\min\{2\alpha, \alpha+2\beta, 3\beta\}
$.

The viewpoint that we develop in the following sections can be read as the proposition of a setting for bridging the two approaches mentioned at the beginning, and may suggests the application of methods based on tropical mathematics to the study of the $\lambda$-calculus and its quantitative extensions.
It also scales to a 
more abstract level, leading to introduce a 
differential operator for continuous functors between \emph{generalized} metric spaces (in the sense of \cite{Lawvere1973}).
We will only sketch some of our main results.% of our our analysis.

\section{Tropical mathematics in a nutshell}

More rigorously, we let the \emph{tropical semiring}, the structure at the heart of tropical mathematics, be $([0,\infty], \min, +)$.
Remark that this coincides with the \emph{Lawvere quantale} $\Lawv=([0,\infty], \geq, +)$ \cite{Hofmann2014, Stubbe2014} (the order is the reverse order $\geq$ on $\BB R$ and the monoid action is addition), the structure at the heart of the categorical study of metric spaces initiated by Lawvere himself \cite{Lawvere1973}, and we will take this point of view in the last section of this contributions.
A tropical polynomial is a piece-wise linear function $f:\Lawv\to \Lawv$ of the form $f(x)=\min_{i_{1},\dots, i_{k}}\{i_{j}x+c_{i_{j}}\}$
where the $i_{j}$ are natural numbers and the $c_{i_{j}}$ are in $\Lawv$.
For example, the polynomials $\varphi_{n}(x)=\min_{i\leq n}\{ix+2^{-i}\}$
are illustrated in Fig.~\ref{fig:plot1} for $n\leq 4$.
A \emph{tropical root} of a tropical polynomials $\varphi$ is a point $x\in \Lawv$ where $\varphi$ is not differentiable. In other words, the roots of $\varphi$ are the points where the minimum defining $\varphi$ is attained at least twice (i.e.~where the slope of $\varphi$ changes). For instance, the tropical roots of $\varphi_{n+1}$ are of the form $2^{-(i+1)}$, for $i \leq n$.
With this definition, tropical roots mimic the usual factorization property of roots: if $x_{0}$ is a root of $f$, this factorizes as
$f(x)=\min\{x,x_{0}\}+ g(x)$. Yet, unlike in standard algebra, tropical roots can be computed in linear time \cite{Noferini2015}.
A \emph{tropical Laurent series} (of one variable $x\in\Lawv$), shortly a \emph{tLs}, is a function that can be expressed as $f(x)=\inf_{n}\{nx+\matr f_{n}\}$, with $\matr  f_{n}$ a sequence in $\Lawv$. In other words, a tLs is a ``limit'' of tropical polynomials of higher and higher degree. For instance, the function $\varphi(x):=\inf_{n\in\N}\set{nx+\frac{1}{2^n}}$ (illustrated in Fig.~\ref{fig:plot1}), that we will take as our running example, 
is the ``limit'' of the polynomials $\varphi_{n}$. Since $\inf$s are not in general $\min$s, the behavior of tLS may be less predictable than that of tropical polynomials. For instance, tropical roots for tLs (see \cite{Porzio2021}) may also include limit points.

\section{Tropical weighted semantics in a nutshell}

The study of matrices with values over the tropical semiring can be seen as a special case of the
\emph{quantitative relational semantics} \cite{Manzo2013}, a well-studied semantics of the $\lambda$-calculus and linear logic:
for a fixed \emph{continuous} semi-ring $Q$ [Def. II.5, \cite{Manzo2013}], the category $\QREL$ has sets as objects and set-indexed matrices with coefficients in $Q$ as morphisms, i.e.~$\QREL(X,Y)=Q^{X\times Y}$ (\cite{Manzo2013} would call it $Q^\Pi$).
As it is expected, $Q^X$ is a $Q$-semimodule and the bijection $\hat{(\cdot)}$ identifies the set of linear maps from $Q^X$ to $Q^Y$ with 
 $\HOM{\QREL}{X}{Y}$.

The tropical relational model is thus provided by the category $\LREL$ of matrices with values over $\Lawv$ (which, being a quantale, is indeed a continuous semi-ring).
%, where $\Lawv$ is the already introduced Lawvere quantale, seen as the idempotent complete semiring $(\BB R_{\geq0}\cup\set{\infty},\inf,\infty,+,0)$.
%The category $\LREL$ is well-defined because $\Lawv$ is a continuous semiring (w.r.t.\ its quantale order $\preceq$.
%This amounts to check that $\min$ and $+$ commute with the $\inf$ (as operations on $\BB R_{\geq0}\cup\set{\infty}$, which is immediate), and that $(\Lawv,\preceq)$ is a cpo with $\infty$ as bottom element (which is immediate since in $\Lawv$ we have $\vee = \inf$) .
It is worth observing that the formula for composition in $\LREL$ %is the tropicalisation of the one defining it in $\QREL$, i.e.
reads as \ $(st)_{a,c}:=\inf\limits_{b\in Y}\set{s_{b,c}+t_{a,b}}$;
 similarly, the linear functions $f:\Lawv^X\to \Lawv^Y$ induced by matrices, which we call \emph{tropical linear}, are exactly those of shape $f(x)_b=\inf\limits_{a\in X} \set{\matr f_{a,b}+x_a}$, for some matrix $\matr f$ from $X$ to $Y$.
%\end{remark}
By applying more or less well-known results (taken from \cite{Manzo2013}, CITARE LEMAYYYY), one obtains that $\LREL$ gives rise to denotational models of several variants of the $\lam$-calculus:

\begin{fact}
 $\LREL$ is a SMCC, thus a model of the linear $\STLC$.

 The coKleisli $\LREL_!$ is CCC, i.e.\ a model of $\STLC$.

The exponential $!$ can be ``decomposed'' into a family of \emph{graded} exponentials $(!_n)_{n\in\N}$ turning $\LREL$, together with $(!_n)_{n\in\N}$, is a model for $\BSTLC$. 

 $\LREL_!$ equipped with $D$, is a CC$\partial$C and the interpretation of the Taylor expansion of a $\STLC$-term $M$, given in \eqref{eq:taylor}, converges to the one of $M$.
\end{fact}

In the previous result, the coKleisli composition of $s\in\Lawv^{!Y\times Z}$ and $t\in\Lawv^{!X\times Y}$ is the matrix $s\circ_! t\in\Lawv^{!X\times Z}$ where $(s\circ_! t)_{\mu,c}$ is:
$
 \inf\limits_{n\in\N, b_1\dots,b_n\in Y, \mu = \mu_1+\cdots +\mu_n}
 \left\{s_{[b_1,\dots,b_n],c} + \sum\limits_{i=1}^n t_{\mu_i,b_i}\right\}$.
and the \emph{tropical differential operator} is the map $D:\HOM{\LREL}{!X}{Y}\to \HOM{\LREL}{!(X+X)}{Y}$ defined as $(Dt)_{\mu\oplus\rho,b}=t_{\rho+\mu,b}$ if $\#\mu=1$ and as $\infty$ otherwise (where a multiset $\nu \in !(X+X)$ is identified with a disjoint sum of $\mu,\rho\in !X$).


In particular, the Taylor formula decomposes an unbounded application as a limit of bounded ones, one might well ask whether it could be possible to see this formula as interpreting  a $\lambda$-term 
as a limit of Lipschitz maps, in some sense, thus bridging the metric and differential approaches.  
Here, a natural direction to look for is the \emph{relational semantics}, i.e.~the somehow canonical ``Taylor'' semantics for $\STDLC$. 
However, in this semantics, terms with bounded applications correspond to \emph{polynomials}, i.e.~to non-Lipschitz functions. 

Yet, what if these polynomials were tropical ones, i.e.~piecewise linear functions? This way, \eqref{eq:taylor} could really be interpreted as a decomposition of $\lambda$-terms via limits (indeed, $\inf$s) of Lipschitz maps. In other words, unbounded term application could be seen 
as a limit of \emph{more and more sensitive} operations. 

\section{Tropical Laurent Series}

 As usual, a matrix $t\in\HOM{\LREL_!}{X}{Y}$ yields a linear map $\Lawv^{!X}\to\Lawv^Y$, but we can also ``express it in the base $X$'', i.e.\ see it as a map $t^!:\Lawv^X\to\Lawv^Y$, by setting 
 $t^!(x):=t\circ_! x$.
%  (we are identifying $\Lawv^X$ with the set $\HOM{\LREL_!}{\emptyset}{X}$ of the \emph{points} of $X$).
 This is the notion of \emph{non-linear} map generated by the CCC-structure of $\LREL_!$.
 Concretely, we have
 \begin{equation}
 t^!(x)_b=\inf\limits_{\mu\in !X} \set{\mu x+ t_{\mu,b}}
 \end{equation}
 where $\mu x:=\sum\limits_{a\in X} \mu(a)x_a$.
 These functions correspond then to tLs with possibly infinitely many variables (in fact, as many as the elements of $X$). 

Therefore, we study in this section the properties of those functions, from the point of view of mathematical analysis.
%In the following we will also refer to them as tLs. 
 
% 
% We will call them simply \emph{tropical Laurent series (tLs)}.
% %Since in the general case of $\QREL$, $t^!$ is a Laurent series with operations in $Q$, let us call \emph{tropical Laurent series} the functions of shape $t^!$ for some $t\in\HOM{\LREL_!}{X}{Y}$.
%\end{remark}
%
%We find the usual notion of tLs of one variable as follows:

%\begin{remark}
Notice that, by identifying $!\set{*}\simeq \N$ and $\Lawv^{\set{*}}\simeq\Lawv$, the tLs generated by the morphisms in $\HOM{\LREL_!}{\set{*}}{\set{*}}$ are exactly the functions $f:\Lawv\to\Lawv$ of shape $f(x)=\inf_{n\in\N}\set{nx+\matr f(n)}$, for some $\matr f:\N\to\Lawv$, i.e.\ usual tLs's of one variable.
In a similar way, the tropical polynomials can be identified with the tLs
$f:\Lawv^X\to\Lawv^Y$ 
 for which the support $\C F=\set{\mu\in!X\mid\matr f_{\mu,b}\neq\infty}$ is \emph{finite}, and which have thus shape $f(x)_b:=\min_{\mu\in\C F} \set{\mu x+ t_{\mu,b}}$.
 This is again the generalisation of usual tropical polynomials to the case of infinitely many variables.


\section{Relations to quantitative properties}



\section{Lawvere's generalised metric spaces}

We already remarked that the tropical semiring $\Lawv$ coincides with the \emph{Lawvere quantale} $\Lawv=([0,\infty], \geq, +)$.
In particular, a (possibly $\infty$) metric on a set $X$ is nothing but a ``$\Lawv$-valued square matrix'' $d:X\times X\to \Lawv$ satisfying axioms like e.g.~the triangular law (indeed, such distance matrices correspond to $\Lawv$-\emph{enriched categories}, a viewpoint we explicitly take in Section \ref{section6}). 

\begin{remark}
 The category $\QREL$ is (equivalent to) a subcategory of the category $Q\SF{Mod}$ of \emph{complete} $Q$-semimodules.
 If $\QREL$ corresponds to considering semimodules (the $Q^X$'s) whose vectors are given in coordinates w.r.t.\ a \emph{fixed base} (the set $X$), $Q\SF{Mod}$ corresponds to considering semimodules in abstract, without fixing a base.
 We take this viewpoint in Section~\ref{section6}.
\end{remark}


%The optional arguments of {\tt $\backslash$documentclass$\{$eptcs$\}$} are
%\begin{itemize}
%\item at most one of
%{\tt adraft},
%{\tt submission} or
%{\tt preliminary},
%\item at most one of {\tt publicdomain} or {\tt copyright},
%\item and optionally {\tt creativecommons},
%  \begin{itemize}
%  \item possibly augmented with
%    \begin{itemize}
%    \item {\tt noderivs}
%    \item or {\tt sharealike},
%    \end{itemize}
%  \item and possibly augmented with {\tt noncommercial}.
% \end{itemize}
%\end{itemize}
%We use {\tt adraft} rather than {\tt draft} so as not to confuse hyperref.
%The style-file option {\tt submission} is for papers that are
%submitted to {\tt $\backslash$event}, where the value of the latter is
%to be filled in in line 2 of the tex-file. Use {\tt preliminary} only
%for papers that are accepted but not yet published. The final version
%of your paper that is to be uploaded at the EPTCS website should have
%none of these style-file options.

%By means of the style-file option
%\href{http://creativecommons.org/licenses/}{creativecommons}
%authors equip their paper with a Creative Commons license that allows
%everyone to copy, distribute, display, and perform their copyrighted
%work and derivative works based upon it, but only if they give credit
%the way you request. By invoking the additional style-file option {\tt
%noderivs} you let others copy, distribute, display, and perform only
%verbatim copies of your work, but not derivative works based upon
%it. Alternatively, the {\tt sharealike} option allows others to
%distribute derivative works only under a license identical to the
%license that governs your work. Finally, you can invoke the option
%{\tt noncommercial} that let others copy, distribute, display, and
%perform your work and derivative works based upon it for
%noncommercial purposes only.

%Authors' (multiple) affiliations and emails use the commands
%{\tt $\backslash$institute} and {\tt $\backslash$email}.
%Both are optional.
%Authors should moreover supply
%{\tt $\backslash$titlerunning} and {\tt $\backslash$authorrunning},
%and in case the copyrightholders are not the authors also
%{\tt $\backslash$copyrightholders}.
%As illustrated above, heuristic solutions may be called for to share
%affiliations. Authors may apply their own creativity here \cite{multipleauthors}.



%\section*{Bibliography}

%We request that you use
%\href{http://eptcs.web.cse.unsw.edu.au/eptcs.bst}
%{\tt $\backslash$bibliographystyle$\{$eptcs$\}$}
%\cite{bibliographystylewebpage}, or one of its variants
%\href{http://eptcs.web.cse.unsw.edu.au/eptcsalpha.bst}{eptcsalpha},
%\href{http://eptcs.web.cse.unsw.edu.au/eptcsini.bst}{eptcsini} or
%\href{http://eptcs.web.cse.unsw.edu.au/eptcsalphaini.bst}{eptcsalphaini}
%\cite{bibliographystylewebpage}. Compared to the original {\LaTeX}
%{\tt $\backslash$biblio\-graphystyle$\{$plain$\}$},
%it ignores the field {\tt month}, and uses the extra bibtex fields {\tt eid}, {\tt doi}, {\tt ee} and {\tt url}. The first is for electronic identifiers (typically the number $n$ indicating the $n^{\rm th}$ paper in an issue) of papers in electronic journals that do not use page numbers. The other three are to refer, with life links, to electronic incarnations of the paper.

%\nocite{*}
%\bibliographystyle{eptcs}
%\bibliography{generic}

%\bibliographystyle{plain}
%\bibliography{tropical.bib}

\end{document}