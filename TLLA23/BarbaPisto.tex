\documentclass[submission,%copyright,creativecommons
]{eptcs}
\providecommand{\event}{TLLA 2023} % Name of the event you are submitting to
%\usepackage{breakurl}             % Not needed if you use pdflatex only.
\usepackage{underscore}           % Only needed if you use pdflatex.

\usepackage{macros}
\usepackage{tikz}
  \usetikzlibrary{calc}
\usepackage{hyperref}
\usepackage{stmaryrd,amssymb,proof,cmll,mathrsfs,amsmath,amsthm}
\usepackage{todonotes}


\title{Tropical methods in Lambda-Calculus}
\author{Davide Barbarossa
\institute{DISI, Universit\`a di Bologna}
\email{davide.barbarossa@unibo.it}
\and
\qquad\qquad Paolo Pistone
\institute{\qquad\qquad\qquad DISI, Universit\`a di Bologna}
\email{\qquad\qquad\qquad paolo.pistone@unibo.it}
}
\def\titlerunning{Trends in Linear Logic and Applications}
\def\authorrunning{D.\ Barbarossa and P.\ Pistone}

\newcommand\eg{\textit{e.g.\ }}
\newcommand\etc{\textit{etc}}


\begin{document}
\maketitle

\begin{abstract}
We propose an interpretation of the $\lambda$-calculus in the framework of tropical mathematics, as a unified framework for both program metrics, based on the analysis of program sensitivity via Lipschitz-conditions, and for resource analysis, based on higher-order program differentiation.
We sketch the relation of this semantics to quantitative properties like differential privacy, convergence logprobabilities and Probabilistic Coherent spaces.
Finally, we study the abstract correspondence between this tropical semantics and Lawvere’s generalised metric spaces.
\end{abstract}

\section*{Motivation}



\section{Tropical Laurent Series}



\section{Relations to quantitative properties}



\section{Lawvere's generalised metric spaces}



%The optional arguments of {\tt $\backslash$documentclass$\{$eptcs$\}$} are
%\begin{itemize}
%\item at most one of
%{\tt adraft},
%{\tt submission} or
%{\tt preliminary},
%\item at most one of {\tt publicdomain} or {\tt copyright},
%\item and optionally {\tt creativecommons},
%  \begin{itemize}
%  \item possibly augmented with
%    \begin{itemize}
%    \item {\tt noderivs}
%    \item or {\tt sharealike},
%    \end{itemize}
%  \item and possibly augmented with {\tt noncommercial}.
% \end{itemize}
%\end{itemize}
%We use {\tt adraft} rather than {\tt draft} so as not to confuse hyperref.
%The style-file option {\tt submission} is for papers that are
%submitted to {\tt $\backslash$event}, where the value of the latter is
%to be filled in in line 2 of the tex-file. Use {\tt preliminary} only
%for papers that are accepted but not yet published. The final version
%of your paper that is to be uploaded at the EPTCS website should have
%none of these style-file options.

%By means of the style-file option
%\href{http://creativecommons.org/licenses/}{creativecommons}
%authors equip their paper with a Creative Commons license that allows
%everyone to copy, distribute, display, and perform their copyrighted
%work and derivative works based upon it, but only if they give credit
%the way you request. By invoking the additional style-file option {\tt
%noderivs} you let others copy, distribute, display, and perform only
%verbatim copies of your work, but not derivative works based upon
%it. Alternatively, the {\tt sharealike} option allows others to
%distribute derivative works only under a license identical to the
%license that governs your work. Finally, you can invoke the option
%{\tt noncommercial} that let others copy, distribute, display, and
%perform your work and derivative works based upon it for
%noncommercial purposes only.

%Authors' (multiple) affiliations and emails use the commands
%{\tt $\backslash$institute} and {\tt $\backslash$email}.
%Both are optional.
%Authors should moreover supply
%{\tt $\backslash$titlerunning} and {\tt $\backslash$authorrunning},
%and in case the copyrightholders are not the authors also
%{\tt $\backslash$copyrightholders}.
%As illustrated above, heuristic solutions may be called for to share
%affiliations. Authors may apply their own creativity here \cite{multipleauthors}.



\section*{Bibliography}

%We request that you use
%\href{http://eptcs.web.cse.unsw.edu.au/eptcs.bst}
%{\tt $\backslash$bibliographystyle$\{$eptcs$\}$}
%\cite{bibliographystylewebpage}, or one of its variants
%\href{http://eptcs.web.cse.unsw.edu.au/eptcsalpha.bst}{eptcsalpha},
%\href{http://eptcs.web.cse.unsw.edu.au/eptcsini.bst}{eptcsini} or
%\href{http://eptcs.web.cse.unsw.edu.au/eptcsalphaini.bst}{eptcsalphaini}
%\cite{bibliographystylewebpage}. Compared to the original {\LaTeX}
%{\tt $\backslash$biblio\-graphystyle$\{$plain$\}$},
%it ignores the field {\tt month}, and uses the extra bibtex fields {\tt eid}, {\tt doi}, {\tt ee} and {\tt url}. The first is for electronic identifiers (typically the number $n$ indicating the $n^{\rm th}$ paper in an issue) of papers in electronic journals that do not use page numbers. The other three are to refer, with life links, to electronic incarnations of the paper.

\nocite{*}
\bibliographystyle{eptcs}
%\bibliography{generic}
\end{document}