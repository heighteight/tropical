

%\paragraph{Taylor Expansion }
\subsection{Proof of  Theorem~\ref{thm:modelsTaylor}}

Theorem~\ref{thm:modelsTaylor} states the validity of Taylor expansion in $\LREL_!$.
We must check for it the following equation, given $f\in \LREL_!(C, B^{A})$ and $g\in \LREL_!(C,A)$:
$$
\mathsf{ev}\circ \langle f,g\rangle= \inf_{n\in \BB N}
\left\{(( \cdots (\Lambda^{-}(f) \underbrace{\star g)\cdots )\star g}_{n\text{ times}})\circ \langle \mathrm{id}, \infty\rangle\right\}
$$
where:
\begin{enumerate}
\item $\mathsf{ev}\in \LREL_!(B^{A}+A, B)$ is the canonical \emph{evaluation} morphism;

\item $\Lambda^{-}(\_):= \mathsf{ev}\circ (\_\times \mathrm{id})$ is the \emph{uncurry} operator;

\item given $f\in \LREL_!(C+A,B)$ and $g\in \LREL_!(C,A)$, 
$f\star g\in \LREL_!(C+A,B)$ is the morphism obtained by differentiating $f$ in its second component and applying $g$ in that component, i.e.~
$$
f\star g =  D(f)\circ \langle \langle \infty, g\circ \pi_{1}\rangle, \mathrm{id}_{C+A}\rangle.
$$ 

\end{enumerate}

We do it in the following $4$ steps.


\begin{enumerate}

\item Let us compute the morphism $\mathsf{ev}$ explicitly: $\mathsf{ev}\in \Lawv^{ \C M_{\mathsf{fin}}(  ( \C M_{\mathsf{fin}}(A)\times B)       +  A    ) \times B  }$ is given by
$$\mathsf{ev}_{\mu,y}=
 \begin{cases}
 0 & \text{ if } \mu=[ \langle\rho, y\rangle]  \oplus \rho \\
 \infty & \text{ otherwise}
 \end{cases}
 $$
and observe that, given $f\in \LREL_!(C, B^{A})$ and $g\in \LREL_!(C,A)$, 
$$
\big(\mathsf{ev}\circ \langle f,g\rangle \big)_{\chi, y}= 
\inf\Big \{ 
\sum_{i=1}^{m}g_{\chi_{i},x_{i}}+
f_{\chi', \langle [x_{1},\dots, x_{m}],y\rangle}
\mid 
x_{1},\dots, x_{m}\in A,
\chi= \chi'+\sum_{i=1}^{m}\chi_{i}, 
\Big \}
$$



\item Let us compute the morphism $\Lambda^{-}$ explicitly: given $g\in \LREL_!(C, B^{A})$, 
$\Lambda^{-}(g)\in \LREL_!(C+A, B)$ is given by 
$$
\big(\Lambda^{-}(g)\big)_{\rho\oplus\mu,y}=g_{\rho, \langle \mu,y\rangle}
$$


\item Let us compute the morphism $\star$ explicitly: $f\star g$ is given by 
$$
(f\star g)_{\rho\oplus\mu,y}=
\inf\Big\{
g_{\rho',x}+
f_{\rho''\oplus(\mu+x)}
\mid
x\in A,
\rho= \rho'+\rho''
\Big\}
$$

\item We can now conclude: given the definition of $\mathsf{ev}\circ \langle f,g\rangle$, to check the Taylor equation it is enough to check that, for all $N\in \BB N$, 
$$
\left((( \cdots (\Lambda^{-}(f) \underbrace{\star g)\cdots )\star g}_{N\text{ times}})\circ \langle \mathrm{id}, \infty\rangle\right)_{\chi,y}=
\inf\Big \{ 
\sum_{i=1}^{N}g_{\chi_{i},x_{i}}+
f_{\chi', \langle [x_{1},\dots, x_{N}],y\rangle}
\mid 
\begin{matrix}
x_{1},\dots, x_{N}\in A,\\
\chi= \chi'+\sum_{i=1}^{N}\chi_{i}
\end{matrix}
\Big \}
$$

Let us show, by induction on $N$, the following equality, from which the desired equality easily descends:
$$
\big(( \cdots (\Lambda^{-}(f) \underbrace{\star g)\cdots )\star g}_{N\text{ times}}\big)_{\chi\oplus\mu,y}=
\inf\Big \{ 
\sum_{i=1}^{N}g_{\chi_{i},x_{i}}+
f_{\chi', \langle\mu+ [x_{1},\dots, x_{N}],y\rangle}
\mid 
\begin{matrix}
x_{1},\dots, x_{N}\in A,\\
\chi= \chi'+\sum_{i=1}^{N}\chi_{i}
\end{matrix}
\Big \}
$$
\begin{itemize}

\item if $N=0$, the right-hand term reduces to 
$f_{\chi, \langle \mu, y\rangle}=(\Lambda^{-}(f))_{\chi\oplus\mu,y}$;

\item otherwise, let $F=(( \cdots (\Lambda^{-}(f) \underbrace{\star g)\cdots )\star g}_{N-1\text{ times}})$, so that by I.H.~we have
$$
F_{\chi\oplus\mu,y}=
\inf\Big \{ 
\sum_{i=1}^{N-1}g_{\chi_{i},x_{i}}+
f_{\chi', \langle\mu+ [x_{1},\dots, x_{N-1}],y\rangle}
\mid 
\begin{matrix}
x_{1},\dots, x_{N-1}\in A,\\
\chi= \chi'+\sum_{i=1}^{N-1}\chi_{i}
\end{matrix}
\Big \}
$$
Then we have
{\small
\begin{align*}
\big( F\star g\big)_{\chi\oplus\mu,y}
&=
\inf \left \{
g_{\chi',x}+F_{\chi''\oplus(\mu+x)}
\mid
x\in A, \chi=\chi'+\chi''
\right\}\\
&=
\inf\left \{ 
g_{\chi',x}+
\inf\left\{
\sum_{i=1}^{N-1}g_{\chi_{i},x_{i}}+
f_{\chi'', \langle\mu+ [x_{1},\dots, x_{N-1}]+x,y\rangle}
\mid 
\begin{matrix}
x_{1},\dots, x_{N-1}\in A,\\
\chi^{*}= \chi''+\sum_{i=1}^{N-1}\chi_{i}
\end{matrix}
\right\}
\ 
\Big\vert \ 
\begin{matrix}
x\in A,\\
\chi=\chi'+\chi^{*}
\end{matrix}
\right \}\\
&=
\inf\Big \{ 
g_{\chi',x}+
\sum_{i=1}^{N-1}g_{\chi_{i},x_{i}}+
f_{\chi'', \langle\mu+ [x_{1},\dots, x_{N-1}]+x,y\rangle}
\mid 
\begin{matrix}
x,x_{1},\dots, x_{N-1}\in A,\\
\chi= \chi'+\chi''+\sum_{i=1}^{N-1}\chi_{i}
\end{matrix}
\Big \}\\
&=
\inf\Big \{ 
\sum_{i=1}^{N}g_{\chi_{i},x_{i}}+
f_{\chi', \langle\mu+ [x_{1},\dots, x_{N}],y\rangle}
\mid 
\begin{matrix}
x_{1},\dots, x_{N}\in A,\\
\chi= \chi'+\sum_{i=1}^{N}\chi_{i}
\end{matrix}
\Big \}.
\end{align*}
}
\end{itemize}
\end{enumerate}


\subsection{Proof of Corollary~\ref{cor:taylor}}


\begin{proof}[Proof of Corollary \ref{cor:taylor}]
Thm.~\ref{thmTLSlocLip} yields the estimate $\max_{\overline{B_{3\delta}(\model{N})}}\model{M}^{!}$. As from Thm.~\ref{thm:taylor} 4.~it follows that $\model{t}^{!}\geq \model{M}^{!}$, we deduce that $K=\max_{\overline{B_{3\delta}(\model{N})}}\model{t}^{!}\geq \max_{\overline{B_{3\delta}(\model{N})}}\model{M}^{!}$ is also a local Lipschitz constant for $\model{M}^{!}$. Moreover, since $\model{t}^{!}$ is concave and non-decreasing, the $\max$ of $\model{t}^{!}$ is attained at the maximum point of $\overline{B_{3\delta}(\model{N})}$, that is, 
$K= \model{t}^{!}(x+3\delta)$. Finally, from $\model{t}^{!}(\model{N})<\infty$ and the continuity of $\model{t}^{!}$ we deduce $K<\infty$.
\end{proof}





