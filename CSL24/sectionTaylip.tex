% !TEX root = /Users/paolopistone/Documents/GitHub/tropicalnew/CSL24/main.tex


We now have all elements necessary to relate the metric and differential analysis of higher-order programs in the tropical relational model. 

The key ingredient is the notion of Taylor expansion $\Te{M}$ of a $\lambda$-term $M$, which is a set of terms of the differential $\lambda$-calculus defined as follows:
%
%
%
% which is defined as follows: 
%
%that we now quickly recall. 
%
%
%Let $M\langle N_{1},\dots, N_{k}\rangle$ be an abbreviation for
%$\mathsf D^{k}[M, N_{1},\dots, N_{k}]0$. A \emph{resource $\lambda$-term} is any term belonging to the grammar $t::= x \mid \lambda x.t \mid t \langle t,\dots, t\rangle$.
%%
%
%The Taylor expansion of a $\lambda$-term is a set of resource terms $\Te{M}$ defined inductively as 
%
$\Te{x}=\{x\}$, $\Te{\lambda x.M}=\{\lambda x.t\mid t\in \Te{M}\}$ and 
$\Te{MN}=\{t\cdot \langle u_{1},\dots, u_{k}\rangle  \mid k\in \mathbb N, t\in \Te{M}, u_{i}\in \Te{N} \}$, where $t\cdot \langle u_{1},\dots, u_{k}\rangle$ is an abbreviation for $\mathsf D^{k}[t, u_{1},\dots, u_{k}] 0$. 
Observe that in the terms appearing in $\Te{M}$ all applications are always \emph{bounded}: they may use an exact number of copies of their input. 
Such terms are usually called \emph{resource $\lambda$-terms} \cite{Pagani2009, Manzo2012}.
Considering the term $M=zxx$ from Example \ref{ex:zxx}, all terms
 $t_{n,m}=z \,\langle x^{n}\rangle\,\langle x^{m}\rangle$, for $n,m\in \mathbb N$, are in $\Te{M}$. 
Observe that the interpretation of $t_{n,m}$ yields a tropical polynomial
$\model{t_{n,m}}^{!}(x)(z)= y_{[n,m]}+(n+m)x$, rather than a tps. 
However, resource $\lambda$-terms need not always yield a tropical polynomial. For instance, consider $y:(*\to*)\to (*\to*), x:(*\to *) \vdash t:  (*\to  *)$
with $t=y\cdot \langle y\cdot \langle x\rangle \rangle\in \Te{y(yx)}$. Then 
$\model{t}^{!}: \Lawv^{  \mathcal M_{\mathrm{fin}}(\mathbb N)\times \mathbb N}
\times \Lawv^{\mathbb N} \to \Lawv^{\mathbb N}$ is given by
$
\model{t}^{!}(y,x)_{i}= \inf_{m,n\in \mathbb N}\big\{    y_{[m],i}  +  y_{[n],m}+  x_{n}\}
$, 
which is not a polynomial. Yet, $\model{t}^{!}$ is Lipschitz, more precisely, 1-Lipschitz in $x$ and 2-Lipschitz in $y$. This is a general fact, as shown below.

%
%$\Te{MN}$ of $MN$ is the set $\set{\Der^{n}[t,u_1,\dots,u_n]0 \mid n\in\N,t\in\Te{M},u_i\in\Te{N}}$.
We have already shown that the tropical differential makes $\LREL_{!}$ a model of the differential $\lambda$-calculus. We now show that it also models the Taylor expansion (this needs not be true for \emph{any} CC$\partial$C).
First, it can be patiently checked that in $(\LREL_!,D)$ all morphisms can be Taylor expanded  (see \cite[Definition 4.22]{Manzo2012}):

\begin{theorem}\label{thm:modelsTaylor}
 For all $t\in\HOM{\LREL_!}{Z}{!X\multimap Y}$, $s\in\HOM{\LREL_!}{Z}{X}$ we have:%, the evaluation of $t$ over $s$ yields 
 \begin{align}\label{eq:taylorcat}
  \RM{ev}\circ_!\langle t,s\rangle =
  \inf\limits_{n\in\N}
  \set{((\dots((\Lambda^- t)\star s)\star \dots)\star s)\circ_! \langle \RM{id},\infty \rangle}.
 \end{align} 
\end{theorem}
%It is worth discussing the formula above a bit more. 
The equation above is a nothing more than a tropical version of the Taylor formula from the Introduction:
%:\HOM{\LREL}{!Z}{X\multimap Y}\to \HOM{\LREL}{!(Z+X)}{Y}$
%$\star:\HOM{\LREL}{!(Z+X)}{Y}\times\HOM{\LREL}{!Z}{X}\to \HOM{\LREL}{!(Z+X)}{Y}$ is defined as 
$u\star s= (Du)\circ_{!} \langle \langle  \infty, s\circ_{!} \pi_{1}\rangle,\mathrm{id}\rangle$ corresponds to the application of the derivative of $u$ on $s$, and $\Lambda^-$ is the uncurry operator.
Hence the right-hand term in \eqref{eq:taylorcat} corresponds to the $\inf$ of the $n$-th derivative of $\Lambda^{-}t$ applied to ``$n$ copies'' of $s$.
%,  i.e.~it coincides with the tropical %interpretation of the
%version of the usual Taylor expansion.
%Moreover, since $\LREL_!$ has countable sums (all $\inf$'s converge), and thanks to equation \eqref{eq:taylorcat}, an immediate adaptation of the proof of [Theorem 4.23, \cite{Manzo2012}] entails that the interpretation of the $\STDLC$-Taylor expansion of a $\STLC$-term $M$ given in \eqref{eq:taylor}, converges to the interpretation of $M$.


Second, since $\LREL_!$ has countable sums (all countable $\inf$s converge), an immediate adaptation of the proof of \cite[Theorem 4.23]{Manzo2012} shows:

\begin{corollary}\label{cor:T(M)=M}
 %For $\Gamma\vdash_{\STLC} M:A$, we have %$\model{\Gamma\vdash_{\STDLC} \Te M:A}:=
$\model{\Gamma\vdash_{\STLC} M:A}=\inf_{t\in\Te{M}} \model{\Gamma\vdash_{\STDLC} t:A}$. %the interpretation of the Taylor expansion of a $\STLC$-term $M$, given in \eqref{eq:taylor}, converges to the one of $M$.
\end{corollary}




Using the results of the previous section, as well as the results above, we can deduce the following properties:

\begin{theorem}\label{thm:taylor}
Let $\mathcal S$ be one of $\RM{PCF}^{\Lawv},\STLC,\BSTLC,\STDLC$. Let $\Gamma\vdash_{\mathcal S}M:A$ and $a\in \model{A}$.
\begin{enumerate}
\item For $\mathcal S=\BSTLC$, $\model{\Gamma\vdash_{\mathcal S}M:A}^{!}_{a}$ is a tropical polynomial, and thus Lipschitz;

\item For $\mathcal S=\STDLC$, if $M\in \Te{M'}$, then $\model{\Gamma\vdash_{\mathcal S}M:A}^{!}_{a}$ is Lipschitz;

\item For $\mathcal S=\STLC,\RM{PCF}^{\Lawv}$, then $\model{\Gamma\vdash_{\mathcal S}M:A}^{!}_{a}$ is locally Lipschitz;

\item For $\mathcal S=\STLC$, $\Te{M}$ decomposes $\model{\Gamma \vdash_{\STLC} M:A}^!_{a}$ as an $\inf_{t\in\Te{M}}\model{\Gamma\vdash_{\STDLC} t:A}^!_{a}$ of {Lipschitz} functions.
\end{enumerate}

\end{theorem}
\begin{proof}
1). As we already observed, the interpretation of a bounded term is a tropical polynomial.
2.) From Proposition \ref{prop:troplinear} 2.~observing that a resource term $t(x)$ may use a variable $x$ a fixed number $n$ times, so that its matrix lies in $\Lawv^{!_{n}X\times Y}$. 
%Now we apply Corollary~\ref{prop:polylip} to each coordinate of the image, and by taking the maximum Lipschitz constant among the finite number $\mathrm{Card}(\model A)$ of them, we obtain the thesis.
3). From Theorem~\ref{thmTLSlocLip}.
4). It follows from \autoref{cor:T(M)=M} plus the fact that, for $(f_n)_{n\in\N}\subseteq\Lawv^{!X\times Y}$, we have $\left(\inf_{n\in\N} f_n\right)^!:\Lawv^X\to\Lawv^Y$, with $\left(\inf_{n\in\N} f_n\right)^!=\inf_{n\in\N} f_n^!$.
\end{proof}




%
%From this we deduce, inductively, the following fact:
%\begin{proposition}
%For any resource $\lambda$-term $t$, if $\Gamma\vdash_{\STDLC}t:A$, then $\model{\Gamma\vdash_{\STDLC}t:A}^{!}$ is a Lipschitz function.
%\end{proposition}
%
%
%This result shows that the interpretation of a program can be approximated via a family of tropical polynomials, and leads to.
% \begin{theorem}\label{thm:taylor}
% $\Te{M}$ decomposes $\model{\Gamma \vdash_{\STLC} M:A}^!_{a}$ as an $\inf_{t\in\Te{M}}\model{\Gamma\vdash_{\STDLC} t:A}^!_{a}$ of \emph{tropical polynomials}, that is,of \emph{Lipschitz} functions.
%\end{theorem}


We conclude our discussion with an application of the Taylor expansion in $\LREL_{!}$: as proved in the previous section, all tps are locally Lipschitz; now, Theorem \ref{thm:taylor} can be used to compute approximations of the Lipschitz constants of an actual higher-order program.

\begin{corollary}
Suppose $x: A \vdash_{\STLC}M:B$ and $\vdash_{\STLC} N:A$. 
Then for all $t\in \Te{M}$ such that $\model{t}^{!}(\model{N})\neq \infty$, and $\delta>0$, the tps $\model{ x:A\vdash_{\STLC}M:B}^{!}$ is $\frac{\model{t}^{!}(\model{N}+2\delta)}{\delta}$-Lipschitz over the open ball
 $B_{\delta}(\model{N})$.
\end{corollary}
\begin{proof}
We apply the Lipschitz estimate from Thm.~\ref{thmTLSlocLip} for $\model{M}^{!}$,  and use the fact that, from Thm.~\ref{thm:taylor} 4.~it follows that $\model{t}^{!}\geq \model{M}^{!}$.
To conclude, since $\model{t}^{!}$ is concave and non-decreasing, $\max_{\overline{B_{3\delta}(x)}}\model{t}^{!}= \model{t}^{!}(x+3\delta)$, which yields the stated Lipschitz estimate. 
\end{proof}
%{\color{red}ADD PROOF SKETCH}

\begin{example}
Consider again the term $M=zxx$ from Example \ref{ex:zxx}-
%, satisfying $x:*, z:(*\to*\to*) \vdash M:  *$. Since $\mathcal M_{\mathrm{fin}}(\{*\})\simeq \mathbb N$, we have that 
%
% $\model{M}^{!}: \Lawv \times \Lawv^{\mathcal M_{\mathrm{fin}}(\mathbb N\times \mathbb N)}\to \Lawv$ is given by
 The (generalized) tps $\model{M}^{!}(x)(y)= \inf_{n,n'\in \mathbb N}\{y_{[(n,n')]}+(n+n')x\}$ is not (globally) Lipschitz: for any 
$L>0$, choose a natural number $N>L$, let $Y\in \Lawv^{\mathcal M_{\mathrm{fin}}(\mathbb N\times \mathbb N)}$ be such that $Y_{\mu}<\infty$ only if $\mu=[(n,n')]$ with $n+n'\geq N$; then $|\model{M}^{!}(x)(Y)- \model{M}^{!}(x+\epsilon)(Y)|\geq N\epsilon > L\epsilon$. 
Now take the approximant $t= z \langle x^{N-1}\rangle \langle x\rangle \in \Te{M}$ (chosen so that $\model t^{!}(x)(Y)<\infty$). Its interpretation is the monomial 
$\model{t}^{!}(x)(Y) = Y_{[(N-1,1)]}+Nx$. We can then compute a Lipschitz-constant for $\model{M}^{!}$ around $\langle x,Y\rangle$
as $\frac{1}{\delta}\model{t}^{!}(\langle x,Y\rangle+\delta)= 3N+3 + \frac{Y_{[(N-1,1)]}+Nx}{\delta}$. 
\end{example}
%    
%    %
%  Consider now $F\in \Lawv^{\mathcal M_{\mathrm{fin}}(\mathbb N\times \mathbb N)}$ given by $F_{\mu}=0$ if $\mu=[(1,1)]$ and $F_{\mu}=\infty$ otherwise. 
%  Then $\model{M}^{!}(\langle x,F\rangle)=2x$
%%We show that we can compute a local 
%More formally, the differential operator $\Der[-]$ transforms a function $M:A\to B$ into a function $\Der[M]: A\to (A\to B)$ which is linear in its first argument. 
%Since $M$ may rather ask for several copies of $N$, this requires a form of non-determinism: 
%For example, if $M$ is the term $\lambda fx.f(fx)$ considered before, $\Der[M]$ takes a first input $N$ and passes it linearly to $M$. Notice that there are two ways of doing so, corresponding to the two bound occurrences of $f$ in $M$: either by applying $N$ to $fx$, or by 
%applying $f$ linearly to $Nx$ (indeed, if $f$ were applied in an unrestricted way, it might duplicate $Nx$, so that $N$ would not be used linearly). This justifies the equation below, in which $\Der[M]$ is identified with the non-deterministic sum of the two possible linear choices:
%\begin{align}
%\Der\left[\lambda f x.f(fx)\right]\cdot N = 
%\lambda fx. N(fx) + \left(\Der[f]\cdot (Nx)\right)(fx)
%\end{align}
%More generally, one can define a notion of $k$-bounded application $\Der^{(k)}[M]\cdot N^{k}$, where $\Der^{(0)}[M]\cdot N^{0}= M$ and $\Der^{(k+1)}[M]\cdot N^{k+1}= \Der[ \Der^{(k)}[M]\cdot N^{k}]\cdot N$, corresponding to passing $N$ to $M$ exactly $k$ times.
%
%
%The name ``differential'' for the operator $\Der[-]$ is justified by the fact that it satisfies many properties of the usual differential operator of analysis $\Der[f]:= \lambda xy. \frac{\mathsf df(y)}{\mathsf dy}\cdot x$. Notably, it is additive in its first variable (i.e.~it commutes with the non-deterministic sum operator), and satisfies the chain rule.
%Most famously, the differential operator can be used to define a Taylor formula for $\lambda$-terms, which decomposes an unrestricted application into a formal non-deterministic sum of bounded applications:

%
%More generally, the relational semantics interprets unbounded programs as \emph{analytic functions}, that is, as functions admitting a representation as power series. For instance, observing that an analytic map $f: \BB R\to \BB R$, where $f(x)=\sum_{n}\widehat f_{n}\cdot x^{n}$ is uniquely determined by the sequence $\widehat f_{n}$, the program $M_{\infty}:=\lambda fx.fx: (\BB R\To \BB R)\To (\BB R\To \BB R)$ is represented by the power series below:
%\begin{align}
%F_{\infty}(f,x)= \sum_{n=0}^{\infty} \widehat f_{n} x^{n}
%\end{align}
%By restricting ourselves to bounded applications, the terms in the power series become finite, that is, the interpretation becomes a \emph{polynomial}: for instance, the program $M_{2}:=\lambda fx. \sum_{i=0}^{2}\Der^{(i)}[f]\cdot x^{i}$, corresponding to passing $x$ \emph{at most twice} to $f$, is represented by the polynomial
%\begin{align}
%F_{\leq 2}(f,x)=\widehat f_{2} x^{2}  + \widehat f_{1}x +  \widehat f_{0} 
%\end{align}
% In this framework the differential operator is naturally represented by formal differentiation of polynomials, where, as one would expect, 
% $\Der[\sum_{n}a_{n}x^{n}]=\sum_{n}\Der[a_{n}x^{n}]$ and $\Der[a_{0}x^{0}]=0$ and $\Der[a_{n+1}x^{n+1}]= (n+1)a_{n+1}x^{n}$, so that power series can be Taylor expanded. 
%
%Let us now see what the results proved in the previous section translate into, when referred to the interpretation of higher-order programs.
%Remark that the metric spaces $(\Lawv^{\mathlarger{+}_{i=1}^n X_i},d_{\infty})$ and $(\prod_{i=1}^n \Lawv^{X_i},\max_{i=1}^n d^{X_i}_\infty)$ are trivially isometric, so we identify them.

% Since $\model{(x_i:A_i)_{i=1}^n\vdash_{\STLC} M:B}\in\HOM{\LREL_!}{\mathlarger{\&}_{i=1}^n \model{A_i}}{\model B}\simeq\Lawv^{\prod_{i=1}^n ! \model {A_i}\times \model B}$, the tps $\model{(x_i:A_i)_{i=1}^n\vdash_{\STLC} M:B}^!:\Lawv^{\mathlarger{+}_{i=1}^n \model{A_i}}\to\Lawv^{\model B} \simeq \prod_{i=1}^n\Lawv^{\model{A_i}}\to\Lawv^{\model B}$ is defined by $t^!(x^1,\dots,x^n)_b:=\inf_{\mu_i\in ! \model{A_i}}\left\{\sum_{i=1}^n\mu_i x^i+t_{(\mu_1,\dots,\mu_n),b}\right\}$.
%Finally, the interpretations of $\BSTLC$-terms are matrices $t\in\HOM{\LREL}{\bigotimes_{i=1}^n !_{n_i}X_i}{Y}=\Lawv^{\prod_{i=1}^n !_{n_i}X_i\times Y}$.In such situation, we define $t^!:\Lawv^{\mathlarger{+}_{i=1}^n X_{i}}\to\Lawv^Y \simeq \prod_{i=1}^n\Lawv^{X_{i}}\to\Lawv^Y$ as $t^!(x^1,\dots,x^n)_b:=\inf_{\mu_i\in !_{n_i} X_i}\left\{\sum_{i=1}^n\mu_i x^i+t_{(\mu_1,\dots,\mu_n),b}\right\}$. Clearly $t^!={\widetilde t}^!$, where $\widetilde t\in\Lawv^{\prod_{i=1}^n ! {X_i}\times Y}$ is the matrix $\widetilde t_{(\mu_1,\dots,\mu_n),b}:=t_{(\mu_1,\dots,\mu_n),b}$ if $\mu_i\in !_{n_i} X_i$ for all $i$, and $:=\infty$ otherwise, which has the same support as $t$.

%\begin{corollary}
%Let $\model A$ be a finite set.
%\begin{enumerate}
%\item $\model{\Gamma \vdash_{\BSTLC} M:B}^!:\prod\limits_{(x_i:_{n_i} A_i)\in\Gamma} \!\!\!\!\!\Lawv^{\model{A_i}} \to \Lawv^{\model B}$ is a \emph{tropical polynomial}, thus (as $\model A$ is finite), a \emph{Lipschitz} function.
%\item $\model{\Gamma \vdash_{\STLC} M:B}^!, \model{\Gamma \vdash_{\STLC_\oplus} M:A}^!:\prod\limits_{(x_i: A_i)\in\Gamma} \!\!\!\!\!\Lawv^{\model{A_i}} \to \Lawv^{\model B}$ are \emph{locally} Lipschitz maps.
%\item $\Te{M}$ decomposes $\model{\Gamma \vdash_{\STLC} M:A}^!$ as an $\inf_{t\in\Te{M}}\model{\Gamma\vdash_{\STDLC} t:A}^!$ of \emph{tropical polynomials}, thus (as $\model A$ is finite), \emph{Lipschitz} functions.
%\end{enumerate}
%\end{corollary} 
%\begin{proof}
%1). Since $\model A$ is finite, also $\model *$ is.
%Thus, as we already observed, the interpretation of a bounded term is a tropical polynomial.
%Now we apply Corollary~\ref{prop:polylip} to each coordinate of the image, and by taking the maximum Lipschitz constant among the finite number $\mathrm{Card}(\model A)$ of them, we obtain the thesis.
%2). It follows immediately from Theorem~\ref{thmTLSlocLip} and the fact that $\model A$ is finite.
%3). It follows from \autoref{cor:T(M)=M} plus the easily checked fact that, for $(f_n)_{n\in\N}\subseteq\Lawv^{!X\times Y}$, we have $\left(\inf_{n\in\N} f_n\right)^!:\Lawv^X\to\Lawv^Y$, with $\left(\inf_{n\in\N} f_n\right)^!=\inf_{n\in\N} f_n^!$.
%\end{proof}
%
%Remark that the restriction $\model A$ finite is without loss of generality, since by Currying all programs can be seen having type $*$, which is natural to interpret as a singleton.
%
%Remark that, if $Y$ is finite and $f:\Lawv^{\mathlarger{+}_{i=1}^n X_i}\to\Lawv^Y$ is $K$-Lipschitz, then $f:\prod_{i=1}^n \Lawv^{X_i}\to \Lawv^Y$ is $K$-Lipschitz on each of the $n$ variable separately.

%A consequence of (3) is that the pointwise distance between two interpretations of programs can always be bounded via Lipschitz tropical polynomial approximants of the initial two programs.
%\begin{corollary}
% Let $\Gamma \vdash_{\STLC} M:A$ and $\Delta\vdash_{\STLC} N:B$.
% For all $\epsilon>0$, $x\in\Lawv^{\model \Gamma}$, $b\in\model A$, there exist $t\in\Te{M}$, $u\in\Te{N}$ s.t.\ $\big| \model{\Gamma \vdash_{\STLC} M:A}^!(x)_b - \model{\Delta \vdash_{\STLC} N:B}^!(x)_b \big| \leq 2\epsilon + \big| \model{\Gamma \vdash_{\STDLC} t:A}^!(x)_b - \model{\Delta \vdash_{\STDLC} u:B}^!(x)_b \big|$.
%\end{corollary}
%
%






