\begin{figure}[h]
\begin{tikzpicture}[thick]
    \coordinate (u) at (-2,0);
    \coordinate (v) at (2,0);
    \coordinate (y) at (-1.2,0);
    \coordinate (z) at (0,0);

    \draw[name path={u--v}] (u) -- (v);
    \node [draw,name path=z] at (z) [circle through={(u)}] {};

    \path[name intersections={of=u--v and z,by={int_u,int_v}}]
      foreach \X in {u,v}{(int_\X) node[below left]{$\X$}};

    \fill   (int_u) circle[radius=2pt]  node [below left] {$u$};
    \fill   (int_v) circle[radius=2pt]  node [below left] {$v$};

    \fill   (z) circle[radius=0pt]  node [xshift=-1cm, yshift=2.2cm] {$\overline{B_{2\delta}(z)}$};

    \fill   (z) circle[radius=2pt]  node [above] {$z$};
    \fill   (y) circle[radius=2pt]  node [above] {$y$};

    \coordinate (x) at (-0.7,0.4);

    \draw   (x) circle[radius=1cm] node [above] {$x$};
    \fill   (x) circle[radius=2pt]  node [xshift=10mm, yshift=10mm] {$\overline{B_{\delta}(x)}$};

    \draw   (x) circle[radius=3cm] node {};
    \fill   (x) circle[radius=0pt]  node [xshift=-1.7cm, yshift=3cm] {$\overline{B_{3\delta}(x)}$};
\end{tikzpicture}
\caption{Drawing of the proof of \autoref{th:locLip}.}
\label{fig:proof_loc_lip}
\end{figure}

\begin{theorem}\label{th:locLip}
Let $f:V\subseteq (\BB R^X,\norm\cdot) \to (\BB R,\absv \cdot)$, with $V$ open and convex and $\norm\cdot$ any norm.
If $f$ is concave and locally bounded, then $f$ is locally Lipschitz.
Moreover, the Lipschitz constant of $f$ on $\overline{B_{\delta}(x)}$ can be chosen to be $\frac{1}{\delta}\max_{\overline{B_{3\delta}(x)}} \absv f$.
\end{theorem}
\begin{proof}
 Call $\overline{B_{\delta}(x)}:=B_1$, $\overline{B_{3\delta}(x)}:=B_3$.
 It suffices to show that for all $x\in V$, there is $\delta>0$ s.t.\ $B_3\subseteq \mathrm{interior}(V)$, $K:=\max_{B_3}  \absv f$ exists and $f$ is $(\frac{1}{\delta}\max_{B_3} \absv f)$-Lipschitz on $B_1$.
 A $\delta$ satisfying the first two conditions exists since $V$ is open and bcause $f$ is locally bounded and $B_3$ is compact.
 We will show that for all such $\delta$, the third condition already holds.

 For that, fix $y,z\in B_1$ and call $r:=\frac{d(y,z)}{2\delta}\in[0,1]$.
 We want to show that $\absv{f(y)-f(z)}\leq \frac{K}{\delta}d(y,z)=2Kr$.
 Wlog $y\neq z$, otherwise there is nothing to show.

 So $r\neq 0$ and we can consider $u:=\frac{1+r}{r}z-\frac{1}{r}y$, $v:=\frac{1}{r}y-\frac{r-1}{r}z$.
 We have $u,v\in \overline{B_{2\delta}(z)}=:B_2$.
 Indeed, $d(u,z)=\norm{u-z}=\norm{\frac{z}{r}+z-\frac{y}{r}-z}=\frac{\norm{z-y}}{r}=2\delta$ and similarly $d(v,z)=2\delta$.
 Geometrically, those are actually the intersections between $B_2$ and the line generated by $y$ and $z$, see \autoref{fig:proof_loc_lip}.
 Now we have the convex combinations $z=\frac{1}{1+r}y+\frac{r}{1+r}u$ and $y=(1-r)z+rv$, so the concavity of $f$ entails on one hand:
 $f(z)\geq \frac{1}{1+r}f(y)+\frac{r}{1+r}f(u)\geq \frac{f(y)}{1+r} - \frac{rK}{1+r}$, i.e.\ $f(y)-f(z)\leq r(K+f(z))\leq 2rK$, and on the other hand:
 $f(y)\geq (1-r)f(z)+rf(v)\geq f(z)-r(f(z)+K)$, i.e.\ $f(z)-f(y)\leq r(f(z)+K)\leq 2rK$.
 In the previous inequalities we have used that $f(u),f(v)\geq -K$.
 This follows because $u,v\in B_2\subseteq B_3$, as it can be immediately checked, thus $\absv{f(u)},\absv{f(v)}\leq K$.
 Putting the final inequalities together, we have $\absv{f(y)-f(z)}\leq 2rK$, i.e.\ the thesis.
\end{proof}

Therefore, since $(0,\infty)^X$ is open and convex in $(\BB R^X,\norm\cdot)$ and all tps are non-negative, we immediately have:

\begin{theorem}[\autoref{thmTLSlocLip}]
All tps $f:\Lawv^X\to\Lawv$ are locally Lipschitz on $(0,\infty)^X$.
Moreover, the Lipschitz constant of $f$ on $\overline{B_{\delta}(x)}$ can be chosen to be $\frac{1}{\delta}\max_{\overline{B_{3\delta}(x)}} f$.
\end{theorem}
