For $Q$ a continuous semiring, one lets the category $\QREL$ (\cite{Manzo2013} calls it $Q^\Pi$) have sets as objects and set-indexed matrices with coefficients in $Q$ as morphisms, i.e.~$\QREL(X,Y):=Q^{X\times Y}$.
%The identity morphism of $\QREL$ is the identity matrix $I\in Q^{X\times X}$,
% given by $i_{a,a}=1$ and $i_{a,b\neq a}=0$, and
The composition $st\in Q^{X\times Z}$ of $t\in Q^{X\times Y}$ and $s\in Q^{Y\times Z}$ is given by $(st)_{a,c}:=\sum_{b\in Y} s_{b,c}t_{a,b}$ (such series converges because $Q$ is continuous).
As expected, $Q^X$ is a $Q$-semimodule and we identify $\HOM{\QREL}{X}{Y}$ with the set of linear maps from $Q^X$ to $Q^Y$, which have shape $f(x)_b=:\sum_{a\in X} \matr f_{a,b}x_a$, for some matrix $\matr f\in Q^{X\times Y}$.
% \begin{remark}
 %Following \cite{Manzo2013, Hofmann2014, Ehrhard2005}, we chose to see a matrix $t$ from $X$ to $Y$ as a map $t:X\times Y\to Q$.
% 
% fix $\HOM{\QREL}{X}{Y}:=Q^{X\times Y}$ with composition $st:X\times Z\to Q$ of $s:Y\times Z\to Q$ and $t:X\times Y\to Q$ defined by $(st)_{a,c}:=\sum_{b\in Y} s_{b,c}t_{a,b}$.
Notice that usual linear algebra conventions correspond to work in $\QREL^{\op}$, %a matrix $X\times Y\to Q$ is usually called a ``$Y\times X$-matrix'', meaning $Y$ rows and $X$ columns
e.g.\ the usual matrix-vector product defines a map $Q^Y\to Q^X$.
Following \cite{Manzo2013, Hofmann2014, Ehrhard2005}, we are instead working with transpose matrices.
%\end{remark}
%
%In $\QREL^{op}$ (which corresponds to systematically taking transpose matrices), composition coincides with the product matrix/matrix and $\hat{(\cdot)}$ with the product matrix/vector.
%In order to avoid confusion, we will refer to a $t\in Q^{X\times Y}$ just as a \emph{matrix from $X$ to $Y$}.
%
%
%
%We must fix a convention for matrices: following \cite{Manzo2013, Hofmann2014, Ehrhard2005}, we fix $\HOM{\QREL}{X}{Y}:=Q^{X\times Y}$ with composition $st:X\times Z\to Q$ of $s:Y\times Z\to Q$ and $t:X\times Y\to Q$ defined by $(st)_{a,c}:=\sum_{b\in Y} s_{b,c}t_{a,b}$.
%In linear algebra, a map $X\times Y\to Q$ is usually called a ``$Y\times X$-matrix'', meaning $Y$ rows and $X$ columns.
%In particular, the product of such a matrix for a vector defines a map $Q^Y\to Q^X$.
%Instead, we prefer to see a $t\in\HOM{\QREL}{X}{Y}$ as giving rise to a map $\hat t:Q^X\to Q^Y$ defined by $\hat t(x)_a:=\sum_{b\in Y} t_{a,b}x_a$.
%In $\QREL^{op}$ (which corresponds to systematically taking transpose matrices), composition coincides with the product matrix/matrix and $\hat{(\cdot)}$ with the product matrix/vector.
%In order to avoid confusion, we will refer to a $t\in Q^{X\times Y}$ just as a \emph{matrix from $X$ to $Y$}.
%As it is expected, $Q^X$ is a $Q$-semimodule and the bijection $\hat{(\cdot)}$ identifies $\HOM{\QREL}{X}{Y}$ with the set of linear maps from $Q^X$ to $Q^Y$.
It is well known that $\QREL$ admits a comonad $!$ which acts, on objects, by taking the finite multisets.
Remember that the coKleisli category $\C C_!$ of a category $\C C$ w.r.t.\ a comonad $!$ is the category whose elements are the same of $\C C$, and $\HOM{\C C_!}{X}{Y}:=\HOM{\C C}{!X}{Y}$, with composition $\circ_!$ defined by making use of the co-multiplication of $!$.
%We will explicit this constructions in our tropical setting in the next lines.

The main actor of this paper is the category $\LREL$. % of sets and matrices with values over $\Lawv$ (which is a continuous semi-ring).
%, where $\Lawv$ is the already introduced Lawvere quantale, seen as the idempotent complete semiring $(\BB R_{\geq0}\cup\set{\infty},\inf,\infty,+,0)$.
%The category $\LREL$ is well-defined because $\Lawv$ is a continuous semiring (w.r.t.\ its quantale order $\preceq$.
%This amounts to check that $\min$ and $+$ commute with the $\inf$ (as operations on $\BB R_{\geq0}\cup\set{\infty}$, which is immediate), and that $(\Lawv,\preceq)$ is a cpo with $\infty$ as bottom element (which is immediate since in $\Lawv$ we have $\vee = \inf$) .
It is worth observing that the composition in $\LREL$ %is the tropicalisation of the one defining it in $\QREL$, i.e.
reads as \ $(st)_{a,c}:=\inf_{b\in Y}\set{s_{b,c}+t_{a,b}}$;
similarly, the linear functions $f:\Lawv^X\to \Lawv^Y$, %induced by matrices
which we call \emph{tropical linear}, are exactly those of shape $f(x)_b=\inf_{a\in X} \set{\matr f_{a,b}+x_a}$, for some $\matr f\in\Lawv^{X\times Y}$.
%\end{remark}

Now, although a matrix $t\in\HOM{\LREL_!}{X}{Y}$ yields a tropical linear map $\Lawv^{!X}\to\Lawv^Y$, by exploiting the coKleisli structure we can also ``express it in the base $X$'', i.e.\ see it as a \emph{non-linear} map, that we call $t^!$.
Tropical Laurent series precisely arise from such general construction:

\begin{definition}
Let $t\in\HOM{\LREL_!}{X}{Y}$.
We define $t^!:\Lawv^X\to\Lawv^Y$ by setting $t^!(x):=t\circ_! x$. %(we are identifying $\Lawv^X$ with the set $\HOM{\LREL_!}{\emptyset}{X}$ of the \emph{points} of $X$).
Concretely, $t^!(x)_b=\inf_{\mu\in !X} \set{\mu x+ t_{\mu,b}}$, where $\mu x:=\sum_{a\in X} \mu(a)x_a$.
\end{definition}
%\begin{remark}%[Tropical Laurent series]
The functions $t^!$ correspond then to the generalisation of tLs with \emph{possibly infinitely} many variables (in fact, as many as the elements of $X$), %In the following we will also refer to them as tLs. 
% We will call them simply \emph{tropical Laurent series (tLs)}.
% %Since in the general case of $\QREL$, $t^!$ is a Laurent series with operations in $Q$, let us call \emph{tropical Laurent series} the functions of shape $t^!$ for some $t\in\HOM{\LREL_!}{X}{Y}$.
%\end{remark}
%
%We find the usual notion of tLs of one variable as follows:
%\begin{remark}
and by identifying $!\set{*}\simeq \N$ and $\Lawv^{\set{*}}\simeq\Lawv$, the tLs generated by the morphisms in $\HOM{\LREL_!}{\set{*}}{\set{*}}$ are exactly the %functions $f:\Lawv\to\Lawv$ of shape $f(x)=\inf_{n\in\N}\set{nx+\matr f(n)}$, for some $\matr f:\N\to\Lawv$, i.e.\ we recover the 
usual tLs's of one variable.
% \end{remark}
In a similar way, tLs
$f:\Lawv^X\to\Lawv^Y$ 
 with \emph{finite} supports $\C F_b=\set{\mu\in!X\mid\matr f_{\mu,b}\neq\infty}$, and which have thus shape $f(x)_b=\min_{\mu\in\C F} \set{\mu x+ t_{\mu,b}}$, are generalisation of usual tropical polynomials to the case of possibly infinitely many variables.

\begin{example}
The $\varphi$ of \autoref{fig:plot1} is indeed of shape $\varphi=t^!$, for $t\in\Lawv^{!\set{*}\times\set{*}}$, $t_{\mu,*}:=2^{-\# \mu}$.
% Therefore $\LREL_!$ is not a full-complete model of $\STLC$.
\end{example}

%\end{align}
%where $+$ is the multiset union.

%Remember that in $\Lawv$ the neutral element for addition is $\infty$ and the neutral for multiplication is $0$, so for instance the evaluation map is the matrix $\RM{eval}^{X,Y}\in\Lawv^{!((X\multimap Y) \times X)\times Y}\simeq\Lawv^{(!!X\times !Y\times !X)\times Y}$ given by $\RM{eval}^{X,Y}_{\rho_1\oplus\rho_2\oplus\mu,b}:=0$ if $\rho_1=[\mu]$ and $\rho_2=[b]$, and $\RM{eval}^{X,Y}_{\rho_1\oplus\rho_2\oplus\mu,b}:=\infty$ otherwise.

\begin{remark}
 Let $X$, $Y$ be sets and let $\langle \_,\_\rangle:X\times Y \to \mathbb{R}$.
 For $f:X\to \mathbb R$, define $f^*:Y\to \mathbb R$ by $f^*(y):= \sup_{x\in X}\{\langle x,y\rangle - f(x)\}$.
 Then for $X=!A$, $Y=\Lawv^A$, where $A$ is a set, and $\langle \mu, y \rangle:= \mu y$, we have that $f^!(y)=(-f)^*(-y)$ for all $f\in\Lawv^{!A}$.
 This is precisely the same formal construction yielding the well-known convex conjugate $f^*$ of a function $f$, by taking $X$ any vector space, $Y$ its dual space, and $\langle \_,\_\rangle$ the application bilinear form (acting as the scalar product on coordinates).
 This construction is in turn a generalisation of the Legendre transformation.
 Despite the formal constructions being the same, we ignore for the moment if these could be connected to the study of high-order programs in our setting.
\end{remark}


