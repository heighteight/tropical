%\begin{remark}\label{rmk:ModelsOfBSTLC}
%Similarly to It is known {\color{red}(([reference??] e dire meglio)} 
For $\BSTLC$, one can see that the exponential $!$ can be ``decomposed'' into a family of \emph{graded} exponentials functors $!_n:\LREL\to\LREL$ ($n\in\BB N$), defined by taking multisets of cardinality \emph{at most} $n$. %  lift to functors 
The sequence $(!_n)_{n\in\N}$ gives rise to a $\N$-graded linear exponential comonad on (the SMC) $\LREL$, %satisfying the adjunction: $\LREL(Z\otimes !_{n}X,Y) \simeq \LREL(Z, !_{n}X\multimap Y)$.
i.e.\ $(\LREL,(!_n)_{n\in\N})$ is a model of $\BSTLC$. 
In particular, bounded arrow types are interpreted via $\model{!_{n}A\multimap B}:= \C M_{\leq n}(\model A) \times \model B$ and so, 
if e.g.\ $\model o=\{\star\}$, then $\model{A}$ is a finite set for any type $A$ of $\BSTLC$.
%\end{remark}
%Bounded types are interesting because of the following proposition:
%\begin{proposition}
%For all bounded types $A,B$, the morphisms from $\model A$ to $\model B$ (in all parametric relational models) correspond to polynomials.
%\end{proposition}
%\begin{proof}
%It suffices to check that $\model A$ is finite for all bounded types $A$. Indeed this implies that a morphism $t:\model A\to \model B$ is a finite matrix $t: \model A \times \model B \to \Lawv$.Hence, its corresponding map $\widehat t:\Lawv^{\model A} \to \Lawv^{\model B}$ is a polynomial.
%\end{proof}

%For example (here $!_{n}(\Lawv^{X}):= \Lawv^{\C M_{\leq n}(X)}$):
%\begin{itemize}
%\item a map $f\in \LREL( !_{1}\Lawv, \Lawv)$ is of the form $f(x)=\min \{x+a,b\}$;
%\item a map $f\in \LREL(!_{2}\Lawv, \Lawv)$ is a ``quadratic'' polynomial $f(x)=\min\{2x+a, x+b, c\}$.
%\end{itemize}

%Until now we simply specialised well-known results in our tropical case, with the intent of showing how things read in this particular case.
%Now we go further, by showing that $\LREL_!$ actually admits a \emph{differential structure}, turning it into a model of the $\STDLC$, i.e.~a $CC\partial C$.
%This viewpoint
%, is where the \emph{metric} and the \emph{differential} viewpoints converge, as explained in the Introduction and Section II, and it % will be further generalised in Section \ref{section6}.

%A model of the $\STDLC$ is usually understood as so-called \emph{Cartesian closed differential categories} (CC$\partial$C), see \cite{Manzo2012} for details.
%In order to treat the $+$ and the constructor $D[\_]\cdot (\_)$ of $\STDLC$, the main features of a CC$\partial$C $\C C$ are that:
%
%1) $\C C$ is a left-additive-CCC, i.e.\ its Homsets are commutative monoids and its Cartesian closed structure is well behaved w.r.t.\ this monoid structure;
%
%2) $\C C$ is equipped with a differential operator map $D:\HOM{\C C}{X}{Y}\to \HOM{\C C}{X\times X}{Y}$ (here $\times$ is the Cartesian product of $\C C$) satisfying $8$ axioms, called D1, ..., D7, D-curry.

%Let us show the differential structure of $\LREL_!$ (remember that the Cartesian product of $\LREL_!$ is the disjoint union $+$).

%\begin{definition} The \emph{tropical differential operator} is the map $D:\HOM{\LREL}{!X}{Y}\to \HOM{\LREL}{!(X+X)}{Y}$ defined as $(Dt)_{\mu\oplus\rho,b}=t_{\rho+\mu,b}$ if $\#\mu=1$ and as $\infty$ otherwise (where a multiset $\nu \in !(X+X)$ is identified with a disjoint sum of $\mu,\rho\in !X$).\end{definition}

For what concerns $\STDLC$, by patiently checking all the needed axioms {\color{red}or by using  LEMAY??}, one sees that
%\begin{proposition}[{\color{red}LEMAY??}]\label{thm:LREL!CCDC}
 $\LREL_!$ becomes a CC$\partial\lambda$C, i.e.\ a model of the $\STDLC$, when equipped with the \emph{tropical differential operator} $D:\HOM{\LREL}{!X}{Y}\to \HOM{\LREL}{!(X+X)}{Y}$ defined as $(Dt)_{\mu\oplus\rho,b}=t_{\rho+\mu,b}$ if $\#\mu=1$ and as $\infty$ otherwise (using the iso $(\mu,\rho)\in !Z\times !Z'\mapsto\mu\oplus\rho \in !(Z+Z')$).
%\end{proposition}
%\begin{remark}For $t\in\HOM{\LREL}{!X}{Y}$, we have: $D^2 t\in\HOM{\LREL_!}{(X+X)+(X+X)}{Y}$, where $(D^2 t)_{(\rho\oplus\rho')\oplus(\nu\oplus\nu'),b}$ equals $t_{\nu+\nu'+\rho',b}$ if $\rho=\emptyset$ and $\#\rho'=1=\#\nu$; it equals $t_{\rho+\nu',b}$ if $\rho'=\emptyset=\nu$ and $\#\rho=1$; it equals $\infty$ otherwise.\end{remark}
%This ensures that one can define a sound interpretation of $\STDLC$-terms in the standard way (see [Section 4.3, \cite{Manzo2010}]).
This model is also well-behaved w.r.t.\ the Taylor expansion, as expressed by the following two results.
As it can be patientely checked, in $(\LREL_!,D)$ all morphisms can be Taylor expanded  (see \cite[Definition 4.22]{Manzo2012}):

\begin{theorem}\label{thm:modelsTaylor}
 For all $t\in\HOM{\LREL_!}{Z}{X\multimap Y}$, $s\in\HOM{\LREL_!}{Z}{X}$ we have:%, the evaluation of $t$ over $s$ yields 
 \begin{align}\label{eq:taylorcat}
  \RM{ev}\circ_!\langle t,s\rangle =
  \inf\limits_{n\in\N}
  \set{((\dots((\Lambda^- t)\star s)\star \dots)\star s)\circ_! \langle \RM{id},\infty \rangle}.
 \end{align} 
\end{theorem}
%It is worth discussing the formula above a bit more. 
Here,
%:\HOM{\LREL}{!Z}{X\multimap Y}\to \HOM{\LREL}{!(Z+X)}{Y}$
%$\star:\HOM{\LREL}{!(Z+X)}{Y}\times\HOM{\LREL}{!Z}{X}\to \HOM{\LREL}{!(Z+X)}{Y}$ is defined as 
$u\star s= (Du)\circ_{!} \langle \langle  \infty, s\circ_{!} \pi_{1}\rangle,\mathrm{id}\rangle$ corresponds to the application of the derivative of $u$ on $s$, and $\Lambda^-$ is the uncurry operator.
Hence the right-hand term in \eqref{eq:taylorcat} corresponds to the $\inf$ of the $n$-th derivative of $\Lambda^{-}t$ applied to ``$n$ copies'' of $s$,  i.e.~it coincides with the tropical %interpretation of the
version of the usual Taylor expansion.
%Moreover, since $\LREL_!$ has countable sums (all $\inf$'s converge), and thanks to equation \eqref{eq:taylorcat}, an immediate adaptation of the proof of [Theorem 4.23, \cite{Manzo2012}] entails that the interpretation of the $\STDLC$-Taylor expansion of a $\STLC$-term $M$ given in \eqref{eq:taylor}, converges to the interpretation of $M$.
Finally, since $\LREL_!$ has countable sums (all countable $\inf$s converge), an immediate adaptation of the proof of [Theorem 4.23, \cite{Manzo2012}] shows:

\begin{corollary}\label{cor:T(M)=M}
 If $\Gamma\vdash_{\STLC} M:A$, then $\model{\Gamma\vdash_{\STDLC} \Te M:A}:=\inf_{t\in\Te{M}} \model{\Gamma\vdash_{\STDLC} t:A}$ converges to $\model{\Gamma\vdash_{\STLC} M:A}$. %the interpretation of the Taylor expansion of a $\STLC$-term $M$, given in \eqref{eq:taylor}, converges to the one of $M$.
\end{corollary}

By straightforward induction on $M$ (using that composition and $D$ easily preserve ``booleaness'' and projections and evaluation of  $\LREL$ are boolean) one can prove that:

\begin{proposition}\label{prop:descrete}
 For all the previous four languages, $\model{\Gamma\vdash M:A}\in\Lawv^{!\model{\Gamma}\times\model{A}}$ is a \emph{boolean} matrix, i.e.\ actually $\model{\Gamma\vdash M:A}\in\{0,\infty\}^{!\model{\Gamma}\times\model{A}}$.
\end{proposition}

The fact that the coefficients are either $0$ or $\infty$ can be seen as the fact that $\STLC$ is relatively trivial, from this point of view.
As we will discuss in \autoref{sec:app}, non-trivial coefficients appear when interpreting more interesting languages, like probabilistic ones.
However, in this paper we shall still focus on $\STLC$, in order to set the basis of the further studies and in order to see some already interesting properties of the interpretation.


