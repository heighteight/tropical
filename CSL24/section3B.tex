For what concerns the probabilistic calculus, let us consider a probabilistic extension of $\STLC$, call it $\STLC_\oplus$, which also extends the first-order language we used in~\autoref{sec:proba}:
we add a ground type $\bool$, terms $\true,\false$ of type $\bool$, terms of shape $M\oplus_p N$ and $pM$, for $p\in[0,1]$, typed via the usual rules and whose operational semantics adds the usual rules:
{\small{\[\begin{array}{cccc}
           \dfrac{\Gamma\vdash M:A \qquad \Gamma\vdash N:A}{\Gamma\vdash M\oplus_p N:A}
           &
           \dfrac{\Gamma\vdash M:A}{\Gamma\vdash pM:A}
           &
           M\oplus_p N \to pM
           &
           M\oplus_p N \to (1-p)N.
          \end{array}\]}}
We use this calculus as a toy example for our purposes.
It can be compiled into $\mathrm{PCF}^{\Lawv}$, an instance of the $\mathrm{PCF}^{\mathcal R}$-languages studied in \cite{Manzo2013}, via the translation $(\_)^\circ$ of terms generated by $(M\oplus_p N)^\circ:=pM^\circ\prog{ or } (1-p)N^\circ$.
It is easy to see that if $\Gamma \vdash_{\STLC_\oplus} M:A$, then $\Gamma \vdash_{\mathrm{PCF}^{\Lawv}} M^\circ:A$.
This is slightly imprecise because in \cite{Manzo2013}, their $\mathrm{PCF}^{\mathcal R}$-languages only have one ground type $\mathrm{int}$, used for modeling $\N$, and has no term $\prog{True}$, $\prog{False}$.
So we are actually compiling $\STLC_\oplus$ into the immediate extension of $\mathrm{PCF}^{\mathcal R}$ which adds two new ground types, $*$ and $\bool$ (and, since we will not consider the integers, we can also drop $\mathrm{int}$), plus the two boolean terms.
This does not invalidate the results (not concerning $\mathrm{int}$) of \cite{Manzo2013}.
Remark also that the operational semantics of $\mathrm{PCF}^{\Lawv}$ \cite[Fig.\ 1]{Manzo2013} simulates the one of $\STLC_\oplus$, potentially using multiple steps, e.g.\ $(M\oplus_p N)^\circ \overset{0}{\rightarrow} pM^\circ \overset{p}{\rightarrow} M^\circ$.
Since $\LREL_!$ is an instance of models for $\mathrm{PCF}^{\mathcal{R}}$, similarly to \cite[Section VI]{Manzo2013}, we use it in order to get a model of $\STLC_\oplus$, by setting $\model{\Gamma \vdash_{\STLC_\oplus} M:A}:=\model{\Gamma \vdash_{\mathrm{PCF}^{\Lawv}} M^\circ:A}\in\HOM{\LREL_!}{\model{\Gamma}}{\model{A}}$.
Here we set $\model{\bool}:=\{0,1\}$ and $\model{\vdash_{\mathrm{PCF}^{\Lawv}} \true:\bool}\in\Lawv^{\{0,1\}}$ gives value $0$ on $1$ and value $\infty$ on $0$, similarly for $\false$.

\begin{remark}
 In the probabilistic case, the matrix interpreting a term need no longer be boolean.
 For instance, $\model{\vdash_{\STLC_\oplus} \true\oplus_p\false :\bool}_i=p$ if $i=1$ and $=1-p$ if $i=0$.
 %In general, one can show that $\model{\Gamma \vdash_{\STLC_\oplus} M:A}\in[0,1]\cup\{\infty\}^{!\model \Gamma \times \model A}$. Coefficients $> 1$ may be considered when dealing with other kinds of effects.
\end{remark}

A precise investigation of the relation between probabilistic programs and their $\LREL_!$ interpretation is left for further studies.

