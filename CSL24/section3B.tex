%\subsubsection{Unbounded duplications}

\begin{comment}

In order to interpret the full $\STLC$, we need a Cartesian closed category (CCC).
It is well-known \cite{Mellies2009} that it is always possible to construct a CCC by taking the \emph{co-Kleisli} $\C C_!$ of a so-called \emph{Lafont category} $\C C$.
%, a construction we now quickly recall.
%defined via a \emph{Lafont exponential} comonad $!$.
%Let us quickly recall the ideas behind these notions.
A SMCC is Lafont when it has finite products and it is equipped with a comonad $!$ (its \emph{Lafont exponential}) which, at level of objects, sends $X$ to an object $!X$ being the free commutative comonoid on $X$.
Such objects $!X$ represent the \emph{bang} connective of linear logic, granting infinite duplications via the infinite product $X^0\otimes X\otimes X^2\otimes X^3\otimes\cdots$, each factor representing a possible number of duplications.
It is well-known that, under mild conditions satisfied by $\QREL$, one can explicit this idea via the fact that the map $X\mapsto \finMS{X}$ (where $\finMS{X}$ is the set of finite multi-sets on $X$) lifts to a functor $!:\QREL\to\QREL$ which is a Lafont-exponential comonad.
Specializing [Corollary III.6, \cite{Manzo2013}] to our case, we have:

\begin{proposition}
 $\LREL$ is Lafont.
\end{proposition}

\end{comment}

Let us recall that the coKleisli category $\C C_!$ of a category $\C C$ w.r.t.\ a comonad $!$ is the category whose elements are the same of $\C C$, and $\HOM{\C C_!}{X}{Y}:=\HOM{\C C}{!X}{Y}$, with composition $\circ_!$ defined by making use of the co-multiplication of $!$.
%We will explicit this constructions in our tropical setting in the next lines.
As well-known, if a SMCC is \emph{Lafont}, then one obtains a CCC from it by defining the exponential objects as $X\to Y:=!X \multimap Y$.
With no surprise $\LREL$ is Lafont (\cite[Corollary III.6]{Manzo2013}) and specialising \cite[Theorem III.7]{Manzo2013} we have indeed that the coKleisli $\LREL_!$ is CCC, i.e.\ a model of $\STLC$, where as usual $!$ acts on objects by taking the set of finite multisets.
%It is instructive what its CCC-structure looks like in our tropical world.
Remark that the cartesian product of $\LREL_!$ (and of $\LREL$) is the disjoint union $+$ of sets, the \emph{evaluation} morphisms are matrices in $\Lawv^{!(!X\times Y)+X)\times Y}$, and the coKleisli composition of $s\in\Lawv^{!Y\times Z}$ and $t\in\Lawv^{!X\times Y}$ is the matrix $s\circ_! t\in\Lawv^{!X\times Z}$ where $(s\circ_! t)_{\mu,c}:=
%\begin{align}
\inf_{n\in\N, b_1\dots,b_n\in Y, \mu = \mu_1+\cdots +\mu_n}
 \left\{s_{[b_1,\dots,b_n],c} + \sum_{i=1}^n t_{\mu_i,b_i}\right\}$.
%\end{align}
%where $+$ is the multiset union.

%Remember that in $\Lawv$ the neutral element for addition is $\infty$ and the neutral for multiplication is $0$, so for instance the evaluation map is the matrix $\RM{eval}^{X,Y}\in\Lawv^{!((X\multimap Y) \times X)\times Y}\simeq\Lawv^{(!!X\times !Y\times !X)\times Y}$ given by $\RM{eval}^{X,Y}_{\rho_1\oplus\rho_2\oplus\mu,b}:=0$ if $\rho_1=[\mu]$ and $\rho_2=[b]$, and $\RM{eval}^{X,Y}_{\rho_1\oplus\rho_2\oplus\mu,b}:=\infty$ otherwise.

%As it is well-known, the Cartesian closed structure %of a category $\C C$  allows to define a sound interpretation $\model{\Gamma\vdash M:A}\in\HOM{\LREL_{!}}{\model \Gamma}{\model A}$ of terms as morphisms.
%In our case, we have:


