
%
%Fino a qui niente a capo
%
%Fino a qui niente a capo
%
%Fino a qui niente a capo
%
%Fino a qui niente a capo
%
%Fino a qui niente a capo
%
%Limite! Alla prossima riga sforo oltre 12 pag.
%

%In the next section we recall some basic ideas from tropical mathematics, and its connection with the study of the Lawvere quantale.
%Since polynomials correspond to piecewise linear (hence Lipschitz) functions in tropical mathematics,
%The reconstruction of the relational semantics over the tropical semi-ring, presented in Section \ref{section3} and Section \ref{section4}, will provide a metric semantics of the differential $\lambda$-calculus, bridging sensitivity and resource analysis. 
%In Section \ref{section5} we suggest potential applications of this approach, relating well-studied applications of program metrics, resource analysis with current uses of tropical mathematics in computer science.  
%Finally, in Section \ref{section6} we show that the connection between the metric and differential analysis of higher-order programs extends well beyond the relational semantics, through a more abstract correspondence between {generalized metric spaces} and modules over the tropical semiring.


%At the basis of our approach is the observation that the \emph{tropical semiring} $([0,\infty], \min, +)$, which is at the heart of tropical mathematics, coincides with the \emph{Lawvere quantale} $\Lawv=([0,\infty], \geq, +)$ \cite{Hofmann2014, Stubbe2014}, the structure at the heart of the categorical study of metric spaces initiated by Lawvere himself \cite{Lawvere1973}.
%Let us recall that a quantale is a complete lattice endowed with a continuous monoid action. In the case of $\Lawv$ the lattice is defined by the reverse order $\geq$ on $\BB R$, and the monoid action is provided by addition. Notice that the lattice join operation of $\Lawv$ coincides with the idempotent semiring operation $\min$. 
%A consequence of these observations is that, as we discussed below, the tropical approach to linear algebra coincides with the study of ``$\Lawv$-valued matrices'', i.e.~of maps of the form $s: X\times Y\to \Lawv$ .
%In particular, a (possibly $\infty$) metric on a set $X$ is nothing but a ``$\Lawv$-valued square matrix'' $d:X\times X\to \Lawv$ satisfying axioms like e.g.~the triangular law (indeed, such distance matrices correspond to $\Lawv$-\emph{enriched categories}, a viewpoint we explicitly take in Section \ref{section6}). 

%The study of matrices with values over the tropical semiring can be seen as a special case of the \emph{quantitative relational semantics} \cite{Manzo2013}, a well-studied semantics of the $\lambda$-calculus. 



A \emph{continuous} semiring \cite[Def. II.5]{Manzo2013} is a cpo equipped with an order-compatible semiring structure.
In a continuous semiring, all series always converge as $\sup$ of its partial sums.
%This is crucial for the definition of the weighted relational models for the $\lambda$-calculus.

\begin{example}
$[0,\infty]$ is a continuous semiring, which we denote $\overline{R_{\geq 0}}$, with usual order and semiring structure.
Notice that $\infty$ is added to ``artificially'' force any infinite series to converge.
The continuous semiring that we will use throughout all this paper is (one of) the so-called \emph{tropical semiring} $\Lawv:=[0,\infty]$ with $\min$ as sum (notice that it is idempotent) and $+$ as product.
Not only $\Lawv$ is at the heart of tropical mathematics, but it also coincides with the \emph{Lawvere quantale} \cite{Hofmann2014, Stubbe2014}
%(remember that a quantale is a complete lattice, for $\Lawv$ the order is $\geq$, endowed with a continuous monoidal action, for $\Lawv$ it is $+$)
, the structure at the heart of the categorical study of metric spaces initiated by Lawvere himself \cite{Lawvere1973}. 
%Notice that the lattice join operation of $\Lawv$ coincides with the idempotent semiring operation $\min$. 
\end{example}

%Since $\Lawv$ is a continuous commutative semiring, [Proposition III.3, \cite{Manzo2013}] immediately applies and gives:

%\begin{proposition}
% $\LREL$ is a linear $\Lawv$-category.
%\end{proposition}

%Unwrapping [Definition II.9, \cite{Manzo2013}], this means that:
%$\HOM{\LREL}{X}{Y}$ is a continuous $\Lawv$-semimodule, with semimodule operations defined pointwise;
%$\LREL$ is a continuous $\Lawv$-category, i.e.\ composition of morphisms commutes with $\inf$'s;
%$\LREL$ is linear, i.e.\ pre- and post-composition with any morphism in any $\HOM{\LREL}{X}{Y}$ are automorphisms on it.

%In the next sections we will see how $\LREL$ gives rise to denotational models of several variants of the $\lam$-calculus.



%
%
%
% 

%
%
%or more generally as a power series $f(x)=\sum_{n}\widehat f_{n}x^{n}$ with coefficient $\widehat f_{n}\in [0,1]$, we can define its \emph{tropicalization} $\trop f: \Lawv \to \Lawv$ as the function 
%\begin{align}
%\trop f(\alpha)= \inf_{n}\left\{ 
%\end{align}
%
%This correspondence can be made precise through the the so-called
% \emph{de Maslov dequantization} \cite{}.
% For each positive real $t$, any polynomial in $\BB R[x]$ can be written under the \emph{$t$-parameterized} form:
% \begin{align}
% p_{t}(x)= \sum_{i=1}^{k}t^{c_{i}}x^{i}
% \end{align}
% with the coefficients $c_{i}$ taken from $\Lawv$. 
% It is clear then that tropical polynomials and $t$-parameterized polynomials admit a one to one correspondence between their presentations.
% 
% Actually, the $\varphi$s and the $p_{t}$s can be related by passing through some intermediate functions $\varphi_{t}$ introduced by Maslov.
%For any $t>1$, the functions $\phi_{t}(x)=-\log_{t}x$ and $\psi_{t}(\alpha)=t^{-\alpha}$ are inverse of each other and define thus continuous (w.r.t.\ the usual topologies) bijections between the space of probabilities $[0,1]$ and $\BB R_{\geq 0}\cup\set{\infty}$ (we write $\log$ for the natural logarithm).
%Moreover, if we set $\alpha \widetilde+ \beta:= \frac{\alpha+\beta}{2}$, $\alpha\sumt{t}\beta=\phi_{t}(\psi_{t}(\alpha)\widetilde{+}\psi_{t}(\beta))=-\log (e^{-\alpha/t}+e^{-\beta/t})-\phi_{t}(2)$ and $\alpha\prodt{t} \beta:=\phi_{t}(\psi_{t}(\alpha)\psi_{t}(\beta))=\alpha+\beta$, it is known that: $\lim_{t\to 0}\alpha\sumt{t} \beta= \min\{\alpha,\beta\}$.
%In this sense, setting $\Lawv_t:=([0,\infty],\sumt{t},\prodt{t})$, one says that $\Lawv_t\to_{t\to 0^+}\Lawv$.
%Moreover, setting $\widetilde\Lawv:=([0,\infty],\widetilde+,\cdot)$, it can be shown that $\Lawv_t\simeq\widetilde\Lawv$ for all $t>0$, so the $\Lawv_t$ are all isomorphic, whereas at the limit we have a discontinuity: it can be shown that $\Lawv_t\not\simeq\Lawv$.
% 

%
%{\color{red}Lista delle cose da dire:}
%
%1) Def di quantale, come lattice e come complete idempotent semiring.
%Among the the so-called \emph{tropical semirings}, we consider the \emph{Lawvere quantale/semiring}.
%
%2) Def di $\Lawv$, the \emph{Lawvere quantale}: seen as the idempotent complete semiring, it is $(\BB R_{\geq0}\cup\set{\infty},\inf,\infty,\cdot,0)$.
%Seen as lattice it is defined by the order $\preceq$, which is the reversed order $\geq$ of the usual order $\leq$ on $\BB R_{\geq0}\cup\set{\infty}$.
%
%3) Maslov dequantisation:
%
%First, let us recall that for any non-negative real $t$, the functions $\phi_{t}(x)=-t\log x$ and $\psi_{t}(\alpha)=e^{-\alpha/t}$ are inverse of each other and define thus continuous (w.r.t.\ the usual topologies) bijections between the space of probabilities $[0,1]$ and $\BB R_{\geq 0}\cup\set{\infty}$ (we write $\log$ for the natural logarithm).
%Moreover, if we set $\alpha \widetilde+ \beta:= \frac{\alpha+\beta}{2}$, $\alpha\sumt{t}\beta=\phi_{t}(\psi_{t}(\alpha)\widetilde{+}\psi_{t}(\beta))=-\log (e^{-\alpha/t}+e^{-\beta/t})-\phi_{t}(2)$ and $\alpha\prodt{t} \beta:=\phi_{t}(\psi_{t}(\alpha)\psi_{t}(\beta))=\alpha+\beta$, it is known that: $\lim_{t\to 0}\alpha\sumt{t} \beta= \min\{\alpha,\beta\}$.
%In this sense, setting $\Lawv_t:=([0,\infty],\sumt{t},\prodt{t})$, one says that $\Lawv_t\to_{t\to 0^+}\Lawv$.
%Moreover, setting $\widetilde\Lawv:=([0,\infty],\widetilde+,\cdot)$, it can be shown that $\Lawv_t\simeq\widetilde\Lawv$ for all $t>0$, so the $\Lawv_t$ are all isomorphic, whereas at the limit we have a discontinuity: it can be shown that $\Lawv_t\not\simeq\Lawv$.
%
%4) Def di $\trop$ di un polinomio/serie, (\`e la stessa formula, dipende solo se gli indici sono finiti/infiniti).
%Come sta scritto sotto, giusto un po' pi\`u formale (per esempio, scriverlo come Definizione).
%
%The fundamental observation that led to the study of mathematics over the \emph{tropical semi-ring} $\Lawv=([0,\infty],\min,+)$ was that, by replacing everywhere the ``$+$'' by the ``$\min$'' and the ``$\times$'' by the ``$+$'', many algebraic and geometric objects becomes combinatorial and their computation simpler. 
%
%For instance, the tropicalization of a cubic polynomial $p(x)=ax^{3}+bx^{2}+cx+d$ yields a piecewise-linear function 
%\begin{align}
%\trop p(\alpha)= \min\{ 3\alpha+a, 2\alpha+b, \alpha+c,d\}
%\end{align}
%Notably, the \emph{tropical roots} (whose definition is recalled in Section \ref{section3}) of $\trop p(\alpha)$ can be found through a rather simple (indeed polytime \cite{}) algorithm, and can be used to \emph{approximate} the actual roots of $p(x)$ \cite{}. 
%More generally, the tropicalization of a power series $f(x)=\sum_{n}\widehat f_{n}x^{n}$ yields a \emph{tropical Laurent series} \cite{} 
%\begin{align}
%\trop f(\alpha)= \inf_{n}\left\{n\alpha+ \widehat f_{n}\right\}
%\end{align}
%a class of functions that we will study in detail in Section \ref{section4}.
%
%%- generalities about tropical maths (tropicalisation $\trop P$  of polynomials and of Laurent series, and their roots -- all that without $\LREL$)
%
