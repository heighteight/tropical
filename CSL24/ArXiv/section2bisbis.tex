% !TEX root = /Users/paolopistone/Documents/GitHub/tropicalnew/CSL24/main.tex

Before discussing how full-scale higher-order programming languages can be interpreted in terms of tropical power series, we highlight how such functions may naturally arise in the study of effectful programming languages.
We will see that, when considering probabilistic and non-deterministic programs, tropical tools can be used to describe the behavior of programs in \emph{the best/worst case}, and may lead to collapse the description of infinitely many possible behaviors into a combinatorial account of the optimal ones.
%Moreover, since, as we'll see, tropical semantics is also a \emph{metric} semantics, it can be used to study how much program behavior is \emph{sensitive} to errors, that is, how a small error in input may be increased in output. 




\subparagraph*{Maximum Likelihood Estimators for Probabilistic Languages}

%
%Tropical methods have been largely applied as a means to solve optimization problems. Typically, to solve a problem of the form 
%$$
%\mathrm{maximize} \ \ p(x_{1},\dots, x_{n})
%$$
%where $p(x_{1},\dots, x_{n})$ is some polynomial function, one can instead try to solve the (generally much simpler) problem of maximizing the associated tropical polynomial $\mathsf tp$ instead.



Let us start with a very basic probabilistic language:
the terms are $M::= \true \mid \false \mid M\oplus_p M$, for $p\in[0,1]$, and the operational semantics is $M\oplus_p N\to pM$ and $M\oplus_p N \to (1-p)N$, so that $M\oplus_p N$ plays the role of a probabilistic coin toss of bias $p$.
Consider the program
$
 M:=(\true \oplus_p\false)\oplus_p((\true\oplus_p\false)\oplus_p(\false\oplus_p\true)).
 $
 Calling $q=1-p$, to each occurrence of $\true$ or $\false$ in $M$, univocally determined by an address
$\omega\in \{l,r\}^{*}$, is associated a monomial $P_{\omega}(p,q)$ which determines the probability of the event ``$M\twoheadrightarrow_{\omega} \true/\false$'', that is, that $M$ reduces to $\true/\false$ according to the choices in $\omega$.
Thinking of $p,q$ as parameters, $P_{\omega}(p,q)$ can thus be read as the \emph{likelihood function} of the event ``$M\twoheadrightarrow_{\omega} \true/\false$''.
 For instance, we have
$P_{rll}(p,q):=qp^2$,
$P_{rrr}(p,q):=q^3$, and 
$P_{rrl}(p,q)=P_{rlr}(p,q):=q^2p$.
The polynomial function $Q_{\true}(p,q):=P_{ll}(p,q)+P_{rll}(p,q)+P_{rrr}(p,q)=p^2+p^2q+q^3$ gives instead the probability of the event ``$M\twoheadrightarrow \true$'', and analogously for $Q_{\false}(p,q):=P_{lr}(p,q)+P_{rrl}(p,q)+P_{rlr}(p,q)=pq+2pq^2$.

This way, the probabilistic evaluation of $M$ is presented as a \emph{hidden Markov model} \cite{Baum1966}, a fundamental statistical model, and notably one to which tropical methods are generally applied \cite{Pachter2004}.
Typical questions in this case would be, for a fixed $\omega_0$:
%
%The tropical point of view allows now to express two natural questions about this situation:
\begin{enumerate}
 \item What is the \emph{maximum likelihood estimator} for the event ``$M\twoheadrightarrow_{\omega_0} \false$''?
 I.e., which is the choice of $p,q$ that maximizes the probability $P_{\omega_0}$?
 \item 
Knowing that $M$ produced $\true$ (similarly for $\false$) , which is the \emph{maximum likelihood estimator} for the event ``$M\twoheadrightarrow_{\omega_0}\false$'', knowing that ``$M\twoheadrightarrow \false$''?
I.e., which is the choice of $p,q$ that makes $\omega_0$ the most likely path among those leading to $\false$ (i.e.\ that maximizes the conditional probability $\BB P(``M\twoheadrightarrow_{\omega_0} \false'' \mid ``M\twoheadrightarrow \false'')$)?
\end{enumerate}


Answering 1) and 2) amounts then at solving a maximization problem related to $P_{\omega_{0}}(p,q)$ or $Q_{\true}(p,q)$. In fact, these problems are more easily solved by passing to the associated \emph{tropical} polynomials.
For 1), the maximum values $x,y$ of $P_{rll}(p,q)$ can be computed by finding the \emph{minimum} values of $\mathsf tP_{rll}(-\log p, -\log q)= -2\log p- \log q$. Notice that the latter is precisely the \emph{negative log-probability} of the event ``$M\twoheadrightarrow_{rll} \false$''. For 2), the maximum values of $Q_{\true}(p,q): [0,1]^{2}\to [0,1]$ can be computed as $e^{-\alpha},e^{-\beta}$, where $\alpha,\beta\in[0,\infty]$ are the \emph{minimum} values of the tropical polynomial 
$\mathsf t Q_{\true}(\alpha,\beta) = \min \{ 2\alpha, 2\alpha+\beta, 3\beta\}$.

As we'll see in Section \ref{section3}, this analysis extends to PCF-style programs. For example, the program $M=\mathbf Y(\lambda x.\true \oplus_{p} x)$ yields the power series $Q_{\true}(p,q)=\sum_{n=0}^{\infty}pq^{n}=\frac{p}{1-q}$ that sums all \emph{infinitely many} ways in which $M$ may reduce to $\true$. Notice that the tropicalised series $\mathsf tQ_{\true}(-\log p,-\log q)=\inf_{n\in \mathbb N}\{-\log p -n\log q\}=-\log p$ collapses onto a single monomial describing the \emph{unique} most likely reduction path of $M$, namely the one that passes through a coin toss only once. 
%
%Similar arguments could be done by replacing $\true$ and $\false$ by, respectively, a converging and diverging term (e.g.~in a $\mathrm{PCF}$-style language), so 1) would be about finding maximum likelihood estimators for the event ``$M$ converges''.


\subparagraph*{Best Case Analysis for Non-Deterministic Languages}

This example is inspired from \cite{Manzo2013}. We consider now a basic non-deterministic language with terms $M::= \true \mid  \mathtt{Gen} \mid  M + M$, with an operation semantics comprising a non-deterministic reduction rule 
$M_{1}+M_{2} \stackrel{\alpha}{\to} M_{i}$ and a generation rule
$\mathtt{Gen}\stackrel{\beta}{\to} \true +\mathtt{Gen}$, 
where in each case the value $\alpha,\beta\in \Lawv$ indicates a \emph{cost} associated with the reduction (e.g.~the estimated clock value for the simulation of each reduction on a given machine model). 
Then, any reduction $\omega: M \twoheadrightarrow N$ of a term to (one of its) normal form is  associated with a tropical monomial $P_{\omega}( \alpha,\beta)$ consisting of the sum of the costs of all reductions in $\omega$. For a given normal form $N$, the reductions $\omega_{i}: M \twoheadrightarrow N$ give rise to a tps $\inf_{i\in I}P_{\omega_{i}}( \alpha,\beta)$. 
For example, consider the non-deterministic term
$
M :=\mathtt{Gen}+  ((\true + \true) + \mathtt{Gen})
$. 
The (infinitely many) reduction paths leading to $\true$ can be grouped as follows:
\begin{itemize}
\item left, then reduce $\mathtt{Gen}$ $n+1$-times, then left;
\item right, then left and then either left or right;
\item right twice, then reduce $\mathtt{Gen}$ $n+1$-times and then left.
\end{itemize}
This leads to the tps 
$\varphi_{M\twoheadrightarrow \true}(\alpha)=\inf_{n\in \mathbb N}\big\{2\alpha+(n+1)\beta, 3\alpha, 3\alpha+(n+1)\beta\big \}= \min\{ 2\alpha+\beta, 3\alpha\}$, which describes all possible behaviors of $M$. Notice that, since $\alpha$ and $\beta$ are always positive, the power series $\varphi_{M\twoheadrightarrow\true}(\alpha)$ is indeed equivalent to the tropical polynomial $\min\{ 2\alpha+\beta, 3\alpha\}$. In other words, of the infinitely many behaviors of $M$, only finitely many have chances to be \emph{optimal}: either left + $\mathtt{Gen}$ + left, or right + left + (left,right). Also in this case, reducing to best-case analysis leads to collapse the infinitary description of \emph{all} behaviors to a purely combinatorial description of the finitely many optimal ones. 

Once reduced $\varphi_{M\twoheadrightarrow\true}$ to a polynomial, the best behavior among these will depend on the values of $\alpha$ and $\beta$, and by studying the tropical polynomial $\varphi_{M\twoheadrightarrow\true}$ one can thus answer questions analogous to 2) above, that is, what are the best choices of costs $\alpha,\beta$ making a \emph{chosen} reduction of $M$ to $\true$ the cheapest one?



















