
%{\color{red}due parole sul fatto che non sappiamo se si posso usare strumenti propri della tropical geo per lo studio dei programmi. Per esempio:}
%\begin{remark}\label{rem:troproots}
% Taking the usual $M$ as example, the $p\in[0,1]$ s.t.\ $(p,1-p)$ is a tropical root of $\trop Q_{\true}$ or of $\trop Q_{\false}$ provide, by definition, the values of the bias of $\oplus_p$ for which there are at least two different equiprobable paths of tossings from $M$ to its normal form.
% Moreover, looking at Equation~\ref{eq:max}, we see that the $p\in[0,1]$ s.t.\ $(-\log p,-\log(r-p))$ is a tropical root of, say, $\trop Q_{\true}$ are the values of the bias of $\oplus_p$ for which there are at least two different equiprobable occurrences of $\true$ that are smapled by $M$ during its tossings to normal form, knowing that some $\true$ was sampled.
%\end{remark}
%
%{\color{red}due parole generalised metric spaces and modules}
%
%
%{\color{red}due parole su PCoh, $\overline{R_{\geq 0}}$Rel e le loro tropicalizzazioni (gia accennato in \autoref{rmk:tropof01Rel})}

%In our opinion,
The main goals of this paper are two. Firstly,  to
demonstrate the existence of a conceptual bridge between two well-studied quantitative approaches to higher-order programs, and to highlight the possibility of transferring results and techniques from one approach to the other. 
Secondly, to suggest that tropical mathematics, a
field which has been largely and successfully applied in computer science, could be used for the quantitative analysis of functional programming languages. 
While the first goal was here developed in detail, and at different levels of abstraction, for the second goal we only sketched a few interesting directions, and we leave their development to a second paper of this series. 

While the main ideas of this article only use basic concepts from the toolbox of tropical mathematics, an exciting direction is that of looking at potential applications of more advanced tools from tropical algebraic and differential geometry (e.g.~Newton polytopes, tropical varieties, tropical differential equations). Another interesting question is how much of our results on tps and their tropical Taylor expansion can be extended to the abstract setting of generalized metric spaces and continuous functors. 
%We believe that exploring these ideas in more depth could be a fruitful direction; moreover, 
%since both generalized metrics and quantale-modules have been largely studied in computer science, 
%a natural question is if the generalized approach of Section \ref{section6} could  lead to new applications of metric and tropical methods to the $\lambda$-calculus.

%Finally, since bounds on the Taylor expansion translate into Lipschitz conditions, 
% two interesting directions to explore are provided by (non-idempotent) intersection types and finiteness spaces \cite{Ehrhard2005}, as both methods are in principle capable of capturing \emph{finitary} bounds on the Taylor expansion. %
%Notably, knowing that the application of a program $M$ to $N$ expands as a finite sum of linear applications may allow one to predict how sensitive $M$ will be ``around $N$''. 

