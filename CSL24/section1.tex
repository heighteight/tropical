
In recent years, more and more interest in the programming language community has been directed towards the study of \emph{quantitative} properties of programs like computing the number of computation steps or convergence probabilities, 
as opposed to purely \emph{qualitative} properties like termination or program equivalence. 
Notably, a significant effort has been made to extend, or adapt, well-established qualitative methods, like type systems, relational logics or denotational semantics, to account for quantitative properties. We can mention, for example, 
intersection type systems aimed at capturing time or space resources \cite{decarvalho2018, Accattoli2022} or convergence probabilities \cite{Breuvart2018, PistoneLICS2022},  relational logics to account for probabilistic properties like e.g.~differential privacy \cite{Barthe_2012} or metric preservation \cite{Reed2010, dallago}, as well as the study of denotational models for 
probabilistic \cite{Ehrhard2011, Staton2017} or differential \cite{difflambda} extensions of the $\lambda$-calculus. 
The main reason to look for methods relying on (quantitative extensions of) type-theory or denotational semantics is that these approaches yield \emph{modular} and \emph{compositional} techniques, that is, allow one to deduce properties of complex programs from the properties of their constituent parts.   
Among such quantitative approaches, two different directions have received considerable attention, which we recall below.
In both approaches the notion of \emph{linearity}, in the sense of linear logic \cite{girardLl} (i.e.~of using inputs exactly once), plays a crucial role.

\paragraph*{Two kind of quantitative approaches}

On the one hand one there is the approach of \emph{program metrics} \cite{Reed2010, Gaboardi2017, Gabo2019} and \emph{quantitative equational theories} \cite{Plotk}: when considering probabilistic or approximate computation, rather than asking whether two programs compute \emph{the same} function, it makes more sense to ask whether they compute functions which do not differ \emph{too much} by controlling the propagation of an error during the computation.
This has motivated the study of denotational frameworks in which types are endowed with a metric, measuring similarity of behavior; this approach has found  applications in e.g.~differential privacy \cite{Reed2010} and coinductive methods \cite{Bonchi2018}, and was recently extended to account for the full $\lambda$-calculus \cite{Geoffroy2020, PistoneLICS, PistoneFSCD2022}.
%As already mentioned, in many situations (e.g.~when dealing with computationally difficult problems, or in differential privacy \cite{Alvim2011, Reed2010}), one %does not look for algorithms to compute a function \emph{exactly}, but rather to approximate it (in an efficient way) within some error bound. In other common situations (e.g.~in differential privacy \cite{Alvim2011, Reed2010}) one needs to verify that an algorithm is not \emph{too sensitive} to errors, that is, a small error in the input will produce a comparably small error in the output. 
%In all these cases, it is common to consider forms of denotational semantics in which types are endowed with a \emph{behavioral metric}, that is, a metric on programs which account for differences in behavior. 
A fundamental insight coming from this line of work is that \emph{affine} programs, i.e.~programs that may use their input at most once, correspond to \emph{non-expansive} (or $1$-Lipschitz) maps, that is, to functions $f$ for which the distances
$d(f(x),f(y))$ produced in output are bounded by the distances $d(x,y)$ in input. 
This is due to the fact that the category $\Met$ of pseudo-metric spaces and non-expansive maps provides a model of the %\emph{linear} simply typed $\lambda$-calculus, being a \emph{symmetric monoidal closed} category, and in fact it also models 
\emph{affine} $\lambda$-calculus (it is symmetric monoidal closed, cartesian and monoidal units coincide). 
The metric approach is not restricted to affine programs, but can be extended to programs with \emph{bounded} duplications \cite{Reed2010, Gaboardi2017}: the idea is that a program duplicating its input $K$ times will give rise to a $K$-\emph{Lipschitz} map.
For instance, the higher-order program $M=\lambda f.\lambda x.f(f(x))$, which duplicates the function input $f$, yields a $2$-Lipschitz map between the metric space $\BB R\multimap \BB R$ of non-expansive real functions and itself. %: if $f,g$ are two non-expansive maps differing by at most $\epsilon$ (i.e.~for which $|f(x)-g(x)|\leq \epsilon$ holds for all $x\in \BB R$), then the application of $M$ to $f$ and $g$ will produce two maps differing by at most $2\epsilon$. 
By observing that a $r$-Lipschitz map between metric spaces $X$ and $Y$ is the same as a non-expansive map between the \emph{re-scaled} space $rX:=(X,%d_{r}(x,y)=
rd(x,y))$ and $Y$, the program $M$ above 
can thus be interpreted as a non-expansive map from $2(\BB R\multimap \BB R)$ to $\BB R\multimap \BB R$.
These observations have led to the study of linear $\lambda$-calculi with \emph{graded} exponentials, like $\mathsf{Fuzz}$ \cite{Reed2010}, inspired from Girard's Bounded Linear Logic \cite{Girard92tcs}, which have been applied to the study of differential privacy \cite{Gaboardi2013, Gaboardi2017}. The types of such systems are defined by combining linear constructors with a \emph{graded linear exponential comonad} $!_{r}(-)$ \cite{Katsumata2018}.

On the other hand, there is the approach based on \emph{differential} \cite{difflambda} or \emph{resource-aware} \cite{Boudol1993} extensions of the $\lambda$-calculus, which is well-connected to the so-called \emph{relational semantics} \cite{Manzo2012, Manzo2013, dill} and has a syntactic counterpart in the study of \emph{non-idempotent} intersection types \cite{decarvalho2018, Mazza2016}. This family of approaches have been exploited to account for higher-order program differentiation \cite{difflambda}, to establish reasonable \emph{cost-models} for the $\lambda$-calculus \cite{Accattoli2021}, and have also been shown suitable for the probabilistic setting \cite{Manzo2013, Breuvart2018, PistoneLICS2022}. 
%A different family of approaches to linearity and duplication arises from the study of the \emph{differential $\lambda$-calculus} \cite{difflambda} (and differential linear logic \cite{dill}) and its categorical models. 
The key ingredient is a \emph{differential operator} $\Der$,  added to the usual syntax of the $\lambda$-calculus. The intuition is that, given $M$ of type $A\to B$ and $N$ of type $A$, the program $\Diff{M}{N}$, still of type $A\to B$, corresponds to the \emph{linear application} of $M$ to $N$: this means that $N$ is passed to $M$ so that the latter may use it exactly once (this is why $\Diff{M}{N}$ still has type $A\to B$, since $M$ might need \emph{other} copies of an input of type $A$).
Another intriguing similarity between program derivatives and 
actual derivatives is provided by the \emph{Taylor expansion} $\C T$:
%In analysis, a sufficiently regular function can be seen (locally around a point) as its Taylor expansion, i.e.\ a series of polynomials weighted via some rational coefficient.
%Now, a polynomial in $x$ of degree $k$ can be thought of as a function which uses its argument $x$ exactly $k$ times.
in $\STDLC$ (even untyped), we can perform a syntactic Taylor expansion of an ordinary $\lambda$-term via an inductively defined map $\Theta$ giving rise to an infinite series of terms, whose crucial case of the definition is
$
\Theta(MN)=\sum_{n\in\mathbb{N}}\frac{1}{n!}\Der^n\left[\Theta(M),\Theta(N)^n\right]0.
$
As in analysis, it decomposes an application as a series of $k$-linear functions, which can be seen as its approximants.
In other words, unbounded duplications correspond to some sort of limit of bounded, but arbitrarily large, ones.
Since we only consider idempotent sum, the factorial coefficient disappears, and the resulting Taylor map is called the \emph{qualitative} Taylor expansion, indicated by $\Te{\_}$. 


%On the one hand one there is the approach of \emph{program metrics} \cite{Reed2010, Gaboardi2017, Gabo2019} and \emph{quantitative equational theories} \cite{Plotk}: when considering probabilistic or approximate computation, rather than asking whether two programs compute \emph{the same} function, it makes more sense to ask   whether they compute functions which do not differ \emph{too much}. This has motivated the study of denotational frameworks in which types are endowed with a metric, measuring similarity of behavior; this approach has found  applications in e.g.~differential privacy \cite{Reed2010} and coinductive methods \cite{Bonchi2018}, and was recently extended to account for the full $\lambda$-calculus \cite{Geoffroy2020, PistoneLICS, PistoneFSCD2022}.

%On the other hand, there is the approach based on \emph{differential} \cite{difflambda} or \emph{resource-aware} \cite{Boudol1993} extensions of the $\lambda$-calculus, which is well-connected to the so-called \emph{relational semantics} \cite{Manzo2012, Manzo2013, dill} and has a syntactic counterpart in the study of \emph{non-idempotent} intersection types \cite{decarvalho2018, Mazza2016}. This family of approaches have been exploited to account for higher-order program differentiation \cite{difflambda}, to establish reasonable \emph{cost-models} for the $\lambda$-calculus \cite{Accattoli2021}, and have also been shown suitable for the probabilistic setting \cite{Manzo2013, Breuvart2018, PistoneLICS2022}. 




% from higher-order programs is based on  soon as one develops  differential semantics in the framework of 
%tropical mathematics.
%
%''
%
%s
%emantics a typical ``duplicating'' map is obtained by composing the diagonal with multiplication:
%$$
%\begin{tikzcd}
%\mathbb R \ar{rrr}{x\mapsto \langle x, x\rangle}
% & &  &
% \mathbb R\times \mathbb R 
% \ar{rrr}{\langle x,y\rangle \mapsto x\cdot y}
% & & & \mathbb R
%\end{tikzcd}
%$$
%yielding the square product function $\lambda x.x^{2}$.
%However, in metric semantics this function needs not even exist (as these models are often restricted to Lipschitz-continuous maps \cite{Gabo2017})! Instead, a typical ``duplicating'' map can be obtained by composing the diagonal with the sum 
%$$
%\begin{tikzcd}
%\mathbb R \ar{rrr}{x\mapsto \langle x, x\rangle}
% & &  &
% \mathbb R\times \mathbb R 
% \ar{rrr}{\langle x,y\rangle \mapsto x+y}
% & & & \mathbb R
%\end{tikzcd}
%$$
%yielding the linear (and Lipschitz) function $\lambda x.2x$.
%
%As this example seems to suggest, there seems to be a sort of ``logarithmic'' gap between the two approaches. Can this be made explicit?



\paragraph*{Tropical mathematics and program semantics } 

Tropical mathematics was introduced in the seventies by the Brazilian mathematician Imre Simon \cite{Simon} as an alternative approach to algebra and geometry where the usual ring structure of numbers based on addition and multiplication is replaced by the semiring structure given, respectively, by ``$\min$'' and ``$+$''.
%
%
% interpreting the usual ``$\times$'' and  ``$+$'' operations by  ``$+$'' by ``$\min$''. It can thus be seen as a sort of ``logarithmic'' version of usual geometry (this idea can be made precise via the so-called \emph{Maslov deformation} \cite{}).
%Tropical mathematics is a form of \emph{idempotent} mathematics, since the role of addition is 
%played by the idempotent operation $\min$.
For instance, the polynomial $p(x,y)=x^{2}+xy^{2}+y^{3}$, when interpreted over the tropical semiring, translates as the piecewise linear function
$
\varphi(x,y)=\min\{2x, x+2y, 3y\}
$.
%This is not a \emph{ad-hoc} setting: 
In the last decades, tropical geometry evolved into a vast and rich research domain, providing a combinatorial counterpart of usual algebraic geometry, with important connections with optimisation theory \cite{Sturmfelds}.
Computationally speaking, working with $\min$ and $+$ is generally easier than working with standard addition and multiplication; for instance, the fundamental (and generally intractable) problem of finding the roots of a polynomial admits a \emph{linear time} algorithm in the tropical case (and, moreover,  the tropical roots can be used to approximate the actual roots \cite{Noferini2015}).
The computational nature of tropical notions explains why these are so widely applied in computer science, notably for convex analysis and machine learning (see \cite{Maragos2021} for a recent survey).

Coming back to our discussion on program linearity, in metric semantics, linear programs correspond to \emph{non-expansive} maps, that is, to functions that do not increase distances; moreover, the possibility of duplicating inputs leads to interpret \emph{bounded} programs (i.e.~programs with a fixed duplication bound) as \emph{Lipschitz-continuous} maps \cite{Gaboardi2017}.
By contrast, in the standard semantics of the differential $\lambda$-calculus, linear programs correspond to linear maps, in the usual algebraic sense, while the possibility of duplicating inputs leads to consider functions defined as \emph{power series}.

A natural question is thus whether these two apparently unrelated ways of interpreting linearity and duplication can be somehow reconciled. At a first glance, there seems to be a  ``logarithmic'' gap between the two approaches:
in metric models $n$ times duplication results in a \emph{linear} (hence Lipschitz) function $n\cdot x$, while in differential models this results in a \emph{polynomial} function $x^{n}$, hence not Lipschitz. The fundamental motivation of this work is then the observation that 
this gap is naturally overcome once we interpret these functions in the framework of tropical mathematics, where, as we'll see, the monomial $x^{n}$ precisely reads as the linear function $n\cdot x$.

Now, the metric approach usually yields Lipschitz conditions for affine programs, but what about ``unbounded'', simply typed $\lambda$-calculus (the ordinary one)?
In such case, one might lose the Lipschitz property, so the programs might be \emph{too} sensitive to errors. 
%For instance, while the functions $M_{k}=\lambda x. k\cdot x: \BB R\to \BB R$ are all Lipschitz-continuous, with Lipschitz constant $k$, the program $M=\lambda x.x^{2}$ obtained by ``duplicating'' $x$ is not Lipschitz anymore: $M$ is, so to say, \emph{too} sensitive to errors. 
It is indeed well-known that the category $\Met$ is \emph{not} Cartesian closed, so it is not a model of $\STLC$ (yet, several Cartesian closed \emph{sub-}categories of $\Met$ do exist, see e.g.~\cite{Clementino2006, PistoneFSCD2022}).
%Still, one might observe that the program $M$ above is actually \emph{locally} Lipschitz-continuous (i.e.~Lipschitz over any open ball).
However, one can still consider Cartesian closed categories of \emph{locally} Lipschitz maps  (i.e.~Lipschitz over any open ball), see e.g.\ \cite{Ehrhard2011, PistoneLICS}, and a new example will be considered in this paper.

On the other hand, since the resource approach decomposes, through the Taylor formula, an unbounded application as a limit of bounded ones, one might well ask whether it could be possible to see this formula as a decomposing the associated locally Lipschitz map by a limit of the associated Lipschitz ones, thus bridging the metric and differential approaches.  
Here, a natural direction to look for is the \emph{relational semantics}, since strictly related with the Taylor expansion $\STDLC$ {\color{red}CITE DeCarvalho ecc}. 
However, in this semantics, terms with bounded applications correspond to \emph{polynomials}, i.e.~to non-Lipschitz functions.
 
Yet, what if these polynomials were tropical ones, i.e.~piecewise linear functions? This way, the Taylor expansion could really be interpreted as a decomposition of $\lambda$-terms via limits (indeed, $\inf$s) of Lipschitz maps: unbounded term application could be seen as a limit of \emph{more and more sensitive} operations. 

%This viewpoint, that we develop in the following sections, not only suggests the application of optimization methods based on tropical mathematics to the study of the $\lambda$-calculus and its quantitative extensions, but it scales to a more abstract level, leading to introduce a differential operator for continuous functors between \emph{generalized} metric spaces (in the sense of \cite{Lawvere1973}).
%
%At this point, 
%by interpreting such polynomials over the tropical semiring 
%
%Our question can thus be reformulated as follows: can we make the relational semantics \emph{Lipschitz}, hence amenable to metric and sensitivity analysis? The goal of this paper is to show that, by appealing to tropical mathematics, this is indeed possible and leads to the somehow unespected discovery of a bridge between the metric and differential study to higher-order programs.

At this point, it is worth mentioning that a tropical variant of relational semantics has already been considered \cite{Manzo2013}, and shown capable of capturing \emph{best-case} quantitative properties, but has not yet been studied in detail. Furthermore, connections between tropical linear algebra and metric spaces have also been observed \cite{Fuji} within the abstract setting of \emph{quantale-enriched} categories \cite{Hofmann2014, Stubbe2014}.
However, a thorough investigation of the interpretation of the $\lambda$-calculus within tropical mathematics has not yet been undertaken. 

{\color{red}Dire proba! Non solo metric vs differential}


\paragraph*{This paper and our contributions}

{\color{red}Dire proba!}
%\subsection{Contributions}

In this paper we demonstrate that the relational interpretation of the $\lambda$-calculus based on tropical mathematics does indeed provides the desired bridge between differential and metric semantics. Moreover, we show that the conceptual unification of these two approaches suggests ways in which techniques from resource-analysis could be used in sensitivity analysis and \emph{vice-versa}, paving the way for new  applications of tropical geometry to the  study of higher-order programs.

Our contributions in this paper are threefold:
\begin{itemize}

\item we study the relational model over the tropical semiring  and we show that the functions interpreting simply-typed lambda terms, which correspond to a generalization of \emph{tropical Laurent series} \cite{Porzio2021}, are locally Lipschitz-continuous, thus yielding a full-scale metric semantics for the $\lambda$-calculus and its bounded fragments. This is in Sections \ref{section3} and \ref{section4}.
%Moreover, we exploit the differential structure of the relational model to study the \emph{tropical Taylor expansion} of a $\lambda$-term, which can be seen as an approximation of the term by way of Lipschitz-continuous maps.


\item Using the relational model as our main source of inspiration,  we suggest a few potential applications of tropical methods to the study of quantitative properties of non-deterministic and probabilistic functional programs, like counting best-case computation steps, 
measuring convergence log-probabilities, and 
differential privacy. This is in Section~\ref{section5}

\item We conclude 
by putting the connection between the 
tropical, differential and metric viewpoints at the right level of generality.
By recalling and suitably extending a well-known correspondence between Lawvere's \emph{generalized metric spaces} \cite{Lawvere1973, Stubbe2014} and modules over the tropical semi-ring \cite{Russo2007}, we show that the category of \emph{complete} generalized metric spaces provides a model of the differential $\lambda$-calculus which extends the tropical relational model. This is in Section~\ref{section6}.
\end{itemize}
%
%\section{Bounded and Differential $\lambda$-Calculi}
%
%
%Bounded Simply Typed $\lambda$-calculus $\BSTLC$:
%$$
%A::= o \mid !_{n}A \multimap A
%$$
%
%
%Resource Simply Typed $\lambda$-calculus $\RSTLC$:
%$$
%A::= o \mid [A, \dots , A] \multimap A
%$$
%
%
%Define a translation of types $(-)^{\C R}$ from $\BSTLC$ to $\RSTLC$ by $o^{\C R}=o$ and $(!_{n}A\multimap B)^{\C R}=
%[\underbrace{A^{\C R},\dots, A^{\C R}}_{n\text{ times}}]\multimap B^{\C R}$.
%
%\begin{proposition}
%$\Gamma \vdash_{\BSTLC} M:A$ implies 
%$\Gamma^{\C R}\vdash_{\RSTLC}M:A^{\C R}$.
%\end{proposition}
%


