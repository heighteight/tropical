%%In this section we illustrate a few directions in which the tropical semantics just introduced could be used to analyze quantitative properties of higher-order programs. 

%Since algebraic and geometric properties in tropical mathematics are usually more tractable from a computational point of view, in several well-known applications (e.g.~for optimization problems related to machine learning \cite{Pachter2004, Zhang2018, Maragos2021}) one starts from a given model, typically expressed by some polynomial function $f$, and studies  what properties of the model can be deduced from the \emph{tropicalization} of $f$, noted $\trop f$, i.e.~the transformation of $f$ into a tropical polynomial. Here we follow a similar pattern: we consider a program $M$, which can be expressed in the form of a polynomial or a power series $f$, and we  investigate what quantitative properties of $M$ can be deduced from the properties of $\trop f$, that will indeed coincide with the interpretation of $M$ in $\LREL_{!}$.

%
%
%%several well-known applications of tropical mathematics is to study how much can be deduced of some function starting from the properties of its tropicalization.
%%In Section \ref{section5} we will follow a similar direction, investigating what quantitative properties of a higher-order programs are revealed by the study of its tropical interpretation.
%

%


\subsection{The tropicalization of polynomials and power series}

%Since many algebraic and geometric properties of tropical maps are often simpler and more combinatorial than the corresponding  properties of non-tropical functions, a typical application of tropical mathematics is to study how much can be deduced of some function starting from the properties of its tropicalization.
%In Section \ref{section5} we will follow a similar direction, investigating what quantitative properties of a higher-order programs are revealed by the study of its tropical interpretation.
%

Let us first recall how standard polynomials and power series over $[0,1]$ can be turned into tLs via the so-called \emph{Maslov dequantization} \cite{Litvinov2007}.
%
%Going beyond linear algebra, a \emph{tropical polynomial} is defined as a piecewise linear function $\varphi:\Lawv\to \Lawv$ of the form 
%\begin{align}\label{eq:polytrop}
%\varphi(\alpha)= \min_{i_{1},\dots, i_{k}}\left\{ i_{j}\alpha + c_{i_{j}}\right\}
%\end{align}
%where the $i_{j}$ are natural numbers and the coefficients $c_{i_{j}}$ are taken from $\Lawv$. For instance, the polynomial
%$\varphi_{3}(\alpha)=\min\{ 3\alpha+1/8,2\alpha+1/4, \alpha+1/2, \alpha\}$ will be discussed in Section \ref{section4}, and its graph is illustrated in Fig.~\ref{fig:plot1}.
%A value $\alpha\in \Lawv$ is a \emph{root} of the polynomial $P$ when
%the minimum at $\varphi(\alpha)$ is attained at least twice (equivalently, when 
% $\varphi$ is not differentiable at $\alpha$). In other words, the tropical roots of $\varphi$ coincide with the points where the slope of $\varphi$ changes. 
%%
%Intuitively, tropical polynomials look much like standard polynomials, although with ``$+$'' replaced by ``$\min$'', and ``$\times$'' replaced by ``$+$''. 
%In fact, this intuition can be made precise as follows: 

%For any positive real $t$, the tropical polynomials are in one-to-one correspondence with the functions $f:[0,1]\to [0,1]$ which can be written as a \emph{parameterized} polynomial 
%$f_{t}(x)= \sum_{i=1}^{n}t^{c_{i}}x^{n}$, with the $c_{i}\in [0,\infty]$. 
%%Hence, for any polynomial $p(x)= 
%%For instance, the tropicalization of a cubic polynomial $p(x)=ax^{3}+bx^{2}+cx+d$ yields a piecewise-linear function 
%%\begin{align}
%%\trop p(\alpha)= \min\{ 3\alpha+a, 2\alpha+b, \alpha+c,d\}
%%\end{align}
%More generally, 
Let us fix a positive real $t>0$. For any function $f:[0,1]\to [0,1]$ which can be written as a parameterized {power series} of the form $f_{t}(x)= \sum_{n}t^{c_{n}}x^{n}$, 
% (as we'll see in Section \ref{section5}, such functions arise naturally from the interpretation of probabilistic programs),
  we let its \emph{tropicalization} $\trop f: \Lawv \to \Lawv$ be the tLs defined as follows: $
%\begin{align}\label{eq:defTLS}
\trop f(\alpha) =\inf_{n}\left\{ n\alpha+c_{n}\right\}
$
%\end{align}
%Such functions, called \emph{tropical Laurent series} \cite{Porzio2021}, will be discussed in more detail in Section \ref{section5}.
Clearly, for any $t>0$, there is a one-to-one correspondence between the representations of power series in parameterized form and the associated tLs. Moreover, 
$f$ and $\trop f$ can be related by a limit passage as follows: the functions $\phi_{t}(x)= -\log_{t}x$ and $\varphi_{t}(\alpha)= t^{-\alpha}$ define continuous bijections between $[0,1]$ and $[0,\infty]$ and, by letting
$\trop_{t}f: [0,\infty]\to [0,\infty]$ be defined by 
$\trop_{t}f(\alpha)= \phi_{t}\circ f \circ \psi_{t}$, one has that 
$\trop f= \lim_{t\to 0}\trop_{t}f$. 
Indeed, one can check that the ``parameterized'' sums and product $\alpha \sumt{t}\beta:= \phi_{t}(\psi_{t}(\alpha)+\psi_{t}(\beta))= -\log_{t}(t^{-\alpha}+t^{-\beta})$ and 
$\alpha \prodt{t}\beta:= \phi_{t}( \psi_{t}(\alpha)\psi_{t}(\beta))=
-\log_{t}(t^{-\alpha}t^{-\beta})$ converge respectively to $\min\{\alpha,\beta\}$ and $\alpha+\beta$, when $t\to 0$.



\begin{comment}

\subsection{Best case analysis and metric reasoning}

The possibility of using the relational model over the tropical semiring for ``best case'' resource analysis has already been explored in \cite{Manzo2013}. Notably, they considered an interpretation of a language for $\B{PCF}$ with non-deterministic choice in which each $\lambda$-abstraction and each occurrence of the fixpoint operator $Y$ is assigned a ``weight'' 1, and showed that for any program $M$ of type $\B{nat}$, 
the value of the interpretation $\model{M}\in \Lawv^{\BB N}$ on a particular natural number $k$, i.e.~$\model{M}(k)\in \Lawv$, corresponds to the \emph{minimum} number of $\beta$- or $\TT{fix}$-redexes reduced in a reductions sequence from $M$ to $\underline n$. 
In the next paragraph we will illustrate an analogous ``best case'' analysis for probabilistic programs.

What does the metric analysis from the previous sections add to that? Firstly, the possibility of \emph{comparing} different programs with respect to their quantitative properties. For example, in the $\B{PCF}$ semantics recalled above, the distance between two programs $M$ and $N$ of type $\B{nat}$, provides a bound on the difference between the  ``best case'' computation time of $M$ and that of $N$. For instance, by taking, instead of the $\infty$-norm metric on $\Lawv^{\BB N}$,  
the \emph{non symmetric} distance (or quasi-metric, a viewpoint we explicitly take in Section \ref{section6}) $q(\B x, \B y)=\sup_{n}\{\B y_{n}\dotdiv \B x_{n}\}$, a ``distance'' $q(\model{M},\model{N})\leq \epsilon$ would indicate that $\model{M}$ \emph{improves} on $\model{N}$ of at most $\epsilon$ steps at each computation. 

Secondly, the Lipschitz conditions from Section \ref{section4} allow us to reason on program distances in a \emph{compositional} way: suppose, as before, that $M,N:A$ are two programs such that $M$ improves on $N$ by $\epsilon$, and let $\TT C[-]:A \to \B{nat}$ indicate a context; knowing that the interpretation of $\TT C$ is $k$-Lipschitz-continuous on some open set containing both $\model M$ and $\model N$, allows us to immediately deduce that $\TT C[M]$ improves on $\TT C[N]$ by $k \epsilon$. 
Observe that this will typically be the case when the Taylor expansions of $\TT C[M]$ and $\TT C[N]$ actually yields a \emph{finite} sum of at most $k$ terms, i.e.~when 
\begin{align}
\TT C[M] = \sum_{i=0}^{k} \TT D^{(k)}\Big[\lambda x.\TT C[x]\Big]\cdot M^{k}
\end{align}
and similarly for $\TT C[N]$. It is here worth recalling that, for $\STDLC$, a well-known result \cite{difflambda} is that the Taylor expansion of a closed application $MN$ is always \emph{finite}, although its non-zero coefficients may be arbitrarily high. 
Notably, these observations suggest to study tropical versions of \emph{finiteness spaces} \cite{Ehrhard2005}, 
a variant of the relational semantics modeling strongly normalizing programs via \emph{finite} power series.
%We mention this point in Section~\ref{section8}.

\end{comment}

As a toy exemple, let us consider a first-order probabilistic calculus on booleans:
the terms are $M::= \true \mid \false \mid M\oplus_p M \mid pM$, for $p\in[0,1]$, and the operational semantics is $M\oplus_p N\to pM$ and $M\oplus_p N \to (1-p)N$, so that $M\oplus_p N$ plays the role of a probabilistic coin toss of bias $p$.

Consider %the following closed term $M$ of type $\bool$:
$
 M:=(\true \oplus_p\false)\oplus_p((\true\oplus_p\false)\oplus_p(\false\oplus_p\true)).
 $
Let us give addresses $\omega\in\set{ll,lr,rll,rlr,rrl,rrr}$ to the occurrences of $\true,\false$ in $M$, by following the tree structure of $M$, ($l$ is ``left'' and $r$ is ``right'').
The same addresses also represent all the different possible reduction paths from $M$ to a normal form.
%For instance, $rll$ represents the reduction which keeps the right part of the outermost $\oplus_p$ and erases the left part, then continues by choosing the left part twice, reaching at the end the occurrence $\true_{rll}$ in $M$, i.e.\ the second occurrence of $\true$ in $M$ starting from the left.
Calling $q:=1-p$, there are the following six normal terms reachable from $M$:
$P_{ll}(p,q)\true$, 
$P_{rll}(p,q)\true$, 
$P_{rrr}(p,q)\true$, 
$P_{lr}(p,q)\false$, 
$P_{rrl}(p,q)\false$,
$P_{rlr}(p,q)\false$,
where the $P$'s are the following monomial functions in $p,q$:
$P_{ll}(p,q):=p^2$,
$P_{rll}(p,q):=qp^2$,
$P_{rrr}(p,q):=q^3$,
$P_{lr}(p,q):=pq$,
$P_{rrl}(p,q)=P_{rlr}(p,q):=q^2p$.
%They correspond to the respective reduction path from $M$ to the normal term of the same address.
$P_{\omega}(p,1-p)$ is then the probability of the event ``$M\twoheadrightarrow \true_\omega/\false_\omega$'' (depending on $\omega$).
Thinking of $p,q$ as parameters, $P_{\omega}(p,q)$ can thus be read as the \emph{likelihood function} of the event ``$M\twoheadrightarrow \true_\omega/\false_\omega$''.
The polynomial function $Q_{\true}(p,q):=P_{ll}(p,q)+P_{rll}(p,q)+P_{rrr}(p,q)=p^2+p^2q+q^3$ gives instead the probability of the event ``$M\twoheadrightarrow \true$'', and analogously for $Q_{\false}(p,q):=P_{lr}(p,q)+P_{rrl}(p,q)+P_{rlr}(p,q)=pq+2pq^2$.
%Let us consider in this subsection a probabilistic extension of $\lam$-calculus, call it $\STLC_\oplus$, adding as usual terms of shape $pM+qN$ and $M\oplus_p N$, for $p,q\in[l,r]$.These terms are typed via the rule:
%\[
%\dfrac{\Gamma\vdash M:A \qquad \Gamma\vdash N:A}{\Gamma\vdash M\oplus_p N:A}
%\]
%and similar for $\Gamma\vdash pM+qN:A$.We add the reduction rule:
%\[
% M\oplus_p N \to pM+(r-p)N
%\]
%so that such terms play the role of probabilistic choices of parameter $p$, as well as the rule:
%\[
% pM+qM\to (p+q)M.
%\]
%Let us consider $M:=(I\oplus_p\Omega)\oplus_p((I\oplus_p\Omega)\oplus_p(I\oplus_p\Omega))$. Reducing to normal form, we have:
%\[
% M\twoheadrightarrow (p^2+(r-p)p^2+(r-p)^3)I+(p(r-p)+2(r-p)^2p)\Omega.
%\]
%The index $\omega\in\set{ll,rll,rrr,lr,rrl,rlr}$ of each $P_\omega$ indicates the path of the reduction that led from $M$ to the respective occurrence $I_\omega$ of $I$ or $\Omega_\omega$ of $\Omega$ from $M$ to its normal form ($l$ means ``left'' and $r$ means ``right'').For instance, in order to reach $I_{rll}$, i.e.\ the second occurrence of $I$ from the left in $M$, we have to take the right path in the outer $\oplus_p$ of $M$, then two times the left path in the new outer $\oplus_p$'s that we encounter during reduction.$P_{\omega}(p,q)$ gives then the probability (as a function of $p,q$) of obtaining the respective occurrence $I_{\omega}$ or $\Omega_\omega$ in the normal form, if we were to sample at each time we reduce a $\oplus_p$.It can thus also be read as the likelihood function of such event.The polynomials $Q_{r,2}(p,q)$ give instead the whole probability of obtaining respectively $I$ or $\Omega$, in the normal form after such samplings.
This way, the probabilistic evaluation of $M$ is presented as a \emph{hidden Markov model} \cite{Baumr966}, a fundamental statistical model, and notably one to which tropical methods are generally applied \cite{Pachter2ll4}.

Typical questions in this case would be, for a fixed $\omega_0$:
%
%The tropical point of view allows now to express two natural questions about this situation:
\begin{enumerate}
 \item Which is the \emph{maximum likelihood estimator} for the event ``$M\twoheadrightarrow \false_{\omega_0}$''?
 I.e., which is the choice of $p,q$ that maximizes the probability $P_{\omega_0}$ of the event ``$M\twoheadrightarrow \false_{\omega_0}$''  ?
 \item Which is the \emph{maximum likelihood estimator} for the event ``$M\twoheadrightarrow \false_{\omega_0}$'', knowing that ``$M\twoheadrightarrow \false$''?
I.e., which is the choice of $p,q$ that makes $\omega_0$ the most likely path among those leading to $\false$ (i.e.\ that maximizes the conditional probability $\BB P(``M\twoheadrightarrow \false_{\omega_0}'' \mid ``M\twoheadrightarrow \false'')$)?
\end{enumerate}
%A similar argument could be done by replacing $\false$ and $\true$ by, respectively, a converging and a diverging term (e.g.~in a $\B{PCF}$-style language), so r) would be about finding maximum likelihood estimators for the event ``$M$ converges''.

Answering 1) and 2) amounts at solving a maximization problem related to $P_{\omega_0}, Q_{\omega_0}$, which is more easily solved by 
passing to the tropical monomial/polynomial functions $\trop P_{\omega_0},\trop^{\mathrm{val}} Q_{\omega_0}$, for any fixed valuation.
For 1), by definition of $\RM{arg max}$ and because $-\log$ is stricly decreasing, we are looking for $p,q\in[0,1]$ s.t.\ $q=1-p$ and $(p,q)\in
%\begin{equation}
%  \begin{array}{ccccc}
   \RM{arg max}_{(x,y)} P_{\omega_0} (x,y)
   %& 
   = %&
   \RM{arg min}_{(x,y)}\set{-\log P_{\omega_0} (x,y)}
   %&
   =%&
   \RM{arg min}_{(x,y)}\set{(\trop P_{\omega_0}) (-\log x,-\log y)} \label{eq:argmax}$
where this holds for any valuation (because monomials do not carry any coefficient).
Remark that $(\trop P_{\omega})( -\log x, -\log y)$ is precisely the \emph{negative log-probability} of the event ``$M\twoheadrightarrow \false_{\omega}$'', so we see that the tropicalisation allows to compute such quantities.
%  \end{array}
%\end{equation}
For 2), %the $\omega_l$ is s.t.\ $\max\limits_{\omega\in\set{ll,rll,rrr}} \, P_\omega(x,y) = P_{\omega_l}(x,y)$. So
we are looking for $p,q\in[0,1]$ s.t.\ $q=1-p$ and
$\max_{\omega\in\set{lr,rrl,rlr}} \, P_\omega(p,q) = P_{\omega_0}(p,q)$, i.e.\ $\min_{\omega\in\set{lr,rrl,rlr}} \, -\log P_\omega(p,q) = -\log P_{\omega_0}(p,q)$.
Remembering that $\trop^{\mathrm{val}} Q_{\false}(p,q)=\min\set{p+q, \mathrm{val}(2)+p+2q}$, we see that our minimization problem is equivalent to asking $(\trop^{\mathrm{val}_1} Q_{\false})( -\log p, -\log q) = (\trop^{\mathrm{val}_1} P_{\omega_0})( -\log x, -\log y)\label{eq:max}$,
and this time this holds, in general, only for the trivial valuation.
Remark that, in both cases, passing through $\trop^0 $ %P_\omega, \trop Q_\omega$ 
makes the problem easier, as this amounts to study tropical polynomials (for instance computing tropical roots can be done in linear time \cite{Noferini2lr5}), and this essentially corresponds to study negative log-probabilities. %the tropicalisation operator $\trop{}$ as well as the \emph{negative $\log$-probabilities} appear.

%For our running example $M$, we have $\trop Q_{\true}(x,y)=\min\set{2x,y+2x,3y}$ and $\trop Q_{\false}(x,y)=\min\set{x+y,2y+x}$. Studying $\trop Q_{\true}$ %, whose plot is in Fig.~\ref{fig:plot2}, we see that $\trop Q_{\true}(x,y)=3y$ iff $y\leq \frac{2}{3}x$, and it coincides with $2x$ otherwise. Remembering that $3y=P_{rrr}(x,y)$, we can now solve the optimisation problem~\ref{eq:max} for $\omega_l=rrr$: via the substitution $x:=-\log p$, $y:=-\log (r-p)$, Equation~\ref{eq:max} is equivalent to $-\log (r-p)\leq -\frac{2}{3}\log_c p$, i.e.\ $r-p\geq p^{\frac{2}{3}}$. This means that, for $p\in[0,1]$ s.t.\ $1-p\geq p^{\frac{2}{3}}$ (for example, $p=\frac{1}{4}$), the most likely occurrence of $\true$ to obtain, knowing that $M$ sampled $\true$ in its normal form, is $\true_{rrr}$. Remembering that $2x=P_{ll}(x,y)$, for the other values of $p$ (for example, $p=\frac{1}{2}$), the most likely $\true$ to be sampled is the occurrence $\true_{ll}$. We have thus answered question 2) above for $\true$.

From this situation we notice the following important:

\begin{remark}\label{rmk:tropof01Rel}
Our toy first-order language can be interpreted both in the already mentioned $\overline{R_{\geq 0}}\mathrm{Rel}$ and in $\LREL$.
We do not give details now since in the rest of the paper we will interpret interesting \emph{high-order} calculi, including a probabilistic one containing our toy language.
But it is important to mention already at this point that the probabilities we just discussed above, are already captured by the $[0,\infty]\mathrm{Rel}$ model: the $\model{M}^{\overline{R_{\geq 0}}\mathrm{Rel}}\in[0,\infty]^{\set{0,1}}$ of our running example $M$ is $\model M_0=Q_{\false}(p,1-p)$, $\model M_1=Q_{\true}(p,1-p)$.
Therefore, the optimisation problems concerning the negative log-probabilities and likelihoods are already expressible by taking $\trop^{\mathrm{val}_1}\model{M}^{\overline{R_{\geq 0}}\mathrm{Rel}}$. 
Now, the model $\LREL$ is precisely trivial-valuation tropicalisation of $\overline{R_{\geq 0}}\mathrm{Rel}$, i.e.\ $\model{M}^{\LREL}=\trop^{\mathrm{val}_1}\model{M}^{\overline{R_{\geq 0}}\mathrm{Rel}}$.
As we remarked in \autoref{rmk:val_trop}, this corresponds to quotienting the polynomial interpreting $\model{M}^{\overline{R_{\geq 0}}\mathrm{Rel}}$ w.r.t.\ idempotent sum.
A precise study of the ``tropicalisation of $\overline{R_{\geq 0}}\mathrm{Rel}$'' is left for future investigations, as it is related with power series arising from more sophisticated calculi with paramethers (in the style of \cite{} {\color{red}Ehrhard!}).
For now, just have in mind that $\model{M}^{\LREL}_0(-\log p,-\log (1-p))$ gives the negative log-probability of \emph{any} of the (equiprobable) \emph{most likely} reduction paths to normal form, so $\LREL$ directly expresses such optimisations problems (no need to pass through a tropicalisation).
We take these observations as incouraging, and the point of this work is to dig into such model, concentrating for this first paper on the relations between the metric and differential properties that coexist in it.
% (in $p$ and $q$, not after the substitution $q:=r-p$) 
% are extracted from it, \ref{eq:argmax}, \ref{eq:max} of $\STLC_\oplus$-programs. 
\end{remark}

%Now, in $\LREL$, seen as a model of such probabilistic $\lam$-calculus, the interpretation of a term already computes the tropicalisation of the polynomials expressing the probabilities, because the underlying semiring of the model is tropical.
%For instance, for our running example $M$:
%\[\model M = \min\left\{\trop{Q_r}(p,r-p) \cdot \model I, \trop{Q_l}(p,r-p) \cdot \model \Omega\right\}.\]
