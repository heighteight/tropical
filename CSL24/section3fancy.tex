% !TEX root = main.tex
In this section we first recall a general and well-known construction that yields, for any \emph{continuous} semiring $Q$, a model $\QREL_{!}$ of effectful extensions of the simply typed $\lambda$-calculus and $\mathrm{PCF}$, and we show how, when $Q=\Lawv$, it captures optimal program behavior; moreover, we discuss how this model adapts to \emph{graded} and \emph{differential} variants of $\STLC$. 
%In the next sections we will instead focus on the peculiar features of the tropical case.

\subparagraph*{Linear/Non-Linear Algebra on $Q$-Modules}
For any \emph{continuous} semiring $Q$ (i.e.~a cpo equipped with an order-compatible semiring structure), one can define a category $\QREL$  (\cite{Manzo2013} calls it $Q^\Pi$) of ``$Q$-valued matrices'' as follows: $\QREL$ has sets as objects and set-indexed matrices with coefficients in $Q$ as morphisms, i.e.~$\QREL(X,Y):=Q^{X\times Y}$.
%The identity morphism of $\QREL$ is the identity matrix $I\in Q^{X\times X}$,
% given by $i_{a,a}=1$ and $i_{a,b\neq a}=0$, and
The composition $st\in Q^{X\times Z}$ of $t\in Q^{X\times Y}$ and $s\in Q^{Y\times Z}$ is given by $(st)_{a,c}:=\sum_{b\in Y} s_{b,c}t_{a,b}$ (observe that this series always converges because $Q$ is continuous).
For any set $X$, $Q^X$ is a $Q$-semimodule and we can identify $\HOM{\QREL}{X}{Y}$ with the set of linear maps from $Q^X$ to $Q^Y$, which have shape $f(x)_b:=\sum_{a\in X} \matr f_{a,b}x_a$, for some matrix $\matr f\in Q^{X\times Y}$.
% \begin{remark}
 %Following \cite{Manzo2013, Hofmann2014, Ehrhard2005}, we chose to see a matrix $t$ from $X$ to $Y$ as a map $t:X\times Y\to Q$.
% 
% fix $\HOM{\QREL}{X}{Y}:=Q^{X\times Y}$ with composition $st:X\times Z\to Q$ of $s:Y\times Z\to Q$ and $t:X\times Y\to Q$ defined by $(st)_{a,c}:=\sum_{b\in Y} s_{b,c}t_{a,b}$.
Notice that usual linear algebra conventions correspond to work in $\QREL^{\op}$, %a matrix $X\times Y\to Q$ is usually called a ``$Y\times X$-matrix'', meaning $Y$ rows and $X$ columns
e.g.\ the usual matrix-vector product defines a map $Q^Y\to Q^X$.
Following \cite{Manzo2013, Hofmann2014, Ehrhard2005}, we are instead working with transpose matrices.

$\QREL$ admits a comonad $!$ which acts on objects by taking the finite multisets.
Remember that the coKleisli category $\C C_!$ of a category $\C C$ w.r.t.\ a comonad $!$ is the category whose objects are the same of $\C C$, and $\HOM{\C C_!}{X}{Y}:=\HOM{\C C}{!X}{Y}$, with composition $\circ_!$ defined via the co-multiplication of $!$.
Now, although a matrix $t\in\HOM{\QREL_!}{X}{Y}$ yields a linear map $\Lawv^{!X}\to\Lawv^Y$, by exploiting the coKleisli structure we can also ``express it in the base $X$'', which leads to the \emph{non-linear} map $t^{!}:Q^{X}\to Q^{Y}$ defined by the power series
$
t^{!}(x)=t\circ_{!}x : b \mapsto \sum_{\mu\in !X}t_{\mu,b}\cdot x^{\mu}
$, 
where $x^{\mu}= \prod_{a\in x}x_{a}^{\mu(a)}$. 

%\end{remark}

When we instantiate $Q=\Lawv$, we obtain the category $\LREL$ of $\Lawv$-valued matrices. As one might expect, this category is tightly related to Lawvere's theory of (generalized) metric spaces. For the moment, let us just observe that a (possibly $\infty$) metric on a set $X$ is nothing but a ``$\Lawv$-valued square matrix'' $d:X\times X\to \Lawv$ satisfying axioms like e.g.~the triangular law.
We will come back to this viewpoint in \autoref{sec:GMS}.

Composition in $\LREL$ %is the tropicalisation of the one defining it in $\QREL$, i.e.
reads as \ $(st)_{a,c}:=\inf_{b\in Y}\set{s_{b,c}+t_{a,b}}$, and the non-linear maps $t^{!}: \Lawv^{X}\to \Lawv^{Y}$ have shape 
$t^!(x)_b=\inf_{\mu\in !X} \set{\mu x+ t_{\mu,b}}$, where $\mu x:=\sum_{a\in X} \mu(a)x_a$. These correspond to the generalisation of tps with \emph{possibly infinitely} many variables (in fact, as many as the elements of $X$).
By identifying $!\set{*}\simeq \N$ and $\Lawv^{\set{*}}\simeq\Lawv$, the tps generated by the morphisms in $\HOM{\LREL_!}{\set{*}}{\set{*}}$ are exactly the %functions $f:\Lawv\to\Lawv$ of shape $f(x)=\inf_{n\in\N}\set{nx+\matr f(n)}$, for some $\matr f:\N\to\Lawv$, i.e.\ we recover the 
usual tps's of one variable. %
%similarly, the linear functions $f:\Lawv^X\to \Lawv^Y$, %induced by matrices
%which we call \emph{tropical linear}, are exactly those of shape $f(x)_b=\inf_{a\in X} \set{\matr f_{a,b}+x_a}$, for some $\matr f\in\Lawv^{X\times Y}$.
%%\end{remark}
For example, the $\varphi$ of \autoref{fig:plot1} is indeed of shape $\varphi=t^!$, for $t\in\Lawv^{!\set{*}\times\set{*}}$, $t_{\mu,*}:=2^{-\# \mu}$.
% Therefore $\LREL_!$ is not a full-complete model of $\STLC$.

\begin{remark}
The operation $f\mapsto f^{!}$ turning a matrix into a function is reminiscent of the well-known operation of taking the 
 \emph{convex conjugate} $f^*$ of a function $f$ defined over a vector space (itself a generalization of the Legendre transformation).
 Indeed, let $X$, $Y$ be sets and let $\langle \_,\_\rangle:X\times Y \to \mathbb{R}$.
 For $f:X\to \mathbb R$, let $f^*:Y\to \mathbb R$ be defined by $f^*(y):= \sup_{x\in X}\{\langle x,y\rangle - f(x)\}$.
 Then for $X=!A$, $Y=\Lawv^A$, where $A$ is a set, and $\langle \mu, y \rangle:= \mu y$, we have $f^!(y)=(-f)^*(-y)$ for all $f\in\Lawv^{!A}$.
%Hence the definition  is the same formal construction yielding the \emph{convex conjugate} $f^*$ of a function $f$, by taking $X$ any vector space, $Y$ its dual space, and $\langle \_,\_\rangle$ the application bilinear form (acting as the scalar product on coordinates).
 %This construction is in turn a generalisation of the Legendre transformation.
% Despite the formal constructions being the same, we ignore for the moment if these could be connected to the study of high-order programs in our setting.
\end{remark}

\subparagraph*{The $\Lawv$-Weighted Relational Model}

The categories $\QREL$ are well-known to yield a model of both probabilistic and non-deterministic versions of $\mathrm{PCF}$ (see e.g.~\cite{Manzo2013, Pagani2018}), which are called \emph{weighted relational models}.
The interpretation of the simply typed $\lambda$-calculus $\STLC$ in $\LREL$ relies on the fact that all categories $\QREL_{!}$ are cartesian closed \cite{Manzo2013}, with cartesian product and exponential objects acting on objects $X,Y$ as, respectively, $X+Y$ and $!X\times Y$. 
Hence, %given an environment $\eta$ associating ground types with sets, 
any typable term $\Gamma \vdash M:A$ gives rise to a morphism 
$\model{\Gamma \vdash M:A}\in \LREL_{!}(\model{\Gamma}, \model{A})$, and thus to a generalized tps $\model{\Gamma \vdash M:A}^{!}:\Lawv^{\model{\Gamma}}\to \Lawv^{\model{A}}$. 
\begin{example}\label{ex:zxx}
The evaluation morphism $\mathsf{ev}\in\LREL_{!}((!X\times Y) + X, Y)$ yields the tps  
$\mathsf{ev}^{!}: \Lawv^{!X\times Y}\times \Lawv^{X}\to \Lawv^{Y}$ given by 
$b\in Y \mapsto \mathsf{ev}^{!}(F,x)_{b}= \inf_{\mu,b}\{ F_{\mu,b}+ \mu x\}$. So, for instance, supposing the ground type $o$ of $\STLC$ is interpreted as the singleton set $\{*\}$, and recalling the identification $!\{*\}\simeq \mathbb N$, the interpretation of the term $x:o, z:(o\to o\to o) \vdash zxx:o$, involving two consecutive evaluations, yields the tps $\varphi:\Lawv\times \Lawv^{\mathcal M_{\mathrm{fin}}(\mathbb N\times \mathbb N)}\to \Lawv$ given by $\varphi(x,z)=\inf_{n,n'}\{z_{[(n,n')]}+(n+n')x\}$. 
\end{example}
This interpretation extends to $\mathrm{PCF}$ by interpreting the fixpoint combinator $\mathbf Y$ via the matrices $\mathrm{fix}^{X}= \inf_{n}\big\{\mathrm{fix}^{X}_{n}\big\} \in \Lawv^{  !(!X\times X) \times X  }$, where 
 $\mathrm{fix}^{X}_{0}=0$ and $\mathrm{fix}^{X}_{n+1}= \RM{ev}\circ_{!}\langle \mathrm{fix}_{n}^{X}, \RM{id} \rangle $.
 
 


One can easily check, by induction on a typing derivation, that for any program of $\STLC$ or $\mathrm{PCF}$, the associated matrix is \emph{discrete}, that is, its values are included in $\{0,\infty\}$. Indeed, as suggested in Section \ref{section2}, the actual interest of tropical semantics lies in the interpretation of effectful programs. 
%
%In particular, the \emph{evaluation} morphisms $\RM{ev}$ in $\LREL$ are discrete matrices in $\Lawv^{!(!X\times Y)+X)\times Y}$ given by 
%$\RM{ev}_{[ (\mu,b) ,\mu ] , b }=0$, and being $\infty$ in all other cases, and the coKleisli composition of $s\in\Lawv^{!Y\times Z}$ and $t\in\Lawv^{!X\times Y}$ is the matrix $s\circ_! t\in\Lawv^{!X\times Z}$, $(s\circ_! t)_{\mu,c}:=
%%\begin{align}
%\inf_{n\in\N, b_1\dots,b_n\in Y, \mu = \mu_1+\cdots +\mu_n}
% \left\{s_{[b_1,\dots,b_n],c} + \sum_{i=1}^n t_{\mu_i,b_i}\right\}$.
% For example, setting $\model{A}=\{\star\}$, the interpretation of the $\lambda$-term $\lambda x.\lambda y.xy: (A\to A)\to (A\to A)$ yields the tps
% $\model{M}^{!}: \Lawv^{\mathbb N}\to \Lawv \to \Lawv$ (recalling that we can identify $\mathcal M_{\mathrm{fin}}(\{\star\})$ with $\mathbb N$) given by 
% $$
% \model{M}^{!}(x)(y)= \RM{ev}^{!}(\langle x,y\rangle)= \inf_{n\in \mathbb N}\big\{ x_{n} + ny\big\}.  
% $$
% Moreover,  
As the homsets $\LREL_{!}(X,Y)$ are $\Lawv$-modules, it is possible to interpret in it  extensions of $\STLC$ and $\mathrm{PCF}$ comprising $\Lawv$-module operations $\alpha \cdot M$ and $M+N$ \cite{Manzo2013}, by 
letting $\model{\Gamma \vdash \alpha\cdot M:A}=\model{\Gamma \vdash M:A}+\alpha$ and 
$\model{\Gamma \vdash M+N:A}=\min\{\model{\Gamma \vdash M:A},\model{\Gamma \vdash N:A}\}$. 
More precisely, \cite{Manzo2013} considers a language $\mathrm{PCF}^{Q}$ corresponding to $\mathrm{PCF}$ extended with $Q$-module operations, with an operational semantics given by rules $M\stackrel{1}{\to}M'$ for each rule $M\to M'$ of $\mathrm{PCF}$ (here $1$ is the monoidal unit of $Q$) as well as 
$M_{1}+M_{2} \stackrel{1}{\to}M_{i}$ and $\alpha\cdot M \stackrel{\alpha}{\to} M $.
Hence, any reduction $\omega=\rho_{1}\dots \rho_{k}: M \twoheadrightarrow N$ is naturally associated with a weight $\mathsf w(\omega)=\sum_{i=1}^{k}\mathsf w(\rho_{i})\in Q$.
In particular, from [Theorem V.6]\cite{Manzo2013} we deduce the following adequation result:
\begin{proposition}
$\model{\vdash_{\mathrm{PCF}^{\Lawv}} M:\mathrm{Nat}}_{n}=  \inf\big\{ \mathsf w(\omega) \ \big \vert \ \omega : M \to \underline n\big \}$ for all $n\in \mathbb N$.
\end{proposition}

The previous result allows to relate the tropical semantics of a program with its best-case operational behavior.  
Observe that the two examples shown in Section \ref{section2} can easily be rephrased in the language $\mathrm{PCF}^{\Lawv}$. 
For instance, for the probabilistic example, %for a given valuation $\RM{val}$, one can use a translation %$(M\oplus_{p}N)^{\RM{val}}= \min\{(M^{\RM{val}}+\RM{val}(p)), N^{\RM{val}}+\RM{val}(1-p)\}$, so that the reductions $:M\oplus_{p}N\stackrel{p}{\to} M$ and $M\oplus_{p}N\stackrel{p-1}{\to} N$ translate into a sequence of two reductions $(M\oplus_{p}N)^{\RM{val}} \stackrel{0}{\to} M^{\RM{val}}\stackrel{\RM{val}(p)}{\to} M$ and $(M\oplus_{p}N)^{\RM{val}} \stackrel{0}{\to} N^{\RM{val}}\stackrel{\RM{val}(1-p)}{\to} M$, with the weight of the latter corresponding to the valuation of the weight of the former. 
one can use the translation $(M\oplus_{p}N)^{\circ}= \min\{M^{\circ}+p, N^{\circ}+(1-p)\}$, so that the reductions $:M\oplus_{p}N\stackrel{p}{\to} M$ and $M\oplus_{p}N\stackrel{1-p}{\to} N$ translate into a sequence of two reductions $(M\oplus_{p}N)^{\circ} \stackrel{0}{\to} M^{\circ}\stackrel{p}{\to} M$ and $(M\oplus_{p}N)^{\circ} \stackrel{0}{\to} N^{\circ}\stackrel{1-p}{\to} N^{\circ}$. %, with the weight of the latter corresponding to the valuation of the weight of the former. 
Let $\mathrm{PPCF}$ (for \emph{probabilistic $\mathrm{PCF}$} \cite{Pagani2018}) be standard $\mathrm{PCF}$ extended with the constructor $M\oplus_{p}N$ ($p\in[0,1]$) and its associated reduction rules. From the above discussion we deduce the following:

\begin{corollary}
%For every closed probabilistic term $M$ of type $\mathrm{Nat}$ in $\mathrm{PPCF}$ and $n\in \mathbb N$, 
Let $\vdash_{\mathrm{PPCF}} M:\mathrm{Nat}$ and $n\in\N$.
Considering its interpretation as a function of $p,1-p$, we have that $\model{\vdash_{\mathrm{PPCF}} M^{\circ}:\mathrm{Nat}}_{n}(-\log(p),-\log(1-p))$ %\in \Lawv$ (cf.~Remark \ref{rmk:val_trop})
is the minimum negative log-probability of any reduction from $M$ to $\underline n$, i.e.\ the negative log-probability of (any of) the (equiprobable) most likely reduction path from $M$ to $n$.
\end{corollary}

Remark that this implies that all solution $p\in[0,1]$ to the equation $-\log \mathsf  w(\omega)= \model{\vdash_{\mathrm{PPCF}^{\Lawv}} M^{\circ}:\mathrm{Nat}}_{n}(-\log(p),-\log(1-p))$ are the values of the probabilistic parameter which make the reduction $\omega$ the most likely.

\begin{remark}
The function $\model{\vdash_{\mathrm{PPCF}} M^{\circ}:\mathrm{Nat}}_{n}(\alpha,\beta)$ is a tps, and Theorem \ref{theorem:fepsilon} ensures that this function coincides \emph{locally} with a tropical polynomial. This means that, for any choice of $p,1-p$, the most likely reduction path of $M$ can be searched for within a \emph{finite} space.
\end{remark}

Finally, \cite{Manzo2013} obtained a similar result for a non-deterministic version of PCF, by translating each term into $\mathrm{PCF}^{\Lawv}$ via $(\lambda x.M)^{\circ}=\lambda x.M^{\circ}+1$ and $(\mathbf YM)^{\circ}= \mathbf Y(M^{\circ}+1)$
(\cite{Manzo2013} considers the discrete tropical semiring $\mathbb N\cup\{\infty\}$, but the result obviously transports to $\Lawv$), and in that case \cite[Corollary VI.10]{Manzo2013} gives that $\model{\vdash M^{\circ}:\mathrm{Nat}}_{n}$ computes the minimum number of $\beta$- and $\mathsf{fix}$- redexes reduced in a reduction sequence from $M $ to $\underline n$. 

%\begin{corollary}[cf.~\cite{Manzo2013}, Corollary VI.10]
%For every closed probabilistic term $M$ of type $\mathrm{Nat}$ in non-deterministic $\mathrm{PCF}$ and $n\in \mathbb N$, $\model{M^{\circ}}_{n}$ is the minimum number of $\beta$- and $\mathsf{fix}$- redexes reduced in a reduction sequence from $M $ to $\underline n$. 
%\end{corollary}

%, an instance of the language $\mathrm{PCF}^{\mathcal R}$ of \cite{Manzo2013}, via the translations 
%$(M\oplus_{p}N)^{\circ} =pM^{\circ}\mathtt{or}(1-p)N^{\circ}$ and $(M+N)^{\circ}=M^{\circ}\mathtt{or}N^{\circ}$.
%In particular, 
 

\subparagraph*{$\mathbb N$-Graded Types}\label{sec:BSTLC}

We now show how to interpret in $\LREL_{!}$ a graded version of $\STLC$, that we call $\BSTLC$, indeed a simplified version of the well-studied language $\mathrm{Fuzz}$ \cite{Reed2010}.
This language is based on a graded exponential $!_{n}A$, corresponding to the possibility of using an element of type $A$ \emph{at most} $n$ times. In particular, if a function $ \lambda x.M$ of type $!_nA\multimap B$, then, for any $N$ of type $A$, $x$ is duplicated \emph{at most} $n$ times in any reduction of $(\lambda x.M)N $ to the normal form.

Graded simple types are defined by $A::= o \ \mid  \ !_{n}A \multimap A$; the contexts of the typing judgements are sets of declarations of the form $x :_{n}A$, with $n\in \mathbb N$;
given two contexts $\Gamma,\Delta$, we define $\Gamma+\Delta$ recursively as follows: if $\Gamma$ and $\Delta$ have no variable in common, then $\Gamma+\Delta=\Gamma\cup \Delta$; otherwise, we let $(\Gamma, x:_{m} A)+( \Delta, x:_{n} A) =  (\Gamma+\Delta), x:_{m+n}A$. 
Moreover, for any context $\Gamma$ and $m\in \mathbb N$, we let $m\Gamma$ be made all $x:_{mn}A$ for $(x:_{n}A) \in \Gamma$.  
The  typing rules of $\BSTLC$ are illustrated in Fig.~\ref{fig:rules}, 


%	\[ \scriptsize \arraycolsep=5pt\def\arraystretch{2.8}
\begin{figure}
\fbox{
	\begin{minipage}{0.9\textwidth}
	\begin{center}
	$\prooftree
	\justifies
	x:_{1}A\vdash x: A
	\endprooftree$
	
	\bigskip
	
	\begin{minipage}{0.42\textwidth}
	\begin{center}
	$\prooftree
		\Gamma \vdash M:A
		\justifies
		\Gamma, x:_{0}B \vdash M:A
		\endprooftree 
		$
		
	\bigskip
	
	$\prooftree
		\Gamma, x:_{n} A\vdash M: B
		\justifies
		\Gamma\vdash \lambda x.M: !_{n}A\multimap B
		\endprooftree$
		
		\medskip
				
	\end{center}
	\end{minipage} \ \ \ \ 
	\begin{minipage}{0.42\textwidth}
	\begin{center}
		
		$\prooftree
		\Gamma, x:_{n}B, y:_{m} B\vdash M:A
		\justifies
		\Gamma, x:_{n+m}B\vdash M\{x/y\}:A
		\endprooftree $
		
		\bigskip
		
		$\prooftree
		\Gamma \vdash M: !_nA\multimap B
		\quad
		\Delta\vdash N: A
		\justifies
		\Gamma +n\Delta\vdash MN: B
		\endprooftree
		 $
		 
		 \medskip
		 
	\end{center}
	\end{minipage}
	\end{center}
	\end{minipage}
	}
	\caption{Typing rules for $\BSTLC$.}
	\label{fig:rules}
	\end{figure}



Now, one can see that the comonad $!$ of $\LREL$ can be ``decomposed'' into a family of ``graded exponentials functors'' $!_n:\LREL\to\LREL$ ($n\in\BB N$), where $!_{n}X$ is the set of multisets on $X$ of cardinality \emph{at most} $n$. %  lift to functors 
The sequence $(!_n)_{n\in\N}$ gives rise to a so-called \emph{$\N$-graded linear exponential comonad} on (the SMC) $\LREL$ \cite{Katsumata2018}. %satisfying the adjunction: $\LREL(Z\otimes !_{n}X,Y) \simeq \LREL(Z, !_{n}X\multimap Y)$.
As such, $(\LREL,(!_n)_{n\in\N})$ yields then a model of $\BSTLC$. Remark that arrow types are interpreted via $\model{!_{n}A\multimap B}:= !_{n}\model A \times \model B$. Notice that, whenever $\model *$ is finite, the set $\model{A}$ is finite for any type $A$ of $\BSTLC$.


\subparagraph*{The Differential $\lambda$-Calculus}\label{sec:STDLC}

We recall the interpretation in $\LREL_{!}$ of the simply typed \emph{differential} $\lambda$-calculus $\STDLC$,  
 an extension of $\STLC$ ensuring exact control of duplications. The syntax of $\STDLC$ (see \cite[Section 3]{Manzo2010}) is made of \emph{terms} $M$ and \emph{sums} $\mathbb T$, mutually generated by: $M::= x\mid \lambda x.M \mid M\mathbb T \mid \Diff{M}{M}$ and $\mathbb T::= 0 \mid M \mid M+\mathbb T$,
quotiented by equations that make $+,0$ form a commutative monoid on the set of sums, %, i.e.\ commutativity and associativity of $+$ and neutrality of $0$ w.r.t.\ $+$;
by linearity of $\lam x.(\_)$, $\Diff{\_}{\_}$ and $(\_)\mathbb T$ (but \emph{not} of $M(\_)$) and by irrelevance of the order of consecutive $\Diff{\_}{\_}$.
%Remark that $M(\_)$ is \emph{not} set to be linear: $\lambda x.0=0\mathbb T=\Diff{0}{N}=\Diff{M}{0}=0$ but $M0\neq0$ in general.
%This is crucial for the definition of the Taylor expansion.
We follow the tradition of quotienting also for the idempotency of $+$.
%Sums are, then, just \emph{finite} sets of terms.
The typing rules are illustrated in Figure~\ref{fig:rules2}, where a context $\Gamma$ is a list of typed variable declarations.
%The axioms are $\Gamma, x:A \vdash x: A$ and $\Gamma\vdash 0:A$.
The main feature of this language is that $\Der^n[\lambda x.M,N^n]0$ has a non-zero normal form iff $x$ is duplicated exactly $n$ times during reduction.

\begin{figure}
\fbox{
	\begin{minipage}{0.9\textwidth}
	\begin{center}
	\begin{minipage}{0.42\textwidth}
	\begin{center}
	$\prooftree
	\justifies
	\Gamma\vdash 0:A
	\endprooftree$
	
	\bigskip
	
	$\prooftree
		\Gamma, x: A\vdash M: B
		\justifies
		\Gamma\vdash \lambda x.M: A\to B
		\endprooftree $
		
		\bigskip
		
		$ \prooftree
		\Gamma \vdash M: A\to B
		\quad
		\Gamma \vdash N: A
		\justifies
		\Gamma \vdash \Diff{M}{N}: A\to B
		\endprooftree$
		
		\medskip
		
	\end{center}
	\end{minipage} \ \ \ \ 
	\begin{minipage}{0.42\textwidth}
	\begin{center}
	$\prooftree 
	\justifies
	\Gamma, x:A \vdash x: A
	\endprooftree $
	
	\bigskip
	
	$\prooftree
		\Gamma \vdash M: A\to B
		\quad
		\Gamma\vdash \mathbb T: A
		\justifies
		\Gamma \vdash M\mathbb T: B
		\endprooftree $
		
		\bigskip
		
		$\prooftree
		\Gamma\vdash M_1: A
		\,\cdots\,
		\Gamma \vdash M_n:A
		\justifies
		\Gamma \vdash M_1+\cdots +M_n : A
		\using (n\geq 2)
		\endprooftree
		$
		
		\medskip
		
	\end{center}
	\end{minipage}
	\end{center}
	\end{minipage}
	}
	\caption{Typing rules of $\STDLC$.}
	\label{fig:rules2}
	\end{figure}

The categorical models of $\STDLC$ are called \emph{cartesian closed
 differential $\lambda$-categories} (CC$\partial\lambda$C)\cite{Manzo2010,Blute2009, Blute2019}. These are CCCs enriched over commutative monoids (i.e.\ morphisms are summable and there is a $0$ morphism), with the cartesian closed structure compatible with the additive structure, 
and equipped with a certain \emph{differential operator} $D$, turning a morphism $f:A\to B$ into a morphism $Df: A\times A\to B$, and 
generalising the usual notion of differential, see e.g.\ \cite{BluteEhrhTass10}.
An example is the CC$\partial\lambda$C of convenient vector spaces with smooth maps, where $D$ is the ``real'' differential of smooth maps.
%$Df:\mathbb{V}\times\mathbb{V}\rightarrow \mathbb{W}$, 
%$Df(x,u):=\dfrac{d}{dt}{\!\Big|_{t=0}} f(x+tu)$, of smooth maps $f:\mathbb{V}\rightarrow \mathbb{W}$.
%More precisely, a cartesian closed category $\C C$ is a $C\partial C$ when:
%\begin{itemize}
%\item $\C C$ is left-additive, i.e.~its hom-sets have the structure of commutative monoids, and the cartesian structure is well-behaved w.r.t.~this monoid structure;
%\item $\C C$ is equipped with a differential operator $D:
%\C C(X,Y)\to \C C(X\times X,Y)$ satisfying some axioms which capture usual properties of differentials (e.g.~the linearity of $D$ in one of its two variables, the chain rule, etc.).
%\end{itemize}

Applying \cite[Theorem 6.1]{lemay2020} one can check that
%\begin{proposition}[{\color{red}LEMAY??}]\label{thm:LREL!CCDC}
 $\LREL_!$ is a CC$\partial\lambda$C (see Section \ref{section5bis}) when equipped with $D:\HOM{\LREL}{!X}{Y}\to \HOM{\LREL}{!(X\& X)}{Y}$ defined as $(Dt)_{\mu\oplus\rho,b}=t_{\rho+\mu,b}$ if $\#\mu=1$ and as $\infty$ otherwise (using the iso $(\mu,\rho)\in !Z\times !Z'\mapsto\mu\oplus\rho \in !(Z+Z')$).
 The differential operator $D$ of $\LREL_{!}$ translates into a differential operator $D_{!}$ turing a tps $f:\Lawv^{X}\to \Lawv^{Y}$ into a tps $D_{!}f:\Lawv^{X}\times \Lawv^{X}\to \Lawv^{Y}$, linear in its first variable, and given by 
$D_{!}f(x,y)_{b}=\inf_{a\in X, \mu\in !X}\left\{\matr f_{\mu+a}+x_{a}+\mu y\right\}$. One can check that, when $f$ is a tropical polynomial, $D_{!}f$ coincides with the standard tropical derivative (see e.g.~\cite{Grigoriev2017}).
%Moreover, the Taylor formula \eqref{eq:taylorcat} yields a ``tropical'' Taylor formula for tps of the form 
%$f(x)=\inf_{n}\left\{D_{!}^{(n)}(f)(!_{n}x,\infty)\right\}$. 
%\end{proposition}
%\begin{remark}For $t\in\HOM{\LREL}{!X}{Y}$, we have: $D^2 t\in\HOM{\LREL_!}{(X+X)+(X+X)}{Y}$, where $(D^2 t)_{(\rho\oplus\rho')\oplus(\nu\oplus\nu'),b}$ equals $t_{\nu+\nu'+\rho',b}$ if $\rho=\emptyset$ and $\#\rho'=1=\#\nu$; it equals $t_{\rho+\nu',b}$ if $\rho'=\emptyset=\nu$ and $\#\rho=1$; it equals $\infty$ otherwise.\end{remark}
%This ensures that one can define a sound interpretation of $\STDLC$-terms in the standard way (see [Section 4.3, \cite{Manzo2010}]).
%As such, $(\LREL_!,D)$ is a model of the differential $\lambda$-calculus $\STDLC$ \cite{ER}.

%In $\BSTLC$ the typing system handles duplications; in $\STDLC$ the syntax with its operational semantics (that we do not give) does it.
\begin{comment}
Writing $\Der^2[\_,(\_)^2]$ as a shortcut for $\Der[\Der[\_,\_],\_]$ and $\Der^1[\_,(\_)^1]$ for $\Diff{\_}{\_}$, the analogue of the previous $\BSTLC$-term is $\vdash_{\STDLC} \lambda {\color{red}z}. \Der^{\color{red}2}[
	\lambda x{\color{green}y}.
		\Der^{\color{violet}1} [
				\Der^{\color{blue}1} [y, x^{\color{blue}1}]
        0, x^{\color{violet}1}
	]0
, z^{\color{red}2}]0
: {\color{red}*}\to ({\color{green}* \to * \to *}) \to *$.
%Here we wrote $\Der^2[\_]\cdot (\_)^2$ as a shortcut for $\Der[\Der[\_]\cdot (\_)]\cdot (\_)$ and $\Der^1[\_]\cdot (\_)^1$ for $\Der[\_]\cdot (\_)$.
In particular, if the \emph{multiplicities} of the arguments (the colored exponents) do not exactly match the number of duplications, e.g.\ in $\vdash_{\STDLC} \lambda z. \Der^{\color{red}3}[
	\lambda xy.
		\Diff{
				\Diff{y}{x}
		0}{x}
	0
, z^{\color{red}3}]0
: *\to (* \to * \to *) \to *$, then the term reduces to the empty sum $0$ (representing an \emph{error}).
\end{comment}
%Correspondingly, the syntax of the simply typed \emph{differential} $\lambda$-calculus ($\STDLC$) is defined by enriching $\STLC$ with a monoid structure $0,+$ over terms, as well as with $\Der$ and a notion of \emph{linear substitution} (see \cite{difflambda} or the Appendix for details).

%Until now we simply specialised well-known results in our tropical case, with the intent of showing how things read in this particular case.
%Now we go further, by showing that $\LREL_!$ actually admits a \emph{differential structure}, turning it into a model of the $\STDLC$, i.e.~a $CC\partial C$.
%This viewpoint
%, is where the \emph{metric} and the \emph{differential} viewpoints converge, as explained in the Introduction and Section II, and it % will be further generalised in Section \ref{section6}.

%A model of the $\STDLC$ is usually understood as so-called \emph{Cartesian closed differential categories} (CC$\partial$C), see \cite{Manzo2012} for details.
%In order to treat the $+$ and the constructor $D[\_]\cdot (\_)$ of $\STDLC$, the main features of a CC$\partial$C $\C C$ are that:
%
%1) $\C C$ is a left-additive-CCC, i.e.\ its Homsets are commutative monoids and its Cartesian closed structure is well behaved w.r.t.\ this monoid structure;
%
%2) $\C C$ is equipped with a differential operator map $D:\HOM{\C C}{X}{Y}\to \HOM{\C C}{X\times X}{Y}$ (here $\times$ is the Cartesian product of $\C C$) satisfying $8$ axioms, called D1, ..., D7, D-curry.

%Let us show the differential structure of $\LREL_!$ (remember that the Cartesian product of $\LREL_!$ is the disjoint union $+$).

%\begin{definition} The \emph{tropical differential operator} is the map $D:\HOM{\LREL}{!X}{Y}\to \HOM{\LREL}{!(X+X)}{Y}$ defined as $(Dt)_{\mu\oplus\rho,b}=t_{\rho+\mu,b}$ if $\#\mu=1$ and as $\infty$ otherwise (where a multiset $\nu \in !(X+X)$ is identified with a disjoint sum of $\mu,\rho\in !X$).\end{definition}

%The models of $\STDLC$ are the cartesian \emph{closed} differential categories ($CC\partial C$), which are defined as $C\partial C$ which are also cartesian closed, and in which the monoid structure and the differential operator are both well-behaved with respect to the closed structure \cite{Manzo2012}. 
%
%By induction on $M$ %(using that composition and $D$ easily preserve ``booleaness'' and projections and evaluation of  $\LREL$ are boolean)
%one can prove that:
%
%\begin{proposition}\label{prop:descrete}
% For $\BSTLC$ and $\STDLC$ (and thus also for its fragment $\STLC$), we have that $\model{\Gamma\vdash M:A}\in\Lawv^{!\model{\Gamma}\times\model{A}}$ is a \emph{boolean} matrix, i.e.\ actually $\model{\Gamma\vdash M:A}\in\{0,\infty\}^{!\model{\Gamma}\times\model{A}}$.
%\end{proposition}
%
%For example, the $\varphi$ of \autoref{fig:plot1} is not the interpretation of a $\lam$-term, because its matrix is not discrete.
%This can be seen as the fact that such calculi are relatively trivial, from this point of view.
%Even if the interpretation of those languages is trivia, we shall still focus on them in order to set the basis of the further studies and in order to see some already interesting properties of their interpretation.{\color{red}SPIEGARE MEGLIO}
%
%





