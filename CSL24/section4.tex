
In this section we establish some topological and metric properties of such functions, this way highlighting the rich and interesting topological and metric structure of the category $\LREL_{!}$.
In analogy with \cite{DanEhrh2011}, we could call them \emph{tropical entire functions}, but since, as we explain in a moment, those are generalised \emph{tropical Laurent series}, we will use the latter terminology, or \emph{tLs} for short.
The literature in tropical mathematics is often recent (e.g.~\cite{Porzio2021}), and several results we prove in this section are, to our knowledge, new.

Remark that the map $(\_)^!$ is a surjective homomorphism (w.r.t.\ coKleisli composition and identity and function composition and identiy) on its image. {\color{red}In Ehrhard-Danos fanno vedere (Lemma 14) che \`e anche iniettivo. Non riesco a capire se anche a noi lo \`e, ma non vedo perch\'e non dovrebbe.}

\begin{remark}
 Since $\model{(x_i:A_i)_{i=1}^n\vdash_{\STLC} M:B}\in\HOM{\LREL_!}{\mathlarger{\&}_{i=1}^n \model{A_i}}{\model B}\simeq\Lawv^{\prod_{i=1}^n ! \model {A_i}\times \model B}$, the tLs $\model{(x_i:A_i)_{i=1}^n\vdash_{\STLC} M:B}^!:\Lawv^{\mathlarger{+}_{i=1}^n \model{A_i}}\to\Lawv^{\model B} \simeq \prod_{i=1}^n\Lawv^{\model{A_i}}\to\Lawv^{\model B}$ is defined by $t^!(x^1,\dots,x^n)_b:=\inf_{\mu_i\in ! \model{A_i}}\left\{\sum_{i=1}^n\mu_i x^i+t_{(\mu_1,\dots,\mu_n),b}\right\}$.
Same for $\STDLC$.
Finally, the interpretations of $\BSTLC$-terms are matrices $t\in\HOM{\LREL}{\bigotimes_{i=1}^n !_{n_i}X_i}{Y}=\Lawv^{\prod_{i=1}^n !_{n_i}X_i\times Y}$.
In such situation, we define $t^!:\Lawv^{\mathlarger{+}_{i=1}^n X_{i}}\to\Lawv^Y \simeq \prod_{i=1}^n\Lawv^{X_{i}}\to\Lawv^Y$ as $t^!(x^1,\dots,x^n)_b:=\inf_{\mu_i\in !_{n_i} X_i}\left\{\sum_{i=1}^n\mu_i x^i+t_{(\mu_1,\dots,\mu_n),b}\right\}$.
Clearly $t^!={\widetilde t}^!$, where $\widetilde t\in\Lawv^{\prod_{i=1}^n ! {X_i}\times Y}$ is the matrix $\widetilde t_{(\mu_1,\dots,\mu_n),b}:=t_{(\mu_1,\dots,\mu_n),b}$ if $\mu_i\in !_{n_i} X_i$ for all $i$, and $:=\infty$ otherwise, which has the same support as $t$.
So, such $t^!$'s are special cases of tLs and thus, \emph{Wlog}, in the following we only consider tLs.
\end{remark}

%
%In this section we study the tropical Laurent series $\Lawv^X\to \Lawv ^Y$ from the viewpoint of analysis.


 \begin{remark}
 Our running example is indeed of shape $\varphi=t^!$, for $t\in\Lawv^{!\set{*}\times\set{*}}$, $t_{\mu,*}:=2^{-\# \mu}$.
 However, it is not the interpretation of a $\lam$-term, because its matrix $t$ is not discrete.
% Therefore $\LREL_!$ is not a full-complete model of $\STLC$.
\end{remark}

If $\model *$ is finite, then the interpretations of $\BSTLC$-terms is a (generalised) tropical polynomial, since then the interpretation of any $\BSTLC$-type $B$ is a finite set, and thus also $!_n \model B$ is.
% 
% , i.e.~for a \emph{finite} set $\C F\subseteq \, !X$.
%%Remark that we also find usual \emph{tropical polynomials} of tropical geometry as a particular case: they 
%correspond to the tLs for which the support $\set{n\in\N\mid\widehat f(n)\neq\infty}$ of $\widehat f$ is \emph{finite}. 
%Actually, for us \emph{tropical polynomial} will mean a function 
%\end{remark}

Looking at Fig~\ref{fig:plot1}, we see that $\varphi$, just like the polynomials $\varphi_{n}$, is non-decreasing and concave.
This is indeed always the case:

\begin{proposition}\label{prop:nondecr+conc}
 Any tLs $f:\Lawv^X\to\Lawv^Y$ is non-decreasing and concave, w.r.t.\ the pointwise order.
\end{proposition}

%In Example~\ref{ex:famous_ex}, 
Again by looking at Fig~\ref{fig:plot1} it appears that, \emph{far from $0$}, $\varphi$ behaves like some polynomials $\varphi_{n}$.
In particular, %for all $\epsilon >0$, $f$ coincides on $[\epsilon,\infty]$ with some polynomial $\varphi_{n}$. More, precisely, 
$\varphi$ coincides on $[\epsilon,\infty]$ with $\varphi_{n}$,
for $\epsilon \geq 2^{-(n+1)}$ (the smallest tropical root of $\varphi_{n}$).
However, at
%
%$x\in [\epsilon,\infty]$, with $\epsilon>0$
%It can be proven by hand that $\varphi(x)$ is a $\min$ for all $x>0$.
 $x=0$ we have that $\varphi(x=0)=\inf_{n\in\N} 2^{-n}=0$, and this is the only point where the $\inf$ is \emph{not} a $\min$.
Also, while the derivative of $f$ is bounded on all $(0,\infty)$, for $x\to 0^+$ it tends to $\infty$.
This phenomenon is reminiscent of \cite[Example 7]{Ehrhard2005},
%Differentials and Distances in Probabilistic Coherence Spaces. FSCD 2019
which actually motivated our first investigations.
In fact, this behaviour is shared by all tLs with \emph{finitely} many variables, as shown by the following result (we identify $!\set{1,\dots,k}\simeq \N^k$, so the matrix of a tLs $f$ with finitely many variables $x=x_1,\dots,x_k$ (and one output variable) is given as a $\matr f:\N^k\to\Lawv$, $f$ having shape $f(x)=\inf_{n\in \N^k}\set{nx+\matr f(n)}$, with $nx$ the scalar product).

\begin{theorem}\label{theorem:fepsilon}
 Let $k\in\N$ and $f:\Lawv^k\to\Lawv$ a tLs with matrix $\matr f:\N^k\to\Lawv$.
 For all $0<\epsilon<\infty$, there is a \emph{finite} $\C F_\epsilon \subseteq \N^k$ such that 
% 
% \begin{enumerate}
%  \item If $\C F_\epsilon=\emptyset$ then $f=\infty$ on all $\Lawv^k$;
%  \item If $f(x_0)=\infty$ for some $x_0\in[0,\infty)^k$, then $\C F_\epsilon=\emptyset$;
  %\item 
$f$ coincides on all $[\epsilon,\infty]^k$ with the tropical \emph{polynomial} $P_\epsilon(x):=\min_{n\in \C F_\epsilon}\set{nx+\matr f(n)}$.
% \end{enumerate}
\end{theorem}
\begin{proof}[Proof sketch]
Let $\C F_\epsilon$ be the set of $n\in\N^{k}$ such that 
$\widehat f(n)<\infty$ and $\widehat f(m)> \widehat f(n)+\epsilon$ holds for all $m\prec n$, where $\preceq$ is the pointwise order on $\N^k$.
The core of the proof is showing that this set is indeed finite and enough for computing $f$.
\end{proof}




\subsection{Continuity of tLs}\label{subsec:cont}%$\Lawv^{X}$ as a normed cone}

The tLs $\varphi$ is continuous on $\BB R_{\geq0}$ (w.r.t.\ the usual norm of real numbers).
By considering the usual norm $\norm{x}_\infty:=\sup_{a\in X} x_a$ on $\Lawv^X$, we can generalise this property, dropping the case of $x$ having some $0$ coordinate:

\begin{theorem}\label{thm:cont}
 All tLs $f:\Lawv^X\to\Lawv$ are continuous on $(0,\infty)^X$, w.r.t.\ to the norm $\norm{\cdot}_\infty$.
\end{theorem}
\begin{proof}
%It follows after adapting \cite[Proposition 4.4]{Cobzas2017} in order to prove that if a real-valued function on a locally convex topological $\BB R$-vector space is, locally around $x$, concave and bounded by a finite constant, then it is continuous at $x$.
\cite[Proposition 4.4.(3)]{Cobzas2017} shows that if a real-valued convex function with domain a convex open subset of a locally convex topological $\BB R$-vector space is, locally around any point, bounded from above by a finite non-zero constant, then it is continuous on all its domain.
Now, our $f$ is concave, so $-f$ is convex.
Also, since $f\geq 0$ on all $\Lawv^X$, we have e.g.\ $-f\leq 1$ on $\Lawv^X$.
%Remark that we cannot immediately apply \cite[Proposition 4.4.(3)]{Cobzas2017}, since $\Lawv^X$ is not even a $\BB R$-vector space.
Now $(0,\infty)^X\subseteq \BB R^X$ is open and convex, and {\color{red}$\BB R^X$ is a locally convex topological $\BB R$-vector space}, so \cite[Proposition 4.4.(3)]{Cobzas2017} entails the continuity of $-f\big|_{(0,\infty)^X}$, hence that of $f\big|_{(0,\infty)^X}$.
\end{proof}

We conclude this subsection by noticing that $\Lawv^X$ with the usual $+$ and the usual $\cdot$ is a $\BB R_{\geq0}$-semimodule.
Together with the norm $\norm{\cdot}_\infty$, it can be proved that it is a Scott-complete \emph{normed cone} (see~\cite{Selinger2004}, or the appendix, for such notions).
Its cone structure induces an order on it, called its \emph{cone order}:
$x\leq y$ iff $y=x+z$ for some (unique) $z\in\Lawv^X$.
Such order actually coincides with the pointwise order on $\Lawv^X$.
This makes it a Scott-continuous dcpo.
Suitable categories of cones have been recently investigated as models of probabilistic computation (\cite{Crubillie2018, EhrPagTas2018, Ehrhard2020}).
In analogy with \cite[Proposition 17]{DanEhrh2011}, we have:

\begin{theorem}\label{thm:ScottCont}
 All monotone (w.r.t.\ pointwise order) and $\norm{\cdot}_{\infty}$-continuous functions $f:(0,\infty)^X\to (0,\infty)$ are Scott-continuous.
 In particular, all tLs $f:\Lawv^X\to\Lawv$ are Scott-continuous on $(0,\infty)^X$ w.r.t.\ the pointwise orders.
\end{theorem}
\begin{proof}
 Use the fact, taken from \cite{Selinger2004}, that in a normed $\R_{\geq 0}$-cone $P$, considered with its cone-order, if every bounded directed net in $P$ admits a sup, and if $(v_i)_{i\in I}$ is a directed net in $P$ with an upper bound $v\in P$, then $\exists\bigvee_{i\in I} v_i \in P$ and, if $\inf_{i\in I} \norm{v-v_i} =0$, one has $\bigvee_{i\in I} v_i = v$.
\end{proof}



\subsection{Lipschitz-continuity of tLs}\label{sec:4C}%$\Lawv^{X}$ as a metric space.}


%The norm $\norm{\cdot}_\infty$ naturally induces a metric $\norm{x-y}_{\infty}$ over the spaces $\Lawv^{X}$.
We will show that tLs satisfy suitable Lipschitz properties w.r.t.\ $\norm{\cdot}_\infty$. 

Let us first look at tropical linear functions:


\begin{proposition}\label{prop:troplinear}
All tropical \emph{linear} functions $f: \Lawv^{X}\to \Lawv^{Y}$ are non-expansive.  
\end{proposition}
%\begin{proof}[Proof sketch]
%Using the fact that $f(\B x)_{b}= \inf_{a\in X}\matr f_{a,b}+\B x_{a}$,
%the problem reduces to checking that $|(\matr f_{a,b}-\B x_{a})- (\matr f_{a,b}-\B y_{a})| = |\B x_{a}-\B y_{a}|\leq \| \B x-\B y\|_{\infty}$.\end{proof}
This result shows that, in analogy with that happens in usual metric semantics, linear programs are interpreted by non-expansive functions. 
%\begin{proof}
%Using $f(\B x)_{b}= \inf_{a\in X}\matr f_{a,b}+\B x_{a}$,
%first observe that $|(\matr f_{a,b}-\B x_{a})- (\matr f_{a,b}-\B y_{a})| = |\B x_{a}-\B y_{a}|\leq \| \B x-\B y\|_{\infty}$; we now have
%$|f(\B x)_{b}-f(\B y)_{b}| \leq |(\inf_{a\in X}\matr f_{a,b}-\B x_{a})-(\inf_{a\in X}\matr f_{a,b}-\B y_{a})| \leq
%\sup_{a\in X}|(\matr f_{a,b}-\B x_{a})- (\matr f_{a,b}-\B y_{a})|\leq 
% \| \B x-\B y\|_{\infty}$.
%\end{proof}

%Before looking at what happens in the case of non-linear programs, let us make the metric structure of $\LREL$ explicit. 
The following proposition provides a useful characterization of the functional metrics in $\LREL$, relying on 
the bijection between $\HOM{\LREL}{X}{Y}$ and set of tropical linear functions from $\Lawv^{X}$ to $\Lawv^{Y}$.

\begin{proposition}
For all tropical linear functions $f,g:\Lawv^{X}\to \Lawv^{Y}$, $d_{\infty}(\matr f,\matr g)=d_\infty(f,g)$.% $\norm{ \matr f-\matr g}_{\infty} =  \sup_{x\in \Lawv^{X}} \norm{ f( x)-g(x)}_{\infty}$.
\end{proposition}

Let us now consider the case of bounded exponentials:
\begin{proposition}\label{prop:boundedlip}
If a tLs $f: \Lawv^{X}\to \Lawv^{Y}$ arises from a matrix $\matr f:!_{n}X\times Y\to \Lawv$, then $f$ is $n$-Lipschitz-continuous.
\end{proposition}
\begin{proof}[Proof sketch]
This follows from Proposition \ref{prop:troplinear} and the remark that, for all $x\in \Lawv^{X}$, $\norm{ !_{n} x-!_{n} y}_{\infty}\leq n\cdot \norm{ x- y}_{\infty}$, where $!_{n} x$ is the restriction of $! x$ to $\C M_{\leq n}(X)$.%
%Using the fact that $f(\B x)_{b}=\inf_{\mu\in \C M_{\leq n}(X)}\{ \matr f_{\mu,b}+ \mu (!_{n}\B x) \}$, where $!_{n}\B x\in \Lawv^{\C M_{\leq n}(X)}$ is given by 
%$(!_{n}\B x)_{[a_{1},\dots, a_{k}]}=\sum_{i=1}^{k}\B x_{a_{i}}$, 
%it suffices to check that $\| (!_{n}\B x)-(!_{n}\B y)\|_{\infty}\leq n\cdot \| \B x-\B y\|_{\infty}$ and apply Proposition \ref{prop:troplinear}.
\end{proof}
This result is perfectly analogous to what happens in the metric models discussed in Section \ref{section2}, the bounded exponentials $!_{n}$ playing the role of the re-scaling trick.

Observe that, for any tropical polynomial $\varphi:\Lawv^{X}\to \Lawv^{Y}$, the associated matrix has shape $!_{\mathrm{deg}(\varphi)}(X)\times Y\to \Lawv$ (as a monomial $\mu_ix+c_{i}$ yields a matrix entry on $!_{\#\mu_i}X\times Y$). Hence, using Proposition \ref{prop:boundedlip}, we have:
\begin{corollary}\label{prop:polylip}
For any tropical polynomial $\varphi:\Lawv^{X}\to\Lawv$, $\varphi$ is $\mathrm{deg}(\varphi)$-Lipschitz continuous.
\end{corollary}

Let us now look at what happens with tLs, i.e.~when considering the full exponential $!$.
As consequence of Theorem~\ref{theorem:fepsilon}, the tLs with \emph{finitely many} variables are always \emph{locally} Lipschitz on all $\BB R_{>0}$.
Actually, we can prove a more general statement, also covering the infinitary case.


\begin{theorem}\label{thmTLSlocLip}
 All tLs $\Lawv^X\to\Lawv$ are locally Lipschitz on $\BB R_{>0}^X$.
\end{theorem}
\begin{proof}[Proof sketch]
The core of the proof is a convex analysis argument (see the Appendix) showing that an arbitrary function $f:\Lawv^X\to\Lawv$ which is non-decreasing, concave and continuous, must be locally Lipschitz. 
\end{proof}


Finally, let us discuss the differential structure. The differential operator $D$ of $\LREL_{!}$ translates into a differential operator $D_{!}$ turing a tLs $f:\Lawv^{X}\to \Lawv^{Y}$ into a tLs $D_{!}f:\Lawv^{X}\times \Lawv^{X}\to \Lawv^{Y}$, linear in its first variable, and given by 
\begin{equation}
D_{!}f(x,y)_{b}=\inf_{a\in X, \mu\in !X}\left\{\matr f_{\mu+a}+x_{a}+\mu y\right\}
\end{equation}
One can check that, when $f$ is a tropical polynomial, $D_{!}f$ coincides with the standard tropical derivative (see e.g.~\cite{Grigoriev2017}).
Moreover, the Taylor formula \eqref{eq:taylorcat} yields a ``tropical'' Taylor formula for tLs of the form 
\begin{equation}
f(x)=\inf_{n}\left\{D_{!}^{(n)}(f)(!_{n}x,\infty)\right\}
\end{equation}
The following result shows that the distance between two tropical maps can be approximated using the terms appearing in their Taylor expansions:
\begin{proposition}
For all tLs $f,g: \Lawv^{X}\to \Lawv^{Y}$, and for all $n\in \BB N$, 
the functions $x\mapsto D_{!}^{(n)}(f)(!_{n}x,\infty)$ are $n$-Lipschitz. Moreover 
$\norm{ \matr f-\matr g}_{\infty}= \sup_{n} \norm{ {\delta^{(n)}f}- {\delta^{(n)}g}}_{\infty}$, 
where $\delta^{(n)}h$ indicates the matrix of $D_{!}^{(n)}h$.
%where $\delta^{(n)}f:( \Lawv^{X})^{n}\to \Lawv^{X}$ is the tropical linear function $\delta^{(n)}f(\B x_{1},\dots, \B x_{n})=
%(\Der^{(n)}f)(\B x_{1},\dots, \B x_{n}, \infty)$. 
\end{proposition} 


The results just presented translate into the following facts about the interpretation of higher-order programs:

\begin{corollary}
Let $\model A$ be a finite set.
\begin{enumerate}
\item $\model{\Gamma \vdash_{\BSTLC} M:A}^!:\Lawv^{\model\Gamma} \to \Lawv^{\model A}$ is a \emph{tropical polynomial}, thus (as $\model A$ is finite), a \emph{Lipschitz} function.
\item $\model{\Gamma \vdash_{\STLC} M:A}^!:\Lawv^{\model\Gamma} \to \Lawv^{\model A}$ is a \emph{locally} Lipschitz map.
\item $\Te{M}$ decomposes $\model{\Gamma \vdash_{\STLC} M:A}^!$ as an $\inf_{t\in\Te{M}}\model{\Gamma\vdash_{\STDLC} t:A}^!$ of \emph{tropical polynomials}, thus (as $\model A$ is finite), \emph{Lipschitz} functions.
\end{enumerate}
\end{corollary} 
\begin{proof}
1). Since $\model A$ is finite, also $\model *$ is.
Thus, as we already observed, the interpretation of a bounded term is a tropical polynomial.
Now we apply Corollary~\ref{prop:polylip} to each coordinate of the image, and by taking the maximum Lipschitz constant among the finite number $\mathrm{Card}(\model A)$ of them, we obtain the thesis.
2). It follows immediately from Theorem~\ref{thmTLSlocLip} and the fact that $\model A$ is finite.
3). It follows from \autoref{cor:T(M)=M} plus the easily checked fact that, for $(f_n)_{n\in\N}\subseteq\Lawv^{!X\times Y}$, we have $\left(\inf_{n\in\N} f_n\right)^!:\Lawv^X\to\Lawv^Y$, with $\left(\inf_{n\in\N} f_n\right)^!=\inf_{n\in\N} f_n^!$.
\end{proof}

Remark that the restriction $\model A$ finite is without loss of generality, since by Currying all programs can be seen having type $*$, which is natural to interpret as a singleton.

A consequence of (3) is that the pointwise distance between two interpretations of programs can always be bounded via Lipschitz tropical polynomial approximants of the initial two programs.
\begin{corollary}
 Let $\Gamma \vdash_{\STLC} M:A$ and $\Delta\vdash_{\STLC} N:B$.
 For all $\epsilon>0$, $x\in\Lawv^{\model \Gamma}$, $b\in\model A$, there exist $t\in\Te{M}$, $u\in\Te{N}$ s.t.\ $\big| \model{\Gamma \vdash_{\STLC} M:A}^!(x)_b - \model{\Delta \vdash_{\STLC} N:B}^!(x)_b \big| \leq 2\epsilon + \big| \model{\Gamma \vdash_{\STDLC} t:A}^!(x)_b - \model{\Delta \vdash_{\STDLC} u:B}^!(x)_b \big|$.
\end{corollary}

\begin{remark}
 Let $X$, $Y$ be sets and let $\langle \_,\_\rangle:X\times Y \to \mathbb{R}$.
 For $f:X\to \mathbb R$, define $f^*:Y\to \mathbb R$ by $f^*(y):= \sup_{x\in X}\{\langle x,y\rangle - f(x)\}$.
 Then for $X=!A$, $Y=\Lawv^A$, where $A$ is a set, and $\langle \mu, y \rangle:= \mu y$, we have that $f^!=(-f)^*$ for all $f\in\Lawv^{!A}$.
 This is precisely the same formal construction yielding the well-known convex conjugate $f^*$ of a function $f$, by taking $X$ any vector space, $Y$ its dual space, and $\langle \_,\_\rangle$ the application bilinear form (acting as the scalar product on coordinates).
 This construction is in turn a generalisation of the Legendre transformation.
 Despite the formal constructions being the same, we ignore for the moment if these could be connected to the study of high-order programs in our setting.
\end{remark}
