\paragraph*{Prohibited duplication/erasure: the \emph{linear} $\STLC$}\label{sec:3A}

$\LREL$ is a Symmetric Monoidal Closed Category (SMCC) \cite[Section III.A]{Manzo2013}, %(actually, even a model of IMLL), 
whose monoidal product $\otimes$ acts on objects as the Cartesian product of sets, %and as the \emph{Kronecker product} of matrices on morphisms.
 %{(\color{red} Ref?)} 
 and its internal hom $X\multimap Y$ is also given by the Cartesian product $X\times Y$ of sets, with the usual evaluation and curry maps.
As such, it is able to interpret the \emph{linear simply typed $\lam$-calculus}, the restriction of the ordinary $\lam$-calculus in which programs ,% following the linear logic mantra, 
can only use their arguments \emph{exactly once}. %(and indeed this calculus can be embedded in its \emph{intuitionistic multiplicative} fragment \emph{IMLL}).
%More precisely, the linear $\lam$-calculus is obtained from the ordinary one by adding the constraint that each $\lam$-abstracted variable appears exactly once in the scope of the $\lam$-abstraction.
%In order to frame it in a category $\C C$, one needs a symmetric tensor product $\otimes$ together with internal hom-set objects $X\multimap Y$ s.t.\ the evaluation and curry maps yield a \emph{symmetric monoidal adjunction}: $\C C(Z\otimes X, Y) \simeq \C C (Z, X\multimap Y)$.

%\subsubsection{Unbounded duplications}

\paragraph*{Allowed duplication/erasure: the $\STLC$}\label{sec:STLC}

\begin{comment}

In order to interpret the full $\STLC$, we need a Cartesian closed category (CCC).
It is well-known \cite{Mellies2009} that it is always possible to construct a CCC by taking the \emph{co-Kleisli} $\C C_!$ of a so-called \emph{Lafont category} $\C C$.
%, a construction we now quickly recall.
%defined via a \emph{Lafont exponential} comonad $!$.
%Let us quickly recall the ideas behind these notions.
A SMCC is Lafont when it has finite products and it is equipped with a comonad $!$ (its \emph{Lafont exponential}) which, at level of objects, sends $X$ to an object $!X$ being the free commutative comonoid on $X$.
Such objects $!X$ represent the \emph{bang} connective of linear logic, granting infinite duplications via the infinite product $X^0\otimes X\otimes X^2\otimes X^3\otimes\cdots$, each factor representing a possible number of duplications.
It is well-known that, under mild conditions satisfied by $\QREL$, one can explicit this idea via the fact that the map $X\mapsto \finMS{X}$ (where $\finMS{X}$ is the set of finite multi-sets on $X$) lifts to a functor $!:\QREL\to\QREL$ which is a Lafont-exponential comonad.
Specializing [Corollary III.6, \cite{Manzo2013}] to our case, we have:

\begin{proposition}
 $\LREL$ is Lafont.
\end{proposition}

\end{comment}

As well-known, if a SMCC is \emph{Lafont}, then one obtains a CCC from it by defining the exponential objects as $X\to Y:=!X \multimap Y$.
With no surprise $\LREL$ is Lafont~\cite[Corollary III.6]{Manzo2013} %and specialising \cite[Theorem III.7]{Manzo2013} 
w.r.t.\ the usual $!$ acting on objects by taking the set of finite multisets.
So the coKleisli $\LREL_!$ is CCC, i.e.\ a model of $\STLC$.
The cartesian product $\&$ of $\LREL_!$ (and of $\LREL$) is the disjoint union $+$ of sets, the \emph{evaluation} morphisms $\RM{ev}$ are matrices in $\Lawv^{!(!X\times Y)+X)\times Y}$, and the coKleisli composition of $s\in\Lawv^{!Y\times Z}$ and $t\in\Lawv^{!X\times Y}$ is the matrix $s\circ_! t\in\Lawv^{!X\times Z}$, $(s\circ_! t)_{\mu,c}:=
%\begin{align}
\inf_{n\in\N, b_1\dots,b_n\in Y, \mu = \mu_1+\cdots +\mu_n}
 \left\{s_{[b_1,\dots,b_n],c} + \sum_{i=1}^n t_{\mu_i,b_i}\right\}$.
