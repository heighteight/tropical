\subsubsection{Concavity}


A function $f:Q^{X}\to Q^{Y}$ is \emph{concave} if for all $\alpha\in [0,1]$, $\B x, \B y\in Q^{X}$ and $b\in Y$ 
$$
f(\alpha\cdot \B x+(1-\alpha)\cdot \B y)_{b} \geq \alpha f(\B x)_{b} + (1-\alpha)f(\B y)_{b}
$$



\begin{proposition}
All tropical functions $f: Q^{X}\to Q^{Y}$ are concave.
\end{proposition}
\begin{proof}
Let us first show that all functions of the form $f(\B x)_{b}= \mu \B x+ c$ are concave:
we have $f(\alpha\B x+ (1-\alpha)\B y)_{b}= \mu(\alpha\B x)+(1-\alpha)\B y)+c=
 \mu(\alpha\B x)+(1-\alpha)\B y)+\alpha c+(1-\alpha)c=
 \alpha(\mu \B x + c)+(1-\alpha)(\mu \B y+c)=\alpha f(\B x)_{b}+(1-\alpha) f(\B x)_{b}$.


To conclude, let us show that if $(f_{i})_{i\in I}$ is a family of concave functions from $Q^{X}$ to $Q^{Y}$, the function $f=\inf_{i\in I}f_{i}$ is also concave: we have
$f(\alpha\B x +(1-\alpha)\B y)_{b}=
\inf_{i\in I}f_{i}(\alpha\B x+(1-\alpha)\B y)_{b} \geq 
\inf_{i\in I}\alpha f_{i}(\B x)_{b}+(1-\alpha)f_{i}(\B y)_{b}
\geq 
\inf_{i\in I}\alpha f_{i}(\B x)_{b} + \inf_{j\in I}(1-\alpha)f_{j}(\B y)_{b}
=
\alpha \cdot (\inf_{i\in I}f_{i}(\B x)_{b})+ (1-\alpha)\cdot( \inf_{j\in I}f_{j}(\B y)_{b})=
\alpha  f(\B x)_{b}+(1-\alpha)f(\B y)_{b}$, where we used the fact that given families $a_{i},b_{i}$ of reals,
$\inf_{i}a_{i}+b_{i}\geq \inf_{i}a_{i}+\inf_{j}b_{j}$.
This follows from the fact that for all $i\in I$, $a_{i}+b_{i}\geq \inf_{i}a_{i}+\inf_{i}b_{i}$.
\end{proof}



Concavity is a useful property, as it leads to establish the following:
\begin{proposition}
For any tropical function $f: Q^{X}\to Q^{Y}$, 
\begin{enumerate}
\item $ \alpha f(\B x)_{b}\leq  f(\alpha\B x)_{b}$;
\item $f(\B x+\B y)_{b}\leq f(\B x)_{b}+f(\B y)_{b}$;
\item $f(\B x+\B y)_{b} \leq f(\B x)_{b} + \phi_{\mu,b}(\B y)$ 
for all $\mu\in \C M_{\mathrm{fin}}(X)$.

\item if $\B x < \B y < \B z$, then 
$$
\frac{f(\B y)-f(\B x)}{|\B y-\B x|} \leq \frac{f(\B z)-f(\B x)}{|\B z-\B x|}
\leq \frac{f(\B z)-f(\B y)}{|\B z-\B y|}
$$

\item the function $R_{b}(\B x, \B y)= \frac{f(\B x)_{b}-f(\B y)_{b}}{| \B x- \B y|}$ is monotonically non-decreasing in $\B x$ (for fixed $\B y$) and in $\B y$ (for fixed $\B x$).

\end{enumerate}
\end{proposition}
\begin{proof}
Since $f$ is concave, we have that for all $\B z$, 
$f(\alpha\cdot\B z)=f(\alpha\cdot\B z+(1-\alpha)\cdot 0)\geq 
\alpha f(\B z)+(1-\alpha)f(0)\geq \alpha f(\B z)$. 

For the second one, 
if $f(\B x)=\mu \B x+c$, then $f(\B x+\B y)= \mu \B x+ \mu \B y + c  \leq
\mu \B x + \mu \B y + 2c = f(\B x)+f(\B y)$.
Moreover, if $(f_{i})_{i\in I}$ is a family of functions such that $f_{i}(\B x+\B y)\leq f_{i}(\B x)+f_{i}(\B y)$, then 
$f=\inf _{i\in I}f_{i}$ also satisfies this property: 
one has $f(\B x+\B y)=\inf_{i\in I}f_{i}(\B x+\B y) \geq \inf_{i} f_{i}(\B x)+f_{i}(\B y) \geq \inf_{i}f_{i}(\B x) + \inf_{i}f_{i}(\B y)=f(\B x)+f(\B y)$.
%Now we can compute
%\begin{align*}
%f(\B x+\B y)_{b}& =
%
%\end{align*}

The third one follows immediately from the second one, since $f(\B y)_{b}\leq \phi_{\mu,b}(\B y)$ for all $\mu\in \C M_{\mathrm{fin}}(X)$.













\end{proof}


\subsubsection{Proof of Theorem \ref{theorem:fepsilon}}




DAVIDE's NOTES!


\subsubsection{Continuity of tLs}




Continuity of cones and metric. DAVIDE'S NOTES
!






\subsubsection{Lipschitz continuity}







\begin{proposition}\label{prop:troplinear}
All tropical linear functions $f: \Lawv^{X}\to \Lawv^{Y}$ are non-expansive.  
\end{proposition}
\begin{proof}[proof TO BE FIXED]
All functions of the form $f(\B x)(b)= \mu_{b} \B x + c_{b}$ are Lipschitz with Lipschitz constant
$\sum_{a\in X}\mu_{b}(a)$: 
one has $| f(\B x)(b)- f(\B y)(b) |= |\mu_{b}\B x+c_{b} -\mu_{b}\B y+c_{b}|=
|\mu_{b}\B x -\mu_{b} \B y  |\leq \left(\sum_{a\in X}\mu_{b}(a)\right)\cdot \sup_{a\in X}|\B x_{a}-\B y_{a}|$.

\end{proof}






\begin{lemma}
Let $f: [0,\infty)^{X}\to Q$ be concave, monotone increasing, and continuous.  
Let $\B x\neq \B y\in [0,\infty)^{X}$, with $\|\B x-\B y\|_{\infty}<\infty$, and let $S(\B x, \B y)=\{ \alpha\B x + (1-\alpha)\B y\mid \alpha \in [0,1]\}$ be the segment generated by $\B x$ and $\B y$. 
Then $f$ is Lipschitz-continuous over $S(\B x, \B y)$.

\end{lemma}
\begin{proof}
Let us prove the lemma under the assumption that for all $a\in X$, $\B y_{a}-\B x_{a}\geq 1$. From the fact that the claim holds under the assumption, we can deduce the claim of the lemma:
indeed for $\alpha\in (0,1)$ large enough we have that $\B y':= \frac{\B y-\alpha\B x}{1-\alpha}$ is such that $\B y\in S(\B x, \B y')$ (and thus $S(\B x, \B y)\subseteq S(\B x, \B y')$) and 
$\B y'_{a}-\B x_{a}\geq 1$. Hence from our proof we deduce that $f$ is Lipschitz-continuous over $S(\B x, \B y')$, and thus a fortiori over $S(\B x, \B y)$ too.


Since $f$ is continuous over $[0,\infty)^{X}$ and $S(\B x, \B y)$ is compact, $f$ admits a maximum $\mathrm{MAX}$ over $S(\B x, \B y)$.
For all $\B z< \B z'\in S(\B x, \B y)$, let $M(\B z, \B z')\in Q^{X}$ be defined by
$$
M(\B z, \B z')_{a}= \frac{f(\B z')-f(\B z)}{\B z'_{a}-\B z_{a}}
$$
Observe that
\begin{align*}
M(\B x, \B y)_{a} & = \frac{f(\B y)-f(\B x)}{\B y_{a}-\B x_{a}} \leq 
f(\B y)-f(\B x) \leq \mathrm{MAX}
\end{align*}
using the fact that $\B y_{a}-\B x_{a}\geq 1$. 

We now claim that $M(\B z, \B z')$ is contravariant in both $\B z$ and $\B z'$. 
Indeed suppose $\B z\leq \B z'' < \B z'$, so that $\B z= \lambda \B z' +(1-\lambda)\B z''$ for some $\lambda \in (0,1)$. Then, using the fact that $f$ is concave, we have 
\begin{align*}
M(\B z, \B z')_{a}&=\frac{f(\B z')-f(\lambda \B x +(1-\lambda)\B z'')}{\B z'_{a}-\lambda \B x_{a} -(1-\lambda)\B z''_{a}} \\
&
\geq 
\frac{f(\B z')-\lambda f(\B z') -(1-\lambda)f(\B z'')}{\B z'_{a}-\lambda \B z'_{a} -(1-\lambda)\B z''_{a}} \\
&
= 
\frac{(1-\lambda)(f(\B z')-f(\B z''))}{(1-\lambda )\B z'_{a}-\B z''_{a}} =M(\B z'', \B z')
\end{align*}
In a similar way it is shown that for $\B z < \B z''\leq \B z'$, $M(\B z,\B z')\leq M(\B z, \B z'')$. 

Therefore, for all $\B z < \B z'\in S(\B x, \B y)$, we have that $M(\B z, \B z')_{a} \leq M(\B x, \B z')_{a} \leq M(\B x, \B y)_{a} \leq \mathrm{MAX}$. 
From this, using the fact that $f$ is monotone increasing, we deduce that 
$|f(\B z')-f(\B z)|=f(\B z')-f(\B z) \leq \mathrm{MAX}\cdot |\B z'_{a}-\B z_{a}|
$ and thus that 
$$|f(\B z')-f(\B z)|\leq \mathrm{MAX}\cdot \|\B z'-\B z\|_{\infty}$$
that is, that $f$ is $\mathrm{MAX}$-Lipschitz over $S(\B x, \B y)$. 
\end{proof}

\begin{proposition}
Let $f: [0,\infty)^{X}\to Q$ be concave, monotone increasing, and continuous.  
For all $\epsilon \in (0,\infty)$ and $\B x\in [0,\infty)^{X}$, $f$ is Lipschitz-continuous over the open ball $B(\B x, \epsilon)$.
\end{proposition}
\begin{proof}
Let $\mathrm{MAX}$ indicate the maximum of $f$ over $B(\B x, \epsilon)$. 
Let $\B y, \B z\in B(\B x, \epsilon)$; then $\|\B y-\B z\|_{\infty}\leq 2\epsilon < \infty$, so by the lemma above $f$ is $K$-Lipschitz over the segment $S(\B y, \B z)$ for some $K\leq \mathrm{MAX}$, so we deduce $|f(\B y)-f(\B z)|\leq \mathrm{MAX}\cdot \|\B y-\B z\|_{\infty}$.
\end{proof}

\begin{theorem}[local Lipschitz-continuity]
Let $f: [0,\infty)^{X}\to Q$ be concave, monotone increasing, and continuous.  
Then $f$ is locally Lipschitz-continuous.
\end{theorem}
\begin{proof}
For all $\B x\in [0,\infty)^{X}$, $f$ is Lipschitz-continuous over the open set $B(\B x, 1)$.
\end{proof}



\subsubsection{Characterizations of the functional metric }



  \begin{proposition}\label{prop:functionalmetric}
For all maps $f,g: Q\langle X\rangle\to Q\langle Y\rangle$,  
$$
  \|\widehat f- \widehat g\|_{\infty}=
  \sup\{\|f(\B x)- g(\B x)\|_{\infty} \mid \B x\in Q\langle X\rangle\}
$$

\end{proposition}
  
Before proving the proposition we need one preliminary lemma:
  
\begin{lemma}
Let $u,v: I\to Q$ and suppose $|u(i)-v(i)|\leq \delta$, for all $i\in I$.
Then $|\inf_{i\in I}u(i)- \inf_{i\in I}v(i)|\leq \delta$. 
\end{lemma}
\begin{proof}
Let $A=\inf_{i\in I}u(i)$ and $B=\inf_{i\in I}v(i)$ and suppose $A\geq B$. 
Suppose by way of contradiction $|A-B|> \delta$; then there exists $i\in I$ such that 
$v(i)<A$ and $|A-v(i)|> \delta$. Indeed, otherwise we would have $|A-B|= \sup\{ |A-v(i)|\mid v(i)\leq A\}\leq \delta$. 
Now, from $|A-v(i)|> \delta$ and $v(i)< A$ we deduce that $|u(j)-v(i)|> \delta$ for all $j\in I$, and thus in particular that $|u(i)-v(i)|>\delta$, against the assumption. We conclude then $|A-B| \leq \delta$.
In case $B\geq A$, we can argue in a similar way. 
\end{proof}

%
%\begin{lemma}
%Let $f,g\in\mathsf{Trop}(1,1)$ (i.e.~$f,g:Q\to Q$), and suppose for all $n\in \BB N$, 
%$|\widehat f_{n}-\widehat g_{n}|\leq \delta$. Then for all $x\in Q$, $|f(x)-g(x)|\leq \delta$.
%
%\end{lemma}
%\begin{proof}
%$|f(x)-g(x)|= |( \inf_{n}nx+\widehat f_{n})- (\inf_{n}nx+\widehat g_{n})|$. 
%Since $|nx+\widehat f_{n} - nx -\widehat g_{n}|= |\widehat f_{n}-\widehat g_{n}|\leq \delta$, by the Lemma above we conclude $|f(x)-g(x)|\leq \delta$. 
%\end{proof}



  
  \begin{proof}[Proof of Proposition \ref{prop:functionalmetric}]
For one side, suppose for all $\mu\in \C  M_{\mathrm{fin}}(X),b\in Y$, 
$|\widehat f_{\mu,b}-\widehat g_{\mu,b}|\leq \delta$. Then for all $\B x\in Q\langle X\rangle$ and $b\in Y$, $|f(\B x)(b)-g(\B x)(b)|\leq \delta$.
Indeed, $|f(\B x)(b)-g(\B x)(b)|= |( \inf_{\mu}\mu \B x+\widehat f_{\mu,b})- (\inf_{\mu}\mu \B x+\widehat g_{\mu,b})|$. 
Since $|\mu \B x+\widehat f_{\mu,b} - \mu \B x -\widehat g_{\mu,b}|= |\widehat f_{\mu,b}-\widehat g_{\mu,b}|\leq \delta$, by the Lemma above we conclude $|f(\B x)(b)-g(\B x)(b)|\leq \delta$. 


For the other side, suppose for some $\mu\in \C M_{\mathrm f}(X)$ and $b\in Y$, $|\widehat f_{\mu,b}-\widehat g_{\mu,b}|> \epsilon$; 
then, by letting $e_{\mu}\in Q\langle !X\rangle$ be the (tropical) characteristic function of $\mu$, we have
 $|f(e_{\mu})({b})-g(e_{\mu})({b})| =
|( \inf_{\mu'}\widehat f_{\mu',b}+ \mu'(e_{\mu}))-
( \inf_{\mu'}\widehat g_{\mu',b}+\mu'(e_{\mu}))|=
|\widehat f_{\mu,b}- \widehat g_{\mu,b}|> \epsilon$, so we deduce that 
$\sup\{\| f(\B x)- g(\B x) \|_{\infty}\mid \B x\in Q\langle X\rangle\}> \epsilon$.
\end{proof}


The second characterization relates distances with the Taylor expansion. Let us briefly discuss the latter, first.
For all $f: Q\langle X\rangle \to Q\langle Y\rangle$, let $\delta^{(n)}f:Q\langle X^{n}\rangle \to Q\langle Y\rangle$ indicate the $n$-linear function given by 
$$
\delta^{(n)}f(\B x^{n})=\mathsf D^{(n)}f(\B x^{n}, \infty)
$$

Notice that $\widehat{\delta^{(n)}f}\in Q^{X^{n}\times Y}$ satisfies 
$\widehat{\delta^{(n)}f}_{a_{1},\dots, a_{n},b}= \widehat f_{[a_{1},\dots, a_{n}],b}$.
In other words, $\delta^{(n)}f$ precisely captures the behavior of $f$ when applied to multisets of length $n$. Moreover, the full behavior of $f$ can be recovered from the functions $\delta^{(n)}f$ using the Taylor expansion which, in its tropical form, reads as:
\begin{equation}
f(\B x)= \inf_{n}\left \{\mathsf D^{(n)}f(\B x^{n},\infty)\right\}= \inf_{n}\left \{\delta^{(n)}f(\B x^{n})\right\}
\tag{Tropical Taylor}
\end{equation}





\begin{proposition}\label{prop:taylormetric}
For all $f,g:Q\langle X\rangle \to Q\langle Y\rangle$, 
$$
\| \widehat f-\widehat g\|_{\infty}= \sup_{n}\| \widehat{\delta^{(n)}f} -\widehat{\delta^{(n)}g}\|_{\infty}
$$
\end{proposition}
\begin{proof}
Let us first show that $\| \widehat f-\widehat g\|_{\infty}\leq \epsilon$ implies 
$\| \delta^{(n)}f-\delta^{(n)}g\|_{\infty}\leq\epsilon$, for all $n\in \BB N$. 
Notice that $\widehat{\delta^{(n)}f}\in Q^{X^{n}\times Y}$ satisfies 
$\widehat{\delta^{(n)}f}_{a_{1},\dots, a_{n},b}= \widehat f_{[a_{1},\dots, a_{n}],b}$. Hence 
from $\| \widehat f-\widehat g\|\leq\epsilon$ it follows that for all $\mu=[a_{1},\dots, a_{n}]$ and $b\in Y$, 
$|\widehat f_{\mu,b}- \widehat g_{\mu,b}|\leq \epsilon$, so 
we deduce that $\| \widehat{\delta^{(n)}f}- \widehat{\delta^{(n)}g}\|\leq \epsilon$.

For the converse direction suppose that, for all $n\in \BB N$, $\| \widehat{\delta^{(n)}f}- \widehat{\delta^{(n)}g}\|\leq \epsilon_{n}$. Then, since the family of coefficients $
F_{n,\mu,b}=
( \widehat{\delta^{(n)}f})_{a_{1},\dots, a_{n},b}$ (where $\mu=[a_{1},\dots, a_{n}]$) is in bijection with the coefficients $\widehat f_{\mu,b}$, we deduce that $|\widehat f_{\mu,b}-\widehat g_{\mu,b}|\leq 
\epsilon_{\sharp \mu  }$, and thus that 
$\| \widehat f-\widehat g\|_{\infty}\leq \sup_{n}\epsilon_{n}$. 
\end{proof}

