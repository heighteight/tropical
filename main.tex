\documentclass[conference]{IEEEtran}
\IEEEoverridecommandlockouts
% The preceding line is only needed to identify funding in the first footnote. If that is unneeded, please comment it out.
\usepackage{cite}
\usepackage{amsmath,amssymb,amsfonts}
\usepackage{algorithmic}
\usepackage{graphicx}
\usepackage{tikz-cd}
\usepackage{textcomp}
\usepackage{xcolor}
\usepackage{amsthm}
\usetikzlibrary{cd}
\usepackage{stmaryrd}



% MATH TEXT STYLES
\newcommand{\B}[1]{\mathbf{#1}}
\newcommand{\BB}[1]{\mathbb{#1}}
\newcommand{\C}[1]{\mathcal{#1}}
\newcommand{\F}[1]{\mathfrak{#1}}
\newcommand{\TT}[1]{\mathtt{#1}}
\newcommand{\RM}[1]{\mathrm{#1}}
\newcommand{\SF}[1]{\mathsf{#1}}



%CATEGORIES

\newcommand{\Met}{\mathsf{Met}}
\newcommand{\Mod}{\Lawv\mathsf{Mod}}
\newcommand{\GMet}{\Lawv\mathsf{CCat}}
\newcommand{\Fun}{\mathsf{Fun}}
\newcommand{\colim}{\mathrm{colim}}
\newcommand{\Yon}{\B{Y}}
\newcommand{\Hom}{\mathrm{Hom}}
\newcommand{\Sym}{\mathrm{Sym}}
\newcommand{\matr}[1]{\hat{#1}}

\newcommand\pfun{\mathrel{\ooalign{\hfil$\mapstochar\mkern5mu$\hfil\cr$\to$\cr}}}



% LAMBDA CALCULI

\newcommand{\lamcalc}{$\lambda$-calculus}
\newcommand{\lam}{\lambda}

\newcommand{\STLC}{\RM{STLC}}
\newcommand{\BSTLC}{\mathsf b\RM{STLC}}
\newcommand{\RSTLC}{\C R\RM{STLC}}
\newcommand{\STDLC}{\RM{ST}\partial\RM{LC}}
\newcommand{\Real}{\SF{Real}}

\newcommand{\Der}{\SF D}
\newcommand{\To}{\Rightarrow}

\newcommand{\finMS}[1]{\C M_{\RM{fin}}(#1)}

\newcommand{\true}{\RM{True}}
\newcommand{\false}{\RM{False}}
\newcommand{\bool}{\SF{Bool}}

\newcommand{\Te}[1]{\C T(#1)}




% METRIC STUFF
\newcommand{\Lawv}{\BB L}
\newcommand{\QualREL}[1]{#1 \SF{Rel}}
\newcommand{\QREL}{\QualREL{Q}}
\newcommand{\LREL}{\QualREL{\Lawv}}
\newcommand{\LCAT}{\Lawv\SF{CCat}}

\newcommand{\op}{\mathrm{op}} 
\newcommand{\sk}{\mathrm{sk}} 
\newcommand{\sym}{\mathrm{sym}} 
\newcommand{\menus}{\dotdiv} 

\newcommand{\norm}[1]{\lVert#1\rVert}
\newcommand{\supnorm}[1]{\lVert#1\rVert_\infty}
\newcommand{\absv}[1]{\left\lvert#1\right\rvert}

% TROPICAL STUFF

\newcommand{\trop}[1]{\SF t #1}
\newcommand{\model}[1]{\llbracket#1\rrbracket}
\newcommand{\nodel}[1]{\langle #1\rangle}
\newcommand{\sumt}[1]{{+}^{#1}}
\newcommand{\prodt}[1]{{\times}^{#1}}


% MISCELLANEOUS

\newcommand{\HOM}[3]{{#1}(#2,#3)}
\newcommand{\N}{\BB N}
\newcommand{\R}{\BB R}
\newcommand{\set}[1]{\{#1\}}
\newcommand{\multiset}{\C M_{\mathrm{fin}}}

\newcommand{\eps}{\epsilon}

\newcommand{\twoheaddownarrow}{\mathrel{\rotatebox[origin=c]{270}{$\twoheadrightarrow$}}\!}


%MATH EVIRONMENTS

\newtheorem{example}{Example}
\newtheorem{definition}{Definition}
\newtheorem{problem}{Problem}
\newtheorem{notation}{Notation}

\newtheorem{remark}{Remark}
\newtheorem{theorem}{Theorem}
\newtheorem{conjecture}{Conjecture}

\newtheorem{lemma}[theorem]{Lemma}

\newtheorem{proposition}[theorem]{Proposition}
\newtheorem{result}{Result}
\newtheorem{fact}{Fact}


\newtheorem{corollary}[theorem]{Corollary}



\def\BibTeX{{\rm B\kern-.05em{\sc i\kern-.025em b}\kern-.08em
    T\kern-.1667em\lower.7ex\hbox{E}\kern-.125emX}}
\begin{document}

\title{Lambda Calculus through Tropical Mathematics}

\author{\IEEEauthorblockN{Davide Barbarossa}
\IEEEauthorblockA{\textit{Universit\`a di Bologna}\\
Bologna, Italy \\
davide.barbarossa@unibo.it}
\and
\IEEEauthorblockN{Paolo Pistone}
\IEEEauthorblockA{\textit{Universit\`a Roma Tre}\\
Rome, Italy\\
paolo.pistone@uniroma3.it}
}

\maketitle

\begin{abstract}

\end{abstract}

\begin{IEEEkeywords}
\end{IEEEkeywords}

\section{Introduction}


In recent years, more and more interest in the programming language community has been directed towards the study of \emph{quantitative} properties of programs like e.g.~computing the number of computation steps, or convergence probabilities, 
as opposed to purely \emph{qualitative} properties like e.g.~termination or program equivalence. 
Notably, a significant effort has been made to extend, or adapt, well-established qualitative methods, like e.g.~type systems, relational logics or denotational semantics, to account for quantitative properties. We can mention, for example, 
intersection type systems aimed at capturing time or space resources \cite{decarvalho2018, Accattoli2022} or convergence probabilities \cite{Breuvart2018, PistoneLICS2022},  relational logics to account for probabilistic properties like e.g.~differential privacy \cite{Barthe_2012} or metric preservation \cite{Reed2010, dallago}, as well as the study of denotational models for 
probabilistic \cite{Ehrhard2011, Staton2017} or differential \cite{difflambda} extensions of the $\lambda$-calculus. 
The main reason to look for methods relying on (quantitative extensions of) type-theory or denotational semantics is that these approaches yield \emph{modular} and \emph{compositional} techniques, that is, allow one to deduce properties of complex programs from the properties of their constituent parts.   

\subsection{Two kinds of quantitative approaches}

Among such quantitative approaches, two different directions have received considerable attention. 

On the one hand one there is the approach of \emph{program metrics} \cite{Reed2010, Gaboardi2017, Gabo2019} and \emph{quantitative equational theories} \cite{Plotk}: when considering probabilistic or approximate computation, rather than asking whether two programs compute \emph{the same} function (which is rarely the case), it makes more sense to ask   whether they compute functions which do not differ \emph{too much}. This has motivated the study of denotational frameworks in which types are endowed with a metric, measuring similarity of behavior; this approach has found  applications in e.g.~differential privacy \cite{Reed2010} and coinductive methods \cite{Bonchi2018}, and was recently extended to account for the full $\lambda$-calculus \cite{Geoffroy2020, PistoneLICS, PistoneFSCD2022}.

On the other hand, there is the approach based on \emph{differential} \cite{difflambda} or \emph{resource-aware} \cite{Boudol1993} extensions of the $\lambda$-calculus, which is well-connected to the so-called \emph{relational semantics} \cite{Manzo2012, Manzo2013, dill} and has a syntactic counterpart in the study of \emph{non-idempotent} intersection types \cite{decarvalho2018, Mazza2016}. This family of approaches have been exploited to account for higher-order program differentiation \cite{difflambda}, to establish reasonable \emph{cost-models} for the $\lambda$-calculus \cite{Accattoli2021}, and have also been shown suitable for the probabilistic setting \cite{Manzo2013, Breuvart2018, PistoneLICS2022}. 


In both approaches the notion of \emph{linearity}, in the sense of linear logic \cite{girardLl} (i.e.~of using inputs exactly once), plays a crucial role.
In metric semantics, linear programs correspond to \emph{non-expansive} maps, that is, to functions that do not increase distances; moreover, the possibility of duplicating inputs leads to interpret \emph{bounded} programs (i.e.~programs with a fixed duplication bound) as \emph{Lipschitz-continuous} maps \cite{Gaboardi2017}.
By contrast, in the standard semantics of the differential $\lambda$-calculus, linear programs correspond to linear maps, in the usual algebraic sense, while the possibility of duplicating inputs leads to consider functions defined as \emph{power series}.


A natural question is thus whether these two apparently unrelated ways of interpreting linearity and duplication can be somehow reconciled. At a first glance, there seems to be a  ``logarithmic'' gap between the two approaches:
in metric models $n$ times duplication results in a \emph{linear} (hence Lipschitz) function $n\cdot x$, while in differential models this results in a \emph{polynomial} function $x^{n}$, hence not Lipschitz. The fundamental motivation of this work is then the observation that 
this gap is naturally overcome once we interpret these functions in the framework of tropical mathematics, where, as we'll see, the monomial $x^{n}$ ``reads as'' the linear function $n\cdot x$.

% from higher-order programs is based on  soon as one develops  differential semantics in the framework of 
%tropical mathematics.
%
%''
%
%s
%emantics a typical ``duplicating'' map is obtained by composing the diagonal with multiplication:
%$$
%\begin{tikzcd}
%\mathbb R \ar{rrr}{x\mapsto \langle x, x\rangle}
% & &  &
% \mathbb R\times \mathbb R 
% \ar{rrr}{\langle x,y\rangle \mapsto x\cdot y}
% & & & \mathbb R
%\end{tikzcd}
%$$
%yielding the square product function $\lambda x.x^{2}$.
%However, in metric semantics this function needs not even exist (as these models are often restricted to Lipschitz-continuous maps \cite{Gabo2017})! Instead, a typical ``duplicating'' map can be obtained by composing the diagonal with the sum 
%$$
%\begin{tikzcd}
%\mathbb R \ar{rrr}{x\mapsto \langle x, x\rangle}
% & &  &
% \mathbb R\times \mathbb R 
% \ar{rrr}{\langle x,y\rangle \mapsto x+y}
% & & & \mathbb R
%\end{tikzcd}
%$$
%yielding the linear (and Lipschitz) function $\lambda x.2x$.
%
%As this example seems to suggest, there seems to be a sort of ``logarithmic'' gap between the two approaches. Can this be made explicit?



\subsection{Tropical mathematics and program semantics } 


Tropical mathematics was introduced in the seventies by the Brazilian mathematician Imre Simon \cite{Simon} as an alternative approach to algebra and geometry where the usual ring structure of numbers based on addition and multiplication is replaced by the semiring structure given, respectively, by ``$\min$'' and ``$+$''.
%
%
% interpreting the usual ``$\times$'' and  ``$+$'' operations by  ``$+$'' by ``$\min$''. It can thus be seen as a sort of ``logarithmic'' version of usual geometry (this idea can be made precise via the so-called \emph{Maslov deformation} \cite{}).
%Tropical mathematics is a form of \emph{idempotent} mathematics, since the role of addition is 
%played by the idempotent operation $\min$.
For instance, the polynomial $p(x,y)=x^{2}+xy^{2}+y^{3}$, when interpreted over the tropical semiring, translates as the piecewise linear function
$
\varphi(x,y)=\min\{2x, x+2y, 3y\}
$.

%This is not a \emph{ad-hoc} setting: 
In the last decades, tropical geometry evolved into a vast and rich research domain, providing a combinatorial counterpart of usual algebraic geometry, with important connections with optimisation theory \cite{Sturmfelds}.
Computationally speaking, working with $\min$ and $+$ is generally easier than working with standard addition and multiplication; for instance, the fundamental (and generally intractable) problem of finding the roots of a polynomial admits a \emph{linear time} algorithm in the tropical case (and, moreover,  the tropical roots can be used to approximate the actual roots \cite{Noferini2015}).
The computational nature of tropical notions explains why these are so widely applied in computer science, notably for convex analysis and machine learning (see \cite{Maragos2021} for a recent survey).

Coming back to our previous discussion on program semantics, tropical geometry might seem to provide precisely what we are looking for, as it turns the monomials $x^{n}$ into the Lipschitz functions $n\cdot x$.
At this point, it is worth mentioning that a tropical variant of relational semantics has already been considered \cite{Manzo2013}, and shown capable of capturing \emph{best-case} quantitative properties, but has not yet been studied in detail. Furthermore, connections between tropical linear algebra and metric spaces (in the abstract setting of \emph{quantale-enriched} categories \cite{Hofmann2014, Stubbe2014}), have also been observed \cite{Fuji}.

In this paper we demonstrate that the relational interpretation of the $\lambda$-calculus based on tropical mathematics does indeed provides the desired bridge between differential and metric semantics. Moreover, we show that the conceptual unification of these two approaches suggests ways in which techniques from resource-analysis could be used in sensitivity analysis and \emph{vice-versa}, paving the way for new  applications of tropical geometry to the  study of higher-order programs.


\subsection{Contributions}

Our contributions in this paper are threefold:
\begin{itemize}

\item we study the relational model over the tropical semiring  and we show that the functions interpreting simply-typed lambda terms, which correspond to a generalization of \emph{tropical Laurent series} \cite{Porzio2021}, are locally Lipschitz-continuous, thus yielding a full-scale metric semantics for the $\lambda$-calculus and its bounded fragments. This is in Sections \ref{section3} and \ref{section4}.
%Moreover, we exploit the differential structure of the relational model to study the \emph{tropical Taylor expansion} of a $\lambda$-term, which can be seen as an approximation of the term by way of Lipschitz-continuous maps.


\item Using the relational model as our main source of inspiration,  we suggest a few potential applications of tropical methods to the study of quantitative properties of non-deterministic and probabilistic functional programs, like counting best-case computation steps, 
measuring convergence log-probabilities, and 
differential privacy. This is in Section~\ref{section5}

\item We conclude 
by putting the connection between the 
tropical, differential and metric viewpoints at the right level of generality.
By recalling and suitably extending a well-known correspondence between Lawvere's \emph{generalized metric spaces} \cite{Lawvere1973, Stubbe2014} and modules over the tropical semi-ring \cite{Russo2007}, we show that the category of \emph{complete} generalized metric spaces provides a model of the differential $\lambda$-calculus which extends the tropical relational model. This is in Section~\ref{section6}.
\end{itemize}
%
%\section{Bounded and Differential $\lambda$-Calculi}
%
%
%Bounded Simply Typed $\lambda$-calculus $\BSTLC$:
%$$
%A::= o \mid !_{n}A \multimap A
%$$
%
%
%Resource Simply Typed $\lambda$-calculus $\RSTLC$:
%$$
%A::= o \mid [A, \dots , A] \multimap A
%$$
%
%
%Define a translation of types $(-)^{\C R}$ from $\BSTLC$ to $\RSTLC$ by $o^{\C R}=o$ and $(!_{n}A\multimap B)^{\C R}=
%[\underbrace{A^{\C R},\dots, A^{\C R}}_{n\text{ times}}]\multimap B^{\C R}$.
%
%\begin{proposition}
%$\Gamma \vdash_{\BSTLC} M:A$ implies 
%$\Gamma^{\C R}\vdash_{\RSTLC}M:A^{\C R}$.
%\end{proposition}
%





\section{Two Quantitative Approaches to the $\lambda$-calculus}




%Recall the two approaches with more details on lambda-calculus and on existing challenges.


%In this section, we discuss in some more detail the two approaches to quantitative semantics we mentioned in the Introduction, at the same time providing an overview of how we aim at bridging them using tropical mathematics.

\paragraph*{Controlled duplication/erasure via graded types: bounded $\lambda$-calculus $\BSTLC$}\label{sec:BSTLC}

One can see that the comonad $!$ of $\LREL$ can be ``decomposed'' {\color{red}REF} into a family of ``graded exponentials functors'' $!_n:\LREL\to\LREL$ ($n\in\BB N$), defined on objects by taking multisets of cardinality \emph{at most} $n$. %  lift to functors 
The sequence $(!_n)_{n\in\N}$ gives rise to a so-called \emph{$\N$-graded linear exponential comonad} on (the SMC) $\LREL$. %satisfying the adjunction: $\LREL(Z\otimes !_{n}X,Y) \simeq \LREL(Z, !_{n}X\multimap Y)$.

As such, $(\LREL,(!_n)_{n\in\N})$ is a model of graded calculi ensuring bounded duplications via graded typing systems, for instance it is a model  \cite{Katsumata2018} of the language $\BSTLC$, a simplified version of the language $\mathrm{Fuzz}$ \cite{Reed2010}, defined as follows: 
the terms are as for the $\STLC$, the types are $A::= * \ \mid  \ !_{n}A \multimap A$, the contexts of the typing judgements are sets of declarations of the form $x :_{n}A$, with $n\in \mathbb N$, and the typing rules are: %given in Fig.~\ref{fig:rules}
	\[ \scriptsize \arraycolsep=5pt\def\arraystretch{2.8}
	\begin{array}{cccc}
		\prooftree
		\Gamma \vdash M:A
		\justifies
		\Gamma, x:_{0}B \vdash M:A
		\endprooftree 
		&
		\prooftree
		\Gamma, x:_{n}B, y:_{m} B\vdash M:A
		\justifies
		\Gamma, x:_{n+m}B\vdash M\{x/y\}:A
		\endprooftree 
		&
		\prooftree
		\Gamma, x:_{n} A\vdash M: B
		\justifies
		\Gamma\vdash \lambda x.M: !_{n}A\multimap B
		\endprooftree
		&
		\prooftree
		\Gamma \vdash M: !_nA\multimap B
		\quad
		\Delta\vdash N: A
		\justifies
		\Gamma +n\Delta\vdash MN: B
		\endprooftree
	\end{array}
	\]
where $\Gamma+\Delta$ is defined letting $(\Gamma, x:_{m} A)+( \Delta, x:_{n} A) =  (\Gamma+\Delta), x:_{m+n}A$, and $m\Gamma$ is made all $x:_{mn}A$ for $(x:_{n}A) \in \Gamma$.  
The axiom is $x:_{1}A\vdash x: A$.
The main feature of this language is that if $\vdash \lambda x.M:\,!_nA\multimap B$, then $x$ is duplicated at most $n$ times in the reduction of $\vdash (\lambda x.M)N :B$ to the normal form.
%E.g., $\vdash_{\BSTLC} \lambda {\color{red}z}.\left( \lambda x{\color{green}y}. yxx \right)z : \, !_{\color{red}2} *\multimap !_{\color{green}1}(!_{\color{violet}1} * \multimap !_{\color{blue}1} * \multimap *) \multimap *$.
%Observe that one can always type an affine term like e.g.~$\lambda xy.x$ with a linear type $A\multimap B\multimap A$. Instead, a term like $\lambda xy.x(xy)$ containing two occurrences of $x$ cannot be given the linear type $(A\multimap A)\multimap (A\multimap A)$ but a type of the form
%$!_{2}(A\multimap A)\multimap (A\multimap A)$. 

%\begin{remark}\label{rmk:ModelsOfBSTLC}
%Similarly to It is known {\color{red}(([reference??] e dire meglio)} 

Remark that since arrow types are interpreted via $\model{!_{n}A\multimap B}:= !_{n}\model A \times \model B$, 
if $\model *$ is finite, then $\model{A}$ is a finite set for any type $A$ of $\BSTLC$.
%\end{remark}
%Bounded types are interesting because of the following proposition:
%\begin{proposition}
%For all bounded types $A,B$, the morphisms from $\model A$ to $\model B$ (in all parametric relational models) correspond to polynomials.
%\end{proposition}
%\begin{proof}
%It suffices to check that $\model A$ is finite for all bounded types $A$. Indeed this implies that a morphism $t:\model A\to \model B$ is a finite matrix $t: \model A \times \model B \to \Lawv$.Hence, its corresponding map $\widehat t:\Lawv^{\model A} \to \Lawv^{\model B}$ is a polynomial.
%\end{proof}

%For example (here $!_{n}(\Lawv^{X}):= \Lawv^{\C M_{\leq n}(X)}$):
%\begin{itemize}
%\item a map $f\in \LREL( !_{1}\Lawv, \Lawv)$ is of the form $f(x)=\min \{x+a,b\}$;
%\item a map $f\in \LREL(!_{2}\Lawv, \Lawv)$ is a ``quadratic'' polynomial $f(x)=\min\{2x+a, x+b, c\}$.
%\end{itemize}

\begin{comment}

\begin{figure*}

	\scriptsize
	
	\[ \arraycolsep=5pt\def\arraystretch{2.8}
	\begin{array}{cccc}
		\prooftree
		\Gamma \vdash M:A
		\justifies
		\Gamma, x:_{0}B \vdash M:A
		\endprooftree 
		&
		\prooftree
		\Gamma, x:_{n}B, y:_{m} B\vdash M:A
		\justifies
		\Gamma, x:_{n+m}B\vdash M\{x/y\}:A
		\endprooftree 
		&
		\prooftree
		\Gamma, x:_{n} A\vdash M: B
		\justifies
		\Gamma\vdash \lambda x.M: !_{n}A\multimap B
		\endprooftree
		&
		\prooftree
		\Gamma \vdash M: !_nA\multimap B
		\quad
		\Delta\vdash N: A
		\justifies
		\Gamma +n\Delta\vdash MN: B
		\endprooftree
		\\
		\\
		\hline
		\\
		\prooftree
		\Gamma, x: A\vdash M: B
		\justifies
		\Gamma\vdash \lambda x.M: A\to B
		\endprooftree 
		&
		\prooftree
		\Gamma \vdash M: A\to B
		\quad
		\Gamma\vdash \mathbb T: A
		\justifies
		\Gamma \vdash M\mathbb T: B
		\endprooftree 
		&
		\prooftree
		\Gamma \vdash M: A\to B
		\quad
		\Gamma \vdash N: A
		\justifies
		\Gamma \vdash \Diff{M}{N}: A\to B
		\endprooftree
		&
		\prooftree
		\Gamma\vdash M_1: A
		\,\cdots\,
		\Gamma \vdash M_n:A
		\justifies
		\Gamma \vdash M_1+\cdots +M_n : A
		\using (n\geq 2)
		\endprooftree
	\end{array}
	\]
	\caption{Typing rules (axiom rules are given in the text) for $\BSTLC$ (top) and $\STDLC$ (bottom).}\label{fig:rules}
\end{figure*}

\end{comment}

\paragraph*{Controlled duplication/erasure via resources: the differential $\lambda$-calculus $\STDLC$}\label{sec:STDLC}

A Cartesian closed differential $\lambda$-category (CC$\partial\lambda$C)\cite{Manzo2010,Blute2009, Blute2019} is a CCC enriched over commutative monoids %(i.e.\ we can add morphisms and there is a $0$ morphism)
%, it is Cartesian closed%(with the closed structure compatible with the additive structure)
and equipped with a certain \emph{differential operator} $D$, generalising the usual notion of differential, see e.g.\ \cite{BluteEhrhTass10}.
%An example is the CC$\partial\lambda$C of convenient vector spaces with smooth maps, where $D$ is the ``real'' differential $Df:\mathbb{V}\times\mathbb{V}\rightarrow \mathbb{W}$, $Df(x,u):=\dfrac{d}{dt}{\!\Big|_{t=0}} f(x+tu)$, of smooth maps $f:\mathbb{V}\rightarrow \mathbb{W}$.
%More precisely, a cartesian category $\C C$ is a $C\partial C$ when:
%\begin{itemize}
%\item $\C C$ is left-additive, i.e.~its hom-sets have the structure of commutative monoids, and the cartesian structure is well-behaved w.r.t.~this monoid structure;
%\item $\C C$ is equipped with a differential operator $D:
%\C C(X,Y)\to \C C(X\times X,Y)$ satisfying some axioms which capture usual properties of differentials (e.g.~the linearity of $D$ in one of its two variables, the chain rule, etc.).
%\end{itemize}
Now, applying \cite[Theorem 6.1]{lemay2020} one can check that
%\begin{proposition}[{\color{red}LEMAY??}]\label{thm:LREL!CCDC}
 $\LREL_!$ becomes a CC$\partial\lambda$C when equipped with the \emph{tropical differential operator} $D:\HOM{\LREL}{!X}{Y}\to \HOM{\LREL}{!(X\& X)}{Y}$ defined as $(Dt)_{\mu\oplus\rho,b}=t_{\rho+\mu,b}$ if $\#\mu=1$ and as $\infty$ otherwise (using the iso $(\mu,\rho)\in !Z\times !Z'\mapsto\mu\oplus\rho \in !(Z+Z')$).
%\end{proposition}
%\begin{remark}For $t\in\HOM{\LREL}{!X}{Y}$, we have: $D^2 t\in\HOM{\LREL_!}{(X+X)+(X+X)}{Y}$, where $(D^2 t)_{(\rho\oplus\rho')\oplus(\nu\oplus\nu'),b}$ equals $t_{\nu+\nu'+\rho',b}$ if $\rho=\emptyset$ and $\#\rho'=1=\#\nu$; it equals $t_{\rho+\nu',b}$ if $\rho'=\emptyset=\nu$ and $\#\rho=1$; it equals $\infty$ otherwise.\end{remark}
%This ensures that one can define a sound interpretation of $\STDLC$-terms in the standard way (see [Section 4.3, \cite{Manzo2010}]).

As such, $(\LREL_!,D)$ is a model of the differential $\lambda$-calculus $\STDLC$ \cite{ER}, a language ensuring exact control of duplications via a notion of linear substitution.
Its syntax (e.g.\ \cite[Section 3]{Manzo2010}) is given by the \emph{terms} $M$ and the \emph{sums} $\mathbb T$, mutually generated by: $M::= x\mid \lambda x.M \mid M\mathbb T \mid \Diff{M}{M}$ and $\mathbb T::= 0 \mid M \mid M+\mathbb T$,
quotiented by $\alpha$-equivalence, by equations that make $+,0$ form a commutative monoid on the set of sums, %, i.e.\ commutativity and associativity of $+$ and neutrality of $0$ w.r.t.\ $+$;
by linearity of $\lam x.(\_)$, $\Diff{\_}{\_}$ and $(\_)\mathbb T$ (but \emph{not} of $M(\_)$) and by irrelevance of the order of consecutive $\Diff{\_}{\_}$.
%Remark that $M(\_)$ is \emph{not} set to be linear: $\lambda x.0=0\mathbb T=\Diff{0}{N}=\Diff{M}{0}=0$ but $M0\neq0$ in general.
%This is crucial for the definition of the Taylor expansion.
We follow the tradition of quotienting also for the idempotency of $+$.
%Sums are, then, just \emph{finite} sets of terms.
The types are $A::= *\mid A\to A$, the typing rules: %in Figure~\ref{fig:rules}
	\[ \scriptsize \arraycolsep=5pt\def\arraystretch{2.8}
	\begin{array}{cccc}
		\prooftree
		\Gamma, x: A\vdash M: B
		\justifies
		\Gamma\vdash \lambda x.M: A\to B
		\endprooftree 
		&
		\prooftree
		\Gamma \vdash M: A\to B
		\quad
		\Gamma\vdash \mathbb T: A
		\justifies
		\Gamma \vdash M\mathbb T: B
		\endprooftree 
		&
		\prooftree
		\Gamma \vdash M: A\to B
		\quad
		\Gamma \vdash N: A
		\justifies
		\Gamma \vdash \Diff{M}{N}: A\to B
		\endprooftree
		&
		\prooftree
		\Gamma\vdash M_1: A
		\,\cdots\,
		\Gamma \vdash M_n:A
		\justifies
		\Gamma \vdash M_1+\cdots +M_n : A
		\using (n\geq 2)
		\endprooftree
	\end{array}
	\]
where a context $\Gamma$ is a list of typed variable declarations.
The axioms are $\Gamma, x:A \vdash x: A$ and $\Gamma\vdash 0:A$.
The main feature of this language is that $\Der^n[\lambda x.M,N^n]0$ has a non-zero normal form iff $x$ is duplicated exactly $n$ times during the reduction to normal form.
%In $\BSTLC$ the typing system handles duplications; in $\STDLC$ the syntax with its operational semantics (that we do not give) does it.
\begin{comment}
Writing $\Der^2[\_,(\_)^2]$ as a shortcut for $\Der[\Der[\_,\_],\_]$ and $\Der^1[\_,(\_)^1]$ for $\Diff{\_}{\_}$, the analogue of the previous $\BSTLC$-term is $\vdash_{\STDLC} \lambda {\color{red}z}. \Der^{\color{red}2}[
	\lambda x{\color{green}y}.
		\Der^{\color{violet}1} [
				\Der^{\color{blue}1} [y, x^{\color{blue}1}]
        0, x^{\color{violet}1}
	]0
, z^{\color{red}2}]0
: {\color{red}*}\to ({\color{green}* \to * \to *}) \to *$.
%Here we wrote $\Der^2[\_]\cdot (\_)^2$ as a shortcut for $\Der[\Der[\_]\cdot (\_)]\cdot (\_)$ and $\Der^1[\_]\cdot (\_)^1$ for $\Der[\_]\cdot (\_)$.
In particular, if the \emph{multiplicities} of the arguments (the colored exponents) do not exactly match the number of duplications, e.g.\ in $\vdash_{\STDLC} \lambda z. \Der^{\color{red}3}[
	\lambda xy.
		\Diff{
				\Diff{y}{x}
		0}{x}
	0
, z^{\color{red}3}]0
: *\to (* \to * \to *) \to *$, then the term reduces to the empty sum $0$ (representing an \emph{error}).
\end{comment}
%Correspondingly, the syntax of the simply typed \emph{differential} $\lambda$-calculus ($\STDLC$) is defined by enriching $\STLC$ with a monoid structure $0,+$ over terms, as well as with $\Der$ and a notion of \emph{linear substitution} (see \cite{difflambda} or the Appendix for details).

%Until now we simply specialised well-known results in our tropical case, with the intent of showing how things read in this particular case.
%Now we go further, by showing that $\LREL_!$ actually admits a \emph{differential structure}, turning it into a model of the $\STDLC$, i.e.~a $CC\partial C$.
%This viewpoint
%, is where the \emph{metric} and the \emph{differential} viewpoints converge, as explained in the Introduction and Section II, and it % will be further generalised in Section \ref{section6}.

%A model of the $\STDLC$ is usually understood as so-called \emph{Cartesian closed differential categories} (CC$\partial$C), see \cite{Manzo2012} for details.
%In order to treat the $+$ and the constructor $D[\_]\cdot (\_)$ of $\STDLC$, the main features of a CC$\partial$C $\C C$ are that:
%
%1) $\C C$ is a left-additive-CCC, i.e.\ its Homsets are commutative monoids and its Cartesian closed structure is well behaved w.r.t.\ this monoid structure;
%
%2) $\C C$ is equipped with a differential operator map $D:\HOM{\C C}{X}{Y}\to \HOM{\C C}{X\times X}{Y}$ (here $\times$ is the Cartesian product of $\C C$) satisfying $8$ axioms, called D1, ..., D7, D-curry.

%Let us show the differential structure of $\LREL_!$ (remember that the Cartesian product of $\LREL_!$ is the disjoint union $+$).

%\begin{definition} The \emph{tropical differential operator} is the map $D:\HOM{\LREL}{!X}{Y}\to \HOM{\LREL}{!(X+X)}{Y}$ defined as $(Dt)_{\mu\oplus\rho,b}=t_{\rho+\mu,b}$ if $\#\mu=1$ and as $\infty$ otherwise (where a multiset $\nu \in !(X+X)$ is identified with a disjoint sum of $\mu,\rho\in !X$).\end{definition}

%The models of $\STDLC$ are the cartesian \emph{closed} differential categories ($CC\partial C$), which are defined as $C\partial C$ which are also cartesian closed, and in which the monoid structure and the differential operator are both well-behaved with respect to the closed structure \cite{Manzo2012}. 


 



\section{Tropical (non-)Linear Algebra and computation}

Di cosa stiamo parlando. Ovvero:

- matrices over a continuous semi-ring

- The tropical semi-ring $\Lawv$ coincides with the Lawvere quantale. 

- tropical linear algebra: the category $\LREL$


\subsection{As a model of linear programs}




- matrices over a continuous semi-ring



- The tropical semi-ring $\Lawv$ coincides with the Lawvere quantale. 


- tropical linear algebra: the category $\LREL$



- linear computation inside $\LREL$: linear lambda-terms through the symmetric monoidal adjunction 
$$
\LREL(Z\otimes X, Y) \simeq \LREL (Z, X\multimap Y)
$$
(with $\LREL(X,Y)\simeq \Lawv^{X\times Y}$.






\subsection{As a model of non-linear programs}

In this section we study the tropical Laurent series $\Lawv^X\to \Lawv ^Y$ from the viewpoint of analysis.

Let us start by recalling Example~\ref{ex:famous_ex}.
By plotting its graph {\color{red}vogliamo plottarlo?}, we see first of all that the function is non-decreasing and concave.
It is easy to see that this is actually always the case:

\begin{proposition}
 Any tropical Laurent series $f:\Lawv^X\to\Lawv^Y$ is non-decreasing and concave, w.r.t.\ the pointwise order.
\end{proposition}

The $f$ of Example~\ref{ex:famous_ex} is continuous on $\BB R_{\geq0}=\Lawv-\set{\infty}$ (w.r.t.\ the usual norm of real numbers).
By considering the usual norm $\norm{x}_\infty:=\sup\limits_{a\in X} \absv{x_a}$ on $\Lawv^X$, we could generalise this property by dropping the case of $x$ having some $0$ coordinate:

\begin{theorem}\label{thm:cont}
 Any tropical Laurent series $f:\Lawv^X\to\Lawv$ is continuous on $\BB R_{>0}$, w.r.t.\ to the norm $\norm{\cdot}_\infty$.
\end{theorem}
\begin{proof}
 The result follows after proving that if a real-valued function on a locally convex topological $\BB R$-vector space is, locally around $x$, concave and bounded by a finite constant, then it is continuous at $x$.
\end{proof}

Not only the $f$ of Example~\ref{ex:famous_ex} is continuous, but it is also locally Lipschitz on $\BB R_{>0}$.
Actually, we can prove the following:

\begin{theorem}
 Let $f:\Lawv\to\Lawv$ a tropical Laurent series with matrix $\widehat f:\N\to\Lawv$.
 For all $0<\epsilon<\infty$, there is a \emph{finite} $\C F_\epsilon \subseteq \N$ s.t.:
 \begin{enumerate}
  \item If $\C F_\epsilon=\emptyset$ then $f=\infty$ on all $\Lawv$;
  \item If $f(x_0)=\infty$ for some $x_0<\infty$, then $\C F_\epsilon=\emptyset$;
  \item On all $[\epsilon,\infty]$, $f$ coincides with the tropical \emph{polynomial} $P_\epsilon(x):=\min\limits_{n\in \C F_\epsilon}\set{nx+\widehat f(n)}$.
 \end{enumerate}
 In particular, $f$ is locally Lipschitz on all $\BB R_{>0}$.
\end{theorem}
\begin{proof}
 We can let $\C F_\epsilon:=\set{n\in\N \mid
 \widehat f(n)<\infty \textit{ and } \widehat f(m)> \widehat f(m)+\epsilon \textit{ for all } m<n}$.
\end{proof}

Remark that, in coherence with the previous result, in Example~\ref{ex:famous_ex} $f(x)$ is indeed a $\min$ for all $x>0$.
At $x=0$ we have $f(x=0)=\inf\limits_{n\in\N} \frac{1}{2^n}=0$ which is \emph{not} a min.
Also, while the derivative of $f$ is bounded on all $\BB R_{>0}$, at $x=0$ it tends to $\infty$.
This phenomenon is reminiscent of [Example 7, PCoh]
%Differentials and Distances in Probabilistic Coherence Spaces. FSCD 2019
, which actually motivated our first investigations.

We could generalise the locally Lipschitz property as follows:
\begin{theorem}
 Let $f:\Lawv^X\to\Lawv$.
 If $f$ is non-decreasing, concave and continuous, then it is locally Lipschitz.
\end{theorem}
\begin{proof}
 Refinement of the arguments used for Theorem~\ref{thm:cont}.
 {\color{red}Che ci scriviamo?}
\end{proof}

Highlight compositional reasoning based on Lipschitz.\\

- the metric on the tensor is the usual tensor metric, the metric on the function space is the usual function metric, the metric on multisets is the multiset metric

{\color{red}
- extend this result to functions $\Lawv^{X}\to \Lawv^{Y}$ with $X,Y$ finite (as this will be useful in section V) Quale ?} 



- discussion of the tropical Taylor expansion: 


- characterization of the functional metric using derivatives\\

Let us end this section by mentioning another point of view on tropical Laurent series.
$\Lawv^X$ with the usual $+$ and the usual $\cdot$ is a $\BB R_{\geq0}$-semimodule.
Together with the norm $\norm{\cdot}_\infty$, it can be proved that it is a Scott-complete normed cone.
The normed cone structure induces an order on it, called its \emph{cone order}, by setting:
$x\leq y$ iff $y=x+z$ for some (unique) $z\in\Lawv^X$.
This order makes it a Scott-continuous dcpo.
Furthermore we have:

\begin{proposition}
  Tropical Laurent series $\Lawv^X\to\Lawv^Y$ are Scott-continuous on $\BB R_{>0}$, w.r.t.\ the cone orders on the domain and codomain.
\end{proposition}



\subsection{As a model of differential programs}

- differential categories

- the tropical differential operator: formal differentiation of tropical Laurent series. Differentiation in $\LREL_{!}$.

- $\LREL_{!}$ is a differential CCC satisfying Taylor. 
Discuss the tropical Taylor expansion $\trop\C T(M)$. 



\section{Analytic Results}

General discussion: optimization properties behave in a Lipschitz way.


- differential privacy and Lipschitzness


- log-probabilities and tropical roots 


- counting computation steps (from Manzonetto, but add relational ``Lipschitz'' reasoning)


- measuring duplications of discrete functions (needs finiteness!)







\section{Applications}

- correspondence between $\Lawv$-modules and $\Lawv$-categories.


- exponential structure of $\LCAT$.


- $\LCAT_{!}$ is a cartesian closed differential category




\section{Quasi-metric Spaces and $\Lawv$-Modules}

The connections between the differential $\lambda$-calculus (and differential linear logic), the relational semantics, and the theory of non-idempotent intersection types is very well-studied (see \cite{}, and more recently, \cite{} for a more abstract perspective, and \cite{} for a 2-categorical, or proof-relevant, extension).
As we said, the relational semantics over the tropical semi-ring was quickly explored in \cite{}, to provide a ``best case'' resource analysis of a $\B{PCF}$-like language with non-deterministic choice. 
\emph{Probabilistic coherent spaces} \cite{}, a variant of  the relational semantics, provide an interpretation of higher-order probabilistic programs
as analytic functions. In \cite{} it was observed that such functions satisfy a local Lipschitz condition somehow reminiscent of our examples in Section \ref{section4}.


The study of linear or bounded type systems for sensitivity analysis was initiated in \cite{} and later developed \cite{}.
As recalled in the paper, the use bounded exponentials ensures that well-typed programs satisfy a Lipschitz condition.
Related approaches, although not based on metrics, are provided by \emph{differential logical relations} \cite{} and \emph{change action} models \cite{}.


More generally, the literature on program metrics in denotational semantics is vast. Since [6] metric spaces, also in Lawvere's generalized sense \cite{}, have been exploited as an alternative framework to standard, domain-theoretic, denotational semantics. 
While standard categories of metric spaces are not models of the full simply typed $\lambda$-calculus, several constructions of cartesian closed categories of metric spaces can be found in the literature. For instance, 
\emph{ultra}-metric spaces \cite{} form a CCC, and have been shown to model $\B{PCF}$ \cite{}.
Also \emph{partial} metrics, introduced in \cite{}, have been shown to provide models of $\STLC$, under suitable generalizations \cite{}.
More generally, \cite{} provides a general characterization of exponentiable objects in categories of (generalized) metric spaces, and \cite{} provides other ways to construct CCCs of
(generalized) metrics, including one based on locally Lipschitz maps, 
using ideas from {differential logical relations}.

Motivated by connections with computer science and fuzzy set-theory, 
the abstract study of generalized metric spaces in the framework of \emph{quantale}- or even \emph{quantaloid}-enriched categories has led to a vast literature in recent years \cite{}, 
and its connections with tropical mathematics are discussed in \cite{}. Moreover, applications of quantale-modules to both logic and computer science have also been studied \cite{}.

Connections between program metrics and the differential $\lambda$-calculus have been already suggested in \cite{}; moreover, \emph{cartesian difference categories} \cite{} have been proposed as a way to relate derivatives in differential categories with those found in change action models.



Finally, applications of tropical mathematics in computer science abound, notably in optimization methods for machine learning \cite{} (typically, for neural networks with peicewise linear activation), linear regression \cite{}, convex analysis \cite{}, as well as finite automata \cite{}.
While our discussion in Section \ref{section5} is inspired by a well-known application of tropical polynomials \cite{},  
the vast literature in this domain lets us think that other ways to 
apply tropical semantics to the analysis of higher-order programs might be studied.

%
%Other connections tropical/metric -> Fuji, ??
%Applications of tropical to computer science.
%Log-probabilities.
%Quantale-modules -> Abramski?
%
%
 











\section{Related Work (vale la pena? O meglio nelle conclusioni?)}



In our opinion, the main goals of this paper are two. Firstly,  to
demonstrate the existence of a conceptual bridge between two different well-studied quantitative approaches to higher-order programs, and to highlight the possibility of transferring results and techniques from one approach to the other. 
Secondly, to suggest that tropical mathematics, a
field which has been largely and successfully applied in computer science, could be used to study quantitative properties of higher-order programs.

While the first goal has been developed in detail, and at different levels of abstraction, for the second goal we only sketched a few interesting directions (best-case analysis, log-probabilities, differential privacy). We believe that exploring in more depth these ideas could be a fruitful direction; moreover, 
since both generalized metrics and quantale-modules have been largely applied in computer science, 
a natural question is whether the generalized approach of Section \ref{section6} could  lead to new applications of metric and tropical methods to the $\lambda$-calculus.

Finally, as suggested in Section \ref{section5}, an interesting direction to look at is the semantics of finiteness spaces \cite{Ehrhard2005}, which allows to model typable terms by means of \emph{finitary} Taylor expansions. Indeed, knowing that the application of a program $M$ to $N$ expands as a finite sum of linear applications (hence by a Lipschitz function in our setting), may allow one to predict how sensitive $M$ will be ``around $N$''.
%
%We have shown that:
%
%- intuitively, the interplay between tropical mathematics and $\lam$-calculus \emph{could} relate the metric and differential approaches on approximations of $\lam$-calc (introduction+section 2)
%
%- this relation \emph{can} take place, because the natural category $\LREL$ and its generalised versions are metric models of the differential $\lam$-calculus (section 3 and 6)
%
%- this relation \emph{seems} to provide applications in different fields (section 5).
%
%Therefore, we mainly set the basis for future interplays between all these areas, hopefully motivating the interest in such an interplay.
%For instance, the general questions are:
%
%- Can we improve the results of section 4 ?
%
%- Can we develop and make the applications of section 5 useful ?
%
%- What does the general setting of section 6 give in terms of theoretical and applied results ?
%
%- Do tools from tropical \emph{geometry} provide something for $\lam$-calculus ? (for instance, the role of tropical roots, tropical varieties,...)
%
%- Finally, there is a last point which we think of interest and we did not mention through the paper: the inclusion of finiteness spaces in the picture.
%Say what could they do and why.

\section{Conclusion}

\input{section9}

\end{document}
