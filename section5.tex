

In this section we illustrate a few directions in which the tropical semantics just introduced could be used to analyze quantitative properties of higher-order programs. 

Since algebraic and geometric properties in tropical mathematics are usually more tractable from a computational point of view, in several well-known applications (e.g.~for optimization problems related to machine learning \cite{Pachter2004, Zhang2018, Maragos2021}) one starts from a given model, typically expressed by some polynomial function $f$, and studies 
what properties of the model can be deduced from the \emph{tropicalization} of $f$, noted $\trop f$, i.e.~the transformation of $f$ into a tropical polynomial.


%
%
%several well-known applications of tropical mathematics is to study how much can be deduced of some function starting from the properties of its tropicalization.
%In Section \ref{section5} we will follow a similar direction, investigating what quantitative properties of a higher-order programs are revealed by the study of its tropical interpretation.
%


Here we follow a similar pattern: we consider a program $M$, which can be expressed in the form of a polynomial or a power series $f$, and we 
investigate what quantitative properties of $M$ can be deduced from the properties of $\trop f$, that will indeed coincide with the interpretation of $M$ in $\LREL_{!}$.





\subsection{The tropicalization of polynomials and power series}

%Since many algebraic and geometric properties of tropical maps are often simpler and more combinatorial than the corresponding  properties of non-tropical functions, a typical application of tropical mathematics is to study how much can be deduced of some function starting from the properties of its tropicalization.
%In Section \ref{section5} we will follow a similar direction, investigating what quantitative properties of a higher-order programs are revealed by the study of its tropical interpretation.
%

Let us first recall how standard polynomials and power series over $[0,1]$ can be turned into tLs via the so-called \emph{Maslov dequantization} \cite{Litvinov2007}.
%
%Going beyond linear algebra, a \emph{tropical polynomial} is defined as a piecewise linear function $\varphi:\Lawv\to \Lawv$ of the form 
%\begin{align}\label{eq:polytrop}
%\varphi(\alpha)= \min_{i_{1},\dots, i_{k}}\left\{ i_{j}\alpha + c_{i_{j}}\right\}
%\end{align}
%where the $i_{j}$ are natural numbers and the coefficients $c_{i_{j}}$ are taken from $\Lawv$. For instance, the polynomial
%$\varphi_{3}(\alpha)=\min\{ 3\alpha+1/8,2\alpha+1/4, \alpha+1/2, \alpha\}$ will be discussed in Section \ref{section4}, and its graph is illustrated in Fig.~\ref{fig:plot1}.
%A value $\alpha\in \Lawv$ is a \emph{root} of the polynomial $P$ when
%the minimum at $\varphi(\alpha)$ is attained at least twice (equivalently, when 
% $\varphi$ is not differentiable at $\alpha$). In other words, the tropical roots of $\varphi$ coincide with the points where the slope of $\varphi$ changes. 
%%
%Intuitively, tropical polynomials look much like standard polynomials, although with ``$+$'' replaced by ``$\min$'', and ``$\times$'' replaced by ``$+$''. 
%In fact, this intuition can be made precise as follows: 

%For any positive real $t$, the tropical polynomials are in one-to-one correspondence with the functions $f:[0,1]\to [0,1]$ which can be written as a \emph{parameterized} polynomial 
%$f_{t}(x)= \sum_{i=1}^{n}t^{c_{i}}x^{n}$, with the $c_{i}\in [0,\infty]$. 
%%Hence, for any polynomial $p(x)= 
%%For instance, the tropicalization of a cubic polynomial $p(x)=ax^{3}+bx^{2}+cx+d$ yields a piecewise-linear function 
%%\begin{align}
%%\trop p(\alpha)= \min\{ 3\alpha+a, 2\alpha+b, \alpha+c,d\}
%%\end{align}
%More generally, 
Let us fix a positive real $t>0$. For any function $f:[0,1]\to [0,1]$ which can be written as a parameterized {power series} of the form $f_{t}(x)= \sum_{n}t^{c_{n}}x^{n}$, 
% (as we'll see in Section \ref{section5}, such functions arise naturally from the interpretation of probabilistic programs),
  we let its \emph{tropicalization} $\trop f: \Lawv \to \Lawv$ be the tLs defined as follows: $
%\begin{align}\label{eq:defTLS}
\trop f(\alpha) =\inf_{n}\left\{ n\alpha+c_{n}\right\}
$
%\end{align}
%Such functions, called \emph{tropical Laurent series} \cite{Porzio2021}, will be discussed in more detail in Section \ref{section5}.
Clearly, for any $t>0$, there is a one-to-one correspondence between the representations of power series in parameterized form and the associated tLs. Moreover, 
$f$ and $\trop f$ can be related by a limit passage as follows: the functions $\phi_{t}(x)= -\log_{t}x$ and $\varphi_{t}(\alpha)= t^{-\alpha}$ define continuous bijections between $[0,1]$ and $[0,\infty]$ and, by letting
$\trop_{t}f: [0,\infty]\to [0,\infty]$ be defined by 
$\trop_{t}f(\alpha)= \phi_{t}\circ f \circ \psi_{t}$, one has that 
$\trop f= \lim_{t\to 0}\trop_{t}f$. 
Indeed, one can check that the ``parameterized'' sums and product $\alpha \sumt{t}\beta:= \phi_{t}(\psi_{t}(\alpha)+\psi_{t}(\beta))= -\log_{t}(t^{-\alpha}+t^{-\beta})$ and 
$\alpha \prodt{t}\beta:= \phi_{t}( \psi_{t}(\alpha)\psi_{t}(\beta))=
-\log_{t}(t^{-\alpha}t^{-\beta})$ converge respectively to $\min\{\alpha,\beta\}$ and $\alpha+\beta$, when $t\to 0$.





\subsection{Best case analysis and metric reasoning}

The possibility of using the relational model over the tropical semiring for ``best case'' resource analysis has already been explored in \cite{Manzo2013}. Notably, they considered an interpretation of a language for $\B{PCF}$ with non-deterministic choice in which each $\lambda$-abstraction and each occurrence of the fixpoint operator $Y$ is assigned a ``weight'' 1, and showed that for any program $M$ of type $\B{nat}$, 
the value of the interpretation $\model{M}\in \Lawv^{\BB N}$ on a particular natural number $k$, i.e.~$\model{M}(k)\in \Lawv$, corresponds to the \emph{minimum} number of $\beta$- or $\TT{fix}$-redexes reduced in a reductions sequence from $M$ to $\underline n$. 
In the next paragraph we will illustrate an analogous ``best case'' analysis for probabilistic programs.

What does the metric analysis from the previous sections add to that? Firstly, the possibility of \emph{comparing} different programs with respect to their quantitative properties. For example, in the $\B{PCF}$ semantics recalled above, the distance between two programs $M$ and $N$ of type $\B{nat}$, provides a bound on the difference between the  ``best case'' computation time of $M$ and that of $N$. For instance, by taking, instead of the $\infty$-norm metric on $\Lawv^{\BB N}$,  
the \emph{non symmetric} distance (or quasi-metric, a viewpoint we explicitly take in Section \ref{section6}) $q(\B x, \B y)=\sup_{n}\{\B y_{n}\dotdiv \B x_{n}\}$, a ``distance'' $q(\model{M},\model{N})\leq \epsilon$ would indicate that $\model{M}$ \emph{improves} on $\model{N}$ of at most $\epsilon$ steps at each computation. 

Secondly, the Lipschitz conditions from Section \ref{section4} allow us to reason on program distances in a \emph{compositional} way: suppose, as before, that $M,N:A$ are two programs such that $M$ improves on $N$ by $\epsilon$, and let $\TT C[-]:A \to \B{nat}$ indicate a context; knowing that the interpretation of $\TT C$ is $k$-Lipschitz-continuous on some open set containing both $\model M$ and $\model N$, allows us to immediately deduce that $\TT C[M]$ improves on $\TT C[N]$ by $k \epsilon$. 
Observe that this will typically be the case when the Taylor expansions of $\TT C[M]$ and $\TT C[N]$ actually yields a \emph{finite} sum of at most $k$ terms, i.e.~when 
\begin{align}
\TT C[M] = \sum_{i=0}^{k} \TT D^{(k)}\Big[\lambda x.\TT C[x]\Big]\cdot M^{k}
\end{align}
and similarly for $\TT C[N]$. It is here worth recalling that, for $\STDLC$, a well-known result \cite{difflambda} is that the Taylor expansion of a closed application $MN$ is always \emph{finite}, although its non-zero coefficients may be arbitrarily high. 
Notably, these observations suggest to study tropical versions of \emph{finiteness spaces} \cite{Ehrhard2005}, 
a variant of the relational semantics modeling strongly normalizing programs via \emph{finite} power series.
%We mention this point in Section~\ref{section8}.


\subsection{Convergence log-probabilities}


Let us consider a probabilistic extension of $\STLC$, call it $\STLC_\oplus$:
we add a ground type $\bool$, terms $\true,\false$ of type $\bool$, terms of shape $M\oplus_p N$ and $pM$, for $p\in[0,1]$, typed via the rules:
\[
 \dfrac{\Gamma\vdash M:A \qquad \Gamma\vdash N:A}{\Gamma\vdash M\oplus_p N:A} 
 \qquad\qquad
 \dfrac{\Gamma\vdash M:A}{\Gamma\vdash pM:A}
\]
We add reduction rules:
\[
 M\oplus_p N \to pM \quad \textit{ and } \quad M\oplus_p N \to (1-p)N
\]
so that $M\oplus_p N$ plays the role of a probabilistic coin toss of bias $p$.
We use this calculus as a toy example for our purposes, and it can be seen as a fragment of the PCF considered in \cite{Manzo2013}.

Consider now the following term $M$ of type $\bool$:
\[
 (\true \oplus_p\false)\oplus_p((\true\oplus_p\false)\oplus_p(\true\oplus_p\false))
\]
Let us give addresses $\omega\in\set{00,01,100,101,110,111}$ to the occurrences of $\true,\false$ in $M$, by following the tree structure of $M$, reading $0$ as ``left'' and $1$ as ``right''.
The same addresses also represent all the different possible reduction paths from $M$ to a normal form.
For instance, $100$ represents the reduction which keeps the right part of the outermost $\oplus_p$ and erases the left part, then continues by choosing the left part twice, reaching at the end the occurrence $\true_{100}$ in $M$, i.e.\ the second occurrence of $\true$ in $M$ starting from the left.
Calling $q:=1-p$, there are the following six normal terms reachable from $M$:
$P_{00}(p,q)\true$, 
$P_{100}(p,q)\true$, 
$P_{111}(p,q)\true$, 
$P_{01}(p,q)\false$, 
$P_{110}(p,q)\false$,
$P_{101}(p,q)\false$,
where the $P$'s are the following monomials in $p,q$:
$P_{00}(p,q):=p^2$,
$P_{100}(p,q):=qp^2$,
$P_{111}(p,q):=q^3$,
$P_{01}(p,q):=pq$,
$P_{110}(p,q)=P_{101}(p,q):=q^2p$.
They correspond to the respective reduction path from $M$ to the normal term of the same address.
$P_{\omega}$ is then the probability (as a function of $p,q$) of obtaining the respective occurrence $\true_{\omega}$ or $\false_\omega$ after all the coin tossings in $M$ are performed.
Thinking of $p,q$ as parameters, $P_{\omega}$ can thus be read as the \emph{likelihood function} of the event $\omega$.
The polynomial $Q_{1}(p,q):=P_{00}(p,q)+P_{100}(p,q)+P_{111}(p,q)$ gives instead the whole probability of obtaining $\true$ after all the tossings, and $Q_{0}(p,q):=P_{01}(p,q)+P_{110}(p,q)+P_{101}(p,q)$ the one for $\false$.


%Let us consider in this subsection a probabilistic extension of $\lam$-calculus, call it $\STLC_\oplus$, adding as usual terms of shape $pM+qN$ and $M\oplus_p N$, for $p,q\in[0,1]$.These terms are typed via the rule:
%\[
%\dfrac{\Gamma\vdash M:A \qquad \Gamma\vdash N:A}{\Gamma\vdash M\oplus_p N:A}
%\]
%and similar for $\Gamma\vdash pM+qN:A$.We add the reduction rule:
%\[
% M\oplus_p N \to pM+(1-p)N
%\]
%so that such terms play the role of probabilistic choices of parameter $p$, as well as the rule:
%\[
% pM+qM\to (p+q)M.
%\]
%Let us consider $M:=(I\oplus_p\Omega)\oplus_p((I\oplus_p\Omega)\oplus_p(I\oplus_p\Omega))$. Reducing to normal form, we have:
%\[
% M\twoheadrightarrow (p^2+(1-p)p^2+(1-p)^3)I+(p(1-p)+2(1-p)^2p)\Omega.
%\]
%The index $\omega\in\set{00,100,111,01,110,101}$ of each $P_\omega$ indicates the path of the reduction that led from $M$ to the respective occurrence $I_\omega$ of $I$ or $\Omega_\omega$ of $\Omega$ from $M$ to its normal form ($0$ means ``left'' and $1$ means ``right'').For instance, in order to reach $I_{100}$, i.e.\ the second occurrence of $I$ from the left in $M$, we have to take the right path in the outer $\oplus_p$ of $M$, then two times the left path in the new outer $\oplus_p$'s that we encounter during reduction.$P_{\omega}(p,q)$ gives then the probability (as a function of $p,q$) of obtaining the respective occurrence $I_{\omega}$ or $\Omega_\omega$ in the normal form, if we were to sample at each time we reduce a $\oplus_p$.It can thus also be read as the likelihood function of such event.The polynomials $Q_{1,2}(p,q)$ give instead the whole probability of obtaining respectively $I$ or $\Omega$, in the normal form after such samplings.

This way, the probabilistic evaluation of $M$ is presented as a \emph{hidden Markov model} \cite{Baum1966}, a fundamental statistical model, and notably one to which tropical methods are generally applied \cite{Pachter2004}.
Typical questions in this case would be:
%
%The tropical point of view allows now to express two natural questions about this situation:
\begin{enumerate}
 \item What is the \emph{maximum likelihood estimator} for the event ``$M$ produces $\true_\omega$'' (similarly for $\false_\omega$)?
 I.e., which is the choice of the parameters $p,q$ that maximizes the probability of getting $\true_\omega$ after all tossings in $M$?
 \item Knowing that $M$ produced $\true$ (similarly for $\false$), which is the choice of the parameters $p,q$ that maximizes the probability that the produced occurrence of $\true$ was a fixed $\true_{\omega_0}$ (i.e. that among the paths leading to $\true$, the one taken was $\omega_0$)?
\end{enumerate}
A similar argument could be done by replacing $\true$ and $\false$ by, respectively, a converging and a diverging term (e.g.~in a $\B{PCF}$-style language), so 1) would be about finding maximum likelihood estimators for the event ``$M$ converges''.

%These are questions much in the style of those asked in {\color{red}Ref?} in the framework of {\color{red}??}.
Answering 1) and 2) amounts at solving a maximization problem related to $P_{\omega}, Q_\omega$, which is more easily solved by 
passing to the tropical monomials/polynomial $\trop P_{\omega},\trop Q_{\omega}$. 
For 1), we are looking for $p,q\in[0,1]$ s.t.\ $q=1-p$ and $(p,q)$ belong to:
\begin{equation}
  \begin{array}{r@{}c@{}l@{}}
   \underset{(x,y)}{\RM{arg max}} P_\omega (x,y)
   & = &
   \underset{(x,y)}{\RM{arg min}}\set{-\log_c P_\omega (x,y)}
   \\
   & = &
   \underset{(x,y)}{\RM{arg max}}\set{(\trop P_\omega) (-\log_c x,-\log_c y)} \label{eq:argmax}
  \end{array}
\end{equation}
where the equalities hold for all $0<c\neq1$.

For 2), %the $\omega_0$ is s.t.\ $\max\limits_{\omega\in\set{00,100,111}} \, P_\omega(x,y) = P_{\omega_0}(x,y)$. So
we are looking for $p,q\in[0,1]$ s.t.\ $q=1-p$ and
$\max\limits_{\omega\in\set{00,100,111}} \, P_\omega(x,y) = P_{\omega_0}(x,y)$.
This last condition is equivalent to ask that:
\begin{equation}
  \begin{array}{l@{}l@{}}
    \min\limits_{\omega\in\set{00,100,111}} \, -\log_c P_\omega(x,y) = -\log_c P_{\omega_0}(x,y)
   & \textit{ i.e.: }
   \\
   (\trop Q_1)( -\log_c x, -\log_c y) = (\trop P_{\omega_0})( -\log_c x, -\log_c y). &  \label{eq:max}
  \end{array}
\end{equation}
In both cases, passing through $\trop $ %P_\omega, \trop Q_\omega$ 
makes the problem easier, as this amounts to study tropical polynomials (for instance computing tropical roots can be done in linear time \cite{Noferini2015}). %the tropicalisation operator $\trop{}$ as well as the \emph{negative $\log$-probabilities} appear.

\begin{remark}
 By adapting [Section IV, \cite{Manzo2013}], it can be seen that $\LREL_!$ is a model of $\STLC_\oplus$.
 In particular, if we set $\model \bool:=\set{0,1}$, our running example $M$ is interpreted as $\model M\in\Lawv^{\set{0,1}}$ giving the tropicalised probabilities: $\model M_0=(\trop Q_0)(p,1-p)$, $\model M_1=(\trop Q_1)(p,1-p)$.
\end{remark}

%Now, in $\LREL$, seen as a model of such probabilistic $\lam$-calculus, the interpretation of a term already computes the tropicalisation of the polynomials expressing the probabilities, because the underlying semiring of the model is tropical.
%For instance, for our running example $M$:
%\[\model M = \min\left\{\trop{Q_1}(p,1-p) \cdot \model I, \trop{Q_0}(p,1-p) \cdot \model \Omega\right\}.\]
Therefore, assuming the interpretation of a term is known, 
so that one can extract the associated tropical polynomials from it,
% (in $p$ and $q$, not after the substitution $q:=1-p$) 
% are extracted from it, 
one could study the optimisation problems \ref{eq:argmax}, \ref{eq:max} \emph{without} having to actually reduce the term. 


%\begin{example}\label{ex:study}
For our running example $M$, we have $\trop Q_1(x,y)=\min\set{2x,y+2x,3y}$ and $\trop Q_0(x,y)=\min\set{x+y,2y+x}$.
Studying $\trop Q_1$, whose plot is in Fig.~\ref{fig:plot2}, we see that $\trop Q_1(x,y)=3y$ iff $y\leq \frac{2}{3}x$, and it coincides with $2x$ otherwise.
Remembering that $3y=P_{111}(x,y)$, we can now solve the optimisation problem~\ref{eq:max} for $\omega_0=111$:
via the substitution $x:=-\log_c p$, $y:=-\log_c (1-p)$, Equation~\ref{eq:max} is equivalent to $-\log_c (1-p)\leq -\frac{2}{3}\log_c p$, i.e.\ $1-p\geq p^{\frac{2}{3}}$.
This means that, for $p\in[0,1]$ s.t.\ $1-p\geq p^{\frac{2}{3}}$ (for example, $p=\frac{1}{4}$), the most likely occurrence of $\true$ to obtain, knowing that $M$ sampled $\true$ in its normal form, is $\true_{111}$.
Remembering that $2x=P_{00}(x,y)$, for the other values of $p$ (for example, $p=\frac{1}{2}$), the most likely $\true$ to be sampled is the occurrence $\true_{00}$.
We have thus answered question 2) above.
%\end{example}

\begin{remark}\label{rem:troproots}
 Taking the usual $M$ as example, the $p\in[0,1]$ s.t.\ $(p,1-p)$ is a tropical root of $\trop Q_1$ or of $\trop Q_0$ provide, by definition, the values of the bias of $\oplus_p$ for which there are at least two different equiprobable paths of tossings from $M$ to its normal form.
 Moreover, looking at Equation~\ref{eq:max}, we see that the $p\in[0,1]$ s.t.\ $(-\log_c p,-\log_c(1-p))$ is a tropical root of, say, $\trop Q_1$ are the values of the bias of $\oplus_p$ for which there are at least two different equiprobable occurrences of $\true$ that are smapled by $M$ during its tossings to normal form, knowing that some $\true$ was sampled.
\end{remark}
