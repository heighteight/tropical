Until now we simply specialised well-known results in our tropical case, with the intent of showing how do things read in this particular case.
Now we go further, by showing that $\LREL_!$ actually admits a \emph{differential structure}, turning it into a model of the $\STDLC$.
This is where the \emph{metric} and the \emph{differential} viewpoints converge, as explained in the Introduction and Section II, and it will be generalised in Section 6.

A model of the $\STDLC$ is usually understood as so-called \emph{Cartesian closed differential categories} (CC$\partial$C), see [Giulio,?] for details.
In order to treat the $+$ and the constructor $D[\_]\cdot (\_)$ of $\STDLC$, the main features of a CC$\partial$C $\C C$ are that:

1) $\C C$ is a left-additive-CCC, i.e.\ its Homsets are commutative monoids and its Cartesian closed structure is well behaved w.r.t.\ this monoid structure;

2) $\C C$ is equipped with a differential operator map $D:\HOM{\C C}{X}{Y}\to \HOM{\C C}{X\times X}{Y}$ (here $\times$ is the Cartesian product of $\C C$) satisfying $8$ axioms, called D1, ..., D7, D-curry.

Let us show the differential structure of $\LREL_!$ (remember that the Cartesian product of $\LREL_!$ is the disjoint union $+$): 

\begin{definition}
 Let $\widetilde{(\cdot)}:\HOM{\LREL}{!X}{Y}\to \HOM{\LREL}{!(X+X)}{Y}$, $\widetilde t_{\mu\oplus\rho,b}:= t_{\mu*\rho,b}$, and $d:\HOM{\LREL}{!(X+X)}{Y}\to \HOM{\LREL}{!(X+X)}{Y}$, $(dt)_{\mu\oplus\rho,b}:=t_{\rho\oplus\mu,b}$ if $\#\mu_1=1$ and $(dt)_{\mu\oplus\rho,b}:=\infty$ otherwise.
 The \emph{tropical differential operator} is the map $D:=d\circ\widetilde{(\cdot)}:\HOM{\LREL}{!X}{Y}\to \HOM{\LREL}{!(X+X)}{Y}$.
\end{definition}

\begin{example}
For $t\in\HOM{\LREL}{!X}{Y}$, we have:
 $(Dt)_{\mu\oplus\rho,b}=t_{\rho*\mu,b}$ if $\#\mu_1=1$ and it is $\infty$ otherwise.
 Analogously, we have:
 $D^2 t\in\HOM{\LREL_!}{(X+X)+(X+X)}{Y}$, where $(D^2 t)_{(\rho\oplus\rho')\oplus(\nu\oplus\nu'),b}$ equals $t_{\nu*\nu'*\rho',b}$ if $\rho=\emptyset$ and $\#\rho'=1=\#\nu$; it equals $t_{\rho*\nu',b}$ if $\rho'=\emptyset=\nu$ and $\#\rho=1$; it equals $\infty$ otherwise.
\end{example}

\begin{theorem}\label{thm:LREL!CCDC}
 $\LREL_!$ equipped with $D$, is a CC$\partial$C.
\end{theorem}
\begin{proof}
 The properties 1) are quite straightforward.
 Properties 2) are tedious technical checks.
\end{proof}

There is more: not only $\LREL_!$ is a model of the $\STDLC$, but this model is also well-behaved w.r.t.\ to the \lamcalc Taylor expansion.
This is expressed by the following property (see [Definition 4.22, Giulio], where it is called the fact of ``modeling the Taylor expansion'').

\begin{theorem}\label{thm:modelsTaylor}
 In $\LREL_!$ equipped with $D$, all morphisms can be Taylor expanded, i.e.\ in $\LREL_!$ we have the following equality:
 \[
  \RM{eval}^{X,Y}\circ_!\langle t,s\rangle =
  \inf\limits_{n\in\N}
  \set{((\dots((\Lambda^- t)\star s)\star \dots)\star s)\circ_! \langle \RM{id},\infty \rangle}.
 \]
 for all $t\in\HOM{\LREL_!}{Z}{X\multimap Y}$, $s\in\HOM{\LREL_!}{Z}{X}$.
\end{theorem}
Here, $\Lambda^-:\HOM{\LREL}{!Z}{X\multimap Y}\to \HOM{\LREL}{!(Z+X)}{Y}$ is the uncarry operator and $\star:\HOM{\LREL}{!(Z+X)}{Y}\times\HOM{\LREL}{!Z}{X}\to \HOM{\LREL}{!(Z+X)}{Y}$ is an operator corresponding to the syntactical \emph{differential substitution} of $\STDLC$.
\begin{proof}
 {\color{red}(Che ci scriviamo ??}
\end{proof}

This property is interesting because of [Theorem 4.23,Giulio], which says that, then, in the model the series corresponding to the Taylor expansion of an ordinary $\lam$-term $M$ converges to $M$.
