Until now we simply specialised well-known results in our tropical case, with the intent of showing how things read in this particular case.
Now we go further, by showing that $\LREL_!$ actually admits a \emph{differential structure}, turning it into a model of the $\STDLC$, i.e.~a $CC\partial C$.
This viewpoint
%, is where the \emph{metric} and the \emph{differential} viewpoints converge, as explained in the Introduction and Section II, and it 
will be further generalised in Section \ref{section6}.

%A model of the $\STDLC$ is usually understood as so-called \emph{Cartesian closed differential categories} (CC$\partial$C), see \cite{Manzo2012} for details.
%In order to treat the $+$ and the constructor $D[\_]\cdot (\_)$ of $\STDLC$, the main features of a CC$\partial$C $\C C$ are that:
%
%1) $\C C$ is a left-additive-CCC, i.e.\ its Homsets are commutative monoids and its Cartesian closed structure is well behaved w.r.t.\ this monoid structure;
%
%2) $\C C$ is equipped with a differential operator map $D:\HOM{\C C}{X}{Y}\to \HOM{\C C}{X\times X}{Y}$ (here $\times$ is the Cartesian product of $\C C$) satisfying $8$ axioms, called D1, ..., D7, D-curry.

Let us show the differential structure of $\LREL_!$ (remember that the Cartesian product of $\LREL_!$ is the disjoint union $+$).

\begin{definition}
 The \emph{tropical differential operator} is the map $D:\HOM{\LREL}{!X}{Y}\to \HOM{\LREL}{!(X+X)}{Y}$ defined as $(Dt)_{\mu\oplus\rho,b}=t_{\rho*\mu,b}$ if $\#\mu=1$ and as $\infty$ otherwise.
\end{definition}

\begin{remark}
For $t\in\HOM{\LREL}{!X}{Y}$, we have:
 $D^2 t\in\HOM{\LREL_!}{(X+X)+(X+X)}{Y}$, where $(D^2 t)_{(\rho\oplus\rho')\oplus(\nu\oplus\nu'),b}$ equals $t_{\nu*\nu'*\rho',b}$ if $\rho=\emptyset$ and $\#\rho'=1=\#\nu$; it equals $t_{\rho*\nu',b}$ if $\rho'=\emptyset=\nu$ and $\#\rho=1$; it equals $\infty$ otherwise.
\end{remark}

\begin{theorem}\label{thm:LREL!CCDC}
 $\LREL_!$ equipped with $D$, is a CC$\partial$C.
\end{theorem}
\begin{proof}
 Left-additivity (with respect to $\min$) is quite straightforward.
 The other axioms of CC$\partial$C's are checked in the Appendix.
\end{proof}

This ensures that one can define a sound interpretation of $\STDLC$-terms in the standard way (see [Section 4.3, \cite{Manzo2010}]).

There is more: this model is also well-behaved w.r.t.\ to the Taylor expansion, as expressed by the following property (see [Definition 4.22, \cite{Manzo2012}]%, where it is called the fact of ``modeling the Taylor expansion''
).

\begin{theorem}\label{thm:modelsTaylor}
 In $\LREL_!$ equipped with $D$, all morphisms can be Taylor expanded, i.e.\ in $\LREL_!$ we have that, for all
$t\in\HOM{\LREL_!}{Z}{X\multimap Y}$ and $s\in\HOM{\LREL_!}{Z}{X}$ we have:%, the evaluation of $t$ over $s$ yields 
 \begin{align}\label{eq:taylorcat}
  \RM{ev}\circ_!\langle t,s\rangle =
  \inf\limits_{n\in\N}
  \set{((\dots((\Lambda^- t)\star s)\star \dots)\star s)\circ_! \langle \RM{id},\infty \rangle}.
 \end{align}
 \end{theorem}
%It is worth discussing the formula above a bit more. 
Here,
%:\HOM{\LREL}{!Z}{X\multimap Y}\to \HOM{\LREL}{!(Z+X)}{Y}$
%$\star:\HOM{\LREL}{!(Z+X)}{Y}\times\HOM{\LREL}{!Z}{X}\to \HOM{\LREL}{!(Z+X)}{Y}$ is defined as 
$u\star s= (Du)\circ_{!} \langle \langle  \infty, s\circ_{!} \pi_{1}\rangle,\mathrm{id}\rangle$ corresponds to linear application of $s$ to $u$, and $\Lambda^-$ is the uncurry operator.
Hence the right-hand term in \eqref{eq:taylorcat} corresponds to the $\inf$ of the $n$-th derivative of $\Lambda^{-}t$ applied to ``$n$ copies'' of $s$,  i.e.~it coincides with the tropical %interpretation of the
version of the usual Taylor expansion.
Since $\LREL_!$ has countable sums (all $\inf$'s converge), and thanks to equation \eqref{eq:taylorcat}, an immediate adaptation of the proof of [Theorem 4.23, \cite{Manzo2012}] entails that the interpretation of the $\STDLC$-Taylor expansion of a $\STLC$-term $M$ given in \eqref{eq:taylor}, converges to the interpretation of $M$.

As a consequence, we have:

\begin{corollary}\label{thm:M=infLip}
 If $\Gamma \vdash_{\STLC}M:A$, then $\model M^!: \model\Gamma \to \model A$ is the $\inf$ of \emph{Lipschitz} functions.
\end{corollary}

%
% corresponding to the syntactical \emph{differential substitution} of $\STDLC$.
%%\begin{proof}
% {\color{red}(Che ci scriviamo ??}
%\end{proof}


