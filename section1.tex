
In recent years, more and more interest in the programming language community has been directed towards the analysis of \emph{quantitative} properties of programs, such as e.g.~number of computation steps, convergence probability, 
approximation bounds, 
as opposed to purely \emph{qualitative} properties like e.g.~termination or equivalence. 
Notably, a significative effort has been made to extend, or adapt, well-established qualitative methods, like e.g.~type systems, relational logics or denotational semantics, to account for quantitative properties. We can mention, for example, 
intersection type systems aimed at capturing time or space resources \cite{decarvalho2018, Accattoli2022}, or convergence probabilities \cite{Breuvart2018, PistoneLICS2022},  relational logics to account for probabilistic properties like e.g.~differential privacy \cite{Barthe_2012} or metric preservation \cite{Reed2010, dallago}, as well as the study of denotational models for 
probabilistic \cite{Ehrhard2011, Staton2017} or differential \cite{difflambda} extensions of the $\lambda$-calculus. 
The main reason to look for methods relying on (quantitative extensions of) type-theory or denotational semantics is that these approaches yield \emph{modular} and \emph{compositional} techniques, that is, allow one to deduce properties of complex programs from the properties of their constituent parts.   

\subsection{Two kinds of quantitative approaches}

Among such quantitative approaches, two different directions have received considerable attention. 

On the one hand one should certainly mention \emph{program metrics} \cite{Reed2010, Gaboardi2017, Gabo2019} and \emph{quantitative equational theories} \cite{Plotk}: when considering probabilistic or approximated computations, rather than knowing whether two programs compute \emph{the same} function (which is rarely the case), it is more useful to know whether they compute functions which do not differ \emph{too much}. This has motivated the study of denotational semantics in which types are endowed with a metric, measuring similarity of behavior; this approach has found  applications in e.g.~differential privacy \cite{Reed2010} and coinductive methods \cite{Bonchi2018}, and was recently extended to account for the the full $\lambda$-calculus \cite{Geoffroy2020, PistoneLICS, PistoneFSCD2022}.

On the other hand there is the approach based on \emph{differential} \cite{difflambda} or \emph{resource-aware} \cite{Boudol1993} extensions of the $\lambda$-calculus, which is well-connected to \emph{relational semantics} \cite{Manzo2012, Manzo2013, dill} and has a syntactic counterpart in the study of \emph{non-idempotent} intersection types \cite{decarvalho2018, Mazza2016}. This family of approaches have been exploited to account for higher-order program differentiation \cite{difflambda}, to establish reasonable \emph{cost-models} for the $\lambda$-calculus \cite{Accattoli2021}, and have also been shown suitable for probabilistic settings \cite{Manzo2013, Breuvart2018, PistoneLICS2022}. 


In both approaches the notion of \emph{linearity}, in the sense of linear logic \cite{girardLl} (i.e.~of using its input exactly once), plays a crucial role.
In metric semantics, linear programs correspond to \emph{non-expansive} maps, i.e.~to functions that do not increase distances; moreover, the possibility of duplicating inputs leads to interpret \emph{bounded} programs (i.e.~programs with a fixed duplication bound) as \emph{Lipschitz-continuous} maps \cite{Gaboardi2017}.
By contrast, in the standard semantics of the differential $\lambda$-calculus, linear programs correspond to linear maps, in the usual algebraic sense, while the possibility of duplicating inputs leads to consider functions defined as \emph{power series}.


A natural question is thus whether these two apparently unrelated ways of interpreting linearity and duplication can be somehow connected. At a first glance, there seems to be a  ``logarithmic'' gap between the two approaches:
in metric models $n$ times duplication results in a \emph{linear} (hence Lipschitz) function $n\cdot x$, while in differential models this results in a \emph{polynomial} $x^{n}$, hence not Lipschitz. The fundamental insight of this article is that a natural way to overcome 
this obstacle and bridge the two viewpoints 
is to develop differential semantics in the framework of 
tropical mathematics.
%
%''
%
%s
%emantics a typical ``duplicating'' map is obtained by composing the diagonal with multiplication:
%$$
%\begin{tikzcd}
%\mathbb R \ar{rrr}{x\mapsto \langle x, x\rangle}
% & &  &
% \mathbb R\times \mathbb R 
% \ar{rrr}{\langle x,y\rangle \mapsto x\cdot y}
% & & & \mathbb R
%\end{tikzcd}
%$$
%yielding the square product function $\lambda x.x^{2}$.
%However, in metric semantics this function needs not even exist (as these models are often restricted to Lipschitz-continuous maps \cite{Gabo2017})! Instead, a typical ``duplicating'' map can be obtained by composing the diagonal with the sum 
%$$
%\begin{tikzcd}
%\mathbb R \ar{rrr}{x\mapsto \langle x, x\rangle}
% & &  &
% \mathbb R\times \mathbb R 
% \ar{rrr}{\langle x,y\rangle \mapsto x+y}
% & & & \mathbb R
%\end{tikzcd}
%$$
%yielding the linear (and Lipschitz) function $\lambda x.2x$.
%
%As this example seems to suggest, there seems to be a sort of ``logarithmic'' gap between the two approaches. Can this be made explicit?



\subsection{Tropical mathematics and program semantics } 


Tropical mathematics was introduced in the seventies by the Brazilian mathematician Imre Simon \cite{Simon} as the study of algebra and geometry where one replaces the usual ring structure of numbers based on addition and multiplication by the semi-ring structure given by ``$\min$'' and ``$+$''.
%
%
% interpreting the usual ``$\times$'' and  ``$+$'' operations by  ``$+$'' by ``$\min$''. It can thus be seen as a sort of ``logarithmic'' version of usual geometry (this idea can be made precise via the so-called \emph{Maslov deformation} \cite{}).
%Tropical mathematics is a form of \emph{idempotent} mathematics, since the role of addition is 
%played by the idempotent operation $\min$.
For instance, a quadratic polynomial like e.g.~$p(x)=ax^{2}+bx+c$, when interpreted over the tropical semi-ring, becomes the piecewise linear function
$
\varphi(\alpha)=\min\{2\alpha + a, \alpha+b, c\}
$.

This is not a \emph{ad-hoc} setting: tropical geometry turns out to be a rich and florid combinatorial counterpart of usual algebraic geometry, with important connections with optimisation theory.
Furthermore, computationally speaking, working with $\min$ and $+$ is generally easier than working with standard addition and multiplication; for instance, the important and generally intractable problem of finding roots of a polynomial, admits a linear time algorithm in the tropical case (and the roots of a tropicalized polynomial can be used to approximate the actual roots of the polynomial \cite{Noferini2015}).
Such a computational nature of tropical concepts explains why these are also widely applied in computer science, e.g.~for convex analysis and machine learning (see \cite{Maragos2021} for a recent survey).

Coming back to our previous discussion of program semantics, tropical geometry might seem to provide precisely what we are looking for, as the monomials $x^{n}$ turn precisely into the Lipschitz functions $nx$.
At this point, it is worth mentioning that a tropical variant of relational semantics has already been considered \cite{Manzo2013}, and shown capable of capturing \emph{best-case} quantitative properties, but has not yet been studied in detail. Furthermore, connections between tropical linear algebra and metric spaces (in the abstract setting of \emph{quantale-enriched} categories \cite{Hofmann2014, Stubbe2014}), have also been observed \cite{Fuji}.

The goal of this paper is to study the tropical semantics of the differential $\lambda$-calculus and use it to connect it with metric semantics. This bridge does not only provide a conceptual unification of two different quantitative approaches to higher-order programs, but it also 
suggests ways in which techniques from resource-analysis could be used in sensitivity analysis and \emph{vice-versa}, opening the way for new potential applications of tropical geometry to the study of quantitative properties of higher-order programs.


\subsection{Contributions}

Our contributions in this paper are threefold:
\begin{itemize}

\item We study the relational model over the tropical semi-ring, showing that
\emph{bounded} simply typed terms are interpreted as tropical polynomials, while general
 simply typed lambda terms correspond to a generalization of \emph{tropical Laurent series} \cite{Porzio2021}. Moreover, we show that such maps are locally Lipschitz-continuous, yielding a full-scale metric semantics for the $\lambda$-calculus.
%Moreover, we exploit the differential structure of the relational model to study the \emph{tropical Taylor expansion} of a $\lambda$-term, which can be seen as an approximation of the term by way of Lipschitz-continuous maps.


\item Using the relational model as our main source of intuition, we suggest a few potential applications of tropical methods to the study of quantitative properties of non-deterministic and probabilistic functional programs like counting best-case computation steps, 
measuring convergenge log-probabilities, and 
differential privacy.

\item We conclude by considering a more general tropical setting between differential and metric properties.
By recalling and suitably extending a well-known correspondence between Lawvere's \emph{generalized metric spaces} (GMS) \cite{Lawvere1973, Stubbe2014} and modules over the tropical semi-ring \cite{Russo2007}, we show that the category of \emph{complete}  GMSs provides a model of the differential $\lambda$-calculus which extends the tropical relational model.
\end{itemize}
%
%\section{Bounded and Differential $\lambda$-Calculi}
%
%
%Bounded Simply Typed $\lambda$-calculus $\BSTLC$:
%$$
%A::= o \mid !_{n}A \multimap A
%$$
%
%
%Resource Simply Typed $\lambda$-calculus $\RSTLC$:
%$$
%A::= o \mid [A, \dots , A] \multimap A
%$$
%
%
%Define a translation of types $(-)^{\C R}$ from $\BSTLC$ to $\RSTLC$ by $o^{\C R}=o$ and $(!_{n}A\multimap B)^{\C R}=
%[\underbrace{A^{\C R},\dots, A^{\C R}}_{n\text{ times}}]\multimap B^{\C R}$.
%
%\begin{proposition}
%$\Gamma \vdash_{\BSTLC} M:A$ implies 
%$\Gamma^{\C R}\vdash_{\RSTLC}M:A^{\C R}$.
%\end{proposition}
%


