
In recent years, more and more interest in the programming language community has been directed towards the study of \emph{quantitative} properties of programs like e.g.~computing the number of computation steps, or convergence probabilities, 
as opposed to purely \emph{qualitative} properties like e.g.~termination or program equivalence. 
Notably, a significant effort has been made to extend, or adapt, well-established qualitative methods, like e.g.~type systems, relational logics or denotational semantics, to account for quantitative properties. We can mention, for example, 
intersection type systems aimed at capturing time or space resources \cite{decarvalho2018, Accattoli2022} or convergence probabilities \cite{Breuvart2018, PistoneLICS2022},  relational logics to account for probabilistic properties like e.g.~differential privacy \cite{Barthe_2012} or metric preservation \cite{Reed2010, dallago}, as well as the study of denotational models for 
probabilistic \cite{Ehrhard2011, Staton2017} or differential \cite{difflambda} extensions of the $\lambda$-calculus. 
The main reason to look for methods relying on (quantitative extensions of) type-theory or denotational semantics is that these approaches yield \emph{modular} and \emph{compositional} techniques, that is, allow one to deduce properties of complex programs from the properties of their constituent parts.   

\subsection{Two kinds of quantitative approaches}

Among such quantitative approaches, two different directions have received considerable attention. 

On the one hand one there is the approach of \emph{program metrics} \cite{Reed2010, Gaboardi2017, Gabo2019} and \emph{quantitative equational theories} \cite{Plotk}: when considering probabilistic or approximate computation, rather than asking whether two programs compute \emph{the same} function (which is rarely the case), it makes more sense to ask   whether they compute functions which do not differ \emph{too much}. This has motivated the study of denotational frameworks in which types are endowed with a metric, measuring similarity of behavior; this approach has found  applications in e.g.~differential privacy \cite{Reed2010} and coinductive methods \cite{Bonchi2018}, and was recently extended to account for the the full $\lambda$-calculus \cite{Geoffroy2020, PistoneLICS, PistoneFSCD2022}.

On the other hand, there is the approach based on \emph{differential} \cite{difflambda} or \emph{resource-aware} \cite{Boudol1993} extensions of the $\lambda$-calculus, which is well-connected to the so-called \emph{relational semantics} \cite{Manzo2012, Manzo2013, dill} and has a syntactic counterpart in the study of \emph{non-idempotent} intersection types \cite{decarvalho2018, Mazza2016}. This family of approaches have been exploited to account for higher-order program differentiation \cite{difflambda}, to establish reasonable \emph{cost-models} for the $\lambda$-calculus \cite{Accattoli2021}, and have also been shown suitable for the probabilistic setting \cite{Manzo2013, Breuvart2018, PistoneLICS2022}. 


In both approaches the notion of \emph{linearity}, in the sense of linear logic \cite{girardLl} (i.e.~of using inputs exactly once), plays a crucial role.
In metric semantics, linear programs correspond to \emph{non-expansive} maps, that is, to functions that do not increase distances; moreover, the possibility of duplicating inputs leads to interpret \emph{bounded} programs (i.e.~programs with a fixed duplication bound) as \emph{Lipschitz-continuous} maps \cite{Gaboardi2017}.
By contrast, in the standard semantics of the differential $\lambda$-calculus, linear programs correspond to linear maps, in the usual algebraic sense, while the possibility of duplicating inputs leads to consider functions defined as \emph{power series}.


A natural question is thus whether these two apparently unrelated ways of interpreting linearity and duplication can be somehow reconciled. At a first glance, there seems to be a  ``logarithmic'' gap between the two approaches:
in metric models $n$ times duplication results in a \emph{linear} (hence Lipschitz) function $n\cdot x$, while in differential models this results in a \emph{polynomial} function $x^{n}$, hence not Lipschitz. The fundamental motivation of this work is then the observation that 
this gap is naturally overcome once we interpret these functions in the framework of tropical mathematics, where, as we'll see, the monomial $x^{n}$ ``reads as'' the linear function $n\cdot x$.

% from higher-order programs is based on  soon as one develops  differential semantics in the framework of 
%tropical mathematics.
%
%''
%
%s
%emantics a typical ``duplicating'' map is obtained by composing the diagonal with multiplication:
%$$
%\begin{tikzcd}
%\mathbb R \ar{rrr}{x\mapsto \langle x, x\rangle}
% & &  &
% \mathbb R\times \mathbb R 
% \ar{rrr}{\langle x,y\rangle \mapsto x\cdot y}
% & & & \mathbb R
%\end{tikzcd}
%$$
%yielding the square product function $\lambda x.x^{2}$.
%However, in metric semantics this function needs not even exist (as these models are often restricted to Lipschitz-continuous maps \cite{Gabo2017})! Instead, a typical ``duplicating'' map can be obtained by composing the diagonal with the sum 
%$$
%\begin{tikzcd}
%\mathbb R \ar{rrr}{x\mapsto \langle x, x\rangle}
% & &  &
% \mathbb R\times \mathbb R 
% \ar{rrr}{\langle x,y\rangle \mapsto x+y}
% & & & \mathbb R
%\end{tikzcd}
%$$
%yielding the linear (and Lipschitz) function $\lambda x.2x$.
%
%As this example seems to suggest, there seems to be a sort of ``logarithmic'' gap between the two approaches. Can this be made explicit?



\subsection{Tropical mathematics and program semantics } 


Tropical mathematics was introduced in the seventies by the Brazilian mathematician Imre Simon \cite{Simon} as an alternative approach to algebra and geometry where the usual ring structure of numbers based on addition and multiplication is replaced by the semiring structure given, respectively, by ``$\min$'' and ``$+$''.
%
%
% interpreting the usual ``$\times$'' and  ``$+$'' operations by  ``$+$'' by ``$\min$''. It can thus be seen as a sort of ``logarithmic'' version of usual geometry (this idea can be made precise via the so-called \emph{Maslov deformation} \cite{}).
%Tropical mathematics is a form of \emph{idempotent} mathematics, since the role of addition is 
%played by the idempotent operation $\min$.
For instance, the polynomial $p(x,y)=x^{2}+xy^{2}+y^{3}$, when interpreted over the tropical semiring, translates as the piecewise linear function
$
\varphi(x,y)=\min\{2x, x+2y, 3y\}
$.

%This is not a \emph{ad-hoc} setting: 
In the last decades, tropical geometry evolved into a vast and rich research domain, providing a combinatorial counterpart of usual algebraic geometry, with important connections with optimisation theory \cite{Sturmfelds}.
Computationally speaking, working with $\min$ and $+$ is generally easier than working with standard addition and multiplication; for instance, the fundamental (and generally intractable) problem of finding the roots of a polynomial admits a \emph{linear time} algorithm in the tropical case (and, moreover,  the tropical roots can be used to approximate the actual roots \cite{Noferini2015}).
The computational nature of tropical notions explains why these are so widely applied in computer science, notably for convex analysis and machine learning (see \cite{Maragos2021} for a recent survey).

Coming back to our previous discussion on program semantics, tropical geometry might seem to provide precisely what we are looking for, as it turns the monomials $x^{n}$ into the Lipschitz functions $n\cdot x$.
At this point, it is worth mentioning that a tropical variant of relational semantics has already been considered \cite{Manzo2013}, and shown capable of capturing \emph{best-case} quantitative properties, but has not yet been studied in detail. Furthermore, connections between tropical linear algebra and metric spaces (in the abstract setting of \emph{quantale-enriched} categories \cite{Hofmann2014, Stubbe2014}), have also been observed \cite{Fuji}.

In this paper we demonstrate that the relational interpretation of the $\lambda$-calculus based on tropical mathematics does indeed provides the desired bridge between differential and metric semantics. Moreover, we show that the conceptual unification of these two approaches suggests ways in which techniques from resource-analysis could be used in sensitivity analysis and \emph{vice-versa}, paving the way for new  applications of tropical geometry to the  study of higher-order programs.


\subsection{Contributions}

Our contributions in this paper are threefold:
\begin{itemize}

\item we study the relational model over the tropical semiring  and we show that the functions interpreting simply-typed lambda terms, which correspond to a generalization of \emph{tropical Laurent series} \cite{Porzio2021}, are locally Lipschitz-continuous, thus yielding a full-scale metric semantics for the $\lambda$-calculus and its bounded fragments. This is in Sections \ref{section3} and \ref{section4}.
%Moreover, we exploit the differential structure of the relational model to study the \emph{tropical Taylor expansion} of a $\lambda$-term, which can be seen as an approximation of the term by way of Lipschitz-continuous maps.


\item Using the relational model as our main source of inspiration,  we suggest a few potential applications of tropical methods to the study of quantitative properties of non-deterministic and probabilistic functional programs, like counting best-case computation steps, 
measuring convergence log-probabilities, and 
differential privacy. This is in Section~\ref{section5}

\item We conclude 
by putting the connection between the 
tropical, differential and metric viewpoints at the right level of generality.
By recalling and suitably extending a well-known correspondence between Lawvere's \emph{generalized metric spaces} \cite{Lawvere1973, Stubbe2014} and modules over the tropical semi-ring \cite{Russo2007}, we show that the category of \emph{complete} generalized metric spaces provides a model of the differential $\lambda$-calculus which extends the tropical relational model. This is in Section~\ref{section6}.
\end{itemize}
%
%\section{Bounded and Differential $\lambda$-Calculi}
%
%
%Bounded Simply Typed $\lambda$-calculus $\BSTLC$:
%$$
%A::= o \mid !_{n}A \multimap A
%$$
%
%
%Resource Simply Typed $\lambda$-calculus $\RSTLC$:
%$$
%A::= o \mid [A, \dots , A] \multimap A
%$$
%
%
%Define a translation of types $(-)^{\C R}$ from $\BSTLC$ to $\RSTLC$ by $o^{\C R}=o$ and $(!_{n}A\multimap B)^{\C R}=
%[\underbrace{A^{\C R},\dots, A^{\C R}}_{n\text{ times}}]\multimap B^{\C R}$.
%
%\begin{proposition}
%$\Gamma \vdash_{\BSTLC} M:A$ implies 
%$\Gamma^{\C R}\vdash_{\RSTLC}M:A^{\C R}$.
%\end{proposition}
%


