

As we have seen, the morphisms of $\LREL$ can be seen as continuous functions between the $\Lawv$-modules $\Lawv^{X}$, when the latter are taken with the metric induced by the $\infty$-norm. This viewpoint gives a metric flavor to $\LREL$, and allowed us to relate differential and metric structure. Yet, how far can this correspondence be pushed?
In particular, is this correspondence restricted to $\Lawv$-modules of the form $\Lawv^{X}$ (i.e.~with a fixed base), or does it hold in some sense for arbitrary $\Lawv$-modules? Is this correspondence restricted to the $\infty$-norm metric, or does it hold for other metrics too?

An answer to these questions comes from an elegant categorical correspondence between tropical linear algebra and the theory of \emph{generalized metric spaces}, initiated by Lawvere's pioneering work \cite{Lawvere1973}, and at the heart of the emergent field of \emph{monoidal topology} \cite{Hofmann2014, Stubbe2014}. 
In this section we first reconstruct this correspondence, by combining some {folklore} results with more abstract ones from recent literature in enriched category theory \cite{Fuji, Stubbe2006, Shen2014}. Then we show that this correspondence can be lifted to a model of the full differential $\lambda$-calculus, by suitably generalizing the construction of the Lafont exponential of $\LREL$.

%
%also extends exploit this correspondence to prove that arbitrary $\Lawv$-modules (equivalently, arbitrary \emph{complete} generalized metric spaces) provide a model of the differential $\lambda$-calculus, hence generalizing the tropical relational model.
%

\subsection{$\Lawv$-modules}


A $\Lawv$-module is a triple $(M,\preceq, \star)$ where $(M, \preceq)$ is a sup-lattice, and $\star: \Lawv \times M \to M$ is a continuous (left-)action of $\Lawv$ on it, where continuous means that $\star$ commutes with both joins in $\Lawv$ and in $M$.% (notice that this is indeed the usual notion of module over the Lawvere quantale $\Lawv$).
A $\Lawv$-module homomorphism is a map $f:M\to N$ commuting with both joins and the $\Lawv$-action. We let $\Mod$ indicate the category of $\Lawv$-modules and their homomorphisms. 
 
 
 

$\Lawv$ is the most basic example of $\Lawv$-module.
Any $\Lawv$-module $M$ has a dual $M^{\op}$, with reversed order and (right-)action $x\multimapinv \epsilon= \bigvee\{y\mid \epsilon \star y\succeq x\}$.
 Other basic examples of $\Lawv$-modules are the sets $\Lawv^{X}$, with order and action defined pointwise. 
 
 
 While the $\Lawv^{X}$ have a fixed base, for an arbitrary $\Lawv$-module one can retrieve a base via the \emph{Yoneda embedding}
$\Yon: M \to \Mod(M^{\op},\Lawv)$, where $\Yon(x)(y)=\inf\{\epsilon\mid \epsilon\star y\succeq x\}$. 


\begin{proposition}[cf.~\cite{Stubbe2014, Shen2014}]\label{prop:yonemod}
For any $\Lawv$-module $M$, the Yoneda embedding has a left-adjoint $\sup(f)=\bigvee_{x\in M}f(x)\star x$.
\end{proposition}


Like $\LREL$, the category $\Mod$ has the relevant structure to interpret the linear $\lambda$-calculus:
\begin{proposition}
$\Mod$ is a SMCC.
\end{proposition}
%\begin{proof}
The hom-sets $\Mod(M,N)$ have a natural $\Lawv$-module structure, defined pointwise. The tensor product of $\Lawv$-modules $M\otimes N$ can be defined as the quotient of the usual tensor product of sup-lattices under the smallest congruence containing all pairs $(\{(\epsilon \star x,y)\},\{(x,\epsilon\star y)\})$ (see e.g.~\cite{Russo2007}).
Notably, any element of $M\otimes N$ can be identified with a join of basic tensors $x\otimes y$, corresponding to the equivalence class of the pair $\langle x,y\rangle$.
Beyond the required adjointness of the internal hom and the tensor, one can check that $\Mod$ is actually \emph{$^{*}$-autonomous}, since it satisfies $(M^{\op})^{\op}\simeq M$ and 
$\Hom(M,N^{\op})\simeq (M\otimes N)^{\op}$.
Finally, $\Mod$ has \emph{biproducts}: products and coproducts are both given by the Cartesian product of the underlying posets, with action defined pointwise.
%
%, similarly to the case of sup-lattices \cite{}, as the quotient of the free sup-lattice $\C P(M\times N)$ under the smallest congruence containing all pairs $((\bigvee A,y),\bigcup_{a\in A}\{(a,y)\})$, for $A\subseteq M, y\in N$, 
%$((x,\bigvee B),\bigcup_{b\in B}\{(x,b)\})$, for all $x\in A$, $B\subseteq N$, and all pairws $\{
%\end{proof}


\begin{remark}
The SMCC structure of $\Mod$ coincides with that of $\LREL$ for the modules $\Lawv^{X}$: we already know that $\Mod(\Lawv^{X},\Lawv^{Y})\simeq \Lawv^{X\times Y} $, and 
one can prove $\Lawv^{X}\otimes \Lawv^{Y}\simeq \Lawv^{X\times Y} $ (cf.~\cite{Russo2007}).% and 
%$\Pi_{i\in I}\Lawv^{X_{i}}\simeq \Lawv^{\coprod_{i\in I}X_{i}}$.
\end{remark}



\subsection{$\Lawv$-categories}

Lawvere was the first to observe that a metric space can be described as a \emph{$\Lawv$-enriched} category. Indeed, spelling out the definition, a $\Lawv$-enriched category (in short, a $\Lawv$-category) is given by a set $X$ together with a ``hom-set'' $X(-,-):X\times X\to \Lawv$ satisfying 
\begin{align}
0  & \geq X(x,x) \tag{$\Lawv$-cat 1}\\
X(y,z)+X(x,y)&\geq  X(x,z) \tag{$\Lawv$-cat 2}
\end{align}
This structure is often referred to as a \emph{generalized metric space} \cite{Lawvere1973, Hofmann2014, Stubbe2014}. %, or as a \emph{quasi-metric space}. 
Notice that a $\Lawv$-enriched \emph{functor} between $\Lawv$-categories is just a non-expansive map $f:X\to Y$, as functoriality reads as $Y(f(x),f(y))\leq X(x,y)$.

A $\Lawv$-category is \emph{skeletal} \cite{Stubbe2014} when 
$X(x,y)=0$ implies $x=y$, and 
 \emph{symmetric} when it coincides with its opposite category $X^{\op}(x,y):=X(y,x)$, i.e.~when $X(x,y)=X(y,x)$. 
 A metric space, in the usual sense, is thus the same as a skeletal and symmetric $\Lawv$-category.
  Notice that any $\Lawv$-category $X$ induces a skeletal category $X^{\sk}$ by quotienting points under $X(x,y)=0$, and a symmetric one by letting $X^{\sym}(x,y)=\max\{X(x,y),X^{\op}(x,y)\}$.
 
 $\Lawv$ has a canonical $\Lawv$-enriched structure given by 
 $\Lawv(r,s)=s \menus r$ (where ``$\dotdiv$'' indicates truncated subtraction), and the Euclidean distance coincides with its symmetrization $\Lawv^{\sym}(x,y)$.
 
 
 
 
 

For any $\Lawv$-category $X$, the presheafs $[X^{\op},\Lawv]$ on $X$ form another $\Lawv$-category, with metric $[X^{\op},\Lawv](f,g)= \sup_{x\in X}\Lawv(f(x),g(x))$.
Notice that, when $X$ has the discrete metric, the metric space $[X,\Lawv]^{\sym}$ coincides with $\Lawv^{X}$ with the $\infty$-norm metric.
The \emph{Yoneda embedding} is the faithful functor $\Yon: X\to [X^{\op},\Lawv]$ given by $\Yon(x)(y)=X(y,x)$.



Actually, an important example of $\Lawv$-categories are precisely the $\Lawv$-modules:

\begin{proposition}
Any $\Lawv$-module $(M,\preceq, \star)$ is a $\Lawv$-category via
\begin{align}
M(x,y) = \inf\left\{ \epsilon \mid \epsilon \star x\geq y\right\}
\end{align}
Moreover, a homomorphism of $\Lawv$-modules is a functor of the associated $\Lawv$-categories.
\end{proposition}

Observe that, since the distance $M(x,y)$ coincides with the Yoneda embedding $\B Y$ in $\Mod$, the latter also coincides with the Yoneda embedding of the associated $\Lawv$-category (this justifies the use of a unique symbol for both embeddings).


\subsection{Complete $\Lawv$-categories correspond to $\Lawv$-modules}

Lawvere also observed that Cauchy-completeness can be expressed in categorical language as the representability in $X$ of certain presheaves in $[X^{\op},\Lawv]$ \cite{Lawvere1973, Rutten1998}. Category theory suggests a yet stronger notion of completeness, corresponding to the existence of all \emph{weighted colimits} (for a comparison of different notions of completeness on $\Lawv$-categories, see \cite{Willerton2013, Rutten1998}).

First, let us recall that functors of shape $\Phi: X\times Y^{\op}\to \Lawv$ are called \emph{distributors} and usually noted $\Phi: Y \pfun X$.


\begin{definition}
Let $X,Y,Z$ be $\Lawv$-categories,
$\Phi: Z\pfun Y$ be a distributor and  $f:Y\to X$ be a functor.
A functor $g:Z\to X$ is the \emph{$\Phi$-weighted colimit of $f$ over $X$}, noted $\colim(\Phi,f)$, if for all $z\in Z$ and $x\in X$
\begin{align}
X(g(z),x)= \sup_{y\in Y}\left\{X(f(y),x)\menus \Phi(y,z)\right\}
\end{align} 
A functor $f:X\to Y$ is \emph{continuous} if it commutes with all existing weighted colimits in $X$, i.e.~$f(\colim(\Phi,g))=\colim(\Phi,f\circ g)$. A $\Lawv$-enriched category 
$X$ is said \emph{categorically complete} (or just \emph{complete}) if all weighted colimits over $X$ exist. 
\end{definition}

We let $\GMet$ indicate the category of complete and skeletal $\Lawv$-categories and continuous functors. 

A useful alternative characterization of complete $\Lawv$-categories is the following:
\begin{proposition}[cf.~\cite{Stubbe2006, Shen2014}]
A $\Lawv$-category is complete iff $\Yon$ has a left-adjoint. 
\end{proposition}
\begin{proof}
For one side, if $X$ is complete, one can define $\sup : [X^{\op},\Lawv]\to X$ as a weighted colimit via $X(\sup x,b)=\sup_{a\in X}
X(a,b)-x_{a}$. Conversely, if a left-adjoint $\sup $ exists, one can define $\mathrm{colim}(\Phi,f):= \sup \Psi$, where $\Psi_{a}=\inf_{b\in X}X(a,f(b))+\Phi_{b}$. 
\end{proof}

Indeed, using this fact, together with Proposition \ref{prop:yonemod}, we arrive at the following:
\begin{proposition}
For any $\Lawv$-module, the associated $\Lawv$-category is complete. 
\end{proposition}

Beyond limits of Cauchy sequences, another important example of colimit is the following:
\begin{definition}
Let $X$ be a $\Lawv$-category, $x\in X$ and $\epsilon \in \Lawv$. The \emph{tensor of $x$ and $\epsilon$}, if it exists, is the colimit $\epsilon \otimes x:= \colim( [\epsilon],\Delta x)$, where
$[\epsilon]: \B 1\pfun \B 1$ is the constantly $\epsilon$ distributor
and $\Delta x:\B 1\to X$ is the constant functor. 
\end{definition}

In a complete $\Lawv$-category all tensors exist and give rise to a $\Lawv$-module structure:
\begin{proposition}
Any complete $\Lawv$-category $X$ is a $\Lawv$-module, with order given by $x\preceq_{X}y $ iff $X(y,x)=0$, and 
action given by tensors $\epsilon \otimes x$. Moreover, a continuous functor between complete $\Lawv$-categories is the same as a homomorphism of the associated $\Lawv$-modules. 
\end{proposition}


To conclude our correspondence between $\Lawv$-modules and complete $\Lawv$-categories, it remains to observe that the 
two constructions leading from one structure to the other are one the inverse of the other: for any $\Lawv$-module $(M,\preceq,\star)$,
$x\preceq_{M}y$ iff $M(y,x)=0$ iff $x\preceq y=0\star y$, and, from  
$M(\epsilon \star x, y)= M(x,y)\dotdiv \epsilon$, we deduce $\epsilon\otimes x=\epsilon \star x$. 
Conversely, 
for any complete $\Lawv$-category $X$ and $x,y\in X$, one can check that 
$X(y,x)=\inf\{ \epsilon \mid X(\epsilon\otimes y,x )=0\}$.

This leads to the following:


\begin{theorem}[cf.~\cite{Stubbe2006}]
$\Mod$ and $\GMet$ are isomorphic categories.
\end{theorem}


Since $\Mod$ (and thus $\GMet$ too) is a SMCC, it is worth making its metric structure explicit. Given complete $\Lawv$-categories $X,Y$, we have that the distance on the hom-set $\Mod(X,Y)$ is given by 
\begin{align}
\Mod(X,Y)(f,g)= \sup_{x\in X}Y(f(x),g(x));
\end{align}
and the distance on $X\otimes Y$ is given by
\begin{align}
(X\otimes Y)(\alpha, \beta)=
\sup_{i}\inf_{j}\left\{X(x_{i},x'_{j})+Y(y_{j},y'_{j}\right\},
\end{align}
 where $\alpha=\bigvee_{i}x_{i}\otimes y_{i}$ and 
$\beta=\bigvee_{j}x'_{j}\otimes y'_{j}$, 
from which we deduce that, for basic tensors
$x\otimes y, x'\otimes y'$, their distance is just $X(x,x')+Y(y,y')$.

%
%
%and is extended to arbitrary joins by continuity, i.e.~
%$(X\otimes Y)(\alpha, \beta)=
%\sup_{i}\inf_{j}X(x_{i},x'_{j})+Y(y_{j},y'_{j}),
%$, where $\alpha=\bigvee_{i}x_{i}\otimes y_{i}$ and 
%$\beta=\bigvee_{j}x'_{j}\otimes y'_{j}$.
%%and thus coincides with the sum-metric over basic tensors;
%\item the distance on the bi-product $\Pi_{i\in i}X_{i}$ is given by
%\begin{align}
%(\Pi_{i\in I}X_{i})(f,g)= \sup_{i\in I}X_{i}(f(x),g(x));
%\end{align}

%\end{itemize}







\subsection{Exponential and differential structure}


We now want to show how the correspondence $\Mod\simeq \GMet$ lifts to a model of the differential $\lambda$-calculus, extending the co-Kleisli category $\LREL_{!}$. 

First, we need to define a Lafont exponential $!$ over $\Mod$, and for this we use a well-known recipe from \cite{Mellies2018, Manzo2013}, that is: we first define a symmetric algebra $\Sym_{n}(M)$ as the equalizer of all permutative actions on $n$-tensors $M\otimes \dots \otimes M$; then, exploiting suitable properties of $\Mod$, we may define $!$ as the product of the symmetric algebras.  

For any $\Lawv$-module $M$, $n\in \BB N$ and permutation $\sigma\in \F S_{n}$, define the homomorphism $\langle \sigma\rangle: M^{\otimes_{n}}\to M^{\otimes_{n}}$ by letting 
$\langle\sigma\rangle (x_{1}\otimes \dots \otimes x_{n})=x_{\sigma(1)}\otimes \dots \otimes x_{\sigma(n)}$ on basic tensors, and extending by continuity on the whole tensor module. 


\begin{definition}[symmetric tensor algebra]

For any $\Lawv$-module $M$ and $n\in \BB N$, let $\Sym_{n}(M)$ indicate the $\Lawv$-module obtained by quotienting 
$M^{\otimes_{n}}$ via the least congruence generated by the action $\langle \sigma \rangle$ of permutations $\sigma\in \F S_{n}$.
\end{definition}


To prove that the map $[x]\mapsto x:\Sym_{n}(M)\to M^{\otimes_{n}} $ is the equalizer of the diagram formed by all morphisms $\langle \sigma\rangle$, it is useful to provide an alternative characterization of it. 

\begin{definition}
Let $M$ be a $\Lawv$-module and $n\in \BB N$. An element $x\in M^{\otimes_{n}}$ is said \emph{permutation-invariant} (in short, \emph{p-invariant}) if for all $\sigma\in \F S_{n}$, 
$\langle \sigma \rangle (x)=x$. 
 A \emph{$\Lawv$-multiset} (with $n$ elements) is an element of $M^{\otimes_{n}}$ of the form 
$ [x_{1},\dots, x_{n}]:= \bigvee_{\sigma\in \F S_{n}}
 x_{\sigma(1)}\otimes \dots \otimes x_{\sigma(n)}$, where $x_{1},\dots, x_{n}\in M$.
\end{definition} 

\begin{proposition}
Any $\Lawv$-multiset is p-invariant. Moreover, the set $!_{n}M$ of p-invariant elements of $M^{\otimes_{n}}$ is a $\Lawv$-submodule of $M^{\otimes_{n}}$, whose elements can be written as joins of $\Lawv$-multisets.
\end{proposition}

Since $!_{n}M$ is included in $M$, using the properties of p-invariant one can easily deduce that $!_{n}M$ provides the desired equalizer. It suffices then to show that $!_{n}M$ is isomorphic to the symmetric algebra.


\begin{proposition}
The inclusion morphism $!_{n}M \to M^{\otimes_{n}}$ is the equalizer of the diagram formed by all $M^{\otimes_{n}}\stackrel{\langle \sigma\rangle}{\to} M^{\otimes_{n}}$. Moreover, 
$!_{n}M \simeq \Sym_{n}(M)$.
\end{proposition}


The module $!_{n}M$ is a complete $\Lawv$-category with distance function defined on $\Lawv$-multisets as below:
\begin{align}
(!_{n}M)(\alpha,\beta)=
\sup_{\sigma\in \F S_{n}}\inf_{\tau\in \F S_{n}}\sum_{i=1}^{n}
X(x_{\sigma(i)},y_{\tau(i)})
\end{align}
where $\alpha=[x_{1},\dots, x_{n}]$ and $\beta= [y_{1},\dots, y_{n}]$.




Using the fact that $\prod_{i}M_{i}\otimes  N\simeq (\prod_{i}M_{i})\otimes N$ holds in $\Mod$ (see e.g.~\cite{Russo2007}), we obtain the following:
\begin{theorem}
The functor $!M:= \prod_{n\in \BB N}!_{n}M$ is a Lafont exponential in $\Mod$.
Hence, $\Mod_{!}$ (equivalently, $\GMet_{!}$) is cartesian closed. 
\end{theorem}

Also in the case of exponentials, the constructions for $\Mod$ generalize those of $\LREL$:
\begin{proposition}
$!_{n}(\Lawv^{X})\simeq \Lawv^{\C M_{\leq n}(X)}$, $!(\Lawv^{X})\simeq \Lawv^{\multiset(X)}$. In particular, $\Mod_{!}(\Lawv^{X},\Lawv^{Y})\simeq \LREL_{!}(X,Y)$.
\end{proposition}

To conclude, we mush show that $\Mod$ is a CC$\partial$C.
Firstly, $\Mod$ is a left-additive-CCC, since each hom-set is a commutative monoid with respect to $\infty$ and $\min$, and the cartesian closed structure is well-behaved with respect to that.

Secondly, $\Mod$ can be equipped with a differential operator $E$ defined, for $f:!M\to N$, by 
\begin{align}\label{eq:dermod}
Ef(\alpha)=
\bigvee\left\{
f(\beta\cup [x]) \ \Big \vert  \ 
\iota_{n}(\beta)\otimes \iota_{1}(x) \leq S(\alpha)
\right\}
\end{align}
where $\iota_{k}: M_{k}\to \prod_{i\in I}M_{i}$ is the injection morphism given by $\iota_{k}(x)( k)=x$ and $\iota_{k}(x)(i\neq k)=\infty$,
and $S: !(M\times N)\to !M\otimes !N$ is the Seely isomorphism \cite{Mellies2018}, and $E$ satisfies all required axioms.
%\end{enumerate}

Notice that, when $f\in \Mod_{!}(\Lawv^{X},\Lawv^{Y})\simeq \LREL_{!}(X,Y)$, its derivative $E f$ coincides with the derivative $D_{!}f$ described in Section \ref{section4}. 

All this leads then to:
\begin{theorem}
$\Mod_{!}$ (equivalently, $\GMet_{!}$), equipped with $E$, is a $CC\partial C$.
\end{theorem}




